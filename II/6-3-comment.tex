\documentclass{article}
\usepackage{mathtools} 
\usepackage{fontspec}
\usepackage[UTF8]{ctex}
\usepackage{amsthm}
\usepackage{mdframed}
\usepackage{xcolor}
\usepackage{amssymb}
\usepackage{amsmath}
\usepackage{hyperref}
\usepackage{mathrsfs}


% 定义新的带灰色背景的说明环境 zremark
\newmdtheoremenv[
  backgroundcolor=gray!10,
  % 边框与背景一致,边框线会消失
  linecolor=gray!10
]{zremark}{注释}

% 通用矩阵命令: \flexmatrix{矩阵名}{元素符号}{行数}{列数}
\newcommand{\flexmatrix}[4]{
  \[
  #1 = \begin{pmatrix}
    #2_{11}     & #2_{12}     & \cdots & #2_{1#4}   \\
    #2_{21}     & #2_{22}     & \cdots & #2_{2#4}   \\
    \vdots      & \vdots      & \ddots & \vdots     \\
    #2_{#31}    & #2_{#32}    & \cdots & #2_{#3#4}
  \end{pmatrix}
  \]
}

% 简化版命令(默认矩阵名为A,元素符号为a): \quickmatrix{行数}{列数}
\newcommand{\quickmatrix}[2]{\flexmatrix{A}{a}{#1}{#2}}


\begin{document}
\title{6.3注释}
\author{张志聪}
\maketitle

\begin{zremark}
  命题3.4 (ii) $\Rightarrow$ (iii)省略的步骤。
\end{zremark}

$\alpha = k\epsilon_i + \epsilon_j$代入上式,
\begin{align*}
  (\mathscr{U}(k\epsilon_i + \epsilon_j), \mathscr{U}(k\epsilon_i + \epsilon_j))
   & = (k\mathscr{U} \epsilon_i + \mathscr{U}\epsilon_j, k\mathscr{U} \epsilon_i + \mathscr{U}\epsilon_j)                                                                                                                               \\
   & = k (\mathscr{U} \epsilon_i, k\mathscr{U} \epsilon_i + \mathscr{U}\epsilon_j) + (\mathscr{U}\epsilon_j, k\mathscr{U} \epsilon_i + \mathscr{U}\epsilon_j)                                                                           \\
   & = k \overline{(k\mathscr{U} \epsilon_i + \mathscr{U}\epsilon_j, \mathscr{U} \epsilon_i)} + \overline{(k\mathscr{U} \epsilon_i + \mathscr{U}\epsilon_j, \mathscr{U}\epsilon_j)}                                                     \\
   & = k \overline{(k\mathscr{U} \epsilon_i,  \mathscr{U} \epsilon_i) + (\mathscr{U}\epsilon_j, \mathscr{U} \epsilon_i)} + \overline{(k\mathscr{U} \epsilon_i, \mathscr{U}\epsilon_j) + (\mathscr{U}\epsilon_j, \mathscr{U}\epsilon_j)} \\
   & = k \overline{k + (\mathscr{U}\epsilon_j, \mathscr{U} \epsilon_i)} + \overline{k(\mathscr{U} \epsilon_i, \mathscr{U}\epsilon_j) + 1}                                                                                               \\
   & = k [\overline{k} + (\mathscr{U} \epsilon_i, \mathscr{U}\epsilon_j)] + \overline{k}(\mathscr{U}\epsilon_j, \mathscr{U} \epsilon_i) + 1                                                                                             \\
   & = k \overline{k} + k(\mathscr{U} \epsilon_i, \mathscr{U}\epsilon_j) + \overline{k}(\mathscr{U}\epsilon_j, \mathscr{U} \epsilon_i) + 1
\end{align*}
又
\begin{align*}
  (k\epsilon_i + \epsilon_j, k\epsilon_i + \epsilon_j)
   & = k (\epsilon_i, k\epsilon_i + \epsilon_j) + (\epsilon_j, k \epsilon_i + \epsilon_j)                                                   \\
   & = k \overline{(k\epsilon_i + \epsilon_j, \epsilon_i)} + \overline{(k\epsilon_i + \epsilon_j, \epsilon_j)}                              \\
   & = k \overline{(k \epsilon_i, \epsilon_i) + (\epsilon_j, \epsilon_i)} + \overline{(k\epsilon_i, \epsilon_j) + (\epsilon_j, \epsilon_j)} \\
   & = k \overline{k} + 1
\end{align*}
于是
\begin{align*}
  k \overline{k} + k(\mathscr{U} \epsilon_i, \mathscr{U}\epsilon_j) + \overline{k}(\mathscr{U}\epsilon_j, \mathscr{U} \epsilon_i) + 1 & = k \overline{k} + 1 \\
  k(\mathscr{U} \epsilon_i, \mathscr{U}\epsilon_j) + \overline{k}(\mathscr{U}\epsilon_j, \mathscr{U} \epsilon_i)                      & = 0
\end{align*}

\begin{zremark}
  (ii) $\Rightarrow$ (iii)的证明方法,对正交变换是否成立,
  即命题2.1中(iv)推(iii)
\end{zremark}

设$\mathscr{A} k\epsilon_i + \epsilon_j \in V (i \neq j)$,
\begin{align*}
  (\mathscr{A} k\epsilon_i + \epsilon_j, \mathscr{A} k\epsilon_i + \epsilon_j)
   & = (k\mathscr{A} \epsilon_i + \mathscr{A}\epsilon_j, k\mathscr{A} \epsilon_i + \mathscr{A}\epsilon_j) \\
   & = (k\mathscr{A} \epsilon_i, k\mathscr{A} \epsilon_i)
  + (k\mathscr{A} \epsilon_i, \mathscr{A}\epsilon_j)
  + (\mathscr{A}\epsilon_j, k\mathscr{A} \epsilon_i)
  + (\mathscr{A}\epsilon_j, \mathscr{A}\epsilon_j)                                                        \\
   & = k \cdot k(\mathscr{A} \epsilon_i, \mathscr{A} \epsilon_i)
  + k(\mathscr{A} \epsilon_i, \mathscr{A}\epsilon_j)
  + k(\mathscr{A}\epsilon_j, \mathscr{A} \epsilon_i)
  + (\mathscr{A}\epsilon_j, \mathscr{A}\epsilon_j)                                                        \\
   & = k\cdot k + 2k(\mathscr{A} \epsilon_i, \mathscr{A}\epsilon_j) + 1
\end{align*}
又
\begin{align*}
  (k\epsilon_i + \epsilon_j, k\epsilon_i + \epsilon_j)
   & = (k\epsilon_i, k\epsilon_i)
  + (k\epsilon_i, \epsilon_j)
  + (\epsilon_j, k\epsilon_i)
  + (\epsilon_j, \epsilon_j)      \\
   & = k \cdot k + 1
\end{align*}
于是
\begin{align*}
  k\cdot k + 2k(\mathscr{A} \epsilon_i, \mathscr{A}\epsilon_j) + 1 & = k \cdot k + 1 \\
  (\mathscr{A} \epsilon_i, \mathscr{A}\epsilon_j)                  & = 0
\end{align*}
又因为
\begin{align*}
  (\mathscr{A} \epsilon_i, \mathscr{A}\epsilon_i)
   & = (\epsilon_i, \epsilon_i) \\
   & = 1
\end{align*}
所以$\mathscr{A} \epsilon_1, \epsilon_2, \cdots, \epsilon_n$是标准正交基。

\begin{zremark}
  $(\mathscr{A}\alpha, \beta) = (\alpha, \mathscr{A}^*\beta)$显然等价于
  \begin{align*}
    (\mathscr{A}\epsilon_i, \epsilon_j) = (\epsilon_i, \mathscr{A}^*\epsilon_j) \ \ \ (i, j = 1, 2, \cdots, n)
  \end{align*}
  的原因。
\end{zremark}
任意$\alpha, \beta \in V$,可表示为
\begin{align*}
  \alpha = x_1 \epsilon_1 + x_2 \epsilon_2 + \cdots + x_n \epsilon_n \\
  \beta = y_1 \epsilon_1 + y_2 \epsilon_2 + \cdots + y_n \epsilon_n
\end{align*}
于是
\begin{align*}
  (\mathscr{A}\alpha, \beta)
   & = (\mathscr{A} x_1 \epsilon_1 + x_2 \epsilon_2 + \cdots + x_n \epsilon_n, y_1 \epsilon_1 + y_2 \epsilon_2 + \cdots + y_n \epsilon_n)                       \\
   & = (x_1 \mathscr{A} \epsilon_1 + x_2 \mathscr{A}\epsilon_2 + \cdots + x_n \mathscr{A}\epsilon_n, y_1 \epsilon_1 + y_2 \epsilon_2 + \cdots + y_n \epsilon_n) \\
   & = \sum \limits_{i = 1}^n \sum \limits_{j = 1}^n x_i (\mathscr{A} \epsilon_i, y_j \epsilon_j)                                                               \\
   & = \sum \limits_{i = 1}^n \sum \limits_{j = 1}^n x_i \overline{(y_j \epsilon_j, \mathscr{A} \epsilon_i)}                                                    \\
   & = \sum \limits_{i = 1}^n \sum \limits_{j = 1}^n x_i \overline{y_j (\epsilon_j, \mathscr{A} \epsilon_i)}                                                    \\
   & = \sum \limits_{i = 1}^n \sum \limits_{j = 1}^n x_i \overline{y_j}\overline{(\epsilon_j, \mathscr{A} \epsilon_i)}                                          \\
   & = \sum \limits_{i = 1}^n \sum \limits_{j = 1}^n x_i \overline{y_j}(\mathscr{A} \epsilon_i, \epsilon_j)
\end{align*}
又
\begin{align*}
  (\alpha, \mathscr{A}\beta)
   & = (x_1 \epsilon_1 + x_2 \epsilon_2 + \cdots + x_n \epsilon_n, \mathscr{A}^* y_1 \epsilon_1 + y_2 \epsilon_2 + \cdots + y_n \epsilon_n)                           \\
   & = (x_1 \epsilon_1 + x_2 \epsilon_2 + \cdots + x_n \epsilon_n, y_1 \mathscr{A}^* \epsilon_1 + y_2 \mathscr{A}^*\epsilon_2 + \cdots + y_n \mathscr{A}^*\epsilon_n) \\
   & = \sum \limits_{i = 1}^n \sum \limits_{j = 1}^n x_i (\epsilon_i, \mathscr{A}^* y_j \epsilon_j)                                                                   \\
   & = \sum \limits_{i = 1}^n \sum \limits_{j = 1}^n x_i \overline{(y_j \mathscr{A}^* \epsilon_j, \epsilon_i)}                                                        \\
   & = \sum \limits_{i = 1}^n \sum \limits_{j = 1}^n x_i \overline{y_j(\mathscr{A}^* \epsilon_j, \epsilon_i)}                                                         \\
   & = \sum \limits_{i = 1}^n \sum \limits_{j = 1}^n x_i \overline{y_j}\overline{(\mathscr{A}^* \epsilon_j, \epsilon_i)}                                              \\
   & = \sum \limits_{i = 1}^n \sum \limits_{j = 1}^n x_i \overline{y_j}(\epsilon_i, \mathscr{A}^* \epsilon_j)                                                         \\
\end{align*}
\begin{itemize}
  \item 左边推右边

        取$i = 1, j = 1$,可得$(\mathscr{A}\epsilon_1, \epsilon_1) = (\epsilon_1, \mathscr{A}^*\epsilon_1)$。 \\
        类似地,推导出剩余项。

  \item 右边推左边

        是平凡的结果。
\end{itemize}
综上,命题得证。



\end{document}
