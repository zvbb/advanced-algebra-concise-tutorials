\documentclass{article}
\usepackage{mathtools} 
\usepackage{fontspec}
\usepackage[UTF8]{ctex}
\usepackage{amsthm}
\usepackage{mdframed}
\usepackage{xcolor}
\usepackage{amssymb}
\usepackage{amsmath}
\usepackage{hyperref}
\usepackage{mathrsfs}


% 定义新的带灰色背景的说明环境 zremark
\newmdtheoremenv[
  backgroundcolor=gray!10,
  % 边框与背景一致,边框线会消失
  linecolor=gray!10
]{zremark}{注释}

% 通用矩阵命令: \flexmatrix{矩阵名}{元素符号}{行数}{列数}
\newcommand{\flexmatrix}[4]{
  \[
  #1 = \begin{pmatrix}
    #2_{11}     & #2_{12}     & \cdots & #2_{1#4}   \\
    #2_{21}     & #2_{22}     & \cdots & #2_{2#4}   \\
    \vdots      & \vdots      & \ddots & \vdots     \\
    #2_{#31}    & #2_{#32}    & \cdots & #2_{#3#4}
  \end{pmatrix}
  \]
}

% 简化版命令(默认矩阵名为A,元素符号为a): \quickmatrix{行数}{列数}
\newcommand{\quickmatrix}[2]{\flexmatrix{A}{a}{#1}{#2}}


\begin{document}
\title{6.2}
\author{张志聪}
\maketitle

\section*{1}

\begin{itemize}
  \item (1)

        \begin{itemize}
          \item 方法一:按照定义

                对任意$\alpha, \beta \in V$,我们有
                \begin{align*}
                  (\mathscr{A} \alpha, \mathscr{A} \beta)
                   & = (\alpha - 2(\eta,\alpha)\eta, \beta - 2(\eta, \beta)\eta)                                                                       \\
                   & = (\alpha, \beta) + (\alpha, - 2(\eta, \beta)\eta) + (- 2(\eta,\alpha)\eta, \beta) + (- 2(\eta,\alpha)\eta, - 2(\eta, \beta)\eta) \\
                   & = (\alpha, \beta) - 2(\eta, \beta) (\alpha, \eta) - 2(\eta, \alpha) (\eta, \beta) + 4 (\eta,\alpha)(\eta, \beta) (\eta, \eta)     \\
                   & = (\alpha, \beta) - 2(\eta, \beta) (\alpha, \eta) - 2(\eta, \alpha) (\eta, \beta) + 4 (\eta,\alpha)(\eta, \beta)                  \\
                   & = (\alpha, \beta) - 4(\alpha, \eta) (\eta, \beta) + 4 (\eta,\alpha)(\eta, \beta)                                                  \\
                   & = (\alpha, \beta)
                \end{align*}
                所以,$\mathscr{A}$是正交变换。

          \item 方法二:按照矩阵角度

                按照命题1.5的推论,我们可以在$\eta$的基础上扩充成一组标准正交基,设为
                \begin{align}
                  \eta, \epsilon_2, \epsilon_3, \cdots, \epsilon_n
                \end{align}
                于是,线性变换$\mathscr{A}$在基$(1)$下矩阵$A$为
                \begin{align*}
                  (\mathscr{A}\eta, \mathscr{A}\epsilon_2, \cdots, \mathscr{A}\epsilon_n)
                   & = (\eta - 2(\eta,\eta)\eta, \epsilon_2 - 2(\eta, \epsilon_2)\eta, \cdots, \epsilon_n - 2(\eta, \epsilon_n)\eta) \\
                   & = (\eta, \epsilon_2, \cdots, \epsilon_n) \begin{bmatrix}
                                                                -1 &                    \\
                                                                   & 1                  \\
                                                                   &   & 1              \\
                                                                   &   &   & \ddots     \\
                                                                   &   &   &        & 1
                                                              \end{bmatrix}
                \end{align*}
                于是
                \begin{align*}
                  A^T A = E
                \end{align*}
                可知$A$是正交矩阵,所以$\mathscr{A}$是正交变换。
        \end{itemize}
  \item (2)

        由(1)的方法二的求解过程可知$|A| = -1$,所以$\mathscr{A}$是第二类的。

  \item (3)

        $A^2 = E$,由第四章命题3.7(ii)可知,$\mathscr{A}^2 = \mathscr{E}$。

  \item (4)

        设$\mathscr{B}$在标准正交基$\eta, \epsilon_2, \epsilon_3, \cdots, \epsilon_n$下的矩阵为$B$,
        由于$\mathscr{B}$是正交变换,所以$B$是正交矩阵,
        因为$A$是正交矩阵,则$A^{-1}$也是正交矩阵,设
        \begin{align*}
          B_1 = A^{-1} B
        \end{align*}
        是正交矩阵,所以$B_1$对应的线性变换$\mathscr{B}_1$是正交变换,
        又因为
        \begin{align*}
          |B_1| = |A^{-1}| |B| = (-1) (-1) = 1
        \end{align*}
        所以,$\mathscr{B}_1$是第一类正交变换,
        而且
        \begin{align*}
          A B_1 = B
        \end{align*}
        所以
        \begin{align*}
          \mathscr{A} \mathscr{B_1} = \mathscr{B}
        \end{align*}
\end{itemize}

\section*{2}

我们有
\begin{align*}
  V = V_1 \oplus V_1^\bot
\end{align*}
由于$dim V_1 = n - 1$,所以$dim V_1^\bot = 1$。

任取单位向量$\eta \in V_{1}^\bot$;
在$V_{1}$取一组标准正交基$\epsilon_2, \epsilon_3, \cdots, \epsilon_n$,
合在一起得到$V$的一组标准正交基$\eta, \epsilon_2, \epsilon_3, \cdots, \epsilon_n$。

因为$\epsilon_i \in V_{1} \ (i = 2, 3, \cdots, n)$,所以
\begin{align*}
  \mathscr{A} \epsilon_i = \epsilon_i \ \ \ (i = 2, 3, \cdots, n)
\end{align*}
因为$\mathscr{A}\eta \in V$,可被线性表示为
\begin{align*}
  \mathscr{A}\eta = k_1 \eta + k_2 \epsilon_2 + \cdots + k_n \epsilon_n
\end{align*}
于是,$\mathscr{A}$在标准正交基$\eta, \epsilon_2, \epsilon_3, \cdots, \epsilon_n$下的矩阵为
\begin{align*}
  A = \begin{bmatrix}
        k_1    & 0      & \cdots & 0      \\
        k_2    & 1      & \cdots & 0      \\
        \vdots & \vdots & \ddots & \vdots \\
        k_n    & 0      & \cdots & 1
      \end{bmatrix}
\end{align*}
因为$\mathscr{A}$是正交变换,所以
\begin{align*}
  |A| = \pm 1
\end{align*}
所以$k_1 = \pm 1$。\\
$\mathscr{A}$是正交变换也可得
\begin{align*}
  (\mathscr{A} \eta, \mathscr{A}\eta)
   & = (\eta, \eta)                   \\
   & = 1                              \\
   & = k_1^2 + k_2^2 + \cdots + k_n^2
\end{align*}
所以,$k_2 = k_3 = \cdots = k_n = 0$。\\
如果$k_1 = 1$,则
\begin{align*}
  \mathscr{A} \eta = \eta \in V_1
\end{align*}
出现矛盾。\\
综上,$k_1 = -1$,于是
\begin{align*}
  A = \begin{bmatrix}
        -1     & 0      & \cdots & 0      \\
        0      & 1      & \cdots & 0      \\
        \vdots & \vdots & \ddots & \vdots \\
        0      & 0      & \cdots & 1
      \end{bmatrix}
\end{align*}

因为
\begin{align*}
  \mathscr{A}(\eta) = -\eta = \eta - 2(\eta,\eta) \eta                        \\
  \mathscr{A}(\epsilon_2) = \epsilon_2 = \epsilon_2 - 2(\eta,\epsilon_2) \eta \\
  \vdots                                                                      \\
  \mathscr{A}(\epsilon_n) = \epsilon_n = \epsilon_n - 2(\eta,\epsilon_n) \eta
\end{align*}
两个线性变换相等,只需考察在一组基上向量相等,所以$\mathscr{A}$是关于$\eta$的镜面反射。

\section*{3}

\begin{itemize}
  \item 必要性

        利用题设和$\mathscr{A}$是正交变换,我们有
        \begin{align*}
          (\alpha_i, \alpha_j) = (\mathscr{A} \alpha_i, \mathscr{A} \alpha_j) = (\beta_i, \beta_j)
        \end{align*}

  \item 充分性

        (i)先证明$\alpha_1, \alpha_2, \cdots, \alpha_s$与
        $\beta_1, \beta_2, \cdots, \beta_s$的秩是相同的。

        取$V$的一组基$\epsilon_1, \epsilon_2, \cdots, \epsilon_n$,
        设
        \begin{align*}
          (\alpha_1, \alpha_2, \cdots, \alpha_s) & = (\epsilon_1, \epsilon_2, \cdots, \epsilon_n) A \\
          (\beta_1, \beta_2, \cdots, \beta_s)    & = (\epsilon_1, \epsilon_2, \cdots, \epsilon_n) B
        \end{align*}
        其中$A, B$都是$n \times s$的矩阵。此时问题变为讨论$A$和$B$的秩。\\
        有题设$(\alpha_i, \alpha_j) = (\beta_i, \beta_j)$可知(标准真正交基下内积为对应坐标相乘在相加)
        \begin{align*}
          A^T A = B^T B
        \end{align*}
        又因为(注:2-5-comment.tex中有证明)
        \begin{align*}
          rank(A^T A) = rank(A)
        \end{align*}
        所以
        \begin{align*}
          rank(A) = rank(B)
        \end{align*}

        (ii)令
        \begin{align*}
          U = L(\alpha_1, \alpha_2, \cdots, \alpha_s) \\
          W = L(\beta_1, \beta_2, \cdots, \beta_s)
        \end{align*}
        两者秩相同,可以构造一个同构映射$f: U \to W$为
        对任意$\alpha \in U$,
        \begin{align*}
          \alpha = k_1 \alpha_1 + k_2 \alpha_2 + \cdots + k_s \alpha_s
        \end{align*}
        定义
        \begin{align*}
          f(\alpha) = k_1 \beta_1 + k_2 \beta_2 + \cdots + k_s \beta_s
        \end{align*}
        此时,我们有
        \begin{align*}
          f(\alpha_i) = \beta_i
        \end{align*}

        取$U^{\bot}, W^{\bot}$的一组基标准正交基
        \begin{align*}
          \epsilon_{r + 1}, \epsilon_{r + 2}, \cdots, \epsilon_{n} \\
          \eta_{r + 1}, \eta_{r + 2}, \cdots, \eta_{n}
        \end{align*}
        扩充$f$,对任意$\alpha \in V$,
        \begin{align*}
          \alpha = k_1 \alpha_1 + k_2 \alpha_2 + \cdots + k_s \alpha_s + k_{r + 1} \epsilon_{r + 1} + \cdots + k_n \epsilon_n
        \end{align*}
        定义
        \begin{align*}
          f(\alpha) = k_1 \beta_1 + k_2 \beta_2 + \cdots + k_s \beta_s + k_{r + 1} \eta_{r + 1} + \cdots + k_n \eta_n
        \end{align*}
        验证$f$是正交变换:
        对任意$\alpha', \alpha'' \in V$,
        \begin{align*}
          \alpha' = a_1 \alpha_1 + \cdots + a_s \alpha_s + a_{r + 1} \epsilon_{r + 1} + \cdots + a_n \epsilon_n \\
          \alpha'' = b_1 \alpha_1 + \cdots + b_s \alpha_s + b_{r + 1} \epsilon_{r + 1} + \cdots + b_n \epsilon_n
        \end{align*}
        因为$(\beta_i, \eta_j) = 0 \ \ \ (1 \leq i \leq r, j \geq r + 1)$,于是
        \begin{align*}
          (f(\alpha'), f(\alpha''))
           & = (a_1 \beta_1 + \cdots + a_{r + 1} \eta_{r + 1} + \cdots + a_n \eta_n, b_1 \beta_1 + \cdots + b_{r + 1} \eta_{r + 1} + \cdots + b_n \eta_n) \\
           & = \sum\limits_{i = 1}^r \sum\limits_{j = 1}^r a_i b_j (\beta_i, \beta_j)                                                                     \\
           & = \sum\limits_{i = 1}^r \sum\limits_{j = 1}^r a_i b_j (\alpha_i, \alpha_j)                                                                   \\
           & = (\alpha', \alpha'')
        \end{align*}
        所以$f$就是目标正交变换。
\end{itemize}

\section*{4}

用代数的方法,算出镜面反射定义中的单位向量$\eta$。
假设已有$\mathscr{A}$满足条件,于是
\begin{align*}
  \mathscr{A} \alpha            & = \beta                \\
  \alpha - 2(\eta, \alpha) \eta & = \beta                \\
  \alpha - \beta                & = 2(\eta, \alpha) \eta
\end{align*}
因为$2(\eta, \alpha)$是标量,于是我们可知$\alpha - \beta$、$\eta$向量的方向相同,所以可得
\begin{align*}
  \eta = \pm \frac{\alpha - \beta}{|\alpha - \beta|}
\end{align*}
假设$\eta = \frac{\alpha - \beta}{|\alpha - \beta|}$,则
\begin{align*}
  2(\eta, \alpha) \eta
   & = 2(\frac{\alpha - \beta}{|\alpha - \beta|}, \alpha) \frac{\alpha - \beta}{|\alpha - \beta|}                          \\
   & = 2 \frac{\alpha - \beta}{|\alpha - \beta|^2}(\alpha - \beta, \alpha)                                                 \\
   & = 2 \frac{\alpha - \beta}{|\alpha - \beta|^2}[(\alpha, \alpha) - (\alpha, \beta)]                                     \\
   & = 2 \frac{\alpha - \beta}{(\alpha - \beta)(\alpha - \beta)}[1 - (\alpha, \beta)]                                      \\
   & = 2 \frac{\alpha - \beta}{(\alpha, \alpha) - (\alpha, \beta) - (\beta, \alpha) + (\beta, \beta)}[1 - (\alpha, \beta)] \\
   & = 2 \frac{\alpha - \beta}{1 - (\alpha, \beta) - (\beta, \alpha) + 1}[1 - (\alpha, \beta)]                             \\
   & = 2 \frac{\alpha - \beta}{2 - 2(\alpha, \beta)}[1 - (\alpha, \beta)]                                                  \\
   & = 2 \frac{\alpha - \beta}{2[1 - (\alpha, \beta)]}[1 - (\alpha, \beta)]                                                \\
   & = \alpha - \beta
\end{align*}
所以,可设$\eta = \frac{\alpha - \beta}{|\alpha - \beta|}$,并在$\eta$上扩展出$V$的一组标准正交基$\eta, \epsilon_2, \epsilon_3, \cdots, \epsilon_n$,
$mathscr{A}$是一个镜面反射,只需令:
\begin{align*}
  \mathscr{A} \eta = - \eta           \\
  \mathscr{A} \epsilon_2 = \epsilon_2 \\
  \vdots                              \\
  \mathscr{A} \epsilon_n = \epsilon_n
\end{align*}
且通过构造过程,我们有
\begin{align*}
  \mathscr{A} \alpha = \beta
\end{align*}

\end{document}
