\documentclass{article}
\usepackage{mathtools} 
\usepackage{fontspec}
\usepackage[UTF8]{ctex}
\usepackage{amsthm}
\usepackage{mdframed}
\usepackage{xcolor}
\usepackage{amssymb}
\usepackage{amsmath}
\usepackage{hyperref}
\usepackage{mathrsfs}


% 定义新的带灰色背景的说明环境 zremark
\newmdtheoremenv[
  backgroundcolor=gray!10,
  % 边框与背景一致,边框线会消失
  linecolor=gray!10
]{zremark}{注释}

% 通用矩阵命令: \flexmatrix{矩阵名}{元素符号}{行数}{列数}
\newcommand{\flexmatrix}[4]{
  \[
  #1 = \begin{pmatrix}
    #2_{11}     & #2_{12}     & \cdots & #2_{1#4}   \\
    #2_{21}     & #2_{22}     & \cdots & #2_{2#4}   \\
    \vdots      & \vdots      & \ddots & \vdots     \\
    #2_{#31}    & #2_{#32}    & \cdots & #2_{#3#4}
  \end{pmatrix}
  \]
}

% 简化版命令(默认矩阵名为A,元素符号为a): \quickmatrix{行数}{列数}
\newcommand{\quickmatrix}[2]{\flexmatrix{A}{a}{#1}{#2}}


\begin{document}
\title{6.2注释}
\author{张志聪}
\maketitle

% \begin{zremark}
%   $\mathscr{A}$是欧氏空间中镜面反射的等价定义:

%   \begin{itemize}
%     \item (1) $\mathscr{A}$正交变换;
%     \item (2) $\mathscr{A}^2 = \mathscr{E}$;
%     \item (3) $\mathscr{A}$的特征值-1的重数为1。
%   \end{itemize}
% \end{zremark}

% \begin{itemize}
%   \item 必要性

%         \begin{itemize}
%           \item (1) 习题1-(1)已经证明。

%           \item (2)

%                 设$\mathscr{A}$的法向量为$\eta$,并在$\eta$的基础上扩充成一组标准正交基,
%                 $\mathscr{A}$在这组基下的矩阵为
%                 \begin{align*}
%                   A = \begin{bmatrix}
%                         -1                    \\
%                          & 1                  \\
%                          &   & 1              \\
%                          &   &   & \ddots     \\
%                          &   &   &        & 1
%                       \end{bmatrix}
%                 \end{align*}
%                 于是$A^2 = E$,所以$\mathscr{A}^2 = \mathscr{E}$。

%           \item (3) 习题1-(2)已经证明。
%         \end{itemize}

%   \item 充分性

%         取$V$中的一组标准正交基,
%         $\mathscr{A}$在这组基下的矩阵为$A$,
%         由(1)可知$A$是正交矩阵,所以$A^T A = E$,即
%         $A^{-1} = A^T$。

%         由(2)可知,$AA = E$,所以$A^{-1} = A$,于是可得$A^T = A^{-1} = A$,
%         于是可知$A$是对称的正交矩阵。

%         由于$A$是正交矩阵,所以是满秩的,可知$A$的秩为$n$,
%         任何对称方程都合同于一个对角矩阵,
%         可以通过可逆矩阵$C$,使得$C^T A C$是对角矩阵(对角元素为$\pm 1$),
%         即$C^{T} A C$是$\mathscr{A}$在另一组基$\epsilon_1, \cdots, \epsilon_n$下的矩阵,于是由(3)可知
%         \begin{align*}
%           det(C^{-1} A C) = det(A) = -1
%         \end{align*}
%         于是
%         \begin{align*}
%           B = C^T A C = \begin{bmatrix}
%                           -1                    \\
%                            & 1                  \\
%                            &   & 1              \\
%                            &   &   & \ddots     \\
%                            &   &   &        & 1
%                         \end{bmatrix}
%         \end{align*}
%         而且,我们有
%         \begin{align*}
%           \mathscr{A} \epsilon_1 = - \epsilon_1 \\
%           \mathscr{A} \epsilon_2 = \epsilon_2 \\
%           \vdots \\
%           \mathscr{A} \epsilon_n = \epsilon_n \\
%         \end{align*}
%         所以,$\mathscr{A}$是镜像反射。

% \end{itemize}


\end{document}
