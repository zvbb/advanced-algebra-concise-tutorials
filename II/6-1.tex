\documentclass{article}
\usepackage{mathtools} 
\usepackage{fontspec}
\usepackage[UTF8]{ctex}
\usepackage{amsthm}
\usepackage{mdframed}
\usepackage{xcolor}
\usepackage{amssymb}
\usepackage{amsmath}
\usepackage{hyperref}


% 定义新的带灰色背景的说明环境 zremark
\newmdtheoremenv[
  backgroundcolor=gray!10,
  % 边框与背景一致,边框线会消失
  linecolor=gray!10
]{zremark}{注释}

% 通用矩阵命令: \flexmatrix{矩阵名}{元素符号}{行数}{列数}
\newcommand{\flexmatrix}[4]{
  \[
  #1 = \begin{pmatrix}
    #2_{11}     & #2_{12}     & \cdots & #2_{1#4}   \\
    #2_{21}     & #2_{22}     & \cdots & #2_{2#4}   \\
    \vdots      & \vdots      & \ddots & \vdots     \\
    #2_{#31}    & #2_{#32}    & \cdots & #2_{#3#4}
  \end{pmatrix}
  \]
}

% 简化版命令(默认矩阵名为A,元素符号为a): \quickmatrix{行数}{列数}
\newcommand{\quickmatrix}[2]{\flexmatrix{A}{a}{#1}{#2}}


\begin{document}
\title{6.1}
\author{张志聪}
\maketitle

\section*{1}

$A$是n阶正定矩阵,所以$A$是对称矩阵。

\begin{itemize}
  \item (1)

        (i)
        \begin{align*}
          (k_1 \alpha_1 + k_2 \alpha_2, \beta)
           & = (k_1 \alpha_1 + k_2 \alpha_2) A \beta^T         \\
           & = k_1 \alpha_1 A \beta^T + k_2 \alpha_2 A \beta^T \\
           & = k_1(\alpha_1, \beta) + k_2(\alpha_2, \beta)
        \end{align*}

        (ii)
        \begin{align*}
          (\beta, \alpha) & = [(\beta, \alpha)]^T  \\
                          & = (\beta A \alpha^T)^T \\
                          & = \alpha A^T \beta^T   \\
                          & = \alpha A \beta^T     \\
                          & = (\alpha, \beta)
        \end{align*}
        注意,因为$(\beta, \alpha)$是标量。

        (iii) 因为$A$是正定矩阵,所以
        \begin{align*}
          (\alpha, \alpha) > 0 \ \ \ (\forall \alpha \in \mathbb{R}^n, \alpha \neq 0)
        \end{align*}
        且$(\alpha, \alpha)$当且仅当$\alpha = 0$。

  \item (2)

        因为$A$
        \begin{align*}
          |(\alpha, \beta)|  & \leq |\alpha| \cdot |\beta|                                \\
          |\alpha A \beta^T| & \leq \sqrt{\alpha A \alpha^T} \cdot \sqrt{\beta A \beta^T}
        \end{align*}
        注意,不能写成对应坐标相乘在相加的形式,因为这里的A不一定是单位矩阵。
\end{itemize}

\section*{4}

\begin{itemize}
  \item 必要性

        $\alpha, \beta$正交,于是
        \begin{align*}
          (\alpha + t \beta, \alpha + t \beta)
           & = (\alpha, \alpha) + 2t(\alpha, \beta) + t^2 (\beta, \beta) \\
           & = (\alpha, \alpha) + t^2 (\beta, \beta)                     \\
           & = |\alpha|^2 + t^2 |\beta|^2                                \\
           & \geq |\alpha|^2
        \end{align*}
        因为$|\alpha|> 0$,所以
        \begin{align*}
          |(\alpha + t \beta, \alpha + t \beta)| & \geq |\alpha|
        \end{align*}

  \item 充分性

        反证法,假设$(\alpha, \beta) \neq 0$。

        由题设可知
        \begin{align*}
          |\alpha + t \beta| \geq |\alpha|                                                \\
          (\alpha, \alpha) + 2t(\alpha, \beta) + t^2 (\beta, \beta) \geq (\alpha, \alpha) \\
          (\beta, \beta) t^2 + 2(\alpha, \beta) t \geq 0
        \end{align*}
        右端是关于$t$的二次多项式,于是判别式为
        \begin{align*}
          \Delta = 4(\alpha, \beta)^2 > 0
        \end{align*}
        于是有个实数根,它在两个实数根之间函数值为负,出现矛盾。

\end{itemize}

\section*{5}

\begin{itemize}
  \item (1)

        我们有
        \begin{align*}
          |\alpha + \beta|^2 & = (\alpha+\beta, \alpha+\beta)                         \\
                             & = (\alpha, \alpha) + 2(\alpha, \beta) + (\beta, \beta) \\
                             & = |\alpha|^2 + 2(\alpha, \beta) + |\beta|^2
        \end{align*}
        利用命题1.1的不等式,我们有
        \begin{align*}
          2|(\alpha, \beta)| \leq 2|\alpha| |\beta|
        \end{align*}
        于是
        \begin{align*}
          |\alpha + \beta|^2 \leq |\alpha|^2 + 2|\alpha| |\beta| + |\beta|^2 = (|\alpha| + |\beta|)^2
        \end{align*}
        这表明
        \begin{align*}
          |\alpha + \beta| \leq |\alpha| + |\beta|
        \end{align*}

  \item (2)

        利用(1)
        \begin{align*}
          |\alpha - \gamma| & = |\alpha + \beta - \beta - \gamma|      \\
                            & = |\alpha - \beta + \beta - \gamma|      \\
                            & \leq |\alpha - \beta| + |\beta - \gamma|
        \end{align*}
        这表明
        \begin{align*}
          d(\alpha, \gamma) \leq d(\alpha, \beta) + d(\beta, \gamma)
        \end{align*}
\end{itemize}

\section*{6}

提示:利用命题1.5的证明思想,得到$L(\alpha, \beta, \gamma)$的正交补。

\section*{7}

\begin{itemize}
  \item (1)

        设
        \begin{align*}
          \beta = k_1 \alpha_1 + k_2 \alpha_2 + \cdots + k_n \alpha_n
        \end{align*}
        于是
        \begin{align*}
          (\beta, \beta)
           & = (\beta, k_1 \alpha_1 + k_2 \alpha_2 + \cdots + k_n \alpha_n)                \\
           & = k_1(\beta, \alpha_1) + k_2(\beta, \alpha_2) + \cdots + k_n(\beta, \alpha_n) \\
           & = 0
        \end{align*}
        所以,$\beta = 0$。

  \item (2)

        我们有
        \begin{align*}
          (\beta_1, \alpha_i) = (\beta_2, \alpha_i)     \\
          (\beta_1, \alpha_i) - (\beta_2, \alpha_i) = 0 \\
          (\beta_1 - \beta_2, \alpha_i) = 0
        \end{align*}
        由(1)可知
        \begin{align*}
          \beta_1 - \beta_2 = 0 \\
          \beta_1 = \beta_2
        \end{align*}
\end{itemize}

\section*{15}

\begin{itemize}
  \item 必要性

        由于$\alpha_1, \alpha_2, \cdots, \alpha_s$线性无关,
        令$M = L(\alpha_1, \alpha_2, \cdots, \alpha_s)$,于是$M$是$V$的线性子空间。
        $(\alpha, \beta)$是$M$内一个对称双线性函数,又因为$M$是有限维线性空间,所以
        $(\alpha, \alpha)$是一个正定二次型函数,即$(\alpha, \beta)$在$M$内的任一组基下的矩阵是正定矩阵。
        所以$|D| > 0$。

  \item 充分性

        反证法,假设$\alpha_1, \alpha_2, \cdots, \alpha_s$线性相关,
        即存在不全为0的$k_1, k_2, \cdots, k_s$,使得
        \begin{align*}
          k_1 \alpha_1 + k_2 \alpha_2 + \cdots + k_s \alpha_s = 0
        \end{align*}
        令
        \begin{align*}
          x = \begin{bmatrix}
                k_1    \\
                k_2    \\
                \vdots \\
                \k_s
              \end{bmatrix}
        \end{align*}
        考虑$D$的任意一个行向量$D_i$,我们有
        \begin{align*}
          D_i x
           & = (\alpha_i, \alpha_1) k_1 + (\alpha_i, \alpha_2) k_2 + \cdots + (\alpha_i, \alpha_s) k_s \\
           & = (\alpha_1, k_1 \alpha_1 + k_2 \alpha_2 + \cdots + k_s \alpha_s)                         \\
           & = (\alpha_1, 0)                                                                           \\
           & = 0
        \end{align*}
        于是
        \begin{align*}
          Dx = 0
        \end{align*}
        因为$x \neq 0$,这与$det(D) \neq 0$矛盾。

\end{itemize}

\section*{17}

设
\begin{align*}
  A = \begin{bmatrix}
        a_{11} & a_{12} & \cdots & a_{1n} \\
        0      & a_{22} & \cdots & a_{2n} \\
        \vdots & \vdots & \ddots & \vdots \\
        0      & 0      & \cdots & a_{nn}
      \end{bmatrix}
\end{align*}
于是
\begin{align*}
  A^T A & =
  \begin{bmatrix}
    a_{11} & 0      & \cdots & 0      \\
    a_{12} & a_{22} & \cdots & 0      \\
    \vdots & \vdots & \ddots & \vdots \\
    a_{1n} & a_{2n} & \cdots & a_{nn}
  \end{bmatrix}
  \begin{bmatrix}
    a_{11} & a_{12} & \cdots & a_{1n} \\
    0      & a_{22} & \cdots & a_{2n} \\
    \vdots & \vdots & \ddots & \vdots \\
    0      & 0      & \cdots & a_{nn}
  \end{bmatrix}                                                                       \\
        & =   \begin{bmatrix}
                a_{11}^2     & a_{11}a_{12}              & \cdots & a_{11}a_{1n}                            \\
                a_{12}a_{11} & a_{12}^2 + a_{22}^2       & \cdots & a_{12}a_{1n} + a_{22}a_{2n}             \\
                \vdots       & \vdots                    & \ddots & \vdots                                  \\
                a_{1n}a_{11} & a_{1n}a_{21}+a_{2n}a_{22} & \cdots & a_{1n}^2 + a_{2n}^2 + \cdots + a_{nn}^2
              \end{bmatrix} \\
        & = \begin{bmatrix}
              1      & 0      & \cdots & 0      \\
              0      & 1      & \cdots & 0      \\
              \vdots & \vdots & \ddots & \vdots \\
              0      & 0      & \cdots & 1
            \end{bmatrix}                                                             \\
\end{align*}
考虑第一行,
\begin{align*}
  a_{11} & = \pm 1                   \\
  a_{1i} & = 0 (i = 2, 3, \cdots, n)
\end{align*}
逐行往下,命题得证。

\section*{18}

$|A| \neq 0$,所以$A$是满秩的,
$A$的列向量可以看做欧式空间$\mathbb{R}^n$的一组基。

由命题1.4可知,正交矩阵$Q$可以看做欧式空间$\mathbb{R}^n$的一组标准正交基。
而施密特正交化方法可以让$A$化为一组标准正交基。

为了讨论的方便,符号与教科书中保持一致(P9)。即$A$的列向量组设为
\begin{align*}
  \alpha_1, \alpha_2, \cdots, \alpha_n
\end{align*}
$Q$的列向量组设为
\begin{align*}
  \epsilon_1, \epsilon_2, \cdots, \epsilon_n
\end{align*}
于是,我们有
\begin{align*}
  \alpha_1 & = |\epsilon_1'| \epsilon_1                                                                                       \\
  \alpha_2 & = \frac{(\alpha_2, \epsilon_1')}{(\epsilon_1', \epsilon_1')} |\epsilon_1'| \epsilon_1 + |\epsilon_2'| \epsilon_2 \\
  \alpha_3 & = \frac{(\alpha_3, \epsilon_1')}{(\epsilon_1', \epsilon_1')} |\epsilon_1'| \epsilon_1 +
  \frac{(\alpha_3, \epsilon_2')}{(\epsilon_2', \epsilon_2')} |\epsilon_2'| \epsilon_2 + |\epsilon_3'| \epsilon_3              \\
  \vdots                                                                                                                      \\
  \alpha_n & = \frac{(\alpha_n, \epsilon_1')}{(\epsilon_1', \epsilon_1')} |\epsilon_1'| \epsilon_1 +
  \frac{(\alpha_n, \epsilon_2')}{(\epsilon_2', \epsilon_2')} |\epsilon_2'| \epsilon_2 + \cdots + |\epsilon_n'| \epsilon_n     \\
\end{align*}
令
\begin{align*}
  T = \begin{bmatrix}
        |\epsilon_1'| & \frac{(\alpha_2, \epsilon_1')}{(\epsilon_1', \epsilon_1')} |\epsilon_1'| & \frac{(\alpha_3, \epsilon_1')}{(\epsilon_1', \epsilon_1')} |\epsilon_1'| & \frac{(\alpha_n, \epsilon_1')}{(\epsilon_1', \epsilon_1')} |\epsilon_1'| \\
        0             & |\epsilon_2'|                                                            & \frac{(\alpha_3, \epsilon_2')}{(\epsilon_2', \epsilon_2')} |\epsilon_2'| & \frac{(\alpha_n, \epsilon_2')}{(\epsilon_2', \epsilon_2')} |\epsilon_2'| \\
        0             & 0                                                                        & |\epsilon_3'|                                                            & \frac{(\alpha_n, \epsilon_2')}{(\epsilon_3', \epsilon_3')} |\epsilon_3'| \\
        \vdots        & \vdots                                                                   & \vdots                                                                   & \vdots                                                                   \\
        0             & 0                                                                        & 0                                                                        & |\epsilon_n'|
      \end{bmatrix}
\end{align*}
综上,
\begin{align*}
  A = Q T
\end{align*}


\end{document}
