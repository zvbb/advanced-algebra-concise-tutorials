\documentclass{article}
\usepackage{mathtools} 
\usepackage{fontspec}
\usepackage[UTF8]{ctex}
\usepackage{amsthm}
\usepackage{mdframed}
\usepackage{xcolor}
\usepackage{amssymb}
\usepackage{amsmath}
\usepackage{hyperref}
\usepackage{mathrsfs}


% 定义新的带灰色背景的说明环境 zremark
\newmdtheoremenv[
  backgroundcolor=gray!10,
  % 边框与背景一致,边框线会消失
  linecolor=gray!10
]{zremark}{注释}

% 通用矩阵命令: \flexmatrix{矩阵名}{元素符号}{行数}{列数}
\newcommand{\flexmatrix}[4]{
  \[
  #1 = \begin{pmatrix}
    #2_{11}     & #2_{12}     & \cdots & #2_{1#4}   \\
    #2_{21}     & #2_{22}     & \cdots & #2_{2#4}   \\
    \vdots      & \vdots      & \ddots & \vdots     \\
    #2_{#31}    & #2_{#32}    & \cdots & #2_{#3#4}
  \end{pmatrix}
  \]
}

% 简化版命令(默认矩阵名为A,元素符号为a): \quickmatrix{行数}{列数}
\newcommand{\quickmatrix}[2]{\flexmatrix{A}{a}{#1}{#2}}


\begin{document}
\title{4.3注释}
\author{张志聪}
\maketitle

\begin{zremark}
  使用命题3.1 证明:行变换不改变矩阵的列秩。
\end{zremark}

\textbf{证明:}

$A$是$m \times n$矩阵,设列向量组为
\begin{align*}
  \beta_1, \beta_2, \cdots, \beta_n \ \ (I)
\end{align*}
为每一个行变换都构造一个从$K^m \to K^m$的单射,
这里以初等行变换:第$i$上的$k$倍,加到第$j$行为例(设$A$经过初等行变换得到$B$):
构造映射$f: K^m \to K^m$如下:
若$\beta = \begin{bmatrix}
    a_1    \\
    a_2    \\
    \vdots \\
    a_i    \\
    \vdots \\
    a_j    \\
    \vdots \\
    a_n
  \end{bmatrix}$,那么
\begin{align*}
  f(\beta) = \begin{bmatrix}
               a_1         \\
               a_2         \\
               \vdots      \\
               a_i         \\
               \vdots      \\
               a_j + k a_i \\
               \vdots      \\
               a_n
             \end{bmatrix}
\end{align*}
容易验证,$f$是线性映射且是单射。
首先设$\beta_{i_1}, \beta_{i_2}, \cdots, \beta_{i_k}$是$(I)$的极大线性无关部分组,
于是,由命题3.1可知,$f(\beta_{i_1}), f(\beta_{i_2}), \cdots, f(\beta_{i_k})$是线性无关的。

对任意$\beta \in (I)$,存在
\begin{align*}
  \beta = k_1 \beta{i_1} + k_2 \beta{i_2} + \cdots + k_k \beta{i_k}
\end{align*}
由于$f$是线性映射,于是
\begin{align*}
  f(\beta) = k_1 f(\beta{i_1}) + k_2 f(\beta{i_2}) + \cdots + k_k f(\beta{i_k})
\end{align*}
所以,$f(\beta_{i_1}), f(\beta_{i_2}), \cdots, f(\beta_{i_k})$是$B$中列向量组的极大线性无关部分组。

综上可得,该行变换不改变矩阵的列秩。


\begin{zremark}
  同构是等价关系。
\end{zremark}

\begin{zremark}
  $U$和$V$是数域$K$上的有限维线性空间,$U$和$V$同构的充分必要条件是
  \begin{align*}
    dim U = dim V
  \end{align*}
\end{zremark}

\textbf{证明:}

\begin{itemize}
  \item 必要性

        $U$和$V$,所以存在一个线性映射$f: U \to V$,且$f$是双射。
        由命题3.2可知
        \begin{align*}
          dim U = dim V
        \end{align*}

  \item 充分性

        令$n = dim U = dim V$,设
        \begin{align*}
          \alpha_1, \alpha_2, \cdots, \alpha_n \ \ (I)
        \end{align*}
        是$U$的一组基,
        \begin{align*}
          \beta_1, \beta_2, \cdots, \beta_n \ \ (II)
        \end{align*}
        是$V$的一组基。

        任意$\alpha \in U$,我们有
        \begin{align*}
          \alpha = k_1 \alpha_1 + k_2 \alpha_2 + \cdots + k_n \alpha_n
        \end{align*}
        构造映射$f: U \to V$如下:
        \begin{align*}
          f(\alpha) = k_1 \beta_1 + k_2 \beta_2 + \cdots + k_n \beta_n
        \end{align*}
        易得,$f$是同构映射。
\end{itemize}

\begin{zremark}
  通过基,证明命题3.5在$U, V$是有限维线性空间的情况下,命题成立。
\end{zremark}

\textbf{证明:}

设$Kerf$的一组基为
\begin{align*}
  \alpha_1, \alpha_2, \cdots, \alpha_r
\end{align*}
然后,以此扩充成$U$的一组基
\begin{align*}
  \alpha_1, \alpha_2, \cdots, \alpha_r, \alpha_{r+1}, \cdots, \alpha_n
\end{align*}
其中
\begin{align*}
  \overline{\alpha_{r+1}}, \overline{\alpha_{r+2}}, \cdots, \overline{\alpha_n}
\end{align*}
是$U/Kerf$的一组基。

因为$Kerf$的一组基,在映射$f$下都是零向量,于是不可能是$Imf$的基,
我们考虑
\begin{align*}
  f(\alpha_{r + 1}), f(\alpha_{r + 2}), \cdots, f(\alpha_n)
\end{align*}
设
\begin{align*}
  k_{r + 1} f(\alpha_{r + 1}) + k_{r + 2} f(\alpha_{r + 2}) + \cdots + k_n f(\alpha_n) = 0 \\
  f(k_{r + 1} \alpha_{r + 1} + k_{r + 2} \alpha_{r + 2} + \cdots + k_n \alpha_n) = 0
\end{align*}
可得$k_{r + 1} \alpha_{r + 1} + k_{r + 2} \alpha_{r + 2} + \cdots + k_n \alpha_n \in Kerf$,
所以存在$k_1, k_2, \cdots, k_r$使得
\begin{align*}
  k_{r + 1} \alpha_{r + 1} + k_{r + 2} \alpha_{r + 2} + \cdots + k_n \alpha_n = k_1 \alpha_1 + k_2 \alpha_2 + \cdots + k_r \alpha_r \\
  -k_1 \alpha_1 - k_2 \alpha_2 - \cdots - k_r \alpha_r + k_{r + 1} \alpha_{r + 1} + k_{r + 2} \alpha_{r + 2} + \cdots + k_n \alpha_n = 0
\end{align*}
由于$ \alpha_1, \alpha_2, \cdots, \alpha_r, \alpha_{r+1}, \cdots, \alpha_n$线性无关,
所以
\begin{align*}
  k_1 = k_2 = \cdots = k_n = 0
\end{align*}
所以,$f(\alpha_{r + 1}), f(\alpha_{r + 2}), \cdots, f(\alpha_n)$线性无关。

任意$\beta \in Imf$,存在
\begin{align*}
  \alpha = a_1 \alpha_1 + a_2 \alpha_2 + \cdots + a_n \alpha_n
\end{align*}
使得
\begin{align*}
  \beta & = f(\alpha)                                                                            \\
        & = a_1 f(\alpha_1) + a_2 f(\alpha_2) + \cdots + a_n f(\alpha_n)                         \\
        & = a_{r + 1} f(\alpha_{r + 1}) + a_{r + 2} f(\alpha_{r + 2}) + \cdots + a_n f(\alpha_n)
\end{align*}
所以,$f(\alpha_{r + 1}), f(\alpha_{r + 2}), \cdots, f(\alpha_n)$是$Imf$的一组基。

于是可得
\begin{align*}
  dim U/Kerf = dim Imf = n - r
\end{align*}
所以,$U/Kerf$与$Imf$同构。

\begin{zremark}
  命题3.6(ii),有一点要特别注意:$dim U$是$n$维的,
  那么$\alpha_1, \alpha_2, \cdots, \alpha_n \in V$也必须是$n$个,
  但命题本身没有说不可以重复。
\end{zremark}

\begin{zremark}
  $f: U \to V$是同构映射,那么任意给定基下的矩阵$A = \sigma(f)$都是满秩矩阵。
\end{zremark}

\textbf{证明:}

设$U, V$的一组基分别是
\begin{align*}
  \epsilon_1, \epsilon_2, \cdots, \epsilon_n \\
  \eta_1, \eta_2, \cdots, \eta_n
\end{align*}
于是,我们有
\begin{align*}
  (f(\epsilon_1), f(\epsilon_2), \cdots, f(\epsilon_n)) = (\eta_1, \eta_2, \cdots, \eta_n) A
\end{align*}
因为$f$是同构映射,按照命题3.1(i),$f(\epsilon_1), f(\epsilon_2), \cdots, f(\epsilon_n)$在$V$中线性无关,
于是可得$A$是满秩的,因为如果$A$不是满秩的,那么存在$A$的某个列向量可以被其他列向量线性表示,即
某个$f(\epsilon_i)$可以被其他
$f(\epsilon_1), f(\epsilon_2), \cdots, f(\epsilon_{i-1}), f(\epsilon_{i+1}), \cdots, f(\epsilon_n)$线性表示,
出现矛盾。

\begin{zremark}
  线性变换$\mathscr{P}$是线性空间$V$某个子空间$M$的投影变换的充分必要条件是$\mathscr{P}^2 = \mathscr{P}$。
\end{zremark}

\textbf{证明:}

\begin{itemize}
  \item 必要性

        设$V = M \oplus N$,
        因为$\mathscr{P}$是线性空间$V$对子空间$M$的投影变换,那么$\forall \alpha \in V$,设
        \begin{align*}
          \alpha = \alpha_1 + \alpha_2 \ \ \ (\alpha_1 \in M, \alpha_2 \in N)
        \end{align*}
        我们有
        \begin{align*}
          \mathscr{P}(\alpha) = \alpha_1
        \end{align*}
        又
        \begin{align*}
          \mathscr{P}^2(\alpha) & = \mathscr{P}(\mathscr{P}(\alpha)) \\
                                & = \mathscr{P}(\alpha_1)            \\
                                & = \alpha_1
        \end{align*}
        所以
        \begin{align*}
          \mathscr{P}^2 = \mathscr{P}
        \end{align*}

  \item 充分性

        令
        \begin{align*}
          M & = Im\mathscr{P}  \\
          N & = Ker\mathscr{P}
        \end{align*}
        由3.5推论1,我们有
        \begin{align*}
          dim M + dim N = dim Im\mathscr{P} + (dim V - dim Im\mathscr{P}) = dim V
        \end{align*}
        又因为$Ker\mathscr{P}, Im\mathscr{P}$都是$V$的子空间,所以$U, V$是直和。

        任意$\alpha \in V$,令
        \begin{align*}
          \alpha = \mathscr{P}(\alpha) + (\alpha - \mathscr{P}(\alpha))
        \end{align*}
        显然$\mathscr{P}(\alpha) \in Im \mathscr{P}$,因为
        \begin{align*}
          \mathscr{P}(\alpha - \mathscr{P}(\alpha))
          & = \mathscr{P}(\alpha) - \mathscr{P}(\mathscr{P}(\alpha)) \\
          & = \mathscr{P}(\alpha) - \mathscr{P}(\alpha) \\
          & = 0 
        \end{align*}
        所以,$\alpha - \mathscr{P}(\alpha) \in Ker\mathscr{P}$。

        综上
        \begin{align*}
          V = Im\mathscr{P} \oplus Ker\mathscr{P} = M \oplus N
        \end{align*}

        由以上的讨论可知,任意$\alpha \in V$,有
        \begin{align*}
          \alpha = \alpha_1 + \alpha_2 \ \ \ (\alpha_1 \in M, \alpha_2 \in N)
        \end{align*}
        且
        \begin{align*}
          \mathscr{P}( \alpha ) = \alpha_1
        \end{align*}
        所以,$\mathscr{P}$是线性空间对子空间$M$的投影变换。
\end{itemize}

\begin{zremark}
  互补投影变换:$\mathscr{E} - \mathscr{P}$
\end{zremark}

\textbf{证明:}

设$\mathscr{P}$是线性空间$V$对子空间$M$(关于直和分解式$V = M \oplus N$)的投影变换。

任意$\alpha \in V$,设
\begin{align*}
  \alpha = \alpha_1 + \alpha_2 \ \ \ (\alpha_1 \in M, \alpha_2 \in N)
\end{align*}
我们有
\begin{align*}
  (\mathscr{E} - \mathscr{P})(\alpha) 
  & = \mathscr{E}(\alpha) - \mathscr{P}(\alpha) \\
  & = \alpha_1 + \alpha_2 - \alpha_1 \\
  & = \alpha_2
\end{align*}

\begin{zremark}
  命题3.8 的一般情况(即线性映射)。
\end{zremark}

\begin{zremark}
  命题3.9 的一般情况(即线性映射)。
\end{zremark}

\end{document}
