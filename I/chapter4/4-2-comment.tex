\documentclass{article}
\usepackage{mathtools} 
\usepackage{fontspec}
\usepackage[UTF8]{ctex}
\usepackage{amsthm}
\usepackage{mdframed}
\usepackage{xcolor}
\usepackage{amssymb}
\usepackage{amsmath}
\usepackage{hyperref}


% 定义新的带灰色背景的说明环境 zremark
\newmdtheoremenv[
  backgroundcolor=gray!10,
  % 边框与背景一致,边框线会消失
  linecolor=gray!10
]{zremark}{注释}

% 通用矩阵命令: \flexmatrix{矩阵名}{元素符号}{行数}{列数}
\newcommand{\flexmatrix}[4]{
  \[
  #1 = \begin{pmatrix}
    #2_{11}     & #2_{12}     & \cdots & #2_{1#4}   \\
    #2_{21}     & #2_{22}     & \cdots & #2_{2#4}   \\
    \vdots      & \vdots      & \ddots & \vdots     \\
    #2_{#31}    & #2_{#32}    & \cdots & #2_{#3#4}
  \end{pmatrix}
  \]
}

% 简化版命令(默认矩阵名为A,元素符号为a): \quickmatrix{行数}{列数}
\newcommand{\quickmatrix}[2]{\flexmatrix{A}{a}{#1}{#2}}


\begin{document}
\title{4.2注释}
\author{张志聪}
\maketitle

\begin{zremark}
  \[
    M_i \cap (\sum \limits_{j = 1; j \neq i}^n{M_j}) = \{0\}  \ (i = 1, 2, 3, \cdots, n)
  \]
  则
  \[
    M_1 \cap M_2 \cap M_3 \cap \cdots \cap M_n = \{0\}
  \]
  反之,不成立。
\end{zremark}

\textbf{证明:}

假设
\[
  M_1 \cap M_2 \cap M_3 \cap \cdots \cap M_n \neq \{0\}
\]
即存在非零向量$\alpha \in M_1 \cap M_2 \cap M_3 \cap \cdots \cap M_n$,
于是
\begin{align*}
  \alpha \in M_i \cap (\sum \limits_{j = 1; j \neq i}^n{M_j}) \ (i = 1, 2, 3, \cdots, n)
\end{align*}
存在矛盾。

反之,只需举一个反例,假设只有$M_k, M_r$存在向量$\alpha$,
且满足
\begin{align*}
  M_1 \cap M_2 \cap M_3 \cap \cdots \cap M_n = \{0\}
\end{align*}
对于
\begin{align*}
  M_k \cap (\sum \limits_{j = 1; j \neq i}^n{M_j}) = \{0, \alpha\}
\end{align*}

\begin{zremark}
  直和的结合律:
  \begin{align*}
    (A \oplus B) \oplus C = A \oplus B \oplus C = A \oplus (B \oplus C)
  \end{align*}
\end{zremark}

\textbf{证明:}

首先$(A + B) + C = A + B + C = A + (B + C)$是显然的。
我们需要着重说明的是直和关系的成立。

\begin{itemize}
  \item 已知$(A \oplus B) \oplus C$可得$A \oplus B \oplus C$。

        $(A \oplus B) \oplus C$,由定理2.2(iii),我们有
        \begin{align*}
          A \cap B = \{0\} \\
          (A + B) \cap C = \{0\}
        \end{align*}
        为了证明$A \oplus B \oplus C$,我们利用定理2.3的(iii),
        即
        \begin{align*}
          A \cap (B + C) = \{0\} \\
          B \cap (A + C) = \{0\} \\
          C \cap (A + B) = \{0\}
        \end{align*}
        最后一条是已知的,且前两条是对称的,所以只需证明一条。

        $\forall \alpha \in A \cap (B + C)$,我们有
        \begin{align*}
          \alpha = a = b + c \\
          a - b = c
        \end{align*}
        因为$a - b \in A + B, c \in C$,利用$(A + B) \cap C = \{0\} $可知
        \begin{align*}
          a - b = c = 0 \\
          a = b         \\
          c = 0
        \end{align*}
        又$A \cap B = \{0\}$,所以
        \begin{align*}
          a = b = 0
        \end{align*}
        所以
        \begin{align*}
          \alpha = a = 0
        \end{align*}
        综上可得
        \begin{align*}
          A \cap (B + C) = \{0\}
        \end{align*}
        所以,$A + B + C$是直和。

  \item 已知$A \oplus B \oplus C$可得$(A \oplus B) \oplus C$。

        为了证明$(A \oplus B) \oplus C$,我们利用定理2.2的(i),
        反证法,假设存在同一向量在$(A \oplus B) \oplus C$中有两种不同的表示。
        \begin{align*}
          \alpha = (a_1 + b_1) + c_1 \\
          \alpha = (a_2 + b_2) + c_2
        \end{align*}
        两式相减得
        \begin{align*}
          (a_1 + b_1) + c_1 - [(a_2 + b_2) + c_2] = 0 \\
          (a_1 - a_2) + (b_1 - b_2) + (c_1 - c_2) = 0
        \end{align*}
        因为$A \oplus B \oplus C$,利用定理2.3(ii)可知
        \begin{align*}
          a_1 - a_2 = 0 \\
          b_1 - b_2 = 0 \\
          c_1 - c_2 = 0
        \end{align*}
        即
        \begin{align*}
          a_1 = a_2 \\
          b_1 = b_2 \\
          c_1 = c_2
        \end{align*}
        与假设矛盾,故$(A + B) + C$中任意向量表法唯一,
        所以$(A + B) + C$是直和。
\end{itemize}
类似地,可证$A \oplus B \oplus C = A \oplus (B \oplus C)$。

\begin{zremark}
  有限多个真子空间$V_i (i = 1, 2, 3, \cdots, n)$做并,无法得到整个线性空间$V$。
\end{zremark}

\textbf{证明:}

对$n$进行归纳。

$n = 1$时,因为$V_1 \subset V$,命题显然成立。

归纳假设$n = m - 1$时,命题成立。

$n = m$时,由归纳假设可知,存在$\alpha \in V$使得
\begin{align*}
  \alpha \not \in V_i \ (i = 1, 2, \cdots, m - 1)
\end{align*}
因为$V_m \subset V$,所以存在$\beta \not \in V_m$。\\
所以$\alpha, \beta$不能同时属于$V_i(i = 1, 2, \cdots, m)$。
进而$\alpha + k_i\beta, \alpha + k \beta $不会同时属于$V_i(i = 1, 2, \cdots, m)$,
如果
\begin{align*}
  \alpha + k_i\beta \in V_i \\
  \alpha + k \beta \in V_i
\end{align*}
那么
\begin{align*}
  (\alpha + k \beta) - (\alpha + k_i\beta)
   & = (k - k_i)\beta
\end{align*}
所以$\beta \in V_i$。\\
又由
\begin{align*}
  \alpha + \beta - \beta = \alpha
\end{align*}
可知$\alpha \in V_i$,出现矛盾。\\
因此若 $\alpha + k_i \beta \in V_i$,则对任意 $k \neq k_i$,必有 $\alpha + k\beta \notin V_i$。

由于数域是无限的,$k_i$ 至多只有 $m$ 个不同的值,我们可以从数域中选取一个
\[
  k \notin \{k_1, k_2, \cdots, k_m\},
\]
从而保证 $\alpha + k \beta \notin V_i \ (i=1,2,\cdots,m)$。

故
\[
  V_1 \cup V_2 \cup \cdots \cup V_m \neq V。
\]

归纳完成,命题得证。

\begin{zremark}
  向量的模M同余和数的模$m$同余的类比。
\end{zremark}

正整数中,
\begin{align*}
  a \equiv b (mod \ m)
\end{align*}
表示$a,b$除以数$m$的余数相同,即
\begin{align*}
  a - b \in \{km | k \in \mathbb{Z}\}
\end{align*}

向量中,
\begin{align*}
  \alpha \equiv \beta (mod \ M)
\end{align*}
按照定义有
\begin{align*}
  \alpha - \beta \in M
\end{align*}

\begin{zremark}
  一个线性空间$V$的商空间$V/M$是不是唯一的。
\end{zremark}

\textbf{证明:}

设$\overline{V} \neq \overline{W}$都是$V$的商空间,
所以存在同余类$\overline{\alpha} \in \overline{V}, \overline{\alpha} \notin \overline{W}$。

又因为$\alpha \in V$,由于商空间是$V$内向量模$M$同余类的全体所成的集合,所以
$\alpha$的模$M$同余类$\overline{\alpha}$有:
\begin{align*}
  \overline{\alpha} \in \overline{V}, \overline{\alpha} \in \overline{W}
\end{align*}
存在矛盾。

\begin{zremark}
  商空间的维数是不是就是其元素个数。
\end{zremark}

数模m的同余类是有限的,比如整数$5$,它的
余数只有$0, 1, 2, 3, 4$。受此启发,那么商空间元素是不是也是有限的。

先给出结论:不是有限的。

只要是线性空间,只要不是零空间(即只有零元素的空间),那么元素个数一定是无限个的。

\begin{zremark}
  商空间作为工具,通常用来干什么?
\end{zremark}

降维,把不关心的维度放入模$M$中。

\begin{zremark}
  商空间与补空间的关系。
\end{zremark}

一个线性空间$V$,它的补空间不唯一,
商空间是唯一的。

\end{document}
