\documentclass{article}
\usepackage{mathtools} 
\usepackage{fontspec}
\usepackage[UTF8]{ctex}
\usepackage{amsthm}
\usepackage{mdframed}
\usepackage{xcolor}
\usepackage{amssymb}
\usepackage{amsmath}
\usepackage{hyperref}
\usepackage{mathrsfs}



% 定义新的带灰色背景的说明环境 zremark
\newmdtheoremenv[
  backgroundcolor=gray!10,
  % 边框与背景一致,边框线会消失
  linecolor=gray!10
]{zremark}{注释}

% 通用矩阵命令: \flexmatrix{矩阵名}{元素符号}{行数}{列数}
\newcommand{\flexmatrix}[4]{
  \[
  #1 = \begin{pmatrix}
    #2_{11}     & #2_{12}     & \cdots & #2_{1#4}   \\
    #2_{21}     & #2_{22}     & \cdots & #2_{2#4}   \\
    \vdots      & \vdots      & \ddots & \vdots     \\
    #2_{#31}    & #2_{#32}    & \cdots & #2_{#3#4}
  \end{pmatrix}
  \]
}

% 简化版命令(默认矩阵名为A,元素符号为a): \quickmatrix{行数}{列数}
\newcommand{\quickmatrix}[2]{\flexmatrix{A}{a}{#1}{#2}}


\begin{document}
\title{4.4}
\author{张志聪}
\maketitle

\section*{3}

(i)$A$的特征多项式为
\begin{align*}
  f(\lambda) = \det(\lambda E - A)
   & = \begin{vmatrix}
         \lambda & -1      &        &         \\
                 & \lambda & -1     &         \\
                 &         & \ddots & \ddots  \\
                 &         &        & \lambda \\
       \end{vmatrix}
   & = \lambda^n
\end{align*}
因为
\begin{align*}
  f(\lambda) = 0 \\
  \lambda^n = 0  \\
  \lambda = 0
\end{align*}
所以,它的特征根仅有0。

(ii)求$\lambda_0 = 0$对应的特征向量.
\begin{align*}
  \lambda E - A = \begin{bmatrix}
                    0 & -1 &        &        \\
                      & 0  & -1     &        \\
                      &    & \ddots & \ddots \\
                      &    &        & 0      \\
                  \end{bmatrix}
\end{align*}
这个齐次线性方程组中,仅有$x_1$是自由未知量,取$x_1 = 1$,得基础解系
\begin{align*}
  \eta_1 = \begin{bmatrix}
             1      \\
             0      \\
             \vdots \\
             0      \\
           \end{bmatrix}
\end{align*}
它对应于$A$的特性向量
\begin{align*}
  x_1 \epsilon_1 + x_2 \epsilon_2 + \cdots + x_n \epsilon_n = \epsilon_1
\end{align*}
于是$V_{\lambda_0} = L(\epsilon)$。

\section*{7}

 (1)反证法,假设存在$\lambda \neq 0$是$\mathscr{A}$的特征值。
于是,存在$\alpha \neq 0$使得
\begin{align*}
  \mathscr{A} \alpha = \lambda \alpha
\end{align*}
于是
\begin{align*}
  \mathscr{A}^k (\alpha)
   & = \mathscr{A}^{k-1}(\mathscr{A} \alpha) \\
   & = \mathscr{A}^{k-1}(\lambda \alpha)     \\
   & = \cdots                                \\
   & = \lambda^k \alpha                      \\
   & \neq 0
\end{align*}
这与题设$\mathscr{A}^k = 0$矛盾。

(2)接下来,证明0是$\mathscr{A}$的特征值。

对任意$\alpha \in V$,我们有
\begin{align*}
  \mathscr{A}^k (\alpha) = 0
\end{align*}
于是,存在$l (1 \leq l \leq k)$使得
\begin{align*}
  \mathscr{A}^{l - 1}(\alpha) \neq 0 \\
  \mathscr{A}^l(\alpha) = \mathscr{A}(\mathscr{A}^{l - 1}(\alpha)) = 0 = 0 \cdot \mathscr{A}^{l - 1}(\alpha)
\end{align*}
所以,$0$是$\mathscr{A}$的特征值,$\mathscr{A}^{l - 1}(\alpha)$是对应的特征向量。

\section*{8}

反证法,假设$\xi_1 + \xi_2$是某个特征向量,那么
存在$\lambda \in K$使得
\begin{align*}
  \mathscr{A} (\xi_1 + \xi_2) = \lambda (\xi_1 + \xi_2)           \\
  \mathscr{A} \xi_1 + \mathscr{A} \xi_2 = \lambda (\xi_1 + \xi_2) \\
  \lambda_1 \xi_1 + \lambda_2 \xi_2 = \lambda (\xi_1 + \xi_2)     \\
  (\lambda_1 - \lambda) \xi_1 + (\lambda_2 - \lambda) \xi_2 = 0
\end{align*}
由命题4.3可知,$\xi_1, \xi_2$是线性无关的,所以
\begin{equation*}
  \begin{cases*}
    \lambda_1 - \lambda = 0 \\
    \lambda_2 - \lambda = 0
  \end{cases*}
\end{equation*}
于是可得
\begin{align*}
  \lambda = \lambda_1 \\
  \lambda = \lambda_2
\end{align*}
有题设可知$\lambda_1 \neq \lambda_2$,出现矛盾,
假设不成立,命题得证。

\section*{8 推广}

$k \xi_1 + l \xi_2$不是$\mathscr{A}$的特征向量。

证明:

因为$k \xi_1 \in V_{\lambda_1}, l \xi_2 \in V_{\lambda_2}$,
利用习题8,命题得证。

\section*{9}

只需证明,这些特征向量属于同一个特征子空间即可。

反证法,假设$\alpha \in V_{\lambda_1}, \beta \in V_{\lambda_2}$,
由习题8可知,
\begin{align*}
  \alpha + \beta
\end{align*}
不是特征向量,与题设矛盾。

\section*{10}

$\mathscr{A}$是线性空间$V$内的可逆线性变换,即存在线性变换$\mathscr{B}$使得
\begin{align*}
  \mathscr{A} \mathscr{B} = \mathscr{B} \mathscr{A} = \mathscr{E}
\end{align*}

任取线性变换$\mathscr{A}, \mathscr{A}$在基$\epsilon_1, \epsilon_2, \cdots, \epsilon_n$
下的矩阵为$A, B$,
我们有
\begin{align*}
  \sigma(AB) = \sigma(A) \sigma(B) = AB = E
\end{align*}
于是可得
\begin{align*}
  rank(A) = rank(B) = rank(E) = n
\end{align*}

\begin{itemize}
  \item (1)

        \begin{itemize}
          \item 方法一

                反证法,假设线性变换$\mathscr{A}$存在为零的特征值,
                那么,我们有
                \begin{align*}
                  |\lambda E - A| = 0 \\
                  |-A| = 0
                \end{align*}
                $A$不满秩,出现矛盾。

          \item 方法二

                $\mathscr{A}$是可逆线性变换,所以它是双射。

                反证法,假设线性变换$\mathscr{A}$存在为零的特征值,
                于是存在$\alpha \in V$使得
                \begin{align*}
                  \mathscr{A} \alpha = 0 \cdot \alpha = 0 \\
                  \mathscr{A} 0 = 0
                \end{align*}
                与$\mathscr{A}$是双射矛盾。
        \end{itemize}

  \item (2)

        $\lambda$是$\mathscr{A}$的特征值,那么,存在$\alpha \in V$使得
        \begin{align*}
          \mathscr{A} \alpha = \lambda \alpha \\
        \end{align*}
        两边同时代入$\mathscr{A}^{-1}$,我们有
        \begin{align*}
          \mathscr{A}^{-1} \mathscr{A} \alpha = \mathscr{A}^{-1} \lambda \alpha \\
          \alpha = \lambda \mathscr{A}^{-1} \alpha
        \end{align*}
\end{itemize}

\section*{10 推广}

 (1)的反方向也是成立的,即:
$\mathscr{A}$的特征值都不为零,则$\mathscr{A}$是可逆线性变换。

证明:

$\mathscr{A}$的在基$\epsilon_1, \epsilon_2, \cdots, \epsilon_n$下的矩阵为$A$,
则$A$的全体特征根的乘积,我们有
\begin{align*}
  \lambda_1 \lambda_2 \cdots \lambda_n = |A| \neq 0
\end{align*}
所以,$A$是满秩。

于是存在$A^{-1}$使得
\begin{align*}
  A^{-1} A = A A^{-1} = E
\end{align*}
于是,由$\sigma: End(V) \to M_n(K)$的双射性,
存在线性变换$\mathscr{B}$在基下的矩阵为$A^{-1}$,所以
\begin{align*}
  E = \sigma(\mathscr{A}) \sigma(\mathscr{B}) = \sigma(\mathscr{A}\mathscr{B}) \\
\end{align*}
可得
\begin{align*}
  \mathscr{A} \mathscr{B} = \mathscr{B} \mathscr{A} = \mathscr{E}
\end{align*}

\section*{11}

提示:P204
\begin{align*}
  A B = B \begin{bmatrix}
            f(\epsilon_1) &               &               &        &                \\
                          & f(\epsilon_2) &               &        &                \\
                          &               & f(\epsilon_3) &        &                \\
                          &               &               & \ddots &                \\
                          &               &               &        & f^{\epsilon_n} \\
          \end{bmatrix}
\end{align*}

\section*{13}

\begin{itemize}
  \item (1)

        \begin{align*}
          \sigma(\mathscr{A}\mathscr{B})
           & = \sigma(\mathscr{A})\sigma(\mathscr{B}) \\
           & = A A^{*}                                \\
           & = |A| E
        \end{align*}
        同时
        \begin{align*}
          \sigma(\mathscr{B}\mathscr{A})
           & = \sigma(\mathscr{B}) \sigma(\mathscr{A}) \\
           & = A^{*} A                                 \\
           & = |A| E
        \end{align*}
        所以
        \begin{align*}
          \sigma(\mathscr{A}\mathscr{B}) = \sigma(\mathscr{B}\mathscr{A})
        \end{align*}

  \item (2)

        问题等价于
        \begin{align*}
          A^{*} x = 0
        \end{align*}
        的解空间的维度和一组基。

        由$0$是$A$的特征值,所以
        \begin{align*}
          |0 E - A| = 0 \\
          |A| = 0
        \end{align*}
        所以,$A$不是满秩的,即$rank(A) < n$。

        利用第三章$\S 3$习题6可知:

        (i) $rank(A) < n - 1$,则$rank(A^*) = 0$,
        于是$A^{*} x$的解空间的维数为$n = n - rank(A^*)$,
        $\epsilon_1, \epsilon_2, \cdots, \epsilon_n$为其一组基。

        (ii) $rank(A) = n - 1$,则$rank(A^*) = 1$,
        于是$A^{*} x$的解空间的维数为$n - 1 = n - rank(A^*)$,
        于是可得$ker(\mathscr{B}) = n - 1$,
        接下来,证明$Ker(\mathscr{B}) = Im(\mathscr{A})$。

        由$x \in V, (A^* A)x = A^* (Ax) = (|A| E)x = 0 $可知,任意$, A(A^* x) = 0$,
        于是可得
        \begin{align*}
          Im(\mathscr{A}) \subseteq \ker(\mathscr{B})
        \end{align*}
        由命题3.5的推论1,我们有
        \begin{align*}
          Im(\mathscr{A}) + ker(\mathscr{A}) = Im(\mathscr{A}) + 1 = n
        \end{align*}
        综上可得
        \begin{align*}
          \ker(\mathscr{B}) = Im(\mathscr{A}) = n - 1
        \end{align*}
\end{itemize}

\section*{16}

设题中的矩阵为$A$。

反证法,假设存在不变子空间爱你$N$使得
\begin{align*}
  V = M \oplus N
\end{align*}
于是,由命题4.5可知,存在一组基,使得$\mathscr{A}$在这组基下的矩阵成准对角的,设为
\begin{align*}
  B = \begin{bmatrix}
        A_1 & 0   \\
        0   & A_2
      \end{bmatrix}
\end{align*}
线性变换$\mathscr{A}$在不同基下的特征多项式相同,所以
\begin{align*}
  |\lambda E - B|
   & = |\lambda E - A_1| |\lambda E - A_2| \\
   & = (\lambda - \lambda_0)^n
\end{align*}
显然,$\lambda_0$是$(\lambda - \lambda_0)^n = 0$的$n$重根,
所以,也是$|\lambda E - A_1| = 0, |\lambda E - A_2| = 0$的根。

综上,$\lambda_0$是$\mathscr{A}|_M, \mathscr{A}|_N$上的特征值,
于是在不变子空间$M,N$中最有各一个特征向量$\alpha_m, \alpha_n$,
又因为$M, N$是直和,所以$\alpha_m, \alpha_n$线性无关。
于是在$\mathscr{A}$中,我们有
\begin{align*}
  \mathscr{A} \alpha_m = \lambda_0 \alpha_m \\
  \mathscr{A} \alpha_n = \lambda_0 \alpha_n
\end{align*}
即线性变换$\mathscr{A}$的属于特征值$\lambda_0$的特性子空间最少是二维的。

而$\lambda_0E - A$的秩,显然是$n - 1$,所以$(\lambda_0E - A)x = 0$的解空间的维数是1,
即线性变换$\mathscr{A}$的属于特征值$\lambda_0$的特性子空间是1维的,
出现矛盾,假设不成立,命题得证。

\section*{17}

反证法,假设$\mathscr{A}$存在非平凡不变子空间$M$,
显然这个平凡子空间是1维的。

设$\eta_1$是$M$的一组基,于是
\begin{align*}
  \mathscr{A} \eta_1 \in M
\end{align*}
于是,$\mathscr{A} \eta_1$可以表示成
\begin{align*}
  \mathscr{A} \eta_1 = \lambda_0 \eta_1 \ (\lambda_0 \in \mathbb{R})
\end{align*}
所以,$\mathscr{A}$有特征值$\lambda_0$,且$\eta_1$是特征向量。

$A$的特征多项式为
\begin{align*}
  |\lambda E - A| = \lambda^2 - 2cos \theta \lambda + 1
\end{align*}
由判别式$\Delta = 4 cos^2 \theta - 4 < 0 (\theta \neq k \pi)$可知
\begin{align*}
  \lambda^2 - 2cos \theta \lambda + 1 = 0
\end{align*}
没有实数解,所以$\mathscr{A}$没有特征值,
出现矛盾,假设不成立,命题得证。

\section*{19}

设$\mathscr{A}$在某组基下的矩阵为$A$,
其特征多项式
\begin{align*}
  f(\lambda) = |\lambda E - A| = 0
\end{align*}
(i)有实数解,即$\mathscr{A}$有特征子空间,
在该特征子空间任取一组基中去一个向量$\epsilon$,
$L(\epsilon)$既是一个不变子空间。

(ii)没有实数解,此时需要把实数域的线性空间,扩展到复数域。

感觉超纲了啊
% 这里插入一个复化空间的定义:
% \begin{zremark}
%   设$V$是实数域上的线性空间,它的复化定义为:
%   \begin{align*}
%     V_{\mathbb{C}}
%   \end{align*}
% \end{zremark}
% 那么$f(\lambda) = 0$有复数根。
% 考察其中任意特征值$\lambda_0$的特征向量$v$。

\section*{20}

任意$\alpha \in V_{\lambda}$,因为
\begin{align*}
  \mathscr{A} \mathscr{B} \alpha
   & = \mathscr{B} \mathscr{A} \alpha \\
   & = \mathscr{B} \lambda \alpha     \\
   & = \lambda \mathscr{B} \alpha
\end{align*}
所以
于是要证明
\begin{align*}
  \mathscr{B} \alpha \in V_{\lambda}
\end{align*}

\section*{21}

因为$\mathscr{A}$的矩阵可对角化,那么对$\mathscr{A}$全部互不相等的特征值
$\lambda_1, \lambda_2, \cdots, \lambda_k$,有
\begin{align*}
  V = V_{\lambda_1} \oplus V_{\lambda_2} \oplus \cdots \oplus V_{\lambda_k}
\end{align*}
因为$\mathscr{A} \mathscr{B} = \mathscr{B} \mathscr{A}$,有习题20可知
$V_{\lambda_1}, V_{\lambda_2}, \cdots , V_{\lambda_k}$都是$\mathscr{B}$的不变子空间。

因为$\mathscr{B}$的矩阵可对角化,$V_{\lambda_i} (1 \leq i \leq k)$是$\mathscr{B}$的不变子空间,
于是由命题4.6可知$\mathscr{B}|_{V_{\lambda_i}}$可对角化,
即$V_{\lambda_i}$中存在一组基$\epsilon_{i_1},\epsilon_{i_2}, \cdots, \epsilon_{i_r}$,
$\mathscr{B}|_{V_{\lambda_i}}$在这组基下的矩阵是对角矩阵,设为
\begin{align*}
  B_i = \begin{bmatrix}
          \lambda_{i_1} & 0             & \cdots & 0             \\
          0             & \lambda_{i_2} & \cdots & 0             \\
          \vdots        & \vdots        & \ddots & \vdots        \\
          0             & 0             & \cdots & \lambda_{i_r}
        \end{bmatrix}
\end{align*}
因为$\epsilon_{i_1},\epsilon_{i_2}, \cdots, \epsilon_{i_r} \in V_{\lambda_i}$,所以
$\mathscr{A}|_{V_{\lambda_i}}$在这组基下的矩阵是对角矩阵为
\begin{align*}
  B_i = \begin{bmatrix}
          \lambda_{i} & 0           & \cdots & 0           \\
          0           & \lambda_{i} & \cdots & 0           \\
          \vdots      & \vdots      & \ddots & \vdots      \\
          0           & 0           & \cdots & \lambda_{i}
        \end{bmatrix}
\end{align*}
综上,我们得到一组基,$\mathscr{A}, \mathscr{B}$在该组基下的矩阵同时成对角形。

\section*{22}

$\mathscr{A}$的矩阵可对角化,即存在一组基$\epsilon_1, \epsilon_2, \cdots, \epsilon_n$,使得
\begin{align*}
  \mathscr{A} \epsilon_i = \lambda_i \epsilon_i (1 \leq i \leq n)
\end{align*}

设$M$是$\mathscr{A}$中不变子空间,那么由命题4.6可知,$\mathscr{A}|_M$的矩阵可对角化,
不妨设$\eta_1, \eta_2, \cdots, \eta_r$是$M$的一组基,使得
\begin{align*}
  \mathscr{A}|_M \eta_i = \lambda'_i \eta_i (1 \leq i \leq r)
\end{align*}
利用替换定理,用$\eta_1, \eta_2, \cdots, \eta_r$替换掉
$\epsilon_1, \epsilon_2, \cdots, \epsilon_n$中$r$个向量,
得到的一组新的基$\eta_1, \eta_2, \cdots, \eta_r, \epsilon_{i_1}, \epsilon_{i_2}, \cdots, \epsilon_{i_{n - r}}$,
令
\begin{align*}
  N = L(\epsilon_{i_1}, \epsilon_{i_2}, \cdots, \epsilon_{i_{n - r}})
\end{align*}
因为这些$\epsilon_{i_j}$本身是$\mathscr{A}$的特征向量,所以
\begin{align*}
  \mathscr{A} \epsilon_{i_j} = \lambda_{i_j} \epsilon_{i_j} \in N
\end{align*}
所以,$N$是$\mathscr{A}$的不变子空间。又显然,
\begin{align*}
  V = M \oplus N
\end{align*}
命题得证。

\section*{23}

对$V$的维数$n$进行归纳。

$n = 1$时,此时线性变换$\mathscr{A}$矩阵可对角化是平凡的。

归纳假设维数$< n$时,$\mathscr{A}$矩阵可对角化。

$V$的维数时$n$时,由$V$是复数域上的线性空间可知,
线性变换$\mathscr{A}$的特征多项式一定有一个复数解$\lambda_0$,
属于特征值$\lambda_0$的特征子空间设为$V_{\lambda_0}$。

因为特征子空间$V_{\lambda_0}$是$\mathscr{A}$的不变子空间,由题设可知存在$\mathscr{A}$
的不变子空间$N$,使得
\begin{align*}
  V = V_{\lambda_0} \oplus N
\end{align*}
因为$dim V = dim V_{\lambda_0} + dim N$,
所以$dim N < n$,由归纳假设可知,
$\mathscr{A}|_N$的矩阵可对角化,设$\mathscr{A}|_N$在
基$\epsilon_{i_1}, \epsilon_{i_2}, \cdots, \epsilon_{i_k}$下的矩阵是对角矩阵
$B$。
另外,取$V_{\lambda_0}$的基为$\epsilon_1, \epsilon_2, \cdots, \epsilon_r$,
两组基合并,得到$V$的一组基。
综上,在这组基下,我们得到一个$\mathscr{A}$的对角矩阵:
\begin{align*}
  \sigma(\mathscr{A}) = \begin{bmatrix}
                          \lambda_0 E &   \\
                                      & B
                        \end{bmatrix}
\end{align*}
归纳完成,命题成立。

\section*{24}


\end{document}
