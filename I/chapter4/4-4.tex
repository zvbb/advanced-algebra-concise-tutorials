\documentclass{article}
\usepackage{mathtools} 
\usepackage{fontspec}
\usepackage[UTF8]{ctex}
\usepackage{amsthm}
\usepackage{mdframed}
\usepackage{xcolor}
\usepackage{amssymb}
\usepackage{amsmath}
\usepackage{hyperref}
\usepackage{mathrsfs}



% 定义新的带灰色背景的说明环境 zremark
\newmdtheoremenv[
  backgroundcolor=gray!10,
  % 边框与背景一致,边框线会消失
  linecolor=gray!10
]{zremark}{注释}

% 通用矩阵命令: \flexmatrix{矩阵名}{元素符号}{行数}{列数}
\newcommand{\flexmatrix}[4]{
  \[
  #1 = \begin{pmatrix}
    #2_{11}     & #2_{12}     & \cdots & #2_{1#4}   \\
    #2_{21}     & #2_{22}     & \cdots & #2_{2#4}   \\
    \vdots      & \vdots      & \ddots & \vdots     \\
    #2_{#31}    & #2_{#32}    & \cdots & #2_{#3#4}
  \end{pmatrix}
  \]
}

% 简化版命令(默认矩阵名为A,元素符号为a): \quickmatrix{行数}{列数}
\newcommand{\quickmatrix}[2]{\flexmatrix{A}{a}{#1}{#2}}


\begin{document}
\title{4.4}
\author{张志聪}
\maketitle

\section*{3}

(i)$A$的特征多项式为
\begin{align*}
  f(\lambda) = \det(\lambda E - A)
   & = \begin{vmatrix}
         \lambda & -1      &        &         \\
                 & \lambda & -1     &         \\
                 &         & \ddots & \ddots  \\
                 &         &        & \lambda \\
       \end{vmatrix}
   & = \lambda^n
\end{align*}
因为
\begin{align*}
  f(\lambda) = 0 \\
  \lambda^n = 0  \\
  \lambda = 0
\end{align*}
所以,它的特征根仅有0。

(ii)求$\lambda_0 = 0$对应的特征向量.
\begin{align*}
  \lambda E - A = \begin{bmatrix}
                    0 & -1 &        &        \\
                      & 0  & -1     &        \\
                      &    & \ddots & \ddots \\
                      &    &        & 0      \\
                  \end{bmatrix}
\end{align*}
这个齐次线性方程组中,仅有$x_1$是自由未知量,取$x_1 = 1$,得基础解系
\begin{align*}
  \eta_1 = \begin{bmatrix}
             1  \\
             0  \\
             \vdots  \\
             0  \\
           \end{bmatrix}
\end{align*}
它对应于$A$的特性向量
\begin{align*}
  x_1 \epsilon_1 + x_2 \epsilon_2 + \cdots + x_n \epsilon_n = \epsilon_1
\end{align*}
于是$V_{\lambda_0} = L(\epsilon)$。

\section*{7}

反证法,假设存在$\lambda \neq 0$是$\mathscr{A}$的特征值。

\end{document}
