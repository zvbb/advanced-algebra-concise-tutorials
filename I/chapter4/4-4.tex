\documentclass{article}
\usepackage{mathtools} 
\usepackage{fontspec}
\usepackage[UTF8]{ctex}
\usepackage{amsthm}
\usepackage{mdframed}
\usepackage{xcolor}
\usepackage{amssymb}
\usepackage{amsmath}
\usepackage{hyperref}
\usepackage{mathrsfs}



% 定义新的带灰色背景的说明环境 zremark
\newmdtheoremenv[
  backgroundcolor=gray!10,
  % 边框与背景一致,边框线会消失
  linecolor=gray!10
]{zremark}{注释}

% 通用矩阵命令: \flexmatrix{矩阵名}{元素符号}{行数}{列数}
\newcommand{\flexmatrix}[4]{
  \[
  #1 = \begin{pmatrix}
    #2_{11}     & #2_{12}     & \cdots & #2_{1#4}   \\
    #2_{21}     & #2_{22}     & \cdots & #2_{2#4}   \\
    \vdots      & \vdots      & \ddots & \vdots     \\
    #2_{#31}    & #2_{#32}    & \cdots & #2_{#3#4}
  \end{pmatrix}
  \]
}

% 简化版命令(默认矩阵名为A,元素符号为a): \quickmatrix{行数}{列数}
\newcommand{\quickmatrix}[2]{\flexmatrix{A}{a}{#1}{#2}}


\begin{document}
\title{4.4}
\author{张志聪}
\maketitle

\section*{3}

(i)$A$的特征多项式为
\begin{align*}
  f(\lambda) = \det(\lambda E - A)
   & = \begin{vmatrix}
         \lambda & -1      &        &         \\
                 & \lambda & -1     &         \\
                 &         & \ddots & \ddots  \\
                 &         &        & \lambda \\
       \end{vmatrix}
   & = \lambda^n
\end{align*}
因为
\begin{align*}
  f(\lambda) = 0 \\
  \lambda^n = 0  \\
  \lambda = 0
\end{align*}
所以,它的特征根仅有0。

(ii)求$\lambda_0 = 0$对应的特征向量.
\begin{align*}
  \lambda E - A = \begin{bmatrix}
                    0 & -1 &        &        \\
                      & 0  & -1     &        \\
                      &    & \ddots & \ddots \\
                      &    &        & 0      \\
                  \end{bmatrix}
\end{align*}
这个齐次线性方程组中,仅有$x_1$是自由未知量,取$x_1 = 1$,得基础解系
\begin{align*}
  \eta_1 = \begin{bmatrix}
             1      \\
             0      \\
             \vdots \\
             0      \\
           \end{bmatrix}
\end{align*}
它对应于$A$的特性向量
\begin{align*}
  x_1 \epsilon_1 + x_2 \epsilon_2 + \cdots + x_n \epsilon_n = \epsilon_1
\end{align*}
于是$V_{\lambda_0} = L(\epsilon)$。

\section*{7}

 (1)反证法,假设存在$\lambda \neq 0$是$\mathscr{A}$的特征值。
于是,存在$\alpha \neq 0$使得
\begin{align*}
  \mathscr{A} \alpha = \lambda \alpha
\end{align*}
于是
\begin{align*}
  \mathscr{A}^k (\alpha)
   & = \mathscr{A}^{k-1}(\mathscr{A} \alpha) \\
   & = \mathscr{A}^{k-1}(\lambda \alpha)     \\
   & = \cdots                                \\
   & = \lambda^k \alpha                      \\
   & \neq 0
\end{align*}
这与题设$\mathscr{A}^k = 0$矛盾。

(2)接下来,证明0是$\mathscr{A}$的特征值。

对任意$\alpha \in V$,我们有
\begin{align*}
  \mathscr{A}^k (\alpha) = 0
\end{align*}
于是,存在$l (1 \leq l \leq k)$使得
\begin{align*}
  \mathscr{A}^{l - 1}(\alpha) \neq 0 \\
  \mathscr{A}^l(\alpha) = \mathscr{A}(\mathscr{A}^{l - 1}(\alpha)) = 0 = 0 \cdot \mathscr{A}^{l - 1}(\alpha)
\end{align*}
所以,$0$是$\mathscr{A}$的特征值,$\mathscr{A}^{l - 1}(\alpha)$是对应的特征向量。

\section*{8}

反证法,假设$\xi_1 + \xi_2$是某个特征向量,那么
存在$\lambda \in K$使得
\begin{align*}
  \mathscr{A} (\xi_1 + \xi_2) = \lambda (\xi_1 + \xi_2)           \\
  \mathscr{A} \xi_1 + \mathscr{A} \xi_2 = \lambda (\xi_1 + \xi_2) \\
  \lambda_1 \xi_1 + \lambda_2 \xi_2 = \lambda (\xi_1 + \xi_2)     \\
  (\lambda_1 - \lambda) \xi_1 + (\lambda_2 - \lambda) \xi_2 = 0
\end{align*}
由命题4.3可知,$\xi_1, \xi_2$是线性无关的,所以
\begin{equation*}
  \begin{cases*}
    \lambda_1 - \lambda = 0 \\
    \lambda_2 - \lambda = 0
  \end{cases*}
\end{equation*}
于是可得
\begin{align*}
  \lambda = \lambda_1 \\
  \lambda = \lambda_2
\end{align*}
有题设可知$\lambda_1 \neq \lambda_2$,出现矛盾,
假设不成立,命题得证。

\section*{8 推广}

$k \xi_1 + l \xi_2$不是$\mathscr{A}$的特征向量。

证明:

因为$k \xi_1 \in V_{\lambda_1}, l \xi_2 \in V_{\lambda_2}$,
利用习题8,命题得证。

\section*{9}

只需证明,这些特征向量属于同一个特征子空间即可。

反证法,假设$\alpha \in V_{\lambda_1}, \beta \in V_{\lambda_2}$,
由习题8可知,
\begin{align*}
  \alpha + \beta
\end{align*}
不是特征向量,与题设矛盾。

\section*{10}

$\mathscr{A}$是线性空间$V$内的可逆线性变换,即存在线性变换$\mathscr{B}$使得
\begin{align*}
  \mathscr{A} \mathscr{B} = \mathscr{B} \mathscr{A} = \mathscr{E}
\end{align*}

任取线性变换$\mathscr{A}, \mathscr{A}$在基$\epsilon_1, \epsilon_2, \cdots, \epsilon_n$
下的矩阵为$A, B$,
我们有
\begin{align*}
  \sigma(AB) = \sigma(A) \sigma(B) = AB = E
\end{align*}
于是可得
\begin{align*}
  rank(A) = rank(B) = rank(E) = n
\end{align*}

\begin{itemize}
  \item (1)

        \begin{itemize}
          \item 方法一

                反证法,假设线性变换$\mathscr{A}$存在为零的特征值,
                那么,我们有
                \begin{align*}
                  |\lambda E - A| = 0 \\
                  |-A| = 0
                \end{align*}
                $A$不满秩,出现矛盾。

          \item 方法二

                $\mathscr{A}$是可逆线性变换,所以它是双射。

                反证法,假设线性变换$\mathscr{A}$存在为零的特征值,
                于是存在$\alpha \in V$使得
                \begin{align*}
                  \mathscr{A} \alpha = 0 \cdot \alpha = 0 \\
                  \mathscr{A} 0 = 0
                \end{align*}
                与$\mathscr{A}$是双射矛盾。
        \end{itemize}

  \item (2)

        $\lambda$是$\mathscr{A}$的特征值,那么,存在$\alpha \in V$使得
        \begin{align*}
          \mathscr{A} \alpha = \lambda \alpha \\
        \end{align*}
        两边同时代入$\mathscr{A}^{-1}$,我们有
        \begin{align*}
          \mathscr{A}^{-1} \mathscr{A} \alpha = \mathscr{A}^{-1} \lambda \alpha \\
          \alpha = \lambda \mathscr{A}^{-1} \alpha
        \end{align*}
\end{itemize}

\section*{10 推广}

(1)的反方向也是成立的,即:
$\mathscr{A}$的特征值都不为零,则$\mathscr{A}$是可逆线性变换。

证明:

$\mathscr{A}$的在基$\epsilon_1, \epsilon_2, \cdots, \epsilon_n$下的矩阵为$A$,
则$A$的全体特征根的乘积,我们有
\begin{align*}
  \lambda_1 \lambda_2 \cdots \lambda_n = |A| \neq 0
\end{align*}
所以,$A$是满秩。

于是存在$A^{-1}$使得
\begin{align*}
  A^{-1} A = A A^{-1} = E
\end{align*}
于是,由$\sigma: End(V) \to M_n(K)$的双射性,
存在线性变换$\mathscr{B}$在基下的矩阵为$A^{-1}$,所以
\begin{align*}
  E = \sigma(\mathscr{A}) \sigma(\mathscr{B}) = \sigma(\mathscr{A}\mathscr{B})\\
\end{align*}
可得
\begin{align*}
  \mathscr{A} \mathscr{B} = \mathscr{B} \mathscr{A} = \mathscr{E}
\end{align*}


\end{document}
