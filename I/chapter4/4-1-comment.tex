\documentclass{article}
\usepackage{mathtools} 
\usepackage{fontspec}
\usepackage[UTF8]{ctex}
\usepackage{amsthm}
\usepackage{mdframed}
\usepackage{xcolor}
\usepackage{amssymb}
\usepackage{amsmath}
\usepackage{hyperref}


% 定义新的带灰色背景的说明环境 zremark
\newmdtheoremenv[
  backgroundcolor=gray!10,
  % 边框与背景一致,边框线会消失
  linecolor=gray!10
]{zremark}{注释}

% 通用矩阵命令: \flexmatrix{矩阵名}{元素符号}{行数}{列数}
\newcommand{\flexmatrix}[4]{
  \[
  #1 = \begin{pmatrix}
    #2_{11}     & #2_{12}     & \cdots & #2_{1#4}   \\
    #2_{21}     & #2_{22}     & \cdots & #2_{2#4}   \\
    \vdots      & \vdots      & \ddots & \vdots     \\
    #2_{#31}    & #2_{#32}    & \cdots & #2_{#3#4}
  \end{pmatrix}
  \]
}

% 简化版命令(默认矩阵名为A,元素符号为a): \quickmatrix{行数}{列数}
\newcommand{\quickmatrix}[2]{\flexmatrix{A}{a}{#1}{#2}}


\begin{document}
\title{4.1注释}
\author{张志聪}
\maketitle

\begin{zremark}
  其他事实(P225基础上的扩展。)
  \begin{itemize}
    \item 7)$k \in K$且$k \neq 0$,$\alpha \in V$且$\alpha \neq 0$,则
          \begin{align*}
            k \alpha \neq 0
          \end{align*}
  \end{itemize}
\end{zremark}

\textbf{证明:}

\begin{itemize}
  \item 7)

        假设$k \alpha = 0$,由6)可知
        \begin{align*}
          k \alpha = 0                         \\
          \frac{1}{k} k \alpha = \frac{1}{k} 0 \\
          \alpha = 0
        \end{align*}
        与题设矛盾,假设成立,命题得证。

\end{itemize}

\begin{zremark}
  只要是线性空间,只要不是零空间(即只有零元素的空间),那么元素个数一定是无限个的。
\end{zremark}

任意非零线性空间$V$,存在$\alpha \in V$且$\alpha \neq 0$,

设任意$k > 1$,由数乘的封闭性,我们有
\begin{align*}
  k \alpha \in V \\
  k \alpha \neq \alpha
\end{align*}
所以,线性空间$V$中有无限多个元素。

\begin{zremark}
  二元数组构成的线性空间一定是二维的吗?
\end{zremark}

\textbf{证明:}

不是;


\end{document}
