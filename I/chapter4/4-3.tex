\documentclass{article}
\usepackage{mathtools} 
\usepackage{fontspec}
\usepackage[UTF8]{ctex}
\usepackage{amsthm}
\usepackage{mdframed}
\usepackage{xcolor}
\usepackage{amssymb}
\usepackage{amsmath}
\usepackage{hyperref}


% 定义新的带灰色背景的说明环境 zremark
\newmdtheoremenv[
  backgroundcolor=gray!10,
  % 边框与背景一致,边框线会消失
  linecolor=gray!10
]{zremark}{注释}

% 通用矩阵命令: \flexmatrix{矩阵名}{元素符号}{行数}{列数}
\newcommand{\flexmatrix}[4]{
  \[
  #1 = \begin{pmatrix}
    #2_{11}     & #2_{12}     & \cdots & #2_{1#4}   \\
    #2_{21}     & #2_{22}     & \cdots & #2_{2#4}   \\
    \vdots      & \vdots      & \ddots & \vdots     \\
    #2_{#31}    & #2_{#32}    & \cdots & #2_{#3#4}
  \end{pmatrix}
  \]
}

% 简化版命令(默认矩阵名为A,元素符号为a): \quickmatrix{行数}{列数}
\newcommand{\quickmatrix}[2]{\flexmatrix{A}{a}{#1}{#2}}


\begin{document}
\title{4.2}
\author{张志聪}
\maketitle

\section*{2}

\begin{itemize}
  \item 方法一

        设$A$的列向量组为$\alpha_1, \alpha_2, \cdots, \alpha_n$,
        $B$的列向量组为$\beta_, \beta_2, \cdots, \beta_n$。
        并设
        \begin{align*}
          rank(A) = rank(B) = r
        \end{align*}

        令$f_A$为$K^n \to K^m$的线性映射:
        \begin{align*}
          f_A(X) = AX
        \end{align*}
        则$AX = 0$的解空间为$Kerf_A$,而$Imf_A = L(\alpha_1, \alpha_2, \cdots, \alpha_n)$,
        故$dim(Imf_A) = rank(A) = r$,按照命题3.5的推论1,有
        \begin{align*}
          dim Kerf_A + dim Imf_A = dim K^n = n
        \end{align*}
        同理可得
        \begin{align*}
          dim Kerf_B + dim Imf_B = dim K^n = n
        \end{align*}
        因为
        \begin{align*}
          dim Imf_A = dim Imf_B
        \end{align*}
        于是可得
        \begin{align*}
          dim Kerf_A = dim Kerf_B
        \end{align*}
        它们的基础解系的维数相等,即
        \begin{align*}
          dim U = dim V
        \end{align*}

        设$U,V$的一组基分别为
        \begin{align*}
          \alpha_{i_1}, \alpha_{i_2}, \cdots, \alpha_{i_r} \ \ \ (I) \\
          \beta_{j_1}, \beta_{j_2}, \cdots, \beta_{j_r} \ \ \ (II)
        \end{align*}
        定义映射$f: U \to V$如下:
        任意$Y \in U$,我们有
        \begin{align*}
          Y = a_1 \alpha_{i_1} + a_2 \alpha_{i_2} + \cdots + a_r \alpha_{i_r}
        \end{align*}
        令
        \begin{align*}
          f(Y) = b_1 \beta_{j_1} + b_2 \beta_{j_2} + \cdots + b_r \beta_{j_r}
        \end{align*}
        因为
        \begin{align*}
          f(\alpha_{i_1}) = \beta_{j_1} \\
          f(\alpha_{i_2}) = \beta_{j_2} \\
          \vdots                        \\
          f(\alpha_{i_r}) = \beta_{j_r}
        \end{align*}
        易证$f$是线性映射,
        且$f$在$U,V$所取定的基下的矩阵为$E$,这就找到了命题中的$T = E$。

  \item 方法二

        因为$rank(A) = rank(B)$,所以$A$与$B$相抵,所以存在$m$阶满秩方阵$P$和$n$阶满秩方阵$Q$,使得
        \begin{align*}
          A = PBQ
        \end{align*}
        任意$\alpha \in U$,我们有
        \begin{align*}
          A \alpha = 0                \\
          PBQ \alpha = 0              \\
          P^{-1} PBQ \alpha = P^{-1}0 \\
          BQ \alpha = 0
        \end{align*}
        令$T = Q$,接下来证明$f(Y) = TY (\forall Y \in U)$是$U$到$V$的同构映射。

        因为$T$是矩阵,$f$显然是线性映射。

        (i)$f$是单射:

        $\forall \alpha, \beta \in U$,使得
        \begin{align*}
          T \alpha = T \beta     \\
          T \alpha - T \beta = 0 \\
          T (\alpha - \beta) = 0
        \end{align*}
        因为$T$满足,$Tx$只有零解,所以存在
        \begin{align*}
          \alpha - \beta = 0 \\
          \alpha = \beta
        \end{align*}

        (ii)$f$是满射。

        任意$\beta \in V$,
        \begin{align*}
          T \alpha = \beta \\
          \alpha = T^{-1} \beta
        \end{align*}
        需要证明$\alpha \in U$。
        \begin{align*}
          A \alpha & = PBQ \alpha       \\
                   & = PBQ T^{-1} \beta \\
                   & = PBQ Q^{-1} \beta \\
                   & = P B \beta 
        \end{align*}
        因为$\beta \ in V$,所以$B \beta = 0$,所以
        \begin{align*}
          A \alpha = 0
        \end{align*}
        综上,$\alpha \in U$,所以$T$是满射。

\end{itemize}

\section*{14}





\end{document}
