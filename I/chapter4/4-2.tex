\documentclass{article}
\usepackage{mathtools} 
\usepackage{fontspec}
\usepackage[UTF8]{ctex}
\usepackage{amsthm}
\usepackage{mdframed}
\usepackage{xcolor}
\usepackage{amssymb}
\usepackage{amsmath}
\usepackage{hyperref}


% 定义新的带灰色背景的说明环境 zremark
\newmdtheoremenv[
  backgroundcolor=gray!10,
  % 边框与背景一致,边框线会消失
  linecolor=gray!10
]{zremark}{注释}

% 通用矩阵命令: \flexmatrix{矩阵名}{元素符号}{行数}{列数}
\newcommand{\flexmatrix}[4]{
  \[
  #1 = \begin{pmatrix}
    #2_{11}     & #2_{12}     & \cdots & #2_{1#4}   \\
    #2_{21}     & #2_{22}     & \cdots & #2_{2#4}   \\
    \vdots      & \vdots      & \ddots & \vdots     \\
    #2_{#31}    & #2_{#32}    & \cdots & #2_{#3#4}
  \end{pmatrix}
  \]
}

% 简化版命令(默认矩阵名为A,元素符号为a): \quickmatrix{行数}{列数}
\newcommand{\quickmatrix}[2]{\flexmatrix{A}{a}{#1}{#2}}


\begin{document}
\title{4.2}
\author{张志聪}
\maketitle

\section*{1}

 (2)提示: $C(A)$是对角矩阵。

\section*{4}

注意:需要考虑$K$是不是有理数,是有理数则是子空间,
否则不是。

\section*{7}

设$M$的秩为$r$,
\begin{align*}
  \alpha_1, \alpha_2, \cdots, \alpha_r
\end{align*}
是它的一组基。

$r = 0$,则$M$是零空间,即线性方程组只有零解,
对于齐次线性方程组:
\begin{equation*}
  \begin{cases*}
    x_1 = 0 \\
    x_2 = 0 \\
    \cdots  \\
    x_n = 0
  \end{cases*}
\end{equation*}
的解空间为零空间($M$),命题成立。

$1 \leq r \leq n$时,设
\begin{align*}
  B = \begin{bmatrix}
        \alpha_1, \alpha_2, \cdots, \alpha_r
      \end{bmatrix}
\end{align*}
对于齐次线性方程组$B^T x = 0$,
因为$rank(B^T) = r$,所以解空间的一组基可设为
\begin{align*}
  \beta_1, \beta_2, \cdots, \beta_{n - r}
\end{align*}
设
\begin{align*}
  A = \begin{bmatrix}
        \beta_1, \beta_2, \cdots, \beta_{n - r}
      \end{bmatrix}
\end{align*}
我们有,
\begin{align*}
  B^T A = 0
\end{align*}
于是
\begin{align*}
  A^T B = 0
\end{align*}
对于齐次线性方程组$A^T x = 0$,
由于$rank(A^T) = n - r$,所以它的解空间的基个数为$r$,
所以$B$中向量可以是$A^T x = 0$解空间的一组基,
即$M$是$A^T x = 0$的解空间,命题成立。

综上,$K^n$上的任一子空间$M$都是数域$K$上某个齐次线性方程组的解空间。

\section*{8}

反证法,假设存在$M_1 \cup M_2 \cup \cdots \cup M_k = V$,
这里是最小状态,即不存在删除某个子空间$M_i$后,
$M_1 \cup M_2 \cup \cdots \cup M_{i - 1} \cup M_{i + 1} \cup \cdots \cup M_k = V$仍然成立。

为了讨论方便,令
\begin{align*}
  M = (M_2 \cup M_3 \cup \cdots \cup M_k)
\end{align*}
于是
\begin{align*}
  V = M_1 \cup M
\end{align*}
因为$M_1 \neq \{0\}, M \neq \{0\}$,
于是
\begin{align*}
  \exists \alpha \in M_1 - M \\
  \exists \beta \in M - M_1
\end{align*}
定义一个无限个数的向量组$(I)$:
\begin{align*}
  \alpha + \beta, 2\alpha + \beta, 3\alpha + \beta, \cdots
\end{align*}
显然,$(I)$中的任意向量都不可能在$M_1$中。

因为$(I)$中向量个数是无限多个,$M$中存在子空间包含无限多个$(I)$中向量,
不妨设为$M_i$包含无限多个$(I)$中向量,任取两个向量$j\alpha + \beta, i\alpha + \beta$,
我们有
\begin{align*}
  (j\alpha + \beta - i\alpha + \beta)
  = (j - i)\alpha
\end{align*}
于是可得,$\alpha \in M_i$,存在矛盾,
假设不成立,命题得证。

\section*{9}

提示:如果函数之间是线性关系,那么函数值之间的线性关系一定成立。

对于行列式
\begin{align*}
  \begin{vmatrix}
    1 & cos x_1 & cos 2 x_1 & cos 3 x_1 \\
    1 & cos x_2 & cos 2 x_2 & cos 3 x_2 \\
    1 & cos x_3 & cos 2 x_3 & cos 3 x_3 \\
    1 & cos x_4 & cos 2 x_4 & cos 3 x_4
  \end{vmatrix}
\end{align*}
由于,$cos kx$都可以表示成$2^{k - 1} (cos x)^k + *$的格式(3-2-comment中有详细证明)。
于是,通过初等列变换可以把非高次项移除
\begin{align*}
  \begin{vmatrix}
    1 & cos x_1 & cos 2 x_1 & cos 3 x_1 \\
    1 & cos x_2 & cos 2 x_2 & cos 3 x_2 \\
    1 & cos x_3 & cos 2 x_3 & cos 3 x_3 \\
    1 & cos x_4 & cos 2 x_4 & cos 3 x_4
  \end{vmatrix}
   & =
  \begin{vmatrix}
    1 & cos x_1 & 2 cos^2 x_1 & 4 cos^3 x_1 \\
    1 & cos x_2 & 2 cos^2 x_2 & 4 cos^3 x_2 \\
    1 & cos x_3 & 2 cos^2 x_3 & 4 cos^3 x_3 \\
    1 & cos x_4 & 2 cos^2 x_4 & 4 cos^3 x_4
  \end{vmatrix} \\
   & = 2 \cdot 4
  \begin{vmatrix}
    1 & cos x_1 & cos^2 x_1 & cos^3 x_1 \\
    1 & cos x_2 & cos^2 x_2 & cos^3 x_2 \\
    1 & cos x_3 & cos^2 x_3 & cos^3 x_3 \\
    1 & cos x_4 & cos^2 x_4 & cos^3 x_4
  \end{vmatrix}
\end{align*}
以上是范德蒙德行列式,于是只要$cos x_1, cos x_2, cos x_3$两两不相等,
则行列式不为零,显然这样的自变量$x_1, x_2, x_3$可以找到,
从而可得,这4个函数是线性无关的。(注意:如果找不到这四个函数线性相关)

综上,子空间$L(1, cos x, cos 2x, cos 3x)$的维数是4,一组基为
\begin{align*}
  1, cos x, cos 2x, cos 3x
\end{align*}

\section*{17}

\begin{align*}
  M_n(K) = M + N
\end{align*}
成立,但不是直和。

\section*{18}

\begin{itemize}
  \item 方法一

        先证明$M_1 + M_2$是直和。
        设$\alpha = \begin{bmatrix}
            a_1    \\
            a_2    \\
            \vdots \\
            a_n
          \end{bmatrix} \in M_1 \cap M_2$,那么,因为$\alpha \in M_1$,
        所以
        \begin{align*}
          a_1 + a_2 + \cdots + a_n = 0
        \end{align*}
        同理,因为$\alpha \in M_2$,
        \begin{align*}
          a_1 = a_2 = \cdots = a_n
        \end{align*}
        于是
        \begin{align*}
          n a_1 = 0 \\
          a_1 = 0
        \end{align*}
        所以
        \begin{align*}
          \alpha = 0
        \end{align*}
        于是可得$M_1 \cap M_2 \neq \{0\}$,
        所以,$M_1 + M_2$是直和。

        再证明$dim M_1 + dim M_2 = n$。
        第一个线性方程组的系数矩阵为
        \begin{align*}
          A = \begin{bmatrix}
                1 & 1 & \cdots & 1
              \end{bmatrix}
        \end{align*}
        $rank(A) = 1$,所以解空间$M_1$维数为
        \begin{align*}
          n - rank(A) = n - 1
        \end{align*}

        易得$M_2$是1维的,且$A$是它的一组基。
        于是可得$dim M_1 + dim M_2 = n$。

        综上,$K^n = M_1 \oplus M_2$。

  \item 方法二

        先证明$K^n = M_1 + M_2$,即对任意$\gamma  = \begin{bmatrix}
            c_1    \\
            c_2    \\
            \vdots \\
            c_n
          \end{bmatrix}$,可以表示成$\gamma = \alpha + \beta$,其中
        \begin{align*}
          \alpha = \begin{bmatrix}
                     a_1    \\
                     a_2    \\
                     \vdots \\
                     a_n
                   \end{bmatrix} \\
          \beta = \begin{bmatrix}
                    b      \\
                    b      \\
                    \vdots \\
                    b
                  \end{bmatrix}
        \end{align*}
        于是,解以下方程组:
        \begin{align*}
          c_1 = a_1 + b \\
          c_2 = a_2 + b \\
          \vdots        \\
          c_n = a_n + b
        \end{align*}
        所有等式相加得
        \begin{align*}
          c_1 + c_2 + \cdots + c_n = a_1 + a_2 + \cdots + a_n + n b
        \end{align*}
        由于$\alpha \in M_1$,所以,我们有
        \begin{align*}
          a_1 + a_2 + \cdots + a_n = 0
        \end{align*}
        于是
        \begin{align*}
          c_1 + c_2 + \cdots + c_n = 0 + n \cdot b \\
          b = \frac{c_1 + c_2 + \cdots + c_n}{n}
        \end{align*}
        带入后可得
        \begin{align*}
          a_1 = c_1 - \frac{c_1 + c_2 + \cdots + c_n}{n} \\
          a_2 = c_2 - \frac{c_1 + c_2 + \cdots + c_n}{n} \\
          \vdots                                         \\
          a_n = c_n - \frac{c_1 + c_2 + \cdots + c_n}{n}
        \end{align*}

        综上,
        \begin{align*}
          \begin{bmatrix}
            c_1    \\
            c_2    \\
            \vdots \\
            c_n
          \end{bmatrix}
          = \begin{bmatrix}
              c_1 - \frac{c_1 + c_2 + \cdots + c_n}{n} \\
              c_2 - \frac{c_1 + c_2 + \cdots + c_n}{n} \\
              \vdots                                   \\
              c_n - \frac{c_1 + c_2 + \cdots + c_n}{n}
            \end{bmatrix}
          + \begin{bmatrix}
              \frac{c_1 + c_2 + \cdots + c_n}{n} \\
              \frac{c_1 + c_2 + \cdots + c_n}{n} \\
              \vdots                             \\
              \frac{c_1 + c_2 + \cdots + c_n}{n}
            \end{bmatrix}
        \end{align*}
        $K^n = M_1 + M_2$得证。

        再证明$M_1 + M_2$是直和。
        通过任意向量$\alpha \in M_1 + M_2$,只有唯一分解方式来证明。

        反证法,假设存在
        \begin{align*}
          \alpha = \alpha_1 + \alpha_2 \\
          \alpha = \beta_1 + \beta_2
        \end{align*}
        其中$\alpha_1, \beta_1 \in M_1, \alpha_2, \beta_2 \in M_2$,
        移项可得
        \begin{align*}
          \alpha_1 - \beta_1 = \beta_2 - \alpha_2
        \end{align*}
        于是$\beta_2 - \alpha_2 \in M_1 \cap M_2$,
        设
        \begin{align*}
          \alpha_2 = \begin{bmatrix}
                       b      \\
                       b      \\
                       \vdots \\
                       b
                     \end{bmatrix}           \\
          \beta_2 = \begin{bmatrix}
                      b^\prime \\
                      b^\prime \\
                      \vdots   \\
                      b^\prime
                    \end{bmatrix}            \\
          \beta_2 - \alpha_2 = \begin{bmatrix}
                                 b^\prime - b \\
                                 b^\prime - b \\
                                 \vdots       \\
                                 b^\prime - b
                               \end{bmatrix} \\
        \end{align*}
        利用$\beta_2 - \alpha_2 \in M_1$可得
        \begin{align*}
          n(b^\prime - b) = 0 \\
          b^\prime - b = 0    \\
          b^\prime = b
        \end{align*}
        于是可得
        \begin{align*}
          \beta_2 = \alpha_2 \\
          \beta_1 = \alpha_1
        \end{align*}
        所以,$\alpha$只有唯一分解方式,于是$M_1 + M_2$是直和。

\end{itemize}

\section*{19}

4-2-comment.tex中有详细证明

\section*{20}

先证明$M + N$是直和。

任意$\alpha \in M \cap N$,我们有
\begin{align*}
  B \alpha = 0 \\
  C \alpha = 0
\end{align*}
于是,我们有
\begin{align*}
  A \alpha = 0
\end{align*}
因为$A$是满秩的,所以只存在零解,所以
\begin{align*}
  \alpha = 0
\end{align*}
于是可得
\begin{align*}
  M \cap N = \{0\}
\end{align*}
所以,$M + N$是直和。

因为
\begin{align*}
  rank(B) = k \\
  rank(C) = n - k
\end{align*}
于是可得
\begin{align*}
  dim M = n - k \\
  dim N = k
\end{align*}
又因为$M, N$是直和(即:它们各自的一组基合并后,是线性无关的,且个数是$n$个),所以
\begin{align*}
  dim M + dim N = dim K^n = n
\end{align*}
所以$K^n = M \oplus N$。

\section*{21}

先证明,$M + (N \cap L)$是直和。
$\forall \alpha \in M \cap (N \cap L)$,我们有
\begin{equation*}
  \begin{cases*}
    \alpha \in M \\
    \alpha \in N \cap L
  \end{cases*}
\end{equation*}
又因为$M \subset N$,所以
\begin{align*}
  \alpha \in M \cap L
\end{align*}
由于$M \oplus L$,所以
\begin{align*}
  \alpha \in M \cap L = \{0\}
\end{align*}
于是可得
\begin{align*}
  \alpha = 0
\end{align*}
综上,$M \oplus (N \cap L)$。

接下里证明$N = M \oplus (N \cap L)$。
因为$M \subseteq N$,于是
\begin{align*}
  M \oplus (N \cap L) \subseteq N
\end{align*}
是显然的。

对任意$\alpha \in N$,
因为$V = M \oplus L$,所以$\alpha$可以唯一表示成
\begin{align*}
  \alpha = m + l
\end{align*}
其中$m \in M, l \in L$。
因为
\begin{align*}
  l = \alpha - m
\end{align*}
由于$\alpha \in N$且$M \subseteq N$,所以
\begin{align*}
  l \in N
\end{align*}
所以
\begin{align*}
  l \in N \cap L
\end{align*}
于是可得
\begin{align*}
  \alpha \in M \oplus (N \cap L)
\end{align*}

\section*{23}

提示
\begin{itemize}
  \item 必要性

        \begin{align*}
          \sum\limits_{j = 1}^{i - 1} M_j \subseteq \sum\limits_{j = 1; j \neq i}^{k} M_j
        \end{align*}

  \item 充分性

        已知
        \begin{align*}
          M_i \cap \left(\sum\limits_{j = 1}^{i - 1} M_j\right) = \{0\}
        \end{align*}
        所以
        \begin{align*}
          M_i \oplus \left(\sum\limits_{j = 1}^{i - 1} M_j\right), i = 1, 2, \cdots, k
        \end{align*}
        于是可得
        \begin{align*}
          dim \left(\sum\limits_{j = 1}^{k} M_j\right)
          & = dim \left(M_k + \sum\limits_{j = 1}^{k-1} M_j\right) \\
          & = dim M_k + dim \left(\sum\limits_{j = 1}^{k-1} M_j\right) \\
          & = dim M_k + dim M_{k-1} + dim \left(\sum\limits_{j = 1}^{k-2} M_j\right) \\
          & \vdots \\
          & = dim M_k + dim M_{k-1} + \cdots + dim M_1
        \end{align*}
        可得$\left(\sum\limits_{j = 1}^{k} M_j\right)$是直和。
\end{itemize}

\section*{24}

提示
\begin{itemize}
  \item 必要性

        显然的。

  \item 充分性

        反证法,假设零向量的表法不唯一,即
        \begin{align*}
          0 = \beta_1 + \beta_2 + \cdots + \beta_k \ (\beta_i \in M_i) \\
          0 = \beta_1' + \beta_2' + \cdots + \beta_k' \ (\beta_i' \in M_i)
        \end{align*}
        存在$\beta_l \neq \beta_l'$。
        于是,我们有
        \begin{align*}
          \alpha = (\alpha_1 + \beta_1) + (\alpha_2 + \beta_2) + \cdots + (\alpha_k + \beta_k) \\
          \alpha = (\alpha_1 + \beta_1') + (\alpha_2 + \beta_2') + \cdots + (\alpha_k + \beta_k')
        \end{align*}
        因为存在$\beta_l \neq \beta_l'$,那么$\alpha$的表法不唯一,导致矛盾。
\end{itemize}

\section*{25}

令$V = K^4$。
易得
\begin{align*}
  dim M = 2
\end{align*}
由命题2.5可知
\begin{align*}
  dim V/M = 4 - 2 = 2
\end{align*}

使用$\alpha_1, \alpha_2$扩充成$V$的一组基。因为
\begin{align*}
  \begin{vmatrix}
    1 & 0 & 1  & 2 \\
    0 & 1 & -1 & 1 \\
    0 & 0 & 1  & 0 \\
    0 & 0 & 1  & 1
  \end{vmatrix}
  = 1 \neq 0
\end{align*}
于是可得
\begin{align*}
  \epsilon_1 + M, \epsilon_2 + M
\end{align*}
是$V/M$的一组基。

\section*{26}

提示:

由于$M_n(K)$的对称矩阵子空间和反对称矩阵子空间的和是直和,
于是反对称矩阵的一组基,加上对称矩阵的一组基,
可以构成$M_n(K)$的一组基。

\section*{27}

提示:

设
\begin{align*}
  \begin{bmatrix}
    * \\
    T_0
  \end{bmatrix}
  = \begin{bmatrix}
      t_{1 \, r+1} & t_{1 \, r+2} & \cdots & t_{1 \, n} \\
      t_{2 \, r+1} & t_{2 \, r+2} & \cdots & t_{2 \, n} \\
      \vdots                                            \\
      t_{n \, r+1} & t_{n \, r+2} & \cdots & t_{n \, n}
    \end{bmatrix}
\end{align*}

由过度矩阵的含义可知
\begin{align*}
  \eta_n = t_{1 \, n} \epsilon_{1} + t_{2 \, n} \epsilon_{2} + \cdots + t_{n \, n} \epsilon_{n}
\end{align*}

于是,由商空间中的运算法则,我们有
\begin{align*}
  \overline{\eta_n} = t_{1 \, n} \overline{\epsilon_{1}} + t_{2 \, n} \overline{\epsilon_{2}} + \cdots + t_{n \, n} \overline{\epsilon_{n}}
\end{align*}
因为在商空间$V/M$中,我们有
\begin{align*}
  \overline{\epsilon_{1}} = \epsilon_{2} = \cdots = \epsilon_{r} = 0
\end{align*}
所以
\begin{align*}
  \overline{\eta_n} = t_{r + 1 \, n} \overline{\epsilon_{r + 1}} + t_{r + 2 \, n} \overline{\epsilon_{r + 2}} + \cdots + t_{n \, n} \overline{\epsilon_{n}}
\end{align*}

\section*{28}

todo 未做出来

\begin{itemize}
  \item (1)

        先证明加法和数乘运算是封闭的。

        \begin{itemize}
          \item $f + g \in P(K)$。

                按照定义的加法,我们有
                \begin{align*}
                   & (f + g) \begin{bmatrix}
                               \cdots & \gamma \alpha + \mu \beta & \cdots
                             \end{bmatrix}               \\
                   & = f \begin{bmatrix}
                           \cdots & \gamma \alpha + \mu \beta & \cdots
                         \end{bmatrix} + g \begin{bmatrix}
                                             \cdots & \gamma \alpha + \mu \beta & \cdots
                                           \end{bmatrix} \\
                   & = \gamma f \begin{bmatrix}
                                  \cdots & \alpha & \cdots
                                \end{bmatrix}
                  + \mu f \begin{bmatrix}
                            \cdots & \beta & \cdots
                          \end{bmatrix}
                  + \gamma g \begin{bmatrix}
                               \cdots & \alpha & \cdots
                             \end{bmatrix}
                  + \mu g \begin{bmatrix}
                            \cdots & \beta & \cdots
                          \end{bmatrix}
                \end{align*}
                又因为列线性函数是数量函数,所以我们有
                \begin{align*}
                   & \gamma f \begin{bmatrix}
                                \cdots & \alpha & \cdots
                              \end{bmatrix}
                  + \mu f \begin{bmatrix}
                            \cdots & \beta & \cdots
                          \end{bmatrix}
                  + \gamma g \begin{bmatrix}
                               \cdots & \alpha & \cdots
                             \end{bmatrix}
                  + \mu g \begin{bmatrix}
                            \cdots & \beta & \cdots
                          \end{bmatrix}                           \\
                   & = \gamma (f \begin{bmatrix}
                                   \cdots & \alpha & \cdots
                                 \end{bmatrix} + g \begin{bmatrix}
                                                     \cdots & \alpha & \cdots
                                                   \end{bmatrix})
                  + \mu (f \begin{bmatrix}
                             \cdots & \beta & \cdots
                           \end{bmatrix} + g \begin{bmatrix}
                                               \cdots & \beta & \cdots
                                             \end{bmatrix})        \\
                   & = \gamma \left((f + g) \begin{bmatrix}
                                              \cdots & \alpha & \cdots
                                            \end{bmatrix}\right)
                  + \mu \left((f + g) \begin{bmatrix}
                                        \cdots & \beta & \cdots
                                      \end{bmatrix}\right)
                \end{align*}
                加法封闭性得证。

          \item $kf \in P(K)$。

                我们有
                \begin{align*}
                   & (kf)\begin{bmatrix}
                           \cdots & \gamma \alpha + \mu \beta & \cdots
                         \end{bmatrix}                  \\
                   & =k f \begin{bmatrix}
                            \cdots & \gamma \alpha + \mu \beta & \cdots
                          \end{bmatrix}                 \\
                   & = k \left(\gamma f \begin{bmatrix}
                                          \cdots & \alpha & \cdots
                                        \end{bmatrix} + \mu f \begin{bmatrix}
                                                                \cdots & \beta & \cdots
                                                              \end{bmatrix}\right) \\
                   & =  \gamma kf \begin{bmatrix}
                                    \cdots & \alpha & \cdots
                                  \end{bmatrix}
                  + \mu kf \begin{bmatrix}
                             \cdots & \beta & \cdots
                           \end{bmatrix}
                \end{align*}

                乘法封闭性得证。
        \end{itemize}

        证明8条运算法则成立:

        \begin{itemize}
          \item 存在零元素。

                零元素$0$具有如下性质: 对任意$A \in M_n(K)$,有
                \begin{align*}
                  0(A) = 0
                \end{align*}
                注意左侧$0$是列线性函数,右侧$0$是数值。

                显然零元素$0 \in P(K)$。

          \item 负元素。

                任意$f \in P(K)$,其负元素为$-f$。

                对任意$A \in M_n(K)$,有
                \begin{align*}
                  (f + (- f))(A) = f(A) + (- f(A)) = 0
                \end{align*}
                所以$f + (- f)$是零元素。
        \end{itemize}

  \item (2)

        任意矩阵$A \in M_n(K)$,设为
        \begin{align*}
          A = \begin{bmatrix}
                a_{11} & a_{12} & \cdots & a_{1n} \\
                a_{21} & a_{22} & \cdots & a_{2n} \\
                \vdots & \vdots & \ddots & \vdots \\
                a_{n1} & a_{n2} & \cdots & a_{nn}
              \end{bmatrix}
        \end{align*}
        即任意列向量$\alpha_i \in A$,表示成
        \begin{align*}
          \alpha_i = \begin{bmatrix}
                       a_{1i} \\
                       a_{2i} \\
                       \vdots \\
                       a_{ni}
                     \end{bmatrix}
        \end{align*}
        $\alpha_i$可以线性表示成
        \begin{align*}
          \alpha_i = a_{1i} \epsilon_1 + a_{2i} \epsilon_2 + \cdots + a_{ni} \epsilon_n
        \end{align*}
        于是,$A$可表示成
        \begin{align*}
          A = \begin{bmatrix}
                a_{11} \epsilon_1 + a_{21} \epsilon_2 + \cdots + a_{n1} \epsilon_n &
                a_{12} \epsilon_1 + a_{22} \epsilon_2 + \cdots + a_{n2} \epsilon_n &
                \cdots                                                             &
                a_{1n} \epsilon_1 + a_{2n} \epsilon_2 + \cdots + a_{nn} \epsilon_n
              \end{bmatrix}
        \end{align*}
        $f$是列线性函数,可以把$A$每列展开,
        具体展开式为
        \begin{align*}
          f(A)
          = \sum\limits_{i_1 = 1}^n \cdots \sum\limits_{i_n = 1}^n a_{i_1 1} \cdots a_{i_n n}
          f[\epsilon_{i_1}  \epsilon_{i_2}  \cdots  \epsilon_{i_n}]
        \end{align*}

  \item (3)

\end{itemize}

\section*{29}

设$F$作为$K$上的线性空间的一组基为
\begin{align*}
  \epsilon_1, \epsilon_2, \cdots, \epsilon_m
\end{align*}
$L$作为$F$上的线性空间的一组基为
\begin{align*}
  \eta_1, \eta_2, \cdots, \eta_n
\end{align*}

对任意$l \in L$,可以表示成
\begin{align*}
  l = f_1 \eta_1 + f_2 \eta_2 + \cdots + f_n \eta_n
\end{align*}
其中$f_1, f_2, \cdots, f_n \in F$,可以表示成
\begin{align*}
   & f_1 = k_{1_1} \epsilon_1 + k_{1_2} \epsilon_2 + \cdots + k_{1_m} \epsilon_m \\
   & f_2 = k_{2_1} \epsilon_1 + k_{2_2} \epsilon_2 + \cdots + k_{2_m} \epsilon_m \\
   & \vdots                                                                      \\
   & f_n = k_{n_1} \epsilon_1 + k_{n_2} \epsilon_2 + \cdots + k_{n_m} \epsilon_m
\end{align*}
于是,$l$可以表示成:
\begin{align*}
  l & = (k_{1_1} \epsilon_1 + k_{1_2} \epsilon_2 + \cdots + k_{1_m} \epsilon_m)\eta_1          \\
    & + (k_{2_1} \epsilon_1 + k_{2_2} \epsilon_2 + \cdots + k_{2_m} \epsilon_m)\eta_2          \\
    & + \cdots + (k_{n_1} \epsilon_1 + k_{n_2} \epsilon_2 + \cdots + k_{n_m} \epsilon_m)\eta_n \\
    & = \sum \limits_{i = 1}^n \left(\sum\limits_{j=1}^m k_{i_j} \epsilon_{j} \eta_i\right)
\end{align*}
其中$k_{i_j} \in K (1 \leq i \leq m, 1 \leq j \leq n)$,
因为$K \subseteq F \subseteq L$,所以$\epsilon_i \eta_j \in L$,
由此可得,任意$l \in L$,可被一组向量$(I)$:
\begin{align*}
  \epsilon_1 \eta_1, \epsilon_1 \eta_2, \cdots \epsilon_1 \eta_n, \\
  \epsilon_2 \eta_1, \epsilon_2 \eta_2, \cdots \epsilon_2 \eta_n, \\
  \vdots                                                          \\
  \epsilon_m \eta_1, \epsilon_m \eta_2, \cdots \epsilon_m \eta_n  \\
\end{align*}
用$K$上的数线性表示。

接下来,我们只要证明$(I)$是线性无关即可完成证明。

存在$k_{i_j} (1 \leq i \leq m, 1 \leq j \leq n)$(一定是存在的,全取0即可)使得:
\begin{align*}
  0 =
  k_{1_1} \epsilon_1 \eta_1 + k_{1_2} \epsilon_1 \eta_2 + \cdots + k_{1_n} \epsilon_1 \eta_n + \\
  k_{2_1} \epsilon_2 \eta_1 + k_{2_2} \epsilon_2 \eta_2 + \cdots + k_{2_n} \epsilon_2 \eta_n + \\
  \vdots                                                                                       \\
  +                                                                                            \\
  k_{m_1} \epsilon_m \eta_1 + k_{m_2} \epsilon_m \eta_2 + \cdots + k_{m_n} \epsilon_m \eta_n
\end{align*}
因为$\epsilon_i \eta_j$是数,所以满足相关运算法则,于是我们有
\begin{align*}
  0 & = (k_{1_1} \epsilon_1 + k_{1_2} \epsilon_2 + \cdots + k_{1_m} \epsilon_m)\eta_1          \\
    & + (k_{2_1} \epsilon_1 + k_{2_2} \epsilon_2 + \cdots + k_{2_m} \epsilon_m)\eta_2          \\
    & + \cdots + (k_{n_1} \epsilon_1 + k_{n_2} \epsilon_2 + \cdots + k_{n_m} \epsilon_m)\eta_n \\
\end{align*}
由于$\eta_1, \eta_2, \cdots, \eta_n$是线性无关,于是可得
\begin{align*}
  0 = k_{1_1} \epsilon_1 + k_{1_2} \epsilon_2 + \cdots + k_{1_m} \epsilon_m \\
  0 = k_{2_1} \epsilon_1 + k_{2_2} \epsilon_2 + \cdots + k_{2_m} \epsilon_m \\
  \vdots                                                                    \\
  0 = k_{n_1} \epsilon_1 + k_{n_2} \epsilon_2 + \cdots + k_{n_m} \epsilon_m
\end{align*}
又因为$\epsilon_1, \epsilon_2, \cdots, \epsilon_m$是线性无关,所以有
\begin{align*}
  k_{i_j} = 0, (1 \leq i \leq m, 1 \leq j \leq n)
\end{align*}
综上,$(I)$是线性无关的。

所以,$L$在$K$上是$mn$维线性空间。

\section*{30}

\begin{itemize}
  \item (1) 略

  \item (2) 略

  \item (3)
\end{itemize}

\end{document}
