\documentclass{article}
\usepackage{mathtools} 
\usepackage{fontspec}
\usepackage[UTF8]{ctex}
\usepackage{amsthm}
\usepackage{mdframed}
\usepackage{xcolor}
\usepackage{amssymb}
\usepackage{amsmath}
\usepackage{hyperref}
\usepackage{mathrsfs}


% 定义新的带灰色背景的说明环境 zremark
\newmdtheoremenv[
  backgroundcolor=gray!10,
  % 边框与背景一致,边框线会消失
  linecolor=gray!10
]{zremark}{注释}

% 通用矩阵命令: \flexmatrix{矩阵名}{元素符号}{行数}{列数}
\newcommand{\flexmatrix}[4]{
  \[
  #1 = \begin{pmatrix}
    #2_{11}     & #2_{12}     & \cdots & #2_{1#4}   \\
    #2_{21}     & #2_{22}     & \cdots & #2_{2#4}   \\
    \vdots      & \vdots      & \ddots & \vdots     \\
    #2_{#31}    & #2_{#32}    & \cdots & #2_{#3#4}
  \end{pmatrix}
  \]
}

% 简化版命令(默认矩阵名为A,元素符号为a): \quickmatrix{行数}{列数}
\newcommand{\quickmatrix}[2]{\flexmatrix{A}{a}{#1}{#2}}


\begin{document}
\title{4.4注释}
\author{张志聪}
\maketitle

\begin{zremark}
  两个$n$阶矩阵$A, B$相似,我们有
  \begin{align*}
    det(A) = det(B) \\
    Trace(A) = Trace(B)
  \end{align*}
\end{zremark}

\textbf{证明:}

todo

\begin{zremark}
  例4.4 的另一种证明方式。
\end{zremark}

todo

\textbf{证明:}


\begin{zremark}
  代数重数 $\geq$ 几何重数
\end{zremark}

\textbf{证明:}

\begin{zremark}
  $\mathscr{A}$是否可以对角化的充分必要条件:

  \begin{align*}
    \textbf{代数重数之和} = \textbf{几何重数之和}
  \end{align*}
  推论:
  \begin{align*}
    \textbf{每个特征值的代数重数} = \textbf{每个特征值的几何重数}
  \end{align*}

\end{zremark}




\end{document}
