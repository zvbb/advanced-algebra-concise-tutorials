\documentclass{article}
\usepackage{mathtools} 
\usepackage{fontspec}
\usepackage[UTF8]{ctex}
\usepackage{amsthm}
\usepackage{mdframed}
\usepackage{xcolor}
\usepackage{amssymb}
\usepackage{amsmath}
\usepackage{hyperref}


% 定义新的带灰色背景的说明环境 zremark
\newmdtheoremenv[
  backgroundcolor=gray!10,
  % 边框与背景一致,边框线会消失
  linecolor=gray!10
]{zremark}{注释}

% 通用矩阵命令: \flexmatrix{矩阵名}{元素符号}{行数}{列数}
\newcommand{\flexmatrix}[4]{
  \[
  #1 = \begin{pmatrix}
    #2_{11}     & #2_{12}     & \cdots & #2_{1#4}   \\
    #2_{21}     & #2_{22}     & \cdots & #2_{2#4}   \\
    \vdots      & \vdots      & \ddots & \vdots     \\
    #2_{#31}    & #2_{#32}    & \cdots & #2_{#3#4}
  \end{pmatrix}
  \]
}

% 简化版命令(默认矩阵名为A,元素符号为a): \quickmatrix{行数}{列数}
\newcommand{\quickmatrix}[2]{\flexmatrix{A}{a}{#1}{#2}}


\begin{document}
\title{4.1}
\author{张志聪}
\maketitle

\section*{8}

维数:2,一组基:
\begin{align*}
  1, w
\end{align*}

\section*{9}

维度:$n + \frac{n^2 - n}{2} = \frac{n^2 + n}{2}$,
一组基:
\begin{itemize}
  \item 对角线上$n$个;
        \begin{align*}
          E_{ij} (1 \leq i=j \leq n)
        \end{align*}

  \item 除了对角线,对称位置上的$\frac{n^2 - n}{2}$个;
        \begin{align*}
          \begin{bmatrix}
            0      & \cdots & 1      & 0      & \cdots & 0      \\
            \vdots & \ddots & \cdots & \cdots & \cdots & \cdots \\
            1      & \cdots & 0      & 0      & \cdots & 0      \\
            0      & \cdots & 0      & 0      & \cdots & 0      \\
            \vdots & \ddots & \cdots & \cdots & \cdots & \cdots \\
            0      & \cdots & 0      & 0      & \cdots & 0
          \end{bmatrix}
        \end{align*}
\end{itemize}

\section*{10}

维度:$n + \frac{n^2 - n}{2} = \frac{n^2 + n}{2}$,
一组基:
\begin{itemize}
  \item 对角线上$n$个;
        \begin{align*}
          E_{ij} (1 \leq i=j \leq n)
        \end{align*}

  \item 除了对角线,对称位置上的$\frac{n^2 - n}{2}$个;
        \begin{align*}
          \begin{bmatrix}
            0      & \cdots & 1      & 0      & \cdots & 0      \\
            \vdots & \ddots & \cdots & \cdots & \cdots & \cdots \\
            -1     & \cdots & 0      & 0      & \cdots & 0      \\
            0      & \cdots & 0      & 0      & \cdots & 0      \\
            \vdots & \ddots & \cdots & \cdots & \cdots & \cdots \\
            0      & \cdots & 0      & 0      & \cdots & 0
          \end{bmatrix}
        \end{align*}
\end{itemize}

\section*{11}

这里就无法一眼看出来了。

需要一个标准的流程找出一组基:

先找出一个非零向量,然后判断这个向量的线性组合是否可以表示线性空间中的所有向量,
如果可以,寻找到此结束。
如果不可以,把不能表示的向量加入到基中,并重复上述过程。

\begin{itemize}
  \item (1)

        第二题(5)的零向量为$(0, 0)$,令
        \begin{align*}
          \epsilon_1 = (1, 0)
        \end{align*}
        因为$\epsilon_1$的线性组合为
        \begin{align*}
          k \circ (1, 0) = [k, \frac{k(k-1)}{2}]
        \end{align*}
        显然,它无法表示$(0, 1)$,于是令
        \begin{align*}
          \epsilon_2 = (0, 1)
        \end{align*}
        它们线性组合为
        \begin{align*}
          k_1 \circ (1, 0) = [k_1, \frac{k_1(k_1-1)}{2}] \\
          k_2 \circ (0, 1) = (0, k_2)
        \end{align*}
        那么,对任意$(x, y)$,我们的以下线性方程组
        \begin{equation*}
          \begin{cases*}
            k_1 = x \\
            k_2 + \frac{k_1(k_1-1)}{2} = y
          \end{cases*}
        \end{equation*}
        可得
        \begin{equation*}
          \begin{cases*}
            k_1 = x \\
            k_2 = y - \frac{x(x-1)}{2}
          \end{cases*}
        \end{equation*}

        于是可得,任意元素$(x, y)$都可以被$(1, 0), (0, 1)$线性表示。

        综上,线性空间的维度是2,一组基为:
        \begin{align*}
          (1, 0), (0, 1)
        \end{align*}

  \item (2)

        第二题(7)的零向量为实数$1$。

        令
        \begin{align*}
          \epsilon_1 = 2
        \end{align*}

        对任意$y > 0$,我们有
        \begin{align*}
          k \circ 2 = 2^{k} = y \\
          k = log_2 y
        \end{align*}

        综上,线性空间的维度是1,一组基为:
        \begin{align*}
          2
        \end{align*}
\end{itemize}

\section*{12}

对任意$n$,我们有
\begin{align*}
  f(A) = a_0 E + a_1 A + a_2 A^2 + \cdots + a_n A^n
\end{align*}

我们有
\begin{align*}
  A \neq A^2 \neq A^3 = E \\
  A^n = A^{n \ mod \ 3}
\end{align*}
其中$n \ mod \ 3$表示$n$除以$3$的余数。

综上,维数为$3$,一组基为
\begin{align*}
  A, A^2, A^3
\end{align*}

\section*{14}

设
\begin{align*}
  E, A, A^2, \cdots, A^{m - 1} \ \ \ (I)
\end{align*}
要想证明$(I)$是一组基,需要证明$(I)$是线性无关的,且可以表示线性空间中的所有向量。

\begin{itemize}
  \item 线性无关性

        假设$(I)$是线性相关的,那么
        \begin{align*}
          k_0 E + k_2 A + k_3 A^2 + \cdots + k_m A^{m - 1} = 0
        \end{align*}
        存在非零解,这与题设中$f(\lambda)$是使$f(A) = 0$的最低次多项式相矛盾。

  \item 可以线性表示表示所有向量。

        只需证明对任意$n$,$A^n$都可以被$(I)$线性表示即可。

        对$n$进行归纳,$n = m$时,有题设可知,$A^n$可以被$(I)$线性表示。

        归纳假设$n = r$时,$A^r$可以被$(I)$线性表示。

        $n = r + 1$时,由归纳假设可知,存在$k_1, k_2, \cdots, k_m \in \mathbb{K}$,使得
        \begin{align*}
          A^r = k_1 E + k_2 A + k_3 A^2 + \cdots + k_m A^{m - 1}
        \end{align*}
        于是,我们有
        \begin{align*}
          A^{r + 1} & = A A^r                                                \\
                    & = A (k_1 E + k_2 A + k_3 A^2 + \cdots + k_m A^{m - 1}) \\
                    & = k_1 A + k_2 A^2 + k_3 A^3 + \cdots + k_m A^{m}
        \end{align*}
        又因为$A^m$也可以被$(I)$线性表示,所以,$A^{r + 1}$可以被$(I)$线性表示。
\end{itemize}
综上,$(I)$是$V$的一组基,从而$dim V = m$。

后边部分的证明略。

\section*{15}

要证明$(A - aE)^k (0 \leq k \leq m - 1)$是一组基,
由于$dim V = m$,于是我们只需证明$(A - aE)^k (0 \leq k \leq m - 1)$线性无关即可。

按二项展开公式,我们有
\begin{align*}
  (A - aE)^j = A^j + *
\end{align*}
由于无法被其他向量$(A - aE)^k (0 \leq k \leq m - 1, k \neq j)$线性表示,
于是,线性无关性得到证明。

\section*{16}

\begin{itemize}
  \item (1)

        \begin{align*}
          T = \begin{bmatrix}
                2  & 0 & 5 & 6 \\
                1  & 3 & 3 & 6 \\
                -1 & 1 & 2 & 1 \\
                1  & 0 & 1 & 3
              \end{bmatrix}
        \end{align*}
\end{itemize}

\section*{17}

设$\xi$在基$\epsilon_1, \epsilon_2, \epsilon_3, \epsilon_4$
中的坐标为$X$,在$\sigma_1, \sigma_2, \sigma_3, \sigma_4$中的坐标为$Y$。
由题设要求$X = Y$,并且
\begin{align*}
  T X = X    \\
  TX - X = 0 \\
  (T - E) X = 0
\end{align*}
由于$T$是已知的,所以需要求解上方的方程组即可。

\section*{18}
todo

\section*{19}

设$<n$的多项式$f(x)$为
\begin{align*}
  f(x) = k_0 + k_1 x + k_2 x^2 + \cdots + k_{n-1} x^{n-1}
\end{align*}
(注:这里不一定是$n-1$次的,因为$k_i(i = 0, 1, \cdots, n-1)$的取值可以为0)。

由题意可得如下线性方程组:
\begin{equation*}
  \begin{cases}
    k_0 + k_1 a_1 + k_2 a_1^2 + k_3 a_1^3 + \cdots + k_{n-1} a_1^{n-1} & = b_1  \\
    k_0 + k_1 a_2 + k_2 a_2^2 + k_3 a_2^3 + \cdots + k_{n-1} a_2^{n-1} & = b_2  \\
                                                                       & \vdots \\
    k_0 + k_1 a_n + k_2 a_n^2 + k_3 a_n^3 + \cdots + k_{n-1} a_n^{n-1} & = b_n
  \end{cases}
\end{equation*}
看做关于$k_0, k_1, \cdots, k_{n-1}$的线性方程组,其系数矩阵为
\begin{align*}
  A = \begin{bmatrix}
        1      & a_1    & a_1^2  & \cdots & a_1^{n-1} \\
        1      & a_2    & a_2^2  & \cdots & a_2^{n-1} \\
        \vdots & \vdots & \vdots & \ddots & \vdots    \\
        1      & a_n    & a_n^2  & \cdots & a_n^{n-1}
      \end{bmatrix}
\end{align*}
$A^T$矩阵是范德蒙德矩阵,且$a_1, a_2, \cdots, a_n$两两不同,
于是,我们有
\begin{align*}
  |A| = |A^T| = \prod\limits_{1 \leq i < j \leq n} (a_j - a_i) \neq 0
\end{align*}
所以,$A$是满秩的,线性方程组只有唯一解。
设
\begin{align*}
  K = \begin{bmatrix}
        k_0    \\
        k_1    \\
        \vdots \\
        k_{n-1}
      \end{bmatrix} \\
  B = \begin{bmatrix}
        b_1    \\
        b_2    \\
        \vdots \\
        b_n
      \end{bmatrix}
\end{align*}
我们有
\begin{align*}
  K = A^{-1} B
\end{align*}
综上,我们有
\begin{align*}
  f(x) & = \begin{bmatrix}
             1 & x & x^2 & \cdots & x^{n-1}
           \end{bmatrix} K \\
       & = \begin{bmatrix}
        1 & x & x^2 & \cdots & x^{n-1}
      \end{bmatrix} A^{-1} B
\end{align*}

\section*{20}

数域$K$上的$n$阶方阵构成一个线性空间$V$,加法和数乘就是矩阵的加法和数乘。
这个空间的基为$E_{ij} (1 \leq i,j \leq n)$,
维数为$n \times n = n^2$。

于是,对于
\begin{align*}
  A^{0}, A, A^{2}, \dots, A^{n^2} \ \ \ (I)
\end{align*}
有$n^2 + 1$个,且都属于$V$,所以$(I)$线性无关的,
从而,存在不全为零的$k_1, k_2, \dots, k_{n^2} \in K$,使得
\begin{align*}
  k_1 A^{0} + k_2 A + \dots + k_{n^2} A^{n^2} = 0
\end{align*}
即存在一个次数$\leq n^2$的多项式$f(x)$使得$f(A) = 0$。

\section*{21}

对于$\beta$,由(iv)可知,存在$\beta^\prime \in V$使得
\begin{align*}
  \beta + \beta^\prime = 0
\end{align*}

于是,我们有
\begin{align*}
  \beta + 0 + \beta^\prime = 0                \\
  \beta + (\alpha + \beta) + \beta^\prime = 0 \\
  \beta + \alpha + (\beta + \beta^\prime) = 0 \\
  \beta + \alpha + 0 = 0                      \\
  \beta + (\alpha + 0) = 0                    \\
  \beta + \alpha  = 0
\end{align*}

\section*{22}
对于$\alpha$,由(iv)可知,存在$\alpha^\prime \in V$使得
\begin{align*}
  \alpha + \alpha^\prime = 0
\end{align*}
利用习题21,我们有
\begin{align*}
  \alpha^\prime + \alpha = 0
\end{align*}
于是
\begin{align*}
  \alpha + 0 = \alpha                        \\
  \alpha + (\alpha^\prime + \alpha) = \alpha \\
  (\alpha + \alpha^\prime) + \alpha = \alpha \\
  0 + \alpha = \alpha
\end{align*}

\section*{23}

对任意的$\alpha, \beta \in V$,
由(iv)可知,存在$\alpha^\prime, \beta^\prime \in V$,使得
\begin{align*}
  \alpha + \alpha^\prime = 0 \\
  \beta + \beta^\prime = 0
\end{align*}
利用习题21,习题22和其他公理,我们有
\begin{align*}
  2(\alpha + \beta) = 2\alpha + 2\beta                                                                  \\
  \alpha + \beta + \alpha + \beta = \alpha + \alpha + \beta + \beta                                     \\
  \alpha^\prime + \alpha + \beta + \alpha + \beta = \alpha^\prime + \alpha + \alpha + \beta + \beta     \\
  (\alpha^\prime + \alpha) + \beta + \alpha + \beta = (\alpha^\prime + \alpha) + \alpha + \beta + \beta \\
  0 + \beta + \alpha + \beta = 0 + \alpha + \beta + \beta                                               \\
  \beta + \alpha + \beta = \alpha + \beta + \beta                                                       \\
  \beta + \alpha + \beta + \beta^\prime = \alpha + \beta + \beta + \beta^\prime                         \\
  \beta + \alpha + 0 = \alpha + \beta + 0                                                               \\
  \beta + \alpha = \alpha + \beta
\end{align*}

\end{document}
