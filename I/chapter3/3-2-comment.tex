\documentclass{article}
\usepackage{mathtools} 
\usepackage{fontspec}
\usepackage[UTF8]{ctex}
\usepackage{amsthm}
\usepackage{mdframed}
\usepackage{xcolor}
\usepackage{amssymb}
\usepackage{amsmath}
\usepackage{hyperref}
\usepackage{tikz}
\usetikzlibrary{shapes.geometric, arrows}
\usetikzlibrary{positioning}


% 定义新的带灰色背景的说明环境 zremark
\newmdtheoremenv[
  backgroundcolor=gray!10,
  % 边框与背景一致,边框线会消失
  linecolor=gray!10
]{zremark}{注释}

% 通用矩阵命令: \flexmatrix{矩阵名}{元素符号}{行数}{列数}
\newcommand{\flexmatrix}[4]{
  \[
  #1 = \begin{pmatrix}
    #2_{11}     & #2_{12}     & \cdots & #2_{1#4}   \\
    #2_{21}     & #2_{22}     & \cdots & #2_{2#4}   \\
    \vdots      & \vdots      & \ddots & \vdots     \\
    #2_{#31}    & #2_{#32}    & \cdots & #2_{#3#4}
  \end{pmatrix}
  \]
}

% 简化版命令(默认矩阵名为A,元素符号为a): \quickmatrix{行数}{列数}
\newcommand{\quickmatrix}[2]{\flexmatrix{A}{a}{#1}{#2}}


\begin{document}
\title{3.2 注释}
\author{张志聪}
\maketitle

\begin{zremark}
  命题2.1的推广 设$f$是$M_n(K) (n \geq 2)$上的列(行)线性函数,那么,下面命题相互等价:
  \begin{itemize}
    \item (0) 如果$A \in M_n(K)$有两列元素相同时,必有$f(A) = 0$。
    \item (1) 设将$A \in M_n(K)$的$i,j$两列(行)互换得出方阵$B$,则$f(B) = - f(A)$。
    \item (2) 设将$A \in M_n(K)$的第$j$列(行)加上其第$i$列(行)的$\lambda$倍($\lambda \in K$)
          得出方阵$B$,则$f(B) = f(A)$。
    \item (3) 对$K$上任何不满秩$n$阶方阵$A$都有$f(A) = 0$。
  \end{itemize}
\end{zremark}

\begin{zremark}
  通过列线性函数定义的行列式$f$,和通过行线性函数定义的行列式$g$,是等价的。
\end{zremark}

\textbf{证明:}

设任意$A \in M_n(K)$。
\begin{itemize}
  \item (1) $A$不满秩。

        由定义可知
        \begin{align*}
          f(A) = 0 = g(A)
        \end{align*}

  \item (2) $A$满秩。

        存在初等矩阵$P_1, P_2, \cdots, P_m$,使得
        \begin{align*}
          A = P_1 P_2 \cdots P_m
        \end{align*}

        如设$P_1, P_2, \cdots, P_m$中有$r$个第一类初等矩阵,$s$个第二类初等矩阵。

        把$A$看做
        \begin{align*}
          A = E P_1 P_2 \cdots P_m
        \end{align*}
        即只进行了初等列变换,那么
        \begin{align*}
          f(A) = (-1)^{r} \lambda_1 \cdots \lambda_s
        \end{align*}

        把$A$看做
        \begin{align*}
          A = P_1 P_2 \cdots P_m E
        \end{align*}
        即只进行了初等行变换,那么
        \begin{align*}
          g(A) = (-1)^{r} \lambda_1 \cdots \lambda_s
        \end{align*}
        所以
        \begin{align*}
          f(A) = g(A)
        \end{align*}
        综上,$f = g$。
\end{itemize}

\begin{zremark}
  $n \geq 2$个不同的自然数,奇排列和偶排列的个数相同。
\end{zremark}

\textbf{证明:}

\begin{itemize}
  \item 方法一:

        习题11

  \item 方法二:

        排列的总个数是$n!$,因为$n \geq 2$,所以总排列数是偶数个。

        设奇排列和偶排列的集合分别为$S_1, S_2$。

        定义映射$f : S_1 \rightarrow S_2$为:将一个奇排列中的第一个数与最后一个数交换。

        接下来,我们证明这个$f$是双射。

        \begin{itemize}
          \item (1) 单射;

                任意$a_1, a_2 \in S_1$且$a_1 \neq a_2$,
                由命题2.5可知,
                \begin{align*}
                  f(a_1), f(a_2) \in S_2
                \end{align*}

                假设$f(a_1) = f(a_2)$,那么
                \begin{align*}
                  a_1 = f(f(a_1)) = f(f(a_2)) = a_2
                \end{align*}
                与$a_1 \neq a_2$矛盾,可得
                \begin{align*}
                  f(a_1) \neq f(a_2)
                \end{align*}
                所以,$f$是单射。

          \item (2) 满射;

                任意$b \in S_2$,由命题2.5可知,
                交换排列$b$中的第一个数和最后一个数所得的$b^\prime \in S_1$,
                又
                \begin{align*}
                  f(b^\prime) = b
                \end{align*}
                所以,$f$是满射。
        \end{itemize}
\end{itemize}

\begin{zremark}
  行列式的常见计算方式
\end{zremark}

\begin{itemize}
  \item (1) 化成阶梯型(注意初等变换对行列式值的影响),通过上(下)对角矩阵行列式,对角线乘积即为行列式值。
  \item (2) 递推关系。
  \item (3) 范德蒙德行列式。
  \item (4) 按某行(列)展开,尤其是零多的情况下。
\end{itemize}

其实,如果是“固定阶”,蛮干也行。

\begin{zremark}
  反对称列线性函数有无穷多个,
  设$f$是反对称称列线性函数,
  那么,存在常数$k$使得$f (A) = k det(A)$的形式。
\end{zremark}

\textbf{证明:}

结合习题3,习题4,习题13

\begin{zremark}
  切比雪夫多项式(不是指某一个具体函数,而是指一类多项式):

  $k \in \mathbb{N}^{+}$,我们有
  \begin{align*}
    cos(k \alpha) = f_k (cos(\alpha))
  \end{align*}
  其中
  \begin{align*}
    f_k (x) = 2^{k - 1} x^k + *
  \end{align*}
  是首项系数为$2^{k - 1}$的$k$次多项式,即$cos(k \alpha)$是关于$cos(\alpha)$的多项式。
\end{zremark}

\textbf{证明:}

对$k$进行归纳,
$k = 1$时,$cos \alpha$显然成立。
归纳假设$k \leq n$时,命题成立。
$k = n+1$时
\begin{align*}
  cos[(n + 1)\alpha] + cos [(n - 1)\alpha]
                     & = 2 cos(n \alpha)cos(\alpha)                       \\
  cos[(n + 1)\alpha] & = 2 cos(n \alpha)cos(\alpha) - cos [(n - 1)\alpha]
\end{align*}
于是由归纳假设可知,$cos(n \alpha)$是关于$cos(\alpha)$的首项系数为$2^{n - 1}$的$n$次多项式,
于是
\begin{align*}
  2 cos(n \alpha)cos(\alpha)
\end{align*}
是关于$cos(\alpha)$的首项系数为$2^{n}$的$n+1$次多项式,
而$ - cos [(n - 1)\alpha]$不会影响最高次项,
综上,$k = n + 1$时,命题成立。

归纳完成。

\end{document}