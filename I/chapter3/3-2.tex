\documentclass{article}
\usepackage{mathtools} 
\usepackage{fontspec}
\usepackage[UTF8]{ctex}
\usepackage{amsthm}
\usepackage{mdframed}
\usepackage{xcolor}
\usepackage{amssymb}
\usepackage{amsmath}
\usepackage{hyperref}


% 定义新的带灰色背景的说明环境 zremark
\newmdtheoremenv[
  backgroundcolor=gray!10,
  % 边框与背景一致,边框线会消失
  linecolor=gray!10
]{zremark}{注释}

% 通用矩阵命令: \flexmatrix{矩阵名}{元素符号}{行数}{列数}
\newcommand{\flexmatrix}[4]{
  \[
  #1 = \begin{pmatrix}
    #2_{11}     & #2_{12}     & \cdots & #2_{1#4}   \\
    #2_{21}     & #2_{22}     & \cdots & #2_{2#4}   \\
    \vdots      & \vdots      & \ddots & \vdots     \\
    #2_{#31}    & #2_{#32}    & \cdots & #2_{#3#4}
  \end{pmatrix}
  \]
}

% 简化版命令(默认矩阵名为A,元素符号为a): \quickmatrix{行数}{列数}
\newcommand{\quickmatrix}[2]{\flexmatrix{A}{a}{#1}{#2}}


\begin{document}
\title{3.1}
\author{张志聪}
\maketitle

\section*{3}

提示:
设
\begin{align*}
  f_m(A) = m det(A)
\end{align*}

\section*{25}

对任意$n$阶方正$A = (a_{ij})$,
对应的$B$为
\begin{align*}
  B = \begin{bmatrix}
        a_{11}b^{1-1} & a_{12}b^{1-2} & \cdots & a_{1n}b^{1-n} \\
        a_{21}b^{2-1} & a_{22}b^{2-2} & \cdots & a_{2n}b^{2-n} \\
        \vdots        & \vdots        & \ddots & \vdots        \\
        a_{n1}b^{n-1} & a_{n2}b^{n-2} & \cdots & a_{nn}b^{n-n}
      \end{bmatrix}
\end{align*}
于是,我们有(先提取行向量的公因子,再提取列向量的公因子)
\begin{align*}
  |B| & = b b^2 \cdots b^{n-1} \begin{vmatrix}
                                 a_{11}b^{-1} & a_{12}b^{-2} & \cdots & a_{1n}b^{-n} \\
                                 a_{21}b^{-1} & a_{22}b^{-2} & \cdots & a_{2n}b^{-n} \\
                                 \vdots       & \vdots       & \ddots & \vdots       \\
                                 a_{n1}b^{-1} & a_{n2}b^{-2} & \cdots & a_{nn}b^{-n}
                               \end{vmatrix}             \\
      & = b b^2 \cdots b^{n-1} \frac{1}{b b^2 \cdots b^{n-1}}\begin{vmatrix}
                                                               a_{11} & a_{12} & \cdots & a_{1n} \\
                                                               a_{21} & a_{22} & \cdots & a_{2n} \\
                                                               \vdots & \vdots & \ddots & \vdots \\
                                                               a_{n1} & a_{n2} & \cdots & a_{nn}
                                                             \end{vmatrix} \\
      & = |A|
\end{align*}

\section*{26}

todo

\section*{27}

把矩阵看做行向量组,
\begin{align*}
  \begin{bmatrix}
    \alpha_0^T \\
    \alpha_1^T \\
    \vdots     \\
    \alpha_{n - 1}^T
  \end{bmatrix}
\end{align*}

我们证明,对任意$\alpha_i$,可以被其他行向量做$i - 1$次初等变换,化为只剩首项系数的形式。

对$i$进行归纳,
$i = 1$时
\begin{align*}
  \alpha_1^T = (a_1 b_1 + c_1 , a_1 b_2 + c_1 , \cdots , a_1 b_n + c_1)
\end{align*}
其中$c_1$是一个定值。
我们有
\begin{align*}
  \alpha_0^T = (a_0, a_0, \cdots, a_0)
\end{align*}
所以,可以通过$\alpha_0^T$,去掉$\alpha_1^T$中的$c_1$,得到
\begin{align*}
  (a_1 b_1 , a_1 b_2 , \cdots , a_1 b_n)
\end{align*}
归纳假设$i = k$时,也成立。

$i = k + 1$时,我们有
\begin{align*}
  \alpha_{k+1}^T = (a_{k+1} b_1^{k + 1} + * b_1^{k} + \cdots + c_{k+1}, a_{k+1} b_2^{k + 1} + * b_2^{k} + \cdots + c_{k+1}, \cdots, a_{k+1} b_n^{k + 1} + * b_n^{k} + \cdots + c_{k+1})
\end{align*}
由归纳假设可知,$\alpha_{k+1}^T$通过$k - 1$次初等变化,可得
\begin{align*}
  (a_{k+1} b_1^{k + 1} + * b_1^{k}, a_{k+1} b_2^{k + 1} + * b_2^{k}, \cdots, a_{k+1} b_n^{k + 1} + * b_n^{k})
\end{align*}
再加上$- \frac{*}{a_k} \alpha_{k}^T$,我们有
\begin{align*}
  (a_{k+1} b_1^{k + 1}, a_{k+1} b_2^{k + 1}, \cdots, a_{k+1} b_n^{k + 1})
\end{align*}
归纳完成。

于是,我们有
\begin{align*}
  \begin{vmatrix}
    f_0(b_1)     & f_0(b_2)     & \cdots & f_0(b_n)     \\
    f_1(b_1)     & f_1(b_2)     & \cdots & f_1(b_n)     \\
    \vdots       & \vdots       & \ddots & \vdots       \\
    f_{n-1}(b_1) & f_{n-1}(b_2) & \cdots & f_{n-1}(b_n)
  \end{vmatrix}
   & = \begin{vmatrix}
         a_0               & a_0              & \cdots & a_0               \\
         a_1 b_1^1         & a_1 b_2^1        & \cdots & a_1 b_n^1         \\
         \vdots            & \vdots           & \ddots & \vdots            \\
         a_{n-1} b_1^{n-1} & a_{n-1}b_2^{n-1} & \cdots & a_{n-1} b_n^{n-1}
       \end{vmatrix}        \\
   & = a_0 a_2 \cdots a_{n - 1} \begin{vmatrix}
                                  1         & 1         & \cdots & 1         \\
                                  b_1^1     & b_2^1     & \cdots & b_n^1     \\
                                  \vdots    & \vdots    & \ddots & \vdots    \\
                                  b_1^{n-1} & b_2^{n-1} & \cdots & b_n^{n-1}
                                \end{vmatrix}      \\
   & = a_0 a_2 \cdots a_{n - 1}  \prod\limits_{1 \leq j < i \leq n} (b_i - b_j)
\end{align*}

\section*{28}

对任意$k \in \mathbb{N}^{+}$,
我们有
\begin{align*}
  cos(k \alpha) = 2^{k - 1} (cos \alpha)^k + *
\end{align*}
这里的$*$是其他次数小于$k$的项。

与习题27类似的方式化简,本题中的行列式可化为
\begin{align*}
   & \begin{vmatrix}
       1                               & 1                               & \cdots & 1                               \\
       cos\alpha_1                     & cos\alpha_2                     & \cdots & cos\alpha_n                     \\
       2 (cos\alpha_1)^2               & 2 (cos\alpha_2)^2               & \cdots & 2 (cos\alpha_n)^2               \\
       \vdots                                                                                                       \\
       2^{n - 2} (cos\alpha_1)^{n - 1} & 2^{n - 2} (cos\alpha_2)^{n - 1} & \cdots & 2^{n - 2} (cos\alpha_n)^{n - 1}
     \end{vmatrix} \\
   & = 2 \cdot 2^{2} \cdots 2^{n - 2}
  \begin{vmatrix}
    1                     & 1                     & \cdots & 1                     \\
    cos\alpha_1           & cos\alpha_2           & \cdots & cos\alpha_n           \\
    (cos\alpha_1)^2       & (cos\alpha_2)^2       & \cdots & (cos\alpha_n)^2       \\
    \vdots                                                                         \\
    (cos\alpha_1)^{n - 1} & (cos\alpha_2)^{n - 1} & \cdots & (cos\alpha_n)^{n - 1}
  \end{vmatrix}                                  \\
   & = 2^{\frac{(n-2)(n-1)}{2}} \prod \limits_{1 \leq j < i \leq n} (cos\alpha_i - cos\alpha_j)
\end{align*}


\end{document}
