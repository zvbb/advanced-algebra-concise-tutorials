\documentclass{article}
\usepackage{mathtools} 
\usepackage{fontspec}
\usepackage[UTF8]{ctex}
\usepackage{amsthm}
\usepackage{mdframed}
\usepackage{xcolor}
\usepackage{amssymb}
\usepackage{amsmath}
\usepackage{hyperref}
\usepackage{tikz}
\usetikzlibrary{shapes.geometric, arrows}
\usetikzlibrary{positioning}


% 定义新的带灰色背景的说明环境 zremark
\newmdtheoremenv[
  backgroundcolor=gray!10,
  % 边框与背景一致,边框线会消失
  linecolor=gray!10
]{zremark}{注释}

% 通用矩阵命令: \flexmatrix{矩阵名}{元素符号}{行数}{列数}
\newcommand{\flexmatrix}[4]{
  \[
  #1 = \begin{pmatrix}
    #2_{11}     & #2_{12}     & \cdots & #2_{1#4}   \\
    #2_{21}     & #2_{22}     & \cdots & #2_{2#4}   \\
    \vdots      & \vdots      & \ddots & \vdots     \\
    #2_{#31}    & #2_{#32}    & \cdots & #2_{#3#4}
  \end{pmatrix}
  \]
}

% 简化版命令(默认矩阵名为A,元素符号为a): \quickmatrix{行数}{列数}
\newcommand{\quickmatrix}[2]{\flexmatrix{A}{a}{#1}{#2}}


\begin{document}
\title{3.3 注释}
\author{张志聪}
\maketitle

\begin{zremark}
  命题 3.6 (命题3.5的加强):

  设$A$是数域$K$上一个$m \times n$矩阵,
  则$rank(A) \geq r$当且仅当
  $A$有一个$r$阶子式不为$0$。
\end{zremark}

\textbf{证明:}

\begin{itemize}
  \item 必要性。

        因为$rank(A) \geq r$,那么对于$A$的列向量组,存在一个极大线性无关部分组,
        从中任选$r$个线性无关列向量,
        得到一个$m \times r$的矩阵$B$,
        矩阵的列秩与行秩相等,于是$B$的行向量组有$r$个线性无关行向量,
        以它们为行向量组成$r \times r$的矩阵$\overline{B}$,因为
        $\overline{B}$满秩,所以$|\overline{B}| \neq 0$,
        这就是要找的,关于$A$的不为零的$r$阶子式。

  \item 充分性。

        设$A$的列向量为$\beta_1, \beta_2, \dots, \beta_n$,并设$A$有$r$阶子式
        以$A$的第$i_1, i_2, \ldots, i_r$行向量和第$j_1, j_2, \ldots, j_r$列向量
        组成$r \times r$矩阵$B$,且$|B| \neq 0$,所以$B$是满秩的,
        于是,我们有$\beta_{j_1}, \beta_{j_2}, \cdots, \beta_{j_r}$是线性无关,
        又$A$列向量组的极大线性无关部分组可以线性表示$\beta_{j_1}, \beta_{j_2}, \cdots, \beta_{j_r}$,
        由替换定理可知$rank(A) \geq r$。
\end{itemize}

\begin{zremark}
  通过 命题3.6 可以证明 命题3.5
\end{zremark}


\end{document}