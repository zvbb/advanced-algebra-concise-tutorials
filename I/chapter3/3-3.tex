\documentclass{article}
\usepackage{mathtools} 
\usepackage{fontspec}
\usepackage[UTF8]{ctex}
\usepackage{amsthm}
\usepackage{mdframed}
\usepackage{xcolor}
\usepackage{amssymb}
\usepackage{amsmath}
\usepackage{hyperref}


% 定义新的带灰色背景的说明环境 zremark
\newmdtheoremenv[
  backgroundcolor=gray!10,
  % 边框与背景一致,边框线会消失
  linecolor=gray!10
]{zremark}{注释}

% 通用矩阵命令: \flexmatrix{矩阵名}{元素符号}{行数}{列数}
\newcommand{\flexmatrix}[4]{
  \[
  #1 = \begin{pmatrix}
    #2_{11}     & #2_{12}     & \cdots & #2_{1#4}   \\
    #2_{21}     & #2_{22}     & \cdots & #2_{2#4}   \\
    \vdots      & \vdots      & \ddots & \vdots     \\
    #2_{#31}    & #2_{#32}    & \cdots & #2_{#3#4}
  \end{pmatrix}
  \]
}

% 简化版命令(默认矩阵名为A,元素符号为a): \quickmatrix{行数}{列数}
\newcommand{\quickmatrix}[2]{\flexmatrix{A}{a}{#1}{#2}}


\begin{document}
\title{3.3}
\author{张志聪}
\maketitle

\section*{1}

证明$\beta_1, \beta_2, \cdots, \beta_s$是线性无关的,
当且仅当
\begin{align*}
  k_1\beta_1 + k_2\beta_2 + \cdots + k_s \beta_s = 0
\end{align*}
只有非零解。

又由题设可知
\begin{align*}
  k_1\beta_1 + k_2\beta_2 + \cdots + k_s \beta_s = 0     \\
  (a_{11} k_1 + a_{21}k_2 + \cdots + a_{s1} k_s)\alpha_1 \\
  +
  (a_{12} k_1 + a_{22}k_2 + \cdots + a_{s2} k_s)\alpha_2 \\
  + \cdots                                               \\
  +
  (a_{1s} k_1 + a_{2s}k_2 + \cdots + a_{ss} k_s)\alpha_s \\
  = 0
\end{align*}
设
\begin{align*}
  X     & = \begin{bmatrix}
              k_1 & k_2 & \cdots & k_s
            \end{bmatrix}          \\
  A     & = \begin{bmatrix}
              a_{11} & a_{12} & \cdots & a_{1s} \\
              a_{21} & a_{22} & \cdots & a_{2s} \\
              \vdots & \vdots & \ddots & \vdots \\
              a_{s1} & a_{s2} & \cdots & a_{ss}
            \end{bmatrix} \\
  A^{T} & = \begin{bmatrix}
              a_{11} & a_{21} & \cdots & a_{s1} \\
              a_{12} & a_{22} & \cdots & a_{s2} \\
              \vdots & \vdots & \ddots & \vdots \\
              a_{1s} & a_{2s} & \cdots & a_{ss}
            \end{bmatrix}
\end{align*}
因为$\alpha_1, \alpha_2, \cdots, \alpha_s$是线性无关的,
那么,$\beta_1, \beta_2, \cdots, \beta_s$线性无关当且仅当
\begin{align*}
  A^T X = 0
\end{align*}
是否只存在非零解,即当且仅当$A^T$满秩,当且仅当$A$满秩。

\section*{4}

提示
\begin{itemize}
  \item (1)

        设线性方程组的系数矩阵为
        \begin{align*}
          A = \begin{bmatrix}
                a_{11}      & a_{12}      & \cdots & a_{1n}      \\
                a_{21}      & a_{22}      & \cdots & a_{2n}      \\
                \vdots      & \vdots      & \ddots & \vdots      \\
                a_{n-1 \ 1} & a_{n-1 \ 2} & \cdots & a_{n-1 \ n}
              \end{bmatrix}
        \end{align*}
        令
        \begin{align*}
          X = \begin{bmatrix}
                M_1    \\
                -M_2   \\
                \vdots \\
                (-1)^{n - 1} M_n
              \end{bmatrix}
        \end{align*}
        我们要证明
        \begin{align*}
          A X = 0
        \end{align*}
        考虑$AX$的第$i$行,我们有
        \begin{align*}
          a_{i1} M_1 + a_{i2} (-M_2) + \cdots + a_{in} [(-1)^{n-1}M_n]
        \end{align*}
        这是以下矩阵的行列式值
        \begin{align*}
          B_i = \begin{bmatrix}
                  a_{11}      & a_{12}      & \cdots & a_{1n}      \\
                  a_{21}      & a_{22}      & \cdots & a_{2n}      \\
                  \vdots      & \vdots      & \ddots & \vdots      \\
                  a_{i1}      & a_{i2}      & \cdots & a_{in}      \\
                  a_{i1}      & a_{i2}      & \cdots & a_{in}      \\
                  \vdots                                           \\
                  a_{n-1 \ 1} & a_{n-1 \ 2} & \cdots & a_{n-1 \ n}
                \end{bmatrix}
        \end{align*}
        由于矩阵$B_i$不是满秩的,所以$det(B_i) = 0$。

  \item (2) 因为基础解系只有一个非零解向量,其他解都是这个解向量的线性组合,
        所以是这个解向量的倍数。

        我们需要证明$(M_1, -M_2, \cdots, (-1)^{n-1}M_n)$是非零向量。
        因为系数矩阵的秩为$n - 1$,那么由命题3.5可知,存在一个非零的$n - 1$阶
        子式,因为$n - 1$阶子式不为零,所以列数不能重复,由于系数矩阵的行数是$n - 1$,
        所以在子式对应的矩阵$B$包含了所有行,而列向量是从系数矩阵的列向量组中去除一个列向量得到的,
        这种选择只有$n$种,所以存在$M_i$可以通过$B$调整行向量的顺序得到(与系数矩阵一致),
        于是,我们有
        \begin{align*}
          det(B) = \pm M_i
        \end{align*}
        即子式和$M_i$只会和$det(B)$相差一个正负号。

        由于$M_i \neq 0$,所以,$(M_1, -M_2, \cdots, (-1)^{n-1}M_n)$是非零向量。

\end{itemize}

\section*{5}


我们有
\begin{align*}
  A A^{*} = det(A) E
\end{align*}

如果$det(A) \neq 0$,那么
\begin{align*}
  det(A A^{*}) = det(det(A) E)        \\
  det(A) det(A^{*}) = det(A)^n det(E) \\
  det(A^{*}) = det(A)^{n - 1}
\end{align*}

如果$det(A) = 0$,所以$A$不满秩,
由习题6可知,$A^{*}$不满秩,所以
\begin{align*}
  det(A^{*}) = 0
\end{align*}

\section*{6}

\begin{itemize}
  \item (1) $rank(A) = n$。

        如果$rank(A^{*})$不满秩,那么
        \begin{align*}
          det(A A^{*}) = 0
        \end{align*}
        这与
        \begin{align*}
          det(A A^{*}) = det(det(A) E) = det(A)^n \neq 0
        \end{align*}
        矛盾。

  \item (2) $rank(A) = n - 1$。

        命题3.5可知,$A$存在一个$n - 1$阶子式不为零,
        由不为零可知子式不会有相同行,
        从$n$个行中选$n - 1$行,
        有$n$种选择。
        同时从$n$个列中选$n - 1$列,
        有$n$种选择。
        因此有$n^2$种选择。

        $A^{*}$的元素不考虑符号,就是这$n^2$中选择,
        于是可得$A^{*}$存在非零元素,所以
        \begin{align*}
          rank(A^{*}) \geq 1
        \end{align*}

        由于
        \begin{align*}
          A A^{*} = |A|E = 0
        \end{align*}
        所以
        \begin{align*}
          rank(A A^{*}) = 0
        \end{align*}
        又因为
        \begin{align*}
          rank(A A^{*}) \geq rank(A) + rank(A^{*}) - n \\
          0 \geq n - 1 + rank(A^{*}) - n               \\
          1 \geq rank(A^{*})
        \end{align*}
        综上
        \begin{align*}
          rank(A^{*}) = 1
        \end{align*}

  \item (3) $rank(A) < n - 1$

        由习题3.5可知,$A$的$n - 1$阶子式都为$0$,
        与$(2)$中类似的分析可知,$A^{*} = 0$,
        所以
        \begin{align*}
          rank(A^{*}) = 0
        \end{align*}

\end{itemize}

\section*{7}

\begin{itemize}
  \item (1)
        \begin{align*}
          |B| & = |T^{-1}AT|       \\
              & = |T^{-1}| |A| |T| \\
              & = |T^{-1}| |T| |A| \\
              & = |T^{-1}T| |A|    \\
              & = |E| |A|          \\
              & = |A|
        \end{align*}

  \item (2)

        矩阵转置行列式不变,于是
        \begin{align*}
          |B| & = |T^{T} A T|     \\
              & = |T^{T}| |A| |T| \\
              & = |T| |A| |T|     \\
              & = |T|^2 |A|
        \end{align*}
        所以,$|A|,|B|$同号。
\end{itemize}

\section*{8}
这是打洞的题。

我们有
\begin{align*}
  \begin{bmatrix}
    E         & 0 \\
    -C A^{-1} & E
  \end{bmatrix}
  \begin{bmatrix}
    A & B \\
    C & D
  \end{bmatrix}
   & = \begin{bmatrix}
         A & B              \\
         0 & D - C A^{-1} B
       \end{bmatrix}
\end{align*}
两边求行列式,
\begin{align*}
  1 \cdot |R| & = |A| \cdot |D - C A^{-1} B| \\
  |R|         & = |A| \cdot |D - C A^{-1} B| \\
\end{align*}

\section*{9}

设
\begin{align*}
  G = \begin{vmatrix}
        a_{11} & a_{12} & b_1 \\
        a_{12} & a_{22} & b_2 \\
        b_1    & b_2    & c
      \end{vmatrix}
\end{align*}
那么,二次曲线可以表示成
\begin{align*}
  \begin{bmatrix}
    x & y & 1
  \end{bmatrix}
  G
  \begin{bmatrix}
    x \\
    y \\
    1
  \end{bmatrix}
  = 0
\end{align*}
设
\begin{align*}
  T = \begin{bmatrix}
        cos\theta & -sin\theta & x_0 \\
        sin\theta & cos\theta  & y_0 \\
        0         & 0          & 1
      \end{bmatrix}
\end{align*}
坐标变换可以表示成
\begin{align*}
  \begin{bmatrix}
    x \\
    y \\
    1 \\
  \end{bmatrix}
  = T \begin{bmatrix}
        x^\prime \\
        y^\prime \\
        1
      \end{bmatrix}
\end{align*}
二次曲线被替换变量$x, y$后,我们有
\begin{align*}
  \begin{bmatrix}
    x^\prime & y^\prime & 1
  \end{bmatrix}
  T^T G T \begin{bmatrix}
            x^\prime \\
            y^\prime \\
            1
          \end{bmatrix}
  = 0
\end{align*}
我们要证明
\begin{align*}
  det(T^T G T) = det(G) = F
\end{align*}
因为
\begin{align*}
  det(T) = 1 = det(T^T)
\end{align*}
所以
\begin{align*}
  det(T^T G T)
   & = det(T^T) det(G) det(T) \\
   & = 1 \cdot det(G) \cdot 1 \\
   & = det(G) = F
\end{align*}

\section*{11}

设$f(x) = c_0 + c_1 x + c_2 x^2 + \cdots + c_{n - 1} x^{n - 1}$。
由$f(a_i) = b_i$,我们得到一个线性方程组,他的系数矩阵为
\begin{align*}
  A = \begin{bmatrix}
        1 & a_1 & a_1^2 & \cdots & a_{1}^{n - 1} \\
        1 & a_2 & a_2^2 & \cdots & a_{2}^{n - 1} \\
        \vdots                                   \\
        1 & a_n & a_n^2 & \cdots & a_{n}^{n - 1} \\
      \end{bmatrix}
\end{align*}
线性方程组可表示成
\begin{align*}
  A \begin{bmatrix}
      c_0    \\
      c_1    \\
      \vdots \\
      c_{n - 1}
    \end{bmatrix}
  = \begin{bmatrix}
      b_1    \\
      b_2    \\
      \vdots \\
      b_{n}
    \end{bmatrix}
\end{align*}
$A^T$是范德蒙当行列式,又因为$a_1, a_2, \cdots, a_n$两两不同,所以
\begin{align*}
  det(A^T) = det(A) \neq 0
\end{align*}
可得$A$是满秩的,因为线性方程组的增广矩阵$\overline{A}$的秩也是$n$(因为不能超过行向量个数),
所以线性方程组一定有解,且解是唯一的。
即$c_0, c_1, \cdots, c_{n - 1}$是唯一的,
进而$f(x)$多项式是唯一的。

\section*{12}

\begin{itemize}
  \item (1)

        $n$阶行列对应的矩阵$A$可以表示成
        \begin{align*}
          \begin{bmatrix}
            x_1 & x_2 & \cdots & x_n
          \end{bmatrix}
          \begin{bmatrix}
            y_1    \\
            y_2    \\
            \vdots \\
            y_n
          \end{bmatrix}
          + \begin{bmatrix}
              1 & 1 & \cdots & 1
            \end{bmatrix}
          \begin{bmatrix}
            1      \\
            1      \\
            \vdots \\
            1
          \end{bmatrix}
        \end{align*}
        以上矩阵秩一矩阵,于是可得
        \begin{align*}
          rank(A) \leq 1 + 1 = 2
        \end{align*}

        于是,$n > 2$时,$det(A) = 0$。

        $n = 1$时,$det(A) = 1 + x_1y_1$;

        $n = 2$时,$det(A) = (1 + x_1y_1)(1 + x_2y_2) - (1 + x_1y_2)(1 + x_2y_1)$

  \item (2)

        $n$阶行列对应的矩阵$A$可以表示成
        \begin{align*}
          \begin{bmatrix}
            1           & 1           & \cdots & 1           \\
            a_1         & a_2         & \cdots & a_n         \\
            \vdots      & \vdots      & \ddots & \vdots      \\
            a_1^{n - 1} & a_2^{n - 1} & \cdots & a_n^{n - 1}
          \end{bmatrix}
          \begin{bmatrix}
            1      & a_1    & \cdots & a_1^{n - 1} \\
            1      & a_2    & \cdots & a_2^{n - 1} \\
            \vdots & \vdots & \ddots & \vdots      \\
            1      & a_n    & \cdots & a_n^{n - 1}
          \end{bmatrix}
        \end{align*}

  \item (3)

        设行列式的矩阵为$A$。

        \begin{itemize}
          \item 方法一:按照例3.3的思想。

                令$\epsilon_k = e^{\frac{2k \pi i}{n}}$,设
                \begin{align*}
                  f(x) = a_1 x^{n - 1} + a_2 x^{n - 2} + \cdots + a_n
                \end{align*}
                构造$n$阶方阵
                \begin{align*}
                  B = \begin{bmatrix}
                        \epsilon_1^{n-1} & \epsilon_2^{n-1} & \cdots & \epsilon_n^{n-1} \\
                        \epsilon_1^{n-2} & \epsilon_2^{n-2} & \cdots & \epsilon_n^{n-2} \\
                        \vdots           & \vdots           & \ddots & \vdots           \\
                        \epsilon_1       & \epsilon_2       & \cdots & \epsilon_n       \\
                        1                & 1                & \cdots & 1                \\
                      \end{bmatrix}
                \end{align*}
                于是
                \begin{align*}
                  A B & = \begin{bmatrix}
                            a_1     & a_2    & \cdots & a_n       \\
                            a_2     & a_3    & \cdots & a_1       \\
                            \vdots  & \vdots & \ddots & \vdots    \\
                            a_{n-1} & a_n    & \cdots & a_{n - 2} \\
                            a_n     & a_1    & \cdots & a_{n - 1} \\
                          \end{bmatrix}
                  \begin{bmatrix}
                    \epsilon_1^{n-1} & \epsilon_2^{n-1} & \cdots & \epsilon_n^{n-1} \\
                    \epsilon_1^{n-2} & \epsilon_2^{n-2} & \cdots & \epsilon_n^{n-2} \\
                    \vdots           & \vdots           & \ddots & \vdots           \\
                    \epsilon_1       & \epsilon_2       & \cdots & \epsilon_n       \\
                    1                & 1                & \cdots & 1                \\
                  \end{bmatrix}                                                   \\
                      & = \begin{bmatrix}
                            f(\epsilon_1)                  & f(\epsilon_2)                  & \cdots & f(\epsilon_n)                  \\
                            \epsilon_1 f(\epsilon_1)       & \epsilon_2 f(\epsilon_2)       & \cdots & \epsilon_n f(\epsilon_n)       \\
                            \vdots                         & \vdots                         & \ddots & \vdots                         \\
                            \epsilon_1^{n-2} f(\epsilon_1) & \epsilon_2^{n-2} f(\epsilon_2) & \cdots & \epsilon_n^{n-2} f(\epsilon_n) \\
                            \epsilon_1^{n-1} f(\epsilon_1) & \epsilon_2^{n-1} f(\epsilon_2) & \cdots & \epsilon_n^{n-1} f(\epsilon_n) \\
                          \end{bmatrix}
                \end{align*}
                于是,我们有
                \begin{align*}
                  |A| \cdot |B| = |A B| = f(\epsilon_1)f(\epsilon_2)\cdots f(\epsilon_n) (-1)^{\frac{(n-1)n}{2}} |B|
                \end{align*}
                因为$|B| \neq 0$,故有
                \begin{align*}
                  |A| = (-1)^{\frac{(n-1)n}{2}} f(\epsilon_1)f(\epsilon_2)\cdots f(\epsilon_n)
                \end{align*}

          \item 方法二:直接利用3.3的结论。

                把矩阵$A$通过初等行变换化为例3.3的样式,具体操作如下,
                对第$n$行依次和$n-1, n - 2, \cdots, 2$进行交换,
                对第$n$行依次和$n-1, n - 2, \cdots, 3$进行交换,
                $\vdots$
                对第$n$与$n-1$进行交换。

                总的交换次数为
                \begin{align*}
                  (n-2) + (n - 3) + \cdots + 3 + 2 + 1 = \frac{(n-2)(n-1)}{2}
                \end{align*}
                利用3.3结论,我们有
                \begin{align*}
                  |A| = (-1)^{\frac{(n-2)(n-1)}{2}} f(\epsilon_1)f(\epsilon_2)\cdots f(\epsilon_n)
                \end{align*}

                这里的结果好像与方法一的不一致,这是因为他们的$f$是不同的。
        \end{itemize}
\end{itemize}

\section*{13}

原题有错误,应该是:
\begin{align*}
  \begin{vmatrix}
    E_m & B   \\
    A   & E_n
  \end{vmatrix}
  = \begin{vmatrix}
      E_n - AB
    \end{vmatrix}
  = \begin{vmatrix}
      E_m - BA
    \end{vmatrix}
\end{align*}

提示:
\begin{align*}
  \begin{bmatrix}
    E_m        & 0   \\
    -AE_m^{-1} & E_n
  \end{bmatrix}
  \begin{bmatrix}
    E_m & B   \\
    A   & E_n
  \end{bmatrix}
  = \begin{bmatrix}
      E_m & B        \\
      0   & E_n - AB
    \end{bmatrix}
\end{align*}

\begin{align*}
  \begin{bmatrix}
    E_m & - B E_n^{-1} \\
    0   & E_n
  \end{bmatrix}
  \begin{bmatrix}
    E_m & B   \\
    A   & E_n
  \end{bmatrix}
  = \begin{bmatrix}
      E_m - BA & 0   \\
      A        & E_n
    \end{bmatrix}
\end{align*}

\section*{14}

\begin{itemize}
  \item 方法一:通过初等列变换

        把$A$的第一列依次与前方的$n, n-1, \cdots, 1$列互换;
        然后是$A$的第二列,依次与$n+1, n, \cdots, 2$列互换;
        重复直到$A$的最后一列,我们得到
        \begin{align*}
          \begin{bmatrix}
            A & C \\
            0 & B
          \end{bmatrix}
        \end{align*}
        因为$A$有$m$列,所以经过了$m n$次初等列变换。
        于是,我们有
        \begin{align*}
          |M| = (-1)^{mn} |A| \cdot |B|
        \end{align*}

  \item 方法二:通过矩阵乘法

        我们有
        \begin{align*}
          \begin{bmatrix}
            C & A \\
            B & 0
          \end{bmatrix}
          \begin{bmatrix}
            0   & E_n \\
            E_m & 0
          \end{bmatrix}
          =
          \begin{bmatrix}
            A & C \\
            0 & B
          \end{bmatrix}
        \end{align*}

        设$D = \begin{bmatrix}
            0   & E_n \\
            E_m & 0
          \end{bmatrix}$为$(a_{ij})$,于是我们有
        \begin{align*}
          |D| = \sum \limits_{(i_1, i_2, \cdots, i_{m+n})}(-1)^{N(i_1, i_2, \cdots, i_{m+n})} a_{i_1 1}a_{i_2 2} \cdots a_{i_{m+n} m+n}
        \end{align*}
        $|D|$是$(n + m)!$项相加,易得只有一项是非零的,即选择$E_n, E_m$有值的行,
        具体的行号为
        \begin{align*}
          n, n+1, \cdots, n+m, 1, 2, \cdots, n
        \end{align*}
        对应的反序数为$nm$,所以
        \begin{align*}
          |D| = (-1)^{nm}
        \end{align*}
        于是
        \begin{align*}
          \begin{vmatrix}
            C & A \\
            B & 0
          \end{vmatrix} = (-1)^{nm} |A| \cdot |B|
        \end{align*}

\end{itemize}

\end{document}
