\documentclass{article}
\usepackage{mathtools} 
\usepackage{fontspec}
\usepackage[UTF8]{ctex}
\usepackage{amsthm}
\usepackage{mdframed}
\usepackage{xcolor}
\usepackage{amssymb}
\usepackage{amsmath}
\usepackage{hyperref}


% 定义新的带灰色背景的说明环境 zremark
\newmdtheoremenv[
  backgroundcolor=gray!10,
  % 边框与背景一致,边框线会消失
  linecolor=gray!10
]{zremark}{注释}

% 通用矩阵命令: \flexmatrix{矩阵名}{元素符号}{行数}{列数}
\newcommand{\flexmatrix}[4]{
  \[
  #1 = \begin{pmatrix}
    #2_{11}     & #2_{12}     & \cdots & #2_{1#4}   \\
    #2_{21}     & #2_{22}     & \cdots & #2_{2#4}   \\
    \vdots      & \vdots      & \ddots & \vdots     \\
    #2_{#31}    & #2_{#32}    & \cdots & #2_{#3#4}
  \end{pmatrix}
  \]
}

% 简化版命令(默认矩阵名为A,元素符号为a): \quickmatrix{行数}{列数}
\newcommand{\quickmatrix}[2]{\flexmatrix{A}{a}{#1}{#2}}


\begin{document}
\title{2.1}
\author{张志聪}
\maketitle

\section*{1-5}
略

\section*{6}

由$\alpha_1, \alpha_2, \cdots, \alpha_s, \beta$线性相关,
我们有
\begin{align*}
  k_1\alpha_1 + k_2\alpha_2 + \cdots + k_s\alpha_s + k_{s + 1} \beta = 0
\end{align*}
存在非零解。

可知$k_{s + 1} \neq 0$,因为如果$k_{s + 1} = 0$,则
\begin{align*}
  k_1\alpha_1 + k_2\alpha_2 + \cdots + k_s\alpha_s = 0
\end{align*}
存在非零解,于是$\alpha_1, \alpha_2, \cdots, \alpha_s$线性相关,
与题设矛盾。

所以,$\beta$可以被$\alpha_1, \alpha_2, \cdots, \alpha_s$线性表示成
\begin{align*}
  \beta = - \frac{1}{k_{s + 1}} (k_1\alpha_1 + k_2\alpha_2 + \cdots + k_s\alpha_s)
\end{align*}

\section*{7}

对任意线性无关的向量组
\begin{align*}
  \alpha_1, \alpha_2, \cdots, \alpha_s \ \ \ (I)
\end{align*}

设其部分组为
\begin{align*}
  \alpha_{i1}, \alpha_{i2}, \cdots, \alpha_{ir} \ \ \ (II)
\end{align*}
其中$1 \leq ir \leq s$。

$ir = 1$时,由于是线性无关的,所以不存在零向量,
于是向量组$\alpha_{i1}$是线性无关的。

$ir > 1$时,假设$\alpha_{i1}, \alpha_{i2}, \cdots, \alpha_{ir}$线性相关,
于是向量组$(II)$中存在$\gamma$向量可以被$(II)$中的其他向量线性表示,
因为$(II)$中的向量也是$(I)$中的向量,
进而可得$(I)$线性相关,与题设矛盾,假设不成立。

综上,命题得证。

\section*{8}

略

\section*{9}

\begin{itemize}
  \item (1)

        反证法,假设$\alpha_1, \alpha_2, \cdots, \alpha_m$线性相关,
        于是
        \begin{align*}
          k_1 \alpha_1 + k_2 \alpha_2 + \cdots + k_m \alpha_m = 0
        \end{align*}
        存在非零解。

        以上可以看做对应方程组的非零解,这个解满足所有方程,每个向量中去掉$i_1, i_2 , \cdots, i_s$个分量,
        会让方程数减少,但这个解还是可以满足剩下的方程,即:
        \begin{align*}
          k_1 \alpha_1^\prime + k_2 \alpha_2\prime + \cdots + k_m \alpha_m\prime = 0
        \end{align*}
        可得$\alpha_1^\prime, \alpha_2^\prime, \cdots, \alpha_m^\prime$是线性相关的,
        与题设矛盾。

  \item (2)

        是(1)中证明的一部分:

        $\alpha_1, \alpha_2, \cdots, \alpha_m$线性相关,
        于是
        \begin{align*}
          k_1 \alpha_1 + k_2 \alpha_2 + \cdots + k_m \alpha_m = 0
        \end{align*}
        存在非零解。

        以上可以看做对应方程组的非零解,这个解满足所有方程,每个向量中去掉$i_1, i_2 , \cdots, i_s$个分量,
        会让方程数减少,但这个解还是可以满足剩下的方程,即:
        \begin{align*}
          k_1 \alpha_1^\prime + k_2 \alpha_2\prime + \cdots + k_m \alpha_m\prime = 0
        \end{align*}
        可得$\alpha_1^\prime, \alpha_2^\prime, \cdots, \alpha_m^\prime$是线性相关的。
\end{itemize}

\section*{10}

就是矩阵的初等变换的另一种阐述。

\section*{11}

\begin{itemize}
  \item 方法一(推荐)

        因为$\alpha_1, \alpha_2, \cdots, \alpha_s$是线性相关的,那么
        \begin{align*}
          k_1 \alpha_1 + k_2 \alpha_2 + \cdots + k_s \alpha_s = 0
        \end{align*}
        存在非零解。

        从右往左,一定可以找到第一个非零的系数$k_t (1 \leq t \leq s)$
        (因为题设中$\alpha_1 \neq 0$,所以找到的第一个非零系数不会是$k_1$),
        于是,$\alpha_t$就可以被之前的向量线性表示。

  \item 方法二(复杂)

        反证法,假设不存在$\alpha_i$可以被$\alpha_1, \alpha_2, \cdots, \alpha_{i - 1}$表示。
        那么,任意$1 < k \leq s$时,$\alpha_{k}$
        也不能被$\alpha_1, \alpha_2, \cdots, \alpha_{k - 1}, \alpha_{k + 1}, \cdots, \alpha_s$线性表示。
        因为如果能够被线性表示成:
        \begin{align*}
          \alpha_{k} = c_1 \alpha_1 + c_2\alpha_2 + \cdots + c_{k - 1}\alpha_{k - 1} + c_{k + 1} \alpha_{k + 1} + \cdots + c_s \alpha_s
        \end{align*}
        由假设可知,$c_{k + 1}, c_{k + 2}, \cdots, c_s$均为0,
        因为如果$c_s \neq 0$,那么
        \begin{align*}
          \alpha_s = \frac{1}{c_s} \alpha_{k} - \frac{1}{c_s} (c_1 \alpha_1 + c_2\alpha_2 + \cdots + c_{k - 1}\alpha_{k - 1} + c_{k + 1} \alpha_{k + 1} + \cdots + c_{s-1} \alpha_{s - 1})
        \end{align*}
        与假设矛盾,所以$c_s = 0$;类似地,可推出其他系数为$0$。

        于是,我们有
        \begin{align*}
          \alpha_{k} = c_1 \alpha_1 + c_2\alpha_2 + \cdots + c_{k - 1}\alpha_{k - 1}
        \end{align*}
        会与假设矛盾。

        由$k$的任意性可知,向量组$\alpha_1, \alpha_2, \cdots, \alpha_s$是线性无关的,
        与题设矛盾,假设不成立,命题得证。
\end{itemize}

\section*{12}

略

\section*{13}

按照极大线性无关部分组的定义证明。

线性无关性,题设已给出,我们只需证明
$\alpha_1, \alpha_2, \cdots, \alpha_s$可以被
$\alpha_{i1}, \alpha_{i2}, \cdots, \alpha_{ir}$线性表示即可。
由条件(2)可知,
\begin{align*}
  a_1 \alpha_i + a_2 \alpha_{i_1} + \cdots + a_{r + 1} \alpha_{i_r} = 0
\end{align*}
存在非零解,这里$a_1 \neq 0$,否者$\alpha_{i_1}, \alpha_{i_2}, \cdots, \alpha_{i_r}$是线性相关的,
与条件(1)矛盾。

所以,$\alpha_i$可以被$\alpha_{i1}, \alpha_{i2}, \cdots, \alpha_{ir}$线性表示,
由$\alpha_i$的任意性可知,满足了极大线性无关部分组的定义,
命题得证。


\section*{14}

设
\begin{align*}
  \alpha_1, \alpha_2, \cdots, \alpha_s \ \ \ (I)
\end{align*}

反证法,假设命题不成立,那么,有$(I)$的线性无关的部分组:
\begin{align*}
  \alpha_{i1}, \alpha_{i2}, \cdots, \alpha_{ir} \ \ \ (II)
\end{align*}
存在$\alpha_t \in (I)$无法被$(II)$线性表示。

于是,我们有
\begin{align*}
  \alpha_{i1}, \alpha_{i2}, \cdots, \alpha_{ir}, \alpha_t  \ \ \ (III)
\end{align*}
是线性无关的(反证法,利用习题6)。

设
\begin{align*}
  \alpha_{j1}, \alpha_{j2}, \cdots, \alpha_{jr} \ \ \ (I^\prime)
\end{align*}
是$(I)$的极大线性无关部分组,因为$(I)$的秩是$r$,所以$(I^\prime)$的个数为$r$。

$(I^\prime)$可以线性表示$(I)$,因为$(III)$是$(I)$的部分组,所以$(I^\prime)$也可以线性表示$(III)$。
由替换定理结论1(2-1-comment.tex中有证,也是命题1.4的逆否命题)可得
\begin{align*}
  r \geq r + 1
\end{align*}
存在矛盾,假设不成立,命题得证。

\section*{15}

设
\begin{align*}
  \alpha_1, \alpha_2, \cdots, \alpha_s \ \ \ (I) \\
  \alpha_{i1}, \alpha_{i2}, \cdots, \alpha_{ir} \ \ \ (II)
\end{align*}

由极大线性无关部分组的定义可知,我们只需证明$(II)$是线性无关的即可。

反证法,假设$(II)$不是线性无关的,那么$\alpha_{i1}, \alpha_{i2}, \cdots, \alpha_{ir}$
中存在可以被其他向量表示的向量,不妨设为$\alpha_{i1}$,从$(II)$
中删除向量$\alpha_{i1}$,得到新的向量组:
\begin{align*}
  \alpha_{i2}, \alpha_{i3}, \cdots, \alpha_{ir} \ \ \ (III)
\end{align*}
可得$(II)$和$(III)$线性等价,于是$(III)$可以线性表示$(I)$。

设
\begin{align*}
  \alpha_{j1}, \alpha_{j2}, \cdots, \alpha_{jr} \ \ \ (I^\prime)
\end{align*}
是$(I)$的极大线性无关部分组,因为$(I)$的秩是$r$,所以$(I^\prime)$的向量个数为$r$。

$(III)$可以线性表示$(I)$,从而可以线性表示$(I^\prime)$。

利用替换定理结论1(2-1-comment.tex中有证,也是命题1.4的逆否命题)可得
\begin{align*}
  r - 1 \geq r
\end{align*}
存在矛盾,假设不成立,命题得证。

\section*{16}

设$(I^\prime)$是$(I)$的极大线性无关部分组,秩为$r$。\\
设$(II^\prime)$是$(II)$的极大线性无关部分组,秩为$s$。

由题设$(II)$可以线性表示$(I)$可得,$(II^\prime)$可以线性表示$(I^\prime)$。

利用替换定理结论1(2-1-comment.tex中有证,也是命题1.4的逆否命题)可得
\begin{align*}
  r \leq s
\end{align*}
命题得证。

\section*{17}

设
\begin{align*}
  \epsilon_1, \epsilon_2, \cdots, \epsilon_n \ \ \ (I)
  \alpha_1, \alpha_2, \cdots, \alpha_n \ \ \ (II)
\end{align*}
易得$(I)$是线性无关的。

假设$(II)$是线性相关的,那么存在$(II)$的某个向量,可以被$(II)$的中的其他向量表示,
不妨设为$\alpha_1$,从$(II)$中删除向量$\alpha_1$,得到新的向量组:
\begin{align*}
  \alpha_2, \alpha_3, \cdots, \alpha_n \ \ \ (III)
\end{align*}
$(III)$与$(II)$线性等价,由题设$(II)$可以线性表示$(I)$可得$(III)$可以线性表示$(I)$。

由替换定理结论1(2-1-comment.tex中有证,也是命题1.4的逆否命题)可得
\begin{align*}
  n - 1 \geq n
\end{align*}
存在矛盾,假设不成立,命题得证。

\section*{18}

设
\begin{align*}
  \alpha_1, \alpha_2, \cdots, \alpha_n \ \ \ (I)
\end{align*}

$n$维坐标向量组:
\begin{align*}
  \epsilon_1, \epsilon_2, \cdots, \epsilon_n \ \ \ (I^\prime)
\end{align*}

\begin{itemize}
  \item 必要性

        已知(I)线性无关,假设存在$n$维向量$\beta$不能被(I)线性表示,
        于是
        \begin{align*}
          \beta, \alpha_1, \alpha_2, \cdots, \alpha_n \ \ \ (II)
        \end{align*}
        也是线性无关的。

        因为任意$n$维向量都可以被$(I^\prime)$线性表示,
        于是$(II)$可以被$(I^\prime)$线性表示,
        且$(II)$是线性无关的,利用替换定理结论1(2-1-comment.tex中有证,也是命题1.4的逆否命题)可得
        \begin{align*}
          n \geq n + 1
        \end{align*}
        存在矛盾,假设不成立。

  \item 充分性

        已知$(I)$可是线性表示任意$n$维向量,
        于是$(I^\prime)$中的任意向量都可以被$(I)$线性表示,
        利用习题17可得,$(I)$是线性无关的。
\end{itemize}

\section*{19}

设

任意向量组为:
\begin{align*}
  \alpha_1, \alpha_2, \cdots, \alpha_n \ \ \ (I)
\end{align*}

(I)的任意线性无关部分组为:
\begin{align*}
  \alpha_{i1}, \alpha_{i2}, \cdots, \alpha_{ir} \ \ \ (II)
\end{align*}

取$(I)$的任意极大线性无关部分组为:
\begin{align*}
  \alpha_{j1}, \alpha_{j2}, \cdots, \alpha_{js} \ \ \ (I^\prime)
\end{align*}

由替换定理结论(2)可知,$(II)$可以适当替换$(I^\prime)$中的向量得到向量组$(II^\prime)$,
且$(II^\prime)$与$(I^\prime)$线性等价。且由替换定理结论(3)可知,
$(II^\prime)$也是线性无关的,综上可得,$(II^\prime)$是$(I)$的极大线性无关部分组,
命题得证。

\section*{20}

设
\begin{align*}
  \alpha_1, \alpha_2, \cdots, \alpha_r                                     & \ \ \ (I)  \\
  \alpha_1, \alpha_2, \cdots, \alpha_r, \alpha_{r + 1}, \cdots, \alpha_{s} & \ \ \ (II)
\end{align*}
假设$(I)$和$(II)$两者不等价(以$(I)$无法线性表示$(II)$中的所有向量为例,其他情况类似),
设两者的极大线性无关部分组分别为$(I^\prime), (II^\prime)$,且由题设可知$(I^\prime)$与$(II^\prime)$向量个数相同,
不妨设为$r$。

由题设知$(I)$是$(II)$的部分组,于是可知$(II^\prime)$是可以线性表示$(I^\prime)$的。
又因为$(I)$无法线性表示$(II)$中的所有向量,从而$(I^\prime)$也无法线性表示$(II)$中的所有向量,
即,存在$\beta \in (II)$无法被$(I^\prime)$线性表示,把$\beta$加入$(I^\prime)$得到
新的向量组$(I^{\prime\prime})$,利用习题6可知$(I^{\prime\prime})$线性无关。

综上,我们有$(II^\prime)$可以线性表示$(I^{\prime\prime})$,
利用替换定理结论1(或命题1.4的逆否命题)可知
\begin{align*}
  r \geq r + 1
\end{align*}
存在矛盾,假设不成立,命题得证。

\section*{21}

设
\begin{align*}
  \alpha_1, \alpha_2, \cdots, \alpha_r \ \ \  & (I)  \\
  \beta_1, \beta_2, \cdots, \beta_r \ \ \     & (II)
\end{align*}
由题设可知$(I)$可以线性表示$(II)$,于是利用习题16可知,
$(I)$的秩大于等于$(II)$的秩。

令
\begin{align*}
  \alpha : = \{ \alpha_1, \alpha_2, \cdots, \alpha_r\}
\end{align*}
于是
\begin{align*}
  \alpha_i = \alpha - \beta_i
\end{align*}
所以
\begin{align*}
  \alpha, \beta_1, \beta_2, \cdots, \beta_r \ \ \  & (III)
\end{align*}
可以线性表示$(I)$。

因为
\begin{align*}
  \beta_1 + \beta_2 + \cdots + \beta_r & = (r - 1) \alpha                                        \\
  \alpha                               & = \frac{1}{r - 1}(\beta_1 + \beta_2 + \cdots + \beta_r)
\end{align*}
可得,$(II)$可以线性表示$(III)$。

于是,$(II)$可以线性表示$(I)$,利用习题16可知,$(II)$的秩大于$(I)$的秩。

综上可得,$(I)$与$(II)$的秩相等。

扩展:证明过程也表明了$(I), (II)$线性等价。

\section*{22}

设$\alpha_1 - \alpha_2, \alpha_2 - \alpha_3, \alpha_3 - \alpha_1$的极大线性无关部分组为$(I^\prime)$。

会不会他本身就是极大线性无关部分组?

我们有
\begin{align*}
  (\alpha_1 - \alpha_2) + (\alpha_2 - \alpha_3) + (\alpha_3 - \alpha_1) = 0
\end{align*}
可知其不是线性无关的,所以$(I^\prime)$的秩小于$3$。

那么$\alpha_1 - \alpha_2, \alpha_2 - \alpha_3$是否线性相关(其他情况类似)?
即
\begin{align*}
  x_1(\alpha_1 - \alpha_2) + x_2(\alpha_2 - \alpha_3) = 0
\end{align*}
是否存在非零解? 假设存在非零解,那么
\begin{align*}
  x_1 \alpha_1 + (x_2 - x_1) \alpha_2 - x_2 \alpha_3 = 0
\end{align*}
成立,这表明$\alpha_1, \alpha_2, \alpha_3$是线性相关的,与假设矛盾,
假设不成立。

由替换定理可知,$(I^\prime)$的秩大于等于$2$,

综上,$(I^\prime)$的秩为2,任取向量组中的两个向量,都是
其极大线性无关部分组。

\section*{23}

设
\begin{align*}
  \alpha_1, \alpha_2, \cdots \alpha_n                            & \ \ \ (I)        \\
  \alpha_{i_1}, \alpha_{i_2}, \cdots \alpha_{i_r}                & \ \ \ (I^\prime) \\
  \alpha - \alpha_1, \alpha - \alpha_2, \cdots \alpha - \alpha_n & \ \ \ (II)
\end{align*}
由习题21可知,$(I),(II)$是线性等价的,那么可得两者的秩都是$r$。

假设
\begin{align*}
  \alpha - \alpha_{i_1}, \alpha - \alpha_{i_2}, \cdots, \alpha - \alpha_{i_r} \ \ \ (II^\prime)
\end{align*}
是$(II)$的极大线性无关部分组。

如果$(I^\prime), (II^\prime)$线性等价,
则$(II^\prime)$可以线性表示$(I)$,进而可以线性表示$(II)$,
又$(II^\prime)$向量的个数$r$,等于$(II)$的秩,利用习题15可以保证$(II^\prime)$的线性无关,
从而是$(II)$的极大线性无关部分组。
于是问题转换为:证明$(I^\prime), (II^\prime)$线性等价。


引入一个临时向量组:
\begin{align*}
  \alpha_{i_1}, \alpha_{i_2}, \cdots \alpha_{i_r}, \alpha \ \ \ (III)
\end{align*}
我们证明$(I^\prime), (III)$线性等价,$(III), (II^\prime)$线性等价,从而得到
$(I^\prime), (II^\prime)$线性等价。

$(I^\prime), (III)$线性等价是显然的。

$(III)$可以线性表示$(II^\prime)$也是显然的,我们主要考虑$(II^\prime)$线性表示$(III)$
的证明。先考虑$\alpha$如何用$(II^\prime)$线性表示。

我们有
\begin{align*}
  k_1 (\alpha - \alpha_{i_1}) + k_2 (\alpha - \alpha_{i_2}) + \cdots + k_r (\alpha - \alpha_{i_r})
   & = (k_1 + k_2 + \cdots + k_r)\alpha - (k_1\alpha_{i_1} + k_2\alpha_{i_2} + \cdots + k_r\alpha_{i_r}) \\
   & = (k_1 + k_2 + \cdots + k_r)\alpha - \alpha                                                         \\
   & = (k_1 + k_2 + \cdots + k_r - 1)\alpha
\end{align*}
由题设可知$k_1 + k_2 + \cdots + k_r - 1 \neq 0$,于是可得
\begin{align*}
  \alpha = \frac{1}{k_1 + k_2 + \cdots + k_r - 1} k_1 (\alpha - \alpha_{i_1}) + k_2 (\alpha - \alpha_{i_2}) + \cdots + k_r (\alpha - \alpha_{i_r})
\end{align*}
所以,$\alpha$可以被$(II^\prime)$线性表示。

至于$(III)$中其他向量的线性表示则是显然的。

综上可得,$(I^\prime), (II^\prime)$线性等价,命题得证。

\section*{24}

反证法,假设$\alpha_1, \alpha_2, \cdots, \alpha_s$线性无关,
那么
\begin{align*}
  k_1 \alpha_1 + k_2 \alpha_2 + \cdots + k_s \alpha_s = 0
\end{align*}
存在非零解。

取$k_1, k_2, \cdots, k_s$中的最大值,不妨设为$k_i$(因为存在非零解,所以$k_i \neq 0$)。

取第$i$行方程,我们有
\begin{align*}
  k_1 a_{1i} + k_2 a_{2i} + \cdots + k_ia_{ii} + \cdots + k_s a_{si}                                           & = 0                \\
  - (k_1 a_{1i} + k_2 a_{2i} + \cdots + k_{i - 1} a_{(i - 1)i} + k_{i+1} a_{(i + 1)i} + k_s a_{si})            & = k_ia_{ii}        \\
  |k_1 a_{1i} + k_2 a_{2i} + \cdots + k_{i-1} a_{(i - 1)i} + k_{i+1} a_{(i + 1)i} + k_s a_{si}|                & = |k_i||a_{ii}|    \\
  |k_1||a_{1i}| + |k_2||a_{2i}| + \cdots + |k_{i-1}||a_{(i - 1)i}| + |k_{i + 1}||a_{(i + 1)i}| + |k_s||a_{si}| & \geq |k_i||a_{ii}| \\
  |k_i|(|a_{1i}| + |a_{2i}| + \cdots + |a_{(i - 1)i}| + |a_{(i + 1)i}| + |a_{si}|) \geq |k_i||a_{ii}|                               \\
  |a_{1i}| + |a_{2i}| + \cdots + |a_{(i - 1)i}| + |a_{(i + 1)i}| + |a_{si}| \geq |a_{ii}|
\end{align*}
与题设矛盾,假设不成立,命题得证。

\section*{25}

设
\begin{align*}
  \eta_1, \eta_2, \cdots, \eta_n             & \ \ \ (I)  \\
  \epsilon_1, \epsilon_2, \cdots, \epsilon_n & \ \ \ (II)
\end{align*}
$(II)$的秩是$n$,接下来,我们只需证明$(I),(II)$线性等价,就能说明$(I)$的秩也为$n$。

引入一个向量$n$维向量
\begin{align*}
  \epsilon = (1, 1, \cdots, 1)
\end{align*}
新建一个向量组
\begin{align*}
  \eta_1, \eta_2, \cdots, \eta_n, \epsilon \ \ \ (III)
\end{align*}
我们只需证明$(I),(II)$都与$(III)$线性等价,则$(I),(II)$线性等价。

$(II), (III)$线性等价是显然的;

$(III)$可以线性表示$(I)$也是显然的;

$(I)$是否可以线性表示$(III)$,关键在于是否可以表示$\epsilon$。

我们有
\begin{align*}
   & \frac{1}{a_1}\eta_1 + \frac{1}{a_2}\eta_2 + \frac{1}{a_n} \eta_n                \\
   & = (\frac{1}{a_1} \epsilon + \epsilon_1) + (\frac{1}{a_2} \epsilon + \epsilon_2)
  + \cdots + (\frac{1}{a_n} \epsilon + \epsilon_n)                                   \\
   & = (\frac{1}{a_1} + \frac{1}{a_2}+\cdots+ \frac{1}{a_n} + 1) \epsilon
\end{align*}
由题设可知
\begin{align*}
  1 + \frac{1}{a_1} + \frac{1}{a_2} + \cdots + \frac{1}{a_n} \neq 0
\end{align*}
于是可得
\begin{align*}
  \epsilon = \frac{1}{1 + \frac{1}{a_1} + \frac{1}{a_2} + \cdots + \frac{1}{a_n}} (\frac{1}{a_1}\eta_1 + \frac{1}{a_2}\eta_2 + \frac{1}{a_n} \eta_n           )
\end{align*}
所以,$\epsilon$可以被$(I)$线性表示,于是我们有$(I), (III)$线性等价。

综上,$(I),(II)$线性等价,于是秩都等于$n$。


\end{document}