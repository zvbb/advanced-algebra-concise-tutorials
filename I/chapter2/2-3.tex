\documentclass{article}
\usepackage{mathtools} 
\usepackage{fontspec}
\usepackage[UTF8]{ctex}
\usepackage{amsthm}
\usepackage{mdframed}
\usepackage{xcolor}
\usepackage{amssymb}
\usepackage{amsmath}
\usepackage{hyperref}


% 定义新的带灰色背景的说明环境 zremark
\newmdtheoremenv[
  backgroundcolor=gray!10,
  % 边框与背景一致,边框线会消失
  linecolor=gray!10
]{zremark}{注释}

% 通用矩阵命令: \flexmatrix{矩阵名}{元素符号}{行数}{列数}
\newcommand{\flexmatrix}[4]{
  \[
  #1 = \begin{pmatrix}
    #2_{11}     & #2_{12}     & \cdots & #2_{1#4}   \\
    #2_{21}     & #2_{22}     & \cdots & #2_{2#4}   \\
    \vdots      & \vdots      & \ddots & \vdots     \\
    #2_{#31}    & #2_{#32}    & \cdots & #2_{#3#4}
  \end{pmatrix}
  \]
}

% 简化版命令(默认矩阵名为A,元素符号为a): \quickmatrix{行数}{列数}
\newcommand{\quickmatrix}[2]{\flexmatrix{A}{a}{#1}{#2}}


\begin{document}
\title{2.3}
\author{张志聪}
\maketitle

\section*{1}

只做第一题,练练手。

先做矩阵消元法
\begin{align*}
  \begin{bmatrix}
    1 & 1 & 1 & 1 & 1  \\
    3 & 2 & 1 & 1 & -3 \\
    0 & 1 & 2 & 2 & 6  \\
    5 & 4 & 3 & 3 & -1
  \end{bmatrix}
  \xrightarrow{\text{处理第一列}}
  \begin{bmatrix}
    1 & 1  & 1  & 1  & 1  \\
    0 & -1 & -2 & -2 & -6 \\
    0 & 1  & 2  & 2  & 6  \\
    0 & -1 & -2 & -2 & -6
  \end{bmatrix}
  \xrightarrow{\text{处理第二列}}
  \begin{bmatrix}
    1 & 1  & 1  & 1  & 1  \\
    0 & -1 & -2 & -2 & -6 \\
    0 & 0  & 0  & 0  & 0  \\
    0 & 0  & 0  & 0  & 0
  \end{bmatrix}
\end{align*}
$rank(A) = 2$,故基础解系中应包含$n - r = 5 - 2 = 3$个向量。
写出阶梯型矩阵的对应方程组
\begin{equation*}
  \begin{cases*}
    x_1 + x_2 + x_3 + x_4 + x_5 = 0 \\
    -x_2 - 2x_3 -2 x_4 -6 x_5 = 0
  \end{cases*}
\end{equation*}
移项,得
\begin{equation*}
  \begin{cases*}
    x_1 + x_2 = - x_3 - x_4 - x_5 \\
    -x_2 = 2x_3 + 2 x_4 + 6 x_5
  \end{cases*}
\end{equation*}
$x_3,x_4,x_5$为自由未知量。
\begin{itemize}
  \item (i) 取$x_3 = 1, x_4 = 0, x_5 = 0$,得一个解向量
        \begin{align*}
          \eta_1 = (1, -2, 1, 0, 0)
        \end{align*}
  \item (ii)取$x_3 = 0, x_4 = 1, x_5 = 0$,得一个解向量
        \begin{align*}
          \eta_2 = (1, -2, 0, 1, 0)
        \end{align*}
  \item (iii)取$x_3 = 0, x_4 = 0, x_5 = 1$,得一个解向量
        \begin{align*}
          \eta_3 = (5, -6, 0, 0, 1)
        \end{align*}
\end{itemize}
于是$\eta_1, \eta_2, \eta_3$为方程组的一个基础解系。方程组的全部解可表为
\begin{align*}
  k_1 \eta_1 + k_2 \eta_2 + k_3 \eta_3
\end{align*}
其中$k_1, k_2, k_3$为数域$K$内任意数。

\section*{2}

设$(I)$是基础解系,$(II)$是与$(I)$线性等价的任意向量组。

按照基础解系的定理,我们要验证三点:
\begin{itemize}
  \item (1)$(II)$中的向量都是解向量。

        任意$\beta \in (II)$都可以被$(I)$线性表示,
        因为$(I)$中都是解向量,他们的线性表示$\beta$,也是解向量。

  \item (2)$(II)$线性无关。

        题设保证的。

  \item (3)解向量都可以被$(II)$线性表示。

        因为任意解向量都可以被$(I)$线性表示,
        $(I)$和$(II)$线性等价,
        于是也能被$(II)$线性表示。
\end{itemize}

\section*{3}

不妨设方程组的基础解系为
\begin{align*}
  \alpha_1, \alpha_2, \cdots \alpha_{n - r} \ \ \ (I)
\end{align*}

满足题设条件的向量组为
\begin{align*}
  \beta_1, \beta_2, \cdots \beta_{n - r} \ \ \ (II)
\end{align*}

\begin{itemize}
  \item 方法一

        利用$\S 1$命题1.4(替换定理)

        假设$(II)$不是基础解系,那么存在解向量$\beta$无法被$(II)$线性表示,
        由命题3.2的逆否命题可知,向量组
        \begin{align*}
          (III) : = (II), \beta
        \end{align*}
        是线性无关的,于是秩为$n - r + 1$。

        因为$(I)$是基础解系,于是$(I)$可以线性表示所有解,于是可以线性表示
        $(III)$,又因为$(III)$线性无关,由$\S 1$命题1.4的逆否命题可知
        \begin{align*}
          n - r \geq n - r + 1
        \end{align*}
        存在矛盾,假设不成立,命题得证。

  \item 方法二

        利用习题2,我们只需证明:$(I), (II)$线性等价即可。

        考虑$(III) : = (I) \cup (II)$的秩。由$\S 2$习题9可知
        \begin{align*}
          n - r \leq rank((III)) \leq 2(n - r)
        \end{align*}
        因为$(II)$中都是解向量,于是都能被$(I)$线性表示,所以
        \begin{align*}
          rank((III)) \leq n - r
        \end{align*}
        综上
        \begin{align*}
          rank((III)) = n - r
        \end{align*}
        由$\S1$习题14可知,$(I),(II)$都是$(III)$的极大线性无关部分组,
        所以$(I), (II)$线性等价。
\end{itemize}

\section*{4}

备注:添加一个方程,对齐次线性方程组解的影响。

反证法,假设$\beta$不可以被$\alpha_1, \alpha_2, \cdots \alpha_{s}$线性表示。

齐次线性方程组对应的矩阵设为
\begin{align*}
  A = \begin{bmatrix}
        \alpha_1 \\
        \alpha_2 \\
        \vdots   \\
        \alpha_s
      \end{bmatrix}
\end{align*}
把方程
\begin{align*}
  b_1 x_1 + b_2 x_2 + \cdots + b_n x_n = 0
\end{align*}
添加到齐次线性方程组,得到新的矩阵:
\begin{align*}
  B = \begin{bmatrix}
        \alpha_1 \\
        \alpha_2 \\
        \vdots   \\
        \alpha_s \\
        \beta
      \end{bmatrix}
\end{align*}

不妨设$rank(A) = r$,那么,我们有
\begin{align*}
  rank(B) = r + 1
\end{align*}
设$A,B$的基础解系分别为$(I),(II)$,由定理3.1可知,基础解系的秩为
\begin{align*}
  rank((I)) = n - r \\
  rank((II)) = n - r - 1
\end{align*}
由题设中对解的描述可知,$(II)$可以线性表示$(I)$,且$(I)$是线性无关的,
于是,利用替换定理可知
\begin{align*}
  n - r - 1 \geq n - r
\end{align*}
存在矛盾,假设不成立,命题得证。

\section*{5}

设两个齐次线性方程组对应的矩阵分别为为
\begin{align*}
  A \\
  B
\end{align*}
把两个齐次线性方程组合并,我们得到新的矩阵
\begin{align*}
  C = \begin{bmatrix}
        A \\
        B
      \end{bmatrix}
\end{align*}
由$\S 2$习题9(更准确的说是行向量版本)可知
\begin{align*}
  max(rank(A), rank(B)) \leq rank(C) \leq rank(A) + rank(B)
\end{align*}
因为$rank(A), rank(B)$都小于$n/2$,我们有
\begin{align*}
  0 \leq rank(C) < n
\end{align*}

如果$rank(C)$等于$0$,说明矩阵$A, B$都是零矩阵,此时任意$x_1, x_2, \cdots, x_n$
都是$A, B$对应的齐次线性方程组的解,命题成立。

$rank(C) < n$时,由定理3.1可知,$C$存在一个秩为$n - rank(C) > 0$的基础解系$(I)$,
因为$(I)$是非空的,$(I)$中的任意向量都是$A, B$的解向量,命题得证。

综上,命题得证。

\section*{6}

\begin{itemize}
  \item 方法一

        对系数$n \times n$矩阵进行初等变换
        \begin{align*}
          \begin{bmatrix}
            0      & 1      & 1      & 1      & \cdots & 1      & 1      \\
            1      & 0      & 1      & 1      & \cdots & 1      & 1      \\
            1      & 1      & 0      & 1      & \cdots & 1      & 1      \\
            1      & 1      & 1      & 0      & \cdots & 1      & 1      \\
            \vdots & \vdots & \vdots & \vdots & \vdots & \vdots & \vdots \\
            1      & 1      & 1      & 1      & \cdots & 1      & 0      \\
          \end{bmatrix}
          \xrightarrow{\text{交换前两列}}
          \begin{bmatrix}
            1      & 0      & 1      & 1      & \cdots & 1      & 1      \\
            0      & 1      & 1      & 1      & \cdots & 1      & 1      \\
            1      & 1      & 0      & 1      & \cdots & 1      & 1      \\
            1      & 1      & 1      & 0      & \cdots & 1      & 1      \\
            \vdots & \vdots & \vdots & \vdots & \vdots & \vdots & \vdots \\
            1      & 1      & 1      & 1      & \cdots & 1      & 0      \\
          \end{bmatrix} \\
          \xrightarrow{\text{先利用$R_1$进行行变换,再利用$C_1$进行列变换}}
          \begin{bmatrix}
            1      & 0      & 0      & 0      & \cdots & 0      & 0      \\
            0      & 1      & 0      & 0      & \cdots & 0      & 0      \\
            0      & 1      & -1     & 0      & \cdots & 0      & 0      \\
            0      & 1      & 0      & -1     & \cdots & 0      & 0      \\
            \vdots & \vdots & \vdots & \vdots & \vdots & \vdots & \vdots \\
            0      & 1      & 0      & 0      & \cdots & 0      & -1     \\
          \end{bmatrix}
        \end{align*}
        有无非零解,就是判断$A$的秩的情况。
        此时,主对角线上都是非零的,且矩阵的右上角都是$0$,观察行向量可得都是线性无关的,
        所以矩阵$rank(A) = n$,于是可得$A$没有非零解。

  \item 方法二

        使用2-3-comment.tex中的结论,直接可得秩为$n$。
\end{itemize}


\section*{7}

不妨设齐次线性方程组$A$的秩为$r$,那么设它的任意基础解系为
\begin{align*}
  \alpha_1, \alpha_2, \cdots, \alpha_{n - r} \ \ \ (I)
\end{align*}

设满足题设要求的线性无关解向量组为
\begin{align*}
  \eta_1, \eta_2, \cdots, \eta_s \ \ \ (II)
\end{align*}

如果$s = n - r$,由习题3可知,$(II)$就是基础解系。

如果$s < n - r$,于是由基础解系的定义可知$(I)$可以线性表示$(II)$,
又因为$(II)$是线性无关的,由替换定理的结论(2)(3)可得,
可以用$(II)$替换掉$(I)$中的部分向量,得到一个与$(I)$线性等价且线性无关的基础解系。

综上,命题得证。

\section*{8}

略

\section*{9}

如果是齐次线性方程组,由解的性质可知,命题是显然成立的。

我们主要考虑非齐次线性方程组,不妨设为$A$。

我们有
\begin{align*}
   & \eta_1 = \eta_1 + (\eta_1 - \eta_1) \\
   & \eta_2 = \eta_1 + (\eta_2 - \eta_1) \\
   & \eta_3 = \eta_1 + (\eta_3 - \eta_1) \\
   & \vdots                              \\
   & \eta_t = \eta_1 + (\eta_t - \eta_1)
\end{align*}
于是
\begin{align*}
  k_1 \eta_1 + k_2 \eta_2 + \cdots + k_t \eta_t
   & = (k_1 + k_2 + \cdots + k_t) \eta_1 + [k_1(\eta_1 - \eta_1) + k_2(\eta_2 - \eta_1) + \cdots + k_t(\eta_t - \eta_1)] \\
   & = \eta_1 + [k_1(\eta_1 - \eta_1) + k_2(\eta_2 - \eta_1) + \cdots + k_t(\eta_t - \eta_1)]
\end{align*}

因为
\begin{align*}
  [k_1(\eta_1 - \eta_1) + k_2(\eta_2 - \eta_1) + \cdots + k_t(\eta_t - \eta_1)]
\end{align*}
是$A$导出方程组的解。
所以
\begin{align*}
  \eta_1 + [k_1(\eta_1 - \eta_1) + k_2(\eta_2 - \eta_1) + \cdots + k_t(\eta_t - \eta_1)]
\end{align*}
是$A$的解。

\section*{10}

设n个平面合并成一个方程组$A$(注意,不一定是齐次的)。

通过一个点,即方程组只有一个解,满足以下条件即可
\begin{align*}
  rank(A) = n = rank(\overline{A})
\end{align*}
其中$\overline{A}$是增广矩阵。

通过同一直线,即导出方程组的基础解系的秩为$1$,满足以下条件即可
\begin{align*}
  rank(A) = n - 1 = rank(\overline{A})
\end{align*}

\section*{11}

不妨设$rank(A) = r$,可设
\begin{align*}
  \alpha_{i_1}, \alpha_{i_2}, \cdots, \alpha_{i_r} \ \ \ (I)
\end{align*}
是$A$列向量组的极大无关部分组。

$(I)$中每个向量加上一个分量$b_j (1 \leq j \leq n)$,得到新的向量组
\begin{align*}
  \beta_{i_1}, \beta_{i_2}, \cdots, \beta_{i_r} \ \ \ (II)
\end{align*}
依然是线性无关的。

由题设$rank(A) = rank(B)$可知,$(II)$是$B$列向量的极大线性无关部分组,
于是$(II)$可以线性表示最后一列,从而线性方程组有解。

\section*{12}

设该方程组的系数矩阵为$A$,增广矩阵为$\overline{A}$,
\begin{align*}
  \beta = (b_1, b_2, \cdots, b_n)
\end{align*}

先对$A$进行行变换,把$n$行放到第$1$行,第$n-1$行放到第二行,以此类推,
于是由2-3-comment.tex中的结论可知:\\
$a = b = 0$时$rank(A) = 0$,
由定理3.2(判别定理)可知,只有当$rank(A) = 0 = rank(\overline{A})$
时,方程组才有解。即$\beta$是零向量才有解。且$K^n$中的任意向量都是它的解。

$a = b = 0$时$rank(A) = 1$,
由定理3.2(判别定理)可知,只有当$rank(A) = 1 = rank(\overline{A})$
时,所有的方程都是一样的,
此时只有当$\beta$中所有分量都相同时,方程组才有解,基础解系的秩为$n - 1$。

$(n - 1)a + b \neq 0$,此时$rank(A) = n$,因为增广矩阵的行只有$n$行,
可得$rank(\overline{A}) = n$,此时,有唯一解且对$\beta$没有要求。

$(n - 1)a + b = 0$,对增广矩阵做行变换:所有的行都加入第一行,得到
\begin{align*}
  \begin{bmatrix}
    0      & 0      & \cdots & 0,     & \sum\limits_{i}^n b_i \\
    a      & b      & \cdots & a,     & b_{n - 1}             \\
    \vdots & \vdots & \ddots & \vdots & \vdots                \\
    a      & a      & \cdots & b      & b_1
  \end{bmatrix}
\end{align*}
已知除去第n行,矩阵的秩为$n - 1$,要想线性方程有解,那么要保证其秩为$n - 1$,
于是需要保证$\sum\limits_{i}^n b_i = 0$。

综上,
\begin{equation*}
  \begin{cases}
    \text{任意解}           & a = b = 0, b_1 = b_2 =\cdots = b_n = 0 \\
    \text{基础解系秩为$n - 1$} & a = b \neq 0, b_1 = b_2 =\cdots = b_n  \\
    \text{基础解系秩为$1$}     & a = b = 0, \sum\limits_{i}^n b_i = 0   \\
    \text{唯一解}           & (n - 1)a + b \neq 0
  \end{cases}
\end{equation*}

\section*{13}
由定理3.3可知,任一解$\gamma$可以表示成
\begin{align*}
  \gamma & = \gamma_0 + k_1\eta_1 + k_2\eta_2 + \cdots + k_s\eta_s                                              \\
         & = \gamma_0 + k_1(\gamma_1 - \gamma_0) + k_2(\gamma_2 - \gamma_0) + \cdots + k_s(\gamma_s - \gamma_0) \\
         & = [1 - (k_1 + k_2 + \cdots + k_s)] \gamma_0 + k_1 \gamma_1 + k_2 \gamma_2 + \cdots + k_s \gamma_s    \\
\end{align*}
令
\begin{align*}
  k_0 & = 1 - (k_1 + k_2 + \cdots + k_s) \\
  \implies                               \\
  1   & = k_0 + k_1 + k_2 + \cdots + k_s
\end{align*}
于是$\gamma$可以表示成
\begin{align*}
  \gamma & = k_0 \gamma_0 + k_1 \gamma_1 + k_2 \gamma_2 + \cdots + k_s \gamma_s
\end{align*}


\section*{14}

不妨设任意非齐次线性方程组的系数矩阵和增广矩阵为$A, \overline{A}$,
因为系数矩阵与增广矩阵的秩相等,由定义3.2(判别定理)可知
方程组有解。

如果$rank(A) = n$,由定理3.3可知,方程组有唯一解。
因为$n - n + 1 = 1$,命题成立。

如果$rank(A) < n$,由定理3.3可知,方程组的解可有某一特殊解$\gamma_0$和它的
导出方程组的一个基础解系表示。
不妨设导出方程组的基础解系为
\begin{align*}
  \eta_1, \eta_2, \cdots, \eta_{n - r}
\end{align*}
于是方程组的解可以表示成
\begin{align*}
  \gamma_0 + k_1 \eta_1 + k_2 \eta_2 + \cdots + k_{n - r} \eta_{n - r}
\end{align*}
我们令
\begin{align*}
  \gamma_0 = \gamma_0          \\
  \gamma_1 = \gamma_0 + \eta_1 \\
  \gamma_2 = \gamma_0 + \eta_2 \\
  \vdots                       \\
  \gamma_{n - r} = \gamma_0 + \eta_{n - r}
\end{align*}
这就是满足题设要求的线性无关解向量组,设为$(I)$。

我们还需证明该向量组是线性无关的。
考虑向量组
\begin{align*}
  \gamma_0, \eta_1, \eta_2, \cdots, \eta_{n - r} \ \ \ (II)
\end{align*}
$(II)$是线性无关的,
因为$\eta_1, \eta_2, \cdots, \eta_{n - r}$是线性无关,
且$\gamma_0$是无法被$\eta_1, \eta_2, \cdots, \eta_{n - r}$
线性表示,否则$\gamma_0$将是导出方程组的解。

易得$(I), (II)$是线性等价的,进而$(I)$是线性无关的。

\section*{15}

\begin{itemize}
  \item (1)

        三点不共线,则向量
        \begin{align*}
          \overrightarrow{AB} = (b_1 - a_1, b_2 - a_2) \\
          \overrightarrow{AC} = (c_1 - a_1, c_2 - a_2)
        \end{align*}
        线性无关。

        对矩阵做初等行列变换
        \begin{align*}
          \begin{bmatrix}
            a_1 & a_2 & 1 \\
            b_1 & b_2 & 1 \\
            c_1 & c_2 & 1
          \end{bmatrix}
          \xrightarrow{\text{行变换}}
          \begin{bmatrix}
            a_1       & a_2       & 1 \\
            b_1 - a_1 & b_2 - a_2 & 0 \\
            c_1 - a_1 & c_2 - a_2 & 0
          \end{bmatrix}
          \xrightarrow{\text{列变换}}
          \begin{bmatrix}
            0         & 0         & 1 \\
            b_1 - a_1 & b_2 - a_2 & 0 \\
            c_1 - a_1 & c_2 - a_2 & 0
          \end{bmatrix} \\
          \xrightarrow{\text{列变换}}
          \begin{bmatrix}
            1 & 0         & 0         \\
            0 & b_1 - a_1 & b_2 - a_2 \\
            0 & c_1 - a_1 & c_2 - a_2
          \end{bmatrix}
        \end{align*}

        充分性: \\
        即$rank(A) = 3$,由于初等变换不改变秩,于是可得
        \begin{align*}
          (0, b_1 - a_1, b_2 - a_2) \\
          (0, c_1 - a_1, c_2 - a_2) \\
        \end{align*}
        线性无关,进而可得
        \begin{align*}
          (b_1 - a_1, b_2 - a_2) \\
          (c_1 - a_1, c_2 - a_2) \\
        \end{align*}
        线性无关,所以三点不共线。

        必要性: \\
        由三点不共线,可知
        \begin{align*}
          (b_1 - a_1, b_2 - a_2) \\
          (c_1 - a_1, c_2 - a_2)
        \end{align*}
        线性无关,从而
        \begin{align*}
          (0, b_1 - a_1, b_2 - a_2) \\
          (0, c_1 - a_1, c_2 - a_2)
        \end{align*}
        线性无关。

        显然,向量
        \begin{align*}
          (1, 0, 0)
        \end{align*}
        不能被
        \begin{align*}
          (0, b_1 - a_1, b_2 - a_2) \\
          (0, c_1 - a_1, c_2 - a_2)
        \end{align*}
        于是可得,秩为$3$。

  \item (2)

        三点不共线,唯一确定一个圆。这是几何结论。

        两条中轴线(以AB,BC为例)相交的点,就是圆心。
        \begin{equation*}
          \begin{cases*}
            2(b_1 - a_1)x + 2(b_2 - a_2)y + (a_1^2 + a_2^2 - b_1^2 - b_2^2) = 0 \\
            2(c_1 - b_1)x + 2(c_2 - b_2)y + (b_1^2 + b_2^2 - c_1^2 - c_2^2) = 0
          \end{cases*}
          \ \ \ (1)
        \end{equation*}
        以上方程一定有唯一的实数解,设为$(x_0, y_0)$。

        接下来,我们需要证明$(x_0, y_0)$是有理数坐标。

        对于方程组$(1)$的系数矩阵,
        因为元素都是有理数,
        我们可以利用初等变换(有理数域内的初等变换)化作
        标准型,由2-3-comment.tex注释2中的讨论可知,
        有理数域扩大到实数域,秩是不会改变的,
        由实数域有唯一解可知,秩为$2$,
        所以在有理数域也有唯一解,
        因为有理数域是实数域的子集,
        从而可知,这个唯一解是有理数解,否则实数域中就有两个解了,存在矛盾。

        综上,$(x_0, y_0)$是有理数坐标。
\end{itemize}

\section*{16}

因为是实数域上的齐次线性方程组,
由2-3-comment.tex注释2中的讨论可知,
秩的情况在实数域还是复数域上是相同的,
题设中齐次线性方程组有非零解,即
实数域中有非零解。
不妨设
\begin{align*}
  (c_1, c_2, \cdots, c_n)
\end{align*}
是一个实数域中的非零解向量。

于是我们有以下方程组

\begin{equation*}
  \begin{cases*}
    \lambda c_1 + a_{12} c_2 + \cdots + a_{1n} c_n = 0 \\
    a_{21} c_1 + \lambda c_2 + \cdots + a_{2n} c_n = 0 \\
    \cdots                                             \\
    a_{n1}c_1 + a_{n2}c_2 + \cdots + \lambda c_n = 0
  \end{cases*}
\end{equation*}
对$i$个方程,乘以$c_i$,于是,我们有
\begin{equation*}
  \begin{cases*}
    \lambda c_1^2 + a_{12} c_1c_2 + \cdots + a_{1n} c_1c_n = 0 \\
    a_{21} c_1c_2 + \lambda c_2^2 + \cdots + a_{2n} c_2c_n = 0 \\
    \cdots                                                     \\
    a_{n1}c_1c_n + a_{n2}c_2c_n + \cdots + \lambda c_n^2 = 0
  \end{cases*}
\end{equation*}
所有方程相加,由于$a_{ij} = - a_{ji}$,我们有
\begin{align*}
  \lambda c_1^2 + \lambda c_2^2 + \cdots + \lambda c_n^2 & = 0 \\
  \lambda (c_1^2 + c_2^2 + \cdots + c_n^2)               & = 0 \\
  \lambda                                                & = 0
\end{align*}
\end{document}
