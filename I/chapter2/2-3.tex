\documentclass{article}
\usepackage{mathtools} 
\usepackage{fontspec}
\usepackage[UTF8]{ctex}
\usepackage{amsthm}
\usepackage{mdframed}
\usepackage{xcolor}
\usepackage{amssymb}
\usepackage{amsmath}
\usepackage{hyperref}


% 定义新的带灰色背景的说明环境 zremark
\newmdtheoremenv[
  backgroundcolor=gray!10,
  % 边框与背景一致,边框线会消失
  linecolor=gray!10
]{zremark}{注释}

% 通用矩阵命令: \flexmatrix{矩阵名}{元素符号}{行数}{列数}
\newcommand{\flexmatrix}[4]{
  \[
  #1 = \begin{pmatrix}
    #2_{11}     & #2_{12}     & \cdots & #2_{1#4}   \\
    #2_{21}     & #2_{22}     & \cdots & #2_{2#4}   \\
    \vdots      & \vdots      & \ddots & \vdots     \\
    #2_{#31}    & #2_{#32}    & \cdots & #2_{#3#4}
  \end{pmatrix}
  \]
}

% 简化版命令(默认矩阵名为A,元素符号为a): \quickmatrix{行数}{列数}
\newcommand{\quickmatrix}[2]{\flexmatrix{A}{a}{#1}{#2}}


\begin{document}
\title{2.3}
\author{张志聪}
\maketitle

\section*{1}

只做第一题,练练手。

先做矩阵消元法
\begin{align*}
  \begin{bmatrix}
    1 & 1 & 1 & 1 & 1  \\
    3 & 2 & 1 & 1 & -3 \\
    0 & 1 & 2 & 2 & 6  \\
    5 & 4 & 3 & 3 & -1
  \end{bmatrix}
  \xrightarrow{\text{处理第一列}}
  \begin{bmatrix}
    1 & 1  & 1  & 1  & 1  \\
    0 & -1 & -2 & -2 & -6 \\
    0 & 1  & 2  & 2  & 6  \\
    0 & -1 & -2 & -2 & -6
  \end{bmatrix}
  \xrightarrow{\text{处理第二列}}
  \begin{bmatrix}
    1 & 1  & 1  & 1  & 1  \\
    0 & -1 & -2 & -2 & -6 \\
    0 & 0  & 0  & 0  & 0  \\
    0 & 0  & 0  & 0  & 0
  \end{bmatrix}
\end{align*}
$rank(A) = 2$,故基础解系中应包含$n - r = 5 - 2 = 3$个向量。
写出阶梯型矩阵的对应方程组
\begin{equation*}
  \begin{cases*}
    x_1 + x_2 + x_3 + x_4 + x_5 = 0 \\
    -x_2 - 2x_3 -2 x_4 -6 x_5 = 0
  \end{cases*}
\end{equation*}
移项,得
\begin{equation*}
  \begin{cases*}
    x_1 + x_2 = - x_3 - x_4 - x_5 \\
    -x_2 = 2x_3 + 2 x_4 + 6 x_5
  \end{cases*}
\end{equation*}
$x_3,x_4,x_5$为自由未知量。
\begin{itemize}
  \item (i) 取$x_3 = 1, x_4 = 0, x_5 = 0$,得一个解向量
        \begin{align*}
          \eta_1 = (1, -2, 1, 0, 0)
        \end{align*}
  \item (ii)取$x_3 = 0, x_4 = 1, x_5 = 0$,得一个解向量
        \begin{align*}
          \eta_2 = (1, -2, 0, 1, 0)
        \end{align*}
  \item (iii)取$x_3 = 0, x_4 = 0, x_5 = 1$,得一个解向量
        \begin{align*}
          \eta_3 = (5, -6, 0, 0, 1)
        \end{align*}
\end{itemize}
于是$\eta_1, \eta_2, \eta_3$为方程组的一个基础解系。方程组的全部解可表为
\begin{align*}
  k_1 \eta_1 + k_2 \eta_2 + k_3 \eta_3
\end{align*}
其中$k_1, k_2, k_3$为数域$K$内任意数。

\section*{2}

设$(I)$是基础解系,$(II)$是与$(I)$线性等价的任意向量组。

按照基础解系的定理,我们要验证三点:
\begin{itemize}
  \item (1)$(II)$中的向量都是解向量。

        任意$\beta \in (II)$都可以被$(I)$线性表示,
        因为$(I)$中都是解向量,他们的线性表示$\beta$,也是解向量。

  \item (2)$(II)$线性无关。

        题设保证的。

  \item (3)解向量都可以被$(II)$线性表示。

        因为任意解向量都可以被$(I)$线性表示,
        $(I)$和$(II)$线性等价,
        于是也能被$(II)$线性表示。
\end{itemize}

\section*{3}

不妨设方程组的基础解系为
\begin{align*}
  \alpha_1, \alpha_2, \cdots \alpha_{n - r} \ \ \ (I)
\end{align*}

满足题设条件的向量组为
\begin{align*}
  \beta_1, \beta_2, \cdots \beta_{n - r} \ \ \ (II)
\end{align*}

\begin{itemize}
  \item 方法一

        利用$\S 1$命题1.4(替换定理)

        假设$(II)$不是基础解系,那么存在解向量$\beta$无法被$(II)$线性表示,
        由命题3.2的逆否命题可知,向量组
        \begin{align*}
         (III) : = (II), \beta
        \end{align*}
        是线性无关的,于是秩为$n - r + 1$。

        因为$(I)$是基础解系,于是$(I)$可以线性表示所有解,于是可以线性表示
        $(III)$,又因为$(III)$线性无关,由$\S 1$命题1.4的逆否命题可知
        \begin{align*}
          n - r \geq n - r + 1
        \end{align*}
        存在矛盾,假设不成立,命题得证。

  \item 方法二

        利用习题2,我们只需证明:$(I), (II)$线性等价即可。

        考虑$(III) : = (I) \cup (II)$的秩。由$\S 2$习题9可知
        \begin{align*}
          n - r \leq rank((III)) \leq 2(n - r)
        \end{align*}
        因为$(II)$中都是解向量,于是都能被$(I)$线性表示,所以
        \begin{align*}
          rank((III)) \leq n - r
        \end{align*}
        综上
        \begin{align*}
          rank((III)) = n - r
        \end{align*}
        由$\S1$习题14可知,$(I),(II)$都是$(III)$的极大线性无关部分组,
        所以$(I), (II)$线性等价。
\end{itemize}

\section*{4}



\end{document}
