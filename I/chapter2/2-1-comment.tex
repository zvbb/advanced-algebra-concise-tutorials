\documentclass{article}
\usepackage{mathtools} 
\usepackage{fontspec}
\usepackage[UTF8]{ctex}
\usepackage{amsthm}
\usepackage{mdframed}
\usepackage{xcolor}
\usepackage{amssymb}
\usepackage{amsmath}
\usepackage{hyperref}


% 定义新的带灰色背景的说明环境 zremark
\newmdtheoremenv[
  backgroundcolor=gray!10,
  % 边框与背景一致,边框线会消失
  linecolor=gray!10
]{zremark}{注释}

% 通用矩阵命令: \flexmatrix{矩阵名}{元素符号}{行数}{列数}
\newcommand{\flexmatrix}[4]{
  \[
  #1 = \begin{pmatrix}
    #2_{11}     & #2_{12}     & \cdots & #2_{1#4}   \\
    #2_{21}     & #2_{22}     & \cdots & #2_{2#4}   \\
    \vdots      & \vdots      & \ddots & \vdots     \\
    #2_{#31}    & #2_{#32}    & \cdots & #2_{#3#4}
  \end{pmatrix}
  \]
}

% 简化版命令(默认矩阵名为A,元素符号为a): \quickmatrix{行数}{列数}
\newcommand{\quickmatrix}[2]{\flexmatrix{A}{a}{#1}{#2}}


\begin{document}
\title{2.1 注释}
\author{张志聪}
\maketitle

\begin{zremark}
  命题1.4的增强形式:替换定理 \\
  给定$K^m$内两个向量组
  \begin{align*}
    \alpha_1, \alpha_2, \cdots, \alpha_r & \ \ \ (I)  \\
    \beta_1, \beta_2, \cdots, \beta_s    & \ \ \ (II)
  \end{align*}
  如果向量组$(I)$中每一个向量都能被$(II)$线性表示,且向量组$(I)$线性无关,
  则有如下结论:
  \begin{itemize}
    \item (1) $r \leq s$。
    \item (2) 向量组$(I)$中的所有向量经过适当替换$(II)$中的向量,
          可以得到与$(II)$线性等价的向量组$(II^\prime)$。
    \item (3) $(II)$线性无关,替换后的向量组$(II^\prime)$也线性无关。
  \end{itemize}
\end{zremark}

\textbf{证明:}

\begin{itemize}
  \item (1)

        这是命题1.4的逆否命题。
  \item (2)

        对$r$进行归纳。

        $\circ$ 归纳基始$r = 1$,
        首先,由$(I)$是线性无关的可知,$\alpha_1 \neq 0$。
        又因为$(I)$中每一个向量都能被$(II)$线性表示,所以存在
        $k_1, k_2, \cdots, k_s$使得
        \begin{align*}
          \alpha_1 = k_1 \beta_1 + k_2 \beta_2 + \cdots + k_s \beta_s
        \end{align*}
        因为$\alpha_1 \neq 0$,可以确定至少存在一个$k_i \neq 0 (1 \leq i \leq s)$(我们取其中的一个即可)。

        我们用$\alpha_1$替换$\beta_i$,于是我们得到新的向量组:
        \begin{align*}
          \beta_1, \beta_2, \cdots, \beta_{i-1}, \alpha_1, \beta_{i+1}, \cdots, \beta_s \ \ \ (II^\prime)
        \end{align*}
        接下来,我们需要证明$(II)$与$(II^\prime)$线性等价。

        $(II^\prime)$可以被$(II)$线性表示是显然的,因为$(II^\prime)$中的向量除了$\alpha_1$外,$(II)$中都存在,
        且$\alpha_1$也是可以被$(II)$线性表示。

        $(II)$和$(II^\prime)$相差一个向量$\alpha_1$,只需关注$\beta_i$是否能被$(II^\prime)$线性表示即可。
        我们有
        \begin{align*}
          \alpha_1 = k_1 \beta_1 + k_2 \beta_2 + \cdots + k_i \beta_i + \cdots + k_s \beta_s                                             \\
          k_i \beta_i = \alpha_1 - k_1 \beta_1 - k_2 \beta_2 - \cdots - k_{i-1} \beta_{i-1} - k_{i+1} \beta_{i+1} - \cdots - k_s \beta_s \\
          \beta_i = \frac{1}{k_i}\alpha_1 - \frac{k_1}{k_i} \beta_1 - \frac{k_2}{k_i} \beta_2 - \cdots - \frac{k_{i-1}}{k_i} \beta_{i-1} - \frac{k_{i+1}}{k_i} \beta_{i+1} - \cdots - \frac{k_s}{k_i} \beta_s
        \end{align*}
        综上,$(II)$与$(II^\prime)$线性等价。

        $\circ$ 归纳假设$r = k - 1$时,命题成立。

        $\circ$ $r = k$时,$(I)$中去除$\alpha_k$的到的$(I^\prime)$也是线性无关的(可以直接用反证法证明)。
        于是利用归纳假设可知,$(I^\prime)$替换$(II)$中的向量得到:
        \begin{align*}
          \alpha_1, \alpha_2, \cdots, \alpha_{k - 1}, \beta_{i1}, \beta_{i2}, \cdots, \beta_{i(s - r + 1)} \ \ \ (II^\prime)
        \end{align*}
        与$(II)$是线性等价的。

        接下来,我们只需证明,把$\alpha_k$替换$\beta_{i1}, \beta_{i2}, \cdots, \beta_{i(s - r + 1)}$中的某一个向量后,
        $(II^\prime)$依然和$(II)$线性等价即可。

        由题设$(I)$可以被$(II)$线性表示可知,因为$(II)$和$II^\prime$是线性等价的,
        于是我们有
        \begin{align*}
          \alpha_k = a_1 \alpha_1 + a_2 \alpha_2 + \cdots + a_{k - 1} \alpha_{k - 1} + a_k \beta_{i1} + \cdots + a_s \beta_{i(s - r + 1)}
        \end{align*}
        我们可以断定$a_k, \cdots, a_s$中必有非零的数
        (可以通过反证法,如果都等于零,那么$\alpha_k$就可以被$(I)$中的其他向量线性表示了,与题设矛盾)。

        设$a_{ij} \neq 0 (k \leq ij \leq s)$,那么,用$\alpha_k$替换掉对应的$\beta_{ij}$,我们得到新的向量组
        \begin{align*}
          \alpha_1, \alpha_2, \cdots, \alpha_{k}, \beta_{i1}, \beta_{i2}, \cdots,
          \beta_{i(j - 1)}, \beta_{i(j + 1)}, \cdots, \beta_{i(s - r + 1)} \ \ \ (III)
        \end{align*}

        易得$(III)$与$(II^\prime)$线性等价(讨论和$r = 1$类似,这里不做赘述。),进而$(II)$与$(III)$线性等价。

        归纳完成,命题成立。

  \item (3)

        对$r$进行归纳。

        $\circ$ 归纳基始$r = 1$,可知$\alpha_1 \neq 0$,
        不妨设替换后的$(II)$为:
        \begin{align*}
          \alpha_1, \beta_2, \beta_3, \cdots, \beta_s \ \ \ (II^\prime)
        \end{align*}

        我们需要证明$(II^\prime)$是线性无关的。反证法,假设$(II^\prime)$是线性有关的,
        那么
        \begin{align*}
          k_1 \alpha_1 + k_2 \beta_2 + \cdots + k_s \beta_s = 0
        \end{align*}
        存在非零解。

        显然,$k_1 \neq 0$,否则
        \begin{align*}
          k_2 \beta_2 + \cdots + k_s \beta_s = 0
        \end{align*}
        这与题设$(II)$线性无关矛盾。

        由$k_1 \neq 0$,有$\alpha_1$可以被$\beta_2, \beta_3, \cdots, \beta_s$
        线性表示为
        \begin{align*}
          \alpha_1 = - \frac{1}{k_1} (k_2 \beta_2 + \cdots + k_s \beta_s)
        \end{align*}

        因为$(II)$与$(II^\prime)$线性等价可知,
        $\beta_1$可以被$(II^\prime)$线性表示:
        \begin{align*}
          \beta_1 & = c_1 \alpha_1 + c_2 \beta_2 + \cdots + c_s \beta_s                                         \\
                  & = - \frac{1}{k_1} (k_2 \beta_2 + \cdots + k_s \beta_s) + c_2 \beta_2 + \cdots + c_s \beta_s
        \end{align*}
        于是我们有$(II)$是线性有关的,与题设矛盾,
        所以假设不成立。

        $\circ$ 归纳假设$r = k - 1$时,命题成立。

        $\circ$ $r = k$时。因为$\alpha_1, \alpha_2, \cdots, \alpha_{k - 1}$是线性无关的,
        所以由归纳假设可知,向量组
        \begin{align*}
          \alpha_1, \alpha_2, \cdots, \alpha_{k - 1}, \beta_{i1}, \beta_{i2}, \cdots, \beta_{i(s - r + 1)} \ \ \ (II^\prime)
        \end{align*}
        是线性无关的。

        按照之前的方法用$\alpha_k$替换掉对应的$\beta_{ij} \ (k \leq ij \leq s)$,
        得到与$(II^\prime)$线性等价的新向量组$(III)$,需证明$(III)$也是线性无关的。

        证明方式与$r = 1$时相同。反证法,假设$(III)$是线性相关的,那么
        \begin{align*}
          c_1 \alpha_1 + c_2 \alpha_2 + \cdots + c_k \alpha_k
          + c_{k + 1} \beta_{i1} + \cdots + c_{k + j - 1} \beta_{i(j-1)}
          + c_{k + j + 1} \beta_{i(j+1)} + \cdots
          + c_s \beta_{i(s - r + 1)} = 0
        \end{align*}
        存在非零解。
        $c_k \neq 0$,否者会与$(II^\prime)$线性无关矛盾。
        于是,我们有$\alpha_k$可以被$(III)$中的其他向量线性表示。

        因为$(II^\prime)$与$(III)$线性等价,所以存在
        \begin{align*}
          \beta_{ij} = d_1 \alpha_1 + d_2 \alpha_2 + \cdots + d_k \alpha_k
          + d_{k + 1} \beta_{i1} + \cdots + d_{k + j - 1} \beta_{i(j-1)}
          + d_{k + j + 1} \beta_{i(j+1)} + \cdots
          + d_s \beta_{i(s - r + 1)}
        \end{align*}
        替换掉$\alpha_k$,就会得到$\beta_{ij}$可以被$(II^\prime)$中的其他向量线性表示,
        这与$(II^\prime)$线性无关矛盾。
        综上,假设不成立,$(III)$是线性无关的。

        归纳完成,命题成立。
\end{itemize}

\begin{zremark}
  筛选法中涉及的命题:\\

  已知,向量组
  \begin{align*}
    \alpha_1, \alpha_2, \cdots, \alpha_n \ \ \ (I)
  \end{align*}
  是线性无关的,如果$\alpha_{n + 1}$无法被向量组线性表示,
  把$\alpha_{n + 1}$加入向量组,则新向量组$(II)$仍然是线性无关的。 
\end{zremark}

\textbf{证明:}

反证法,假设新的向量组线性相关,于是方程组
\begin{align*}
  k_1 \alpha_1 + k_2 \alpha_2 + \cdots + k_n \alpha_n + k_{n + 1} \alpha_{n + 1} = 0
\end{align*}
存在非零解。

这里$k_{n + 1} = 0$,否则会与$\alpha_{n + 1}$不能被$(I)$线性表示矛盾。

于是,我们有
\begin{align*}
  k_1 \alpha_1 + k_2 \alpha_2 + \cdots + k_n \alpha_n = 0
\end{align*}
存在非零解,这与题设$(I)$线性无关矛盾。

综上,假设不成立,命题得证。

\begin{zremark}
  关于集合$S$上的二元“关系”(这里用$\thicksim$表示)的定义:\\
  对任意$a, b \in S$,要么$a, b$存在这种“关系”($a \thicksim b$),
  要么不存在这种“关系”($a \not\thicksim b$),两者必选其一。
\end{zremark}

\begin{zremark}
  一个重要关系。

  设,向量组为
  \begin{align*}
    \alpha_1, \alpha_2, \cdots, \alpha_2 \ \ \ (I)
  \end{align*}
  $(I)$的秩是$r$,
  存在一个部分组为
  \begin{align*}
    \alpha_{i1}, \alpha_{i2}, \cdots, \alpha_{ik} \ \ \ (II)
  \end{align*}
  那么,以下任意两个结论可以推出另一个结论:
  \begin{itemize}
    \item (1) $(II)$线性无关;
    \item (2) $(II)$可以线性表示$(I)$;
    \item (3) $k = r$。
  \end{itemize}
\end{zremark}

\textbf{证明:}

习题14,15

\end{document}