\documentclass{article}
\usepackage{mathtools} 
\usepackage{fontspec}
\usepackage[UTF8]{ctex}
\usepackage{amsthm}
\usepackage{mdframed}
\usepackage{xcolor}
\usepackage{amssymb}
\usepackage{amsmath}
\usepackage{hyperref}


% 定义新的带灰色背景的说明环境 zremark
\newmdtheoremenv[
  backgroundcolor=gray!10,
  % 边框与背景一致,边框线会消失
  linecolor=gray!10
]{zremark}{注释}

% 通用矩阵命令: \flexmatrix{矩阵名}{元素符号}{行数}{列数}
\newcommand{\flexmatrix}[4]{
  \[
  #1 = \begin{pmatrix}
    #2_{11}     & #2_{12}     & \cdots & #2_{1#4}   \\
    #2_{21}     & #2_{22}     & \cdots & #2_{2#4}   \\
    \vdots      & \vdots      & \ddots & \vdots     \\
    #2_{#31}    & #2_{#32}    & \cdots & #2_{#3#4}
  \end{pmatrix}
  \]
}

% 简化版命令(默认矩阵名为A,元素符号为a): \quickmatrix{行数}{列数}
\newcommand{\quickmatrix}[2]{\flexmatrix{A}{a}{#1}{#2}}


\begin{document}
\title{2.5 总结}
\author{张志聪}
\maketitle

\begin{zremark}
  特殊$n$阶方阵的封闭性。
\end{zremark}

\begin{itemize}
  \item (1) 对角矩阵。

        两个对角矩阵$A, B$,又如下封闭性:
        \begin{itemize}
          \item 加法封闭: $A + B$也是对角矩阵;
          \item 数乘封闭: $kA$也是对角矩阵;
          \item 乘法封闭: $AB$也是对角矩阵。
          \item 逆矩阵封闭: $A$可逆,则$A^{-1}$也是对角矩阵(是上三角矩阵的特例)。
        \end{itemize}


  \item (2) (反)对称矩阵。

        两个(反)对称矩阵$A, B$,又如下封闭性:
        \begin{itemize}
          \item 加法封闭: $A + B$也是(反)对称矩阵;
          \item 数乘封闭: $kA$也是(反)对称矩阵;
          \item 乘法有条件的封闭: 充分必要条件“A, B可交换,即$AB = BA$”(习题16)。
          \item 逆矩阵封闭: $A^{-1}$也是(反)对称矩阵(习题18-1)。
        \end{itemize}

  \item (3) 上(下)三角矩阵。

        两个上(下)三角矩阵$A, B$,又如下封闭性:
        \begin{itemize}
          \item 加法封闭: $A + B$也是上(下)三角矩阵;
          \item 数乘封闭: $kA$也是上(下)三角矩阵;
          \item 乘法封闭: $AB$也是上(下)三角矩阵;
          \item 逆矩阵封闭: $A$可逆,则$A^{-1}$也是上(下)三角矩阵(习题18-2)。
        \end{itemize}
\end{itemize}

\begin{zremark}
  任何$n$阶方阵,总能表示成对称矩阵和反称矩阵之和。
\end{zremark}

\textbf{证明:}

设$A$为任意$n$阶方阵,那么
\begin{equation*}
  \begin{cases*}
    A = B + C \\
    A^T = B^T + C^T = B - C
  \end{cases*}
\end{equation*}
解方程的可得
\begin{equation*}
  \begin{cases*}
    B = \frac{1}{2}(A + A^{T}) \\
    C = \frac{1}{2}(A - A^{T})
  \end{cases*}
\end{equation*}
所以
\begin{align*}
  A = \frac{1}{2}(A+A^{T}) + \frac{1}{2}(A-A^{T})
\end{align*}

\begin{zremark}
  $Trace$具有一些优良的性质:习题23
\end{zremark}

\begin{zremark}
  $n$阶矩阵$A$通过初等变换化为$Y$,已知
  $Y$存在逆矩阵$Y^{-1}$,那么$A^{-1}$与
  $Y^{-1}$的关系。
\end{zremark}

设$n$阶初等矩阵$P_1, P_2, \cdots, P_s$;$Q_1, Q_2, \cdots, Q_t$,使
\begin{align*}
  P_s \cdots P_2 P_1 A Q_1 Q_2 \cdots Q_t = Y                                         \\
  Q_t^{-1} \cdots Q_2^{-1} Q_1^{-1} A^{-1} P_1^{-1} P_2^{-1} \cdots P_s^{-1} = Y^{-1} \\
  A^{-1} = Q_1 Q_2 \cdots Q_t Y^{-1} P_s \cdots P_2P_1
\end{align*}

\begin{zremark}
  对任意矩阵$A$,我们有
  \begin{align*}
    rank(A) = rank(A^T A)
  \end{align*}
\end{zremark}

\textbf{证明:}

通过两者的基础解系相同来证明。

如果$Ax = 0$,则$A^T A x = 0$;

如果$A^T A x = 0$,则
\begin{align*}
  0 = x^T A^T A x = (A x)^T (A x) \\
\end{align*}
所以$Ax = 0$(可设$Ax = \alpha$,于是$\alpha^T \alpha = 0$,即$\alpha = 0$)。

由两者的基础解系相同,可知
\begin{align*}
  rank(A) = rank(A^T A)
\end{align*}

\end{document}
