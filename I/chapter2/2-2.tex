\documentclass{article}
\usepackage{mathtools} 
\usepackage{fontspec}
\usepackage[UTF8]{ctex}
\usepackage{amsthm}
\usepackage{mdframed}
\usepackage{xcolor}
\usepackage{amssymb}
\usepackage{amsmath}
\usepackage{hyperref}


% 定义新的带灰色背景的说明环境 zremark
\newmdtheoremenv[
  backgroundcolor=gray!10,
  % 边框与背景一致,边框线会消失
  linecolor=gray!10
]{zremark}{注释}

% 通用矩阵命令: \flexmatrix{矩阵名}{元素符号}{行数}{列数}
\newcommand{\flexmatrix}[4]{
  \[
  #1 = \begin{pmatrix}
    #2_{11}     & #2_{12}     & \cdots & #2_{1#4}   \\
    #2_{21}     & #2_{22}     & \cdots & #2_{2#4}   \\
    \vdots      & \vdots      & \ddots & \vdots     \\
    #2_{#31}    & #2_{#32}    & \cdots & #2_{#3#4}
  \end{pmatrix}
  \]
}

% 简化版命令(默认矩阵名为A,元素符号为a): \quickmatrix{行数}{列数}
\newcommand{\quickmatrix}[2]{\flexmatrix{A}{a}{#1}{#2}}


\begin{document}
\title{2.2}
\author{张志聪}
\maketitle

\section*{1-3}
略

\section*{4}

纠错:$n$阶方程,改为$n \times n$矩阵。

矩阵元素一共有$n^2$个,所以非零元素个数为
\begin{align*}
  n^2 - (n^2 - n + 1) = n - 1
\end{align*}

于是可得,$rank(A) \leq n - 1 < n$,并且$rank(A)$的最大值为$n - 1$。

\section*{5}

\begin{itemize}
  \item (1)

        等价关系,需要满足如下条件:
        \begin{itemize}
          \item (i) 反身性:对任意$m \times n$矩阵$A$,有$A \thicksim A$。

                这显然是成立的。

          \item (ii) 对称性:对任意$m \times n$矩阵$A, B$,若$A \thicksim B$,则$B \thicksim A$。

                由于矩阵的初等行(列)变换都是可逆的,所以命题成立。

          \item (iii) 传递性:对任意$m \times n$矩阵$A, B, C$,若$A \thicksim B$且$B \thicksim C$,则$A \thicksim C$。

                这也是显然的,矩阵$A$先初等变换到矩阵$B$,再从矩阵$B$初等变换到矩阵$C$。
        \end{itemize}
        故是等价关系。

  \item (2)


        等价关系的证明与(1)相同。我们证明后半部。

        因为$rank(A) = m$,可得$n \geq m$(否则与列秩和行秩相等矛盾)。

        在矩阵中可以找到$m$个线性无关的列向量(例2.3已经告诉我们如何找了),
        然后进行初等列变换,把这$m$个列向量移动到矩阵的左侧
        (此时,左侧形成一个$m \times m$线性无关的子矩阵,设为$A^\prime$。)

        依次对主对角线上的元素进行如下处理:

        步骤1:\\
        如果元素$a_{ii} \ (1 \leq i \leq m)$不等于0,则该列乘以$\frac{1}{a_{ii}}$,使该元素等于1。
        如果元素$a_{ii}$等于0,则从该行中选一个不为零的列,移动到$i$列,
        ( 因为左侧的小矩阵$A^\prime$的秩是$m$,所以行向量不可能存在零行)
        然后乘以对应的倒数,使的$a_{ii}$等于1。

        步骤2:\\
        该行的其他元素,整列加上$i$列相应的倍数,化为0。

        步骤3:\\
        对$a{i + 1, i + 1}$进行相同的操作。

        进行$m$次迭代,直到$m$行,此时矩阵$A^\prime$变成主对角线上的元素都是1,
        其他元素都是0。对大于$m$的列,借助矩阵$A^\prime$可以化为全零的列。

  \item (3)

        与(2)类似,不做赘述。

\end{itemize}

\section*{6}

因为$F$中不为零的行向量组(设为$(I)$)是线性无关,
且零向量,也可以被$(I)$向量组表示,
所以$(I)$是$F$行向量组的极大线性无关部分组,
于是$rank(F) = \text{不为零的行数}$,
而矩阵初等行变换是改变矩阵的秩,所以$rank(A) = rank(F)$。

\section*{7}

交换$i$列与$j$列,矩阵变成主对角线元素是1,其他元素为0的矩阵(也叫单位矩阵)。
易得秩为$n$。

\section*{8}

\begin{align*}
  \begin{bmatrix}
    1      & 0      & \cdots & \cdots & \cdots & 0      & 1      \\
    1      & 1      & 0      & \cdots & \cdots & \cdots & 0      \\
    0      & 1      & 1      & \ddots & \ddots & \ddots & \vdots \\
    \vdots & \ddots & \ddots & \ddots & \ddots & \ddots & \vdots \\
    \vdots & \ddots & \ddots & \ddots & \ddots & \ddots & \vdots \\
    \vdots & \ddots & \ddots & \ddots & \ddots & \ddots & 0      \\
    0      & \cdots & \cdots & \cdots & 0      & 1      & 1
  \end{bmatrix}
  \xrightarrow{\text{第n列移到最前面}}
  \begin{bmatrix}
    1 & 1      & 0      & \cdots & \cdots & \cdots & 0      \\
    0 & 1      & 1      & 0      & \cdots & \cdots & \cdots \\
    0 & 0      & 1      & 1      & \ddots & \ddots & \ddots \\
    0 & \vdots & \ddots & \ddots & \ddots & \ddots & \ddots \\
    0 & \vdots & \ddots & \ddots & \ddots & \ddots & 0      \\
    0 & \vdots & \ddots & \ddots & \ddots & \ddots & 1      \\
    1 & 0      & \cdots & \cdots & \cdots & 0      & 1
  \end{bmatrix} \\
  \xrightarrow{R_n - R_1}
  \begin{bmatrix}
    1 & 1      & 0      & \cdots & \cdots & \cdots & 0      \\
    0 & 1      & 1      & 0      & \cdots & \cdots & \cdots \\
    0 & 0      & 1      & 1      & \ddots & \ddots & \ddots \\
    0 & \vdots & \ddots & \ddots & \ddots & \ddots & \ddots \\
    0 & \vdots & \ddots & \ddots & \ddots & \ddots & 0      \\
    0 & \vdots & \ddots & \ddots & \ddots & \ddots & 1      \\
    0 & -1     & \cdots & \cdots & \cdots & 0      & 1
  \end{bmatrix}
  \xrightarrow{R_n + R_2}
  \begin{bmatrix}
    1 & 1      & 0      & \cdots & \cdots & \cdots & 0      \\
    0 & 1      & 1      & 0      & \cdots & \cdots & \cdots \\
    0 & 0      & 1      & 1      & \ddots & \ddots & \ddots \\
    0 & \vdots & \ddots & \ddots & \ddots & \ddots & \ddots \\
    0 & \vdots & \ddots & \ddots & \ddots & \ddots & 0      \\
    0 & \vdots & \ddots & \ddots & \ddots & \ddots & 1      \\
    0 & 0      & 1      & \cdots & \cdots & 0      & 1
  \end{bmatrix} \\
  \xrightarrow{R_n - R_3}
  \begin{bmatrix}
    1 & 1      & 0      & \cdots & \cdots & \cdots & 0      \\
    0 & 1      & 1      & 0      & \cdots & \cdots & \cdots \\
    0 & 0      & 1      & 1      & \ddots & \ddots & \ddots \\
    0 & \vdots & \ddots & \ddots & \ddots & \ddots & \ddots \\
    0 & \vdots & \ddots & \ddots & \ddots & \ddots & 0      \\
    0 & \vdots & \ddots & \ddots & \ddots & \ddots & 1      \\
    0 & 0      & 0      & -1     & \cdots & 0      & 1
  \end{bmatrix} \\
\end{align*}
观察初等行变换对最后一行的影响: \\
第0次:$a_{n0} = 1$; \\
第1次:$a_{n1} = -1$;\\
第2次:$a_{n2} = 1$;\\
因为接下来的操作一致,我们可得在偶数次$j$变化时$a_{nj} = 1$,奇数次$j$变化$a_{nj} = -1$。

于是$n - 1$是偶数时,经过$n - 1$次变化,$a_{n, n - 1} = 1$,即
\begin{align*}
  \begin{bmatrix}
    1 & 1      & 0      & \cdots & \cdots & \cdots & 0      \\
    0 & 1      & 1      & 0      & \cdots & \cdots & \cdots \\
    0 & 0      & 1      & 1      & \ddots & \ddots & \ddots \\
    0 & \vdots & \ddots & \ddots & \ddots & \ddots & \ddots \\
    0 & \vdots & \ddots & \ddots & \ddots & \ddots & 0      \\
    0 & \vdots & \ddots & \ddots & \ddots & 1      & 1      \\
    0 & 0      & 0      & \cdots & 0      & 1      & 1
  \end{bmatrix}
  \xrightarrow{R_n - R_{n - 1}}
  \begin{bmatrix}
    1 & 1      & 0      & \cdots & \cdots & \cdots & 0      \\
    0 & 1      & 1      & 0      & \cdots & \cdots & \cdots \\
    0 & 0      & 1      & 1      & \ddots & \ddots & \ddots \\
    0 & \vdots & \ddots & \ddots & \ddots & \ddots & \ddots \\
    0 & \vdots & \ddots & \ddots & \ddots & \ddots & 0      \\
    0 & \vdots & \ddots & \ddots & \ddots & 1      & 1      \\
    0 & 0      & 0      & \cdots & 0      & 0      & 0
  \end{bmatrix}
\end{align*}
可得秩为$n - 1$。

$n - 1$是奇数时,$n - 1$次变化后,$a_{n, n - 1} = -1$,即
\begin{align*}
  \begin{bmatrix}
    1 & 1      & 0      & \cdots & \cdots & \cdots & 0      \\
    0 & 1      & 1      & 0      & \cdots & \cdots & \cdots \\
    0 & 0      & 1      & 1      & \ddots & \ddots & \ddots \\
    0 & \vdots & \ddots & \ddots & \ddots & \ddots & \ddots \\
    0 & \vdots & \ddots & \ddots & \ddots & \ddots & 0      \\
    0 & \vdots & \ddots & \ddots & \ddots & 1      & 1      \\
    0 & 0      & 0      & \cdots & 0      & -1     & 1
  \end{bmatrix}
  \xrightarrow{R_n + R_{n - 1}}
  \begin{bmatrix}
    1 & 1      & 0      & \cdots & \cdots & \cdots & 0      \\
    0 & 1      & 1      & 0      & \cdots & \cdots & \cdots \\
    0 & 0      & 1      & 1      & \ddots & \ddots & \ddots \\
    0 & \vdots & \ddots & \ddots & \ddots & \ddots & \ddots \\
    0 & \vdots & \ddots & \ddots & \ddots & \ddots & 0      \\
    0 & \vdots & \ddots & \ddots & \ddots & 1      & 1      \\
    0 & 0      & 0      & \cdots & 0      & 0      & 2
  \end{bmatrix}
\end{align*}
可得秩为$n$。

综上,矩阵秩的情况如下:
\begin{equation*}
  \begin{cases*}
    n     & n \text{是偶数} \\
    n - 1 & n \text{是奇数}
  \end{cases*}
\end{equation*}

\section*{9}
这道题主要利用了以下命题:\\
\textbf{向量组$B$可以线性表示向量组$A$,那么$B$的秩大于$A$的秩。}

证明:

以列向量的角度考虑,
矩阵A的列向量组的任意极大线性无关部分组设为$(I)$,
矩阵B的列向量组的任意极大线性无关部分组设为$(II)$。

可得$(I) \cup (II)$和可以线性表示矩阵$A$和矩阵$B$的列向量组,
且$(I) \cup (II)$是矩阵$C$的列向量组的部分组,于是我们有
\begin{align*}
  max(rank(A), rank(B)) \leq rank(C)
\end{align*}

因为$C$中的任意列向量,一定可以被$(I) \cup (II)$表示,
而$(I) \cup (II)$不一定是线性无关的,于是我们有
\begin{align*}
  rank(C) \leq rank(A) + rank(B)
\end{align*}

\section*{10}

利用习题9。

设矩阵$D = (AB)$,那么矩阵$D$的列向量的极大线性无关部分组$(I)$,
可以线性表示$C$的任意列向量。所以,我们有
\begin{align*}
  rank(C) \leq rank(D)
\end{align*}
由习题9可知
\begin{align*}
  rank(D) \leq rank(A) + rank(B)
\end{align*}
综上
\begin{align*}
  rank(C) \leq rank(A) + rank(B)
\end{align*}

\section*{11}

由于初等行变换后,不改变原线性方程组的解,于是可得以$A$为系数举证的齐次线性方程组和
以$B$为系数矩阵的齐次线性方程组同解,即如果
\begin{align}
  k_1 \alpha_1 + k_2 \alpha_2 + \cdots + k_n \alpha_n = 0
\end{align}
那么
\begin{align*}
  k_1 \beta_1 + k_2 \beta_2 + \cdots + k_n \beta_n = 0
\end{align*}

\section*{12}

我们有
\begin{align*}
  rank(B) \leq rank(A) \leq m
\end{align*}
所以
\begin{align*}
  rank(B) - s \leq rank(A) - s \leq m - s
\end{align*}
又因为
\begin{align*}
  0 \leq rank(B) \leq s
\end{align*}
于是
\begin{align*}
  rank(A) - rank(B) \leq rank(A) - s \leq m - s
\end{align*}
综上
\begin{align*}
  rank(A) - rank(B) \leq m - s \\
  rank(B) \geq rank(A) + s - m
\end{align*}

\section*{13}

设$A$的行向量为
\begin{align*}
  \alpha_1, \alpha_2, \cdots, \alpha_n
\end{align*}

$rank(A) = 0$,说明$A$是零矩阵,那么,$K$内的数
\begin{align*}
  a_1, a_2, \cdots, a_m; b_1, b_2, \cdots, b_n
\end{align*}
都为零即可。

$rank(A) = 1$,那么对于$A$的行向量组来说,极大线性无关部分组$(I)$中只有一个向量,
不妨设为$\alpha_i$,取
\begin{align*}
  b_1 = \alpha_{i1} \\
  b_2 = \alpha_{i2} \\
  \vdots            \\
  b_n = \alpha_{in}
\end{align*}
因为$A$中的任意行向量,都可以被$(I)$线性表示,即可以被$\alpha_i$乘以某个常数$k$表示,
可设
\begin{align*}
  \alpha_1 = k_1 \alpha_i \\
  \alpha_2 = k_2 \alpha_i \\
  \vdots                  \\
  \alpha_m = k_m \alpha_i
\end{align*}
取
\begin{align*}
  a_1 = k_1 \\
  a_2 = k_2 \\
  \vdots \\
  a_m = k_m
\end{align*}

\end{document}