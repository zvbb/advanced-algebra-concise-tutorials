\documentclass{article}
\usepackage{mathtools} 
\usepackage{fontspec}
\usepackage[UTF8]{ctex}
\usepackage{amsthm}
\usepackage{mdframed}
\usepackage{xcolor}
\usepackage{amssymb}
\usepackage{amsmath}
\usepackage{hyperref}


% 定义新的带灰色背景的说明环境 zremark
\newmdtheoremenv[
  backgroundcolor=gray!10,
  % 边框与背景一致,边框线会消失
  linecolor=gray!10
]{zremark}{注释}

% 通用矩阵命令: \flexmatrix{矩阵名}{元素符号}{行数}{列数}
\newcommand{\flexmatrix}[4]{
  \[
  #1 = \begin{pmatrix}
    #2_{11}     & #2_{12}     & \cdots & #2_{1#4}   \\
    #2_{21}     & #2_{22}     & \cdots & #2_{2#4}   \\
    \vdots      & \vdots      & \ddots & \vdots     \\
    #2_{#31}    & #2_{#32}    & \cdots & #2_{#3#4}
  \end{pmatrix}
  \]
}

% 简化版命令(默认矩阵名为A,元素符号为a): \quickmatrix{行数}{列数}
\newcommand{\quickmatrix}[2]{\flexmatrix{A}{a}{#1}{#2}}


\begin{document}
\title{2.4}
\author{张志聪}
\maketitle

\section*{8}

设
\begin{align*}
  A = [\alpha_1 \ \alpha_2 \ \cdots \  \alpha_n]
\end{align*}

由题设可知
\begin{align*}
  \alpha_1, \alpha_2, \cdots, \alpha_n \ \ \ (I)
\end{align*}
是线性无关的。

设$D = AB = AC$,那么对任意列向量$col_j(D) (1 \leq j \leq s)$,
我们有
\begin{align*}
  col_j(D) = A col_j(B) = A col_j(C)
\end{align*}
即$col_j(D)$可以被$(I)$线性表示:
\begin{align*}
  col_j(D) = k_1 \alpha_1 + k_2 \alpha_2 + \cdots + k_n \alpha_n
\end{align*}
由于$(I)$是线性无关的,利用命题3.1可知,表示法是唯一的,
即$k_1, k_2, \cdots, k_n$是唯一的。
于是可得$col_j(B) = col_j(C)$,所以$B = C$。

\section*{9}

 (1) $k$的值;

需要保证$rank(A) = 2$,有两种方法确定这一点:
\begin{itemize}
  \item (1) 利用习题7

        因为$B$不是零矩阵,所以$rank(B) \geq 1$,
        利用习题7可得
        \begin{align*}
          rank(A) \leq 3 - 1 = 2
        \end{align*}
        又$A$第二列与第三列已经线性无关了,所以第一列
        一定要能被其他列线性表示,否者$rank(A) = 3$,会导致矛盾。

  \item (2) 矩阵乘法的整体理解。

        设$C = AB$,于是如果写$A = (\alpha_1, \alpha_2, \alpha_3)$,那么
        对$col_j(C)$,有
        \begin{align*}
          col_j(C) = 0 & = A \begin{bmatrix}
                               b_{1j} \\
                               b_{2j} \\
                               b_{3j}
                             \end{bmatrix}                                 \\
                       & = (\alpha_1, \alpha_2, \alpha_3) \begin{bmatrix}
                                                            b_{1j} \\
                                                            b_{2j} \\
                                                            b_{3j}
                                                          \end{bmatrix}    \\
                       & = b_{1j}\alpha_1 + b_{2j}\alpha_2 + b_{3j}\alpha_3
        \end{align*}
        如果$rank(A) = 3$,$\alpha_1, \alpha_2, \alpha_3$线性无关,那么
        \begin{align*}
          \begin{bmatrix}
            b_1j \\
            b_2j \\
            b_3j
          \end{bmatrix} = 0
        \end{align*}
        这与题设中$B \neq 0$矛盾。
\end{itemize}

于是可得$k = \frac{1}{3}$。

(2) $B$的值。

写
\begin{align*}
  B = [b_1 b_2 b_3]
\end{align*}

所以
\begin{align*}
  A b_1 = 0 \\
  A b_2 = 0 \\
  A b_3 = 0
\end{align*}

对于线性方程组$AX = 0$,
由于$rank(A) = 2$,所以方程组的基础解系存在,
且基础解系中向量个数为$3 - 2 = 1$,此时已经确定了$B$的存在性,
接下来,就是计算$AX$的基础解系,这个步骤略。

\section*{10}

由命题4.4可知
\begin{align*}
  rank(A + B) \leq rank(A) + rank(B) < n
\end{align*}
记
\begin{align*}
  C = [c_1 \, c_2 \, c_3 \, \cdots \, c_n]
\end{align*}
所以
\begin{align*}
  (A + B)C = 0
\end{align*}
可以表示成
\begin{align*}
  (A + B) c_1 = 0 \\
  (A + B) c_2 = 0 \\
  \vdots          \\
  (A + B) c_n = 0
\end{align*}
问题转变成线性方程组$(A+B)X = 0$是否有解,因为$rank(A+B) < n$,
于是方程组的基础解系存在,
所以,
可以通过基础解系中向量的线性组合构成$C$,使得$(A + B)C = 0$。

\section*{11}

因为存在非零的$C$使得$AC = 0$可得
$rank(A) < n$,否则$C$只能是零矩阵。
于是可得$rank(AB) \leq min\{rank(A), rank(B)\} < n$。

接下来的证明与习题10类似,这里不做赘述。

\section*{12}

\section*{13}

记
\begin{align*}
  A = \begin{bmatrix}
        \eta_1^T \\
        \eta_1^T \\
        \vdots   \\
        \eta_s^T
      \end{bmatrix}
\end{align*}
考虑线性方程组$Ax = 0$的解。
因为$rank(A) = s \neq n$,所以方程组的基础解系存在,
且基础解系中向量个数为$n - s$,不妨设
\begin{align*}
  \alpha_1, \alpha_2, \cdots, \alpha_{n - s}
\end{align*}
是方程组的一个基础解系。
记
\begin{align*}
  B = \begin{bmatrix}
        \alpha_1 & \alpha_2 & \cdots & \alpha_{n - s}
      \end{bmatrix}
\end{align*}
于是,我们有
\begin{align*}
  A B = 0
\end{align*}
又
\begin{align*}
  (AB)^T = B^T A^T = 0
\end{align*}
这里
\begin{align*}
  A^T = \begin{bmatrix}
          \eta_1 & \eta_2 & \cdots & \eta_s
        \end{bmatrix}
\end{align*}
因为$rank(A^T) = s, rank(B^T) = n - s$,$A^T$中的$s$个列向量都是线性无关的,
且$A^T$的任意列向量$\eta$都有$B^T \eta = 0$,
于是由基础解系的定义可知$A^T$的列向量组是$B^T$对应的齐次线性方程的基础解系。

\section*{14}



\end{document}
