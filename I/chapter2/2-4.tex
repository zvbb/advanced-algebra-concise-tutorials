\documentclass{article}
\usepackage{mathtools} 
\usepackage{fontspec}
\usepackage[UTF8]{ctex}
\usepackage{amsthm}
\usepackage{mdframed}
\usepackage{xcolor}
\usepackage{amssymb}
\usepackage{amsmath}
\usepackage{hyperref}


% 定义新的带灰色背景的说明环境 zremark
\newmdtheoremenv[
  backgroundcolor=gray!10,
  % 边框与背景一致,边框线会消失
  linecolor=gray!10
]{zremark}{注释}

% 通用矩阵命令: \flexmatrix{矩阵名}{元素符号}{行数}{列数}
\newcommand{\flexmatrix}[4]{
  \[
  #1 = \begin{pmatrix}
    #2_{11}     & #2_{12}     & \cdots & #2_{1#4}   \\
    #2_{21}     & #2_{22}     & \cdots & #2_{2#4}   \\
    \vdots      & \vdots      & \ddots & \vdots     \\
    #2_{#31}    & #2_{#32}    & \cdots & #2_{#3#4}
  \end{pmatrix}
  \]
}

% 简化版命令(默认矩阵名为A,元素符号为a): \quickmatrix{行数}{列数}
\newcommand{\quickmatrix}[2]{\flexmatrix{A}{a}{#1}{#2}}


\begin{document}
\title{2.4}
\author{张志聪}
\maketitle

\section*{8}

\begin{itemize}
  \item 方法一

        设
        \begin{align*}
          D = A B = A C
        \end{align*}
        设$B$的列向量组为$\beta_1, \beta_2, \cdots, \beta_s$,
        $C$的列向量组为$\gamma_1, \gamma_2, \cdots, \gamma_s$,
        $D$的列向量组为$\eta_1, \eta_2, \cdots, \eta_s$,
        那么按照矩阵乘法,任意$1 \leq i \leq s$,应有
        \begin{align*}
          \eta_i = A \beta_i = A \gamma_i \\
          A (\beta_i - \gamma_i) = 0
        \end{align*}
        由于$rank(A) = n$,于是方程组$A x = 0$只有零解,
        所以
        \begin{align*}
          \beta_i - \gamma_i = 0 \\
          \beta_i = \gamma_i
        \end{align*}
        由于$i$是任意的,所以
        \begin{align*}
          B = C
        \end{align*}

  \item 方法二

        设
        \begin{align*}
          A = [\alpha_1 \ \alpha_2 \ \cdots \  \alpha_n]
        \end{align*}

        由题设可知
        \begin{align*}
          \alpha_1, \alpha_2, \cdots, \alpha_n \ \ \ (I)
        \end{align*}
        是线性无关的。

        设$D = AB = AC$,那么对任意列向量$col_j(D) (1 \leq j \leq s)$,
        我们有
        \begin{align*}
          col_j(D) = A col_j(B) = A col_j(C)
        \end{align*}
        即$col_j(D)$可以被$(I)$线性表示:
        \begin{align*}
          col_j(D) = k_1 \alpha_1 + k_2 \alpha_2 + \cdots + k_n \alpha_n
        \end{align*}
        由于$(I)$是线性无关的,利用命题3.1可知,表示法是唯一的,
        即$k_1, k_2, \cdots, k_n$是唯一的。
        于是可得$col_j(B) = col_j(C)$,所以$B = C$。
\end{itemize}


\section*{9}

 (1) $k$的值;

需要保证$rank(A) = 2$,有两种方法确定这一点:
\begin{itemize}
  \item (1) 利用习题7

        因为$B$不是零矩阵,所以$rank(B) \geq 1$,
        利用习题7可得
        \begin{align*}
          rank(A) \leq 3 - 1 = 2
        \end{align*}
        又$A$第二列与第三列已经线性无关了,所以第一列
        一定要能被其他列线性表示,否者$rank(A) = 3$,会导致矛盾。

  \item (2) 矩阵乘法的整体理解。

        设$C = AB$,于是如果写$A = (\alpha_1, \alpha_2, \alpha_3)$,那么
        对$col_j(C)$,有
        \begin{align*}
          col_j(C) = 0 & = A \begin{bmatrix}
                               b_{1j} \\
                               b_{2j} \\
                               b_{3j}
                             \end{bmatrix}                                 \\
                       & = (\alpha_1, \alpha_2, \alpha_3) \begin{bmatrix}
                                                            b_{1j} \\
                                                            b_{2j} \\
                                                            b_{3j}
                                                          \end{bmatrix}    \\
                       & = b_{1j}\alpha_1 + b_{2j}\alpha_2 + b_{3j}\alpha_3
        \end{align*}
        如果$rank(A) = 3$,$\alpha_1, \alpha_2, \alpha_3$线性无关,那么
        \begin{align*}
          \begin{bmatrix}
            b_1j \\
            b_2j \\
            b_3j
          \end{bmatrix} = 0
        \end{align*}
        这与题设中$B \neq 0$矛盾。
\end{itemize}

于是可得$k = \frac{1}{3}$。

(2) $B$的值。

写
\begin{align*}
  B = [b_1 b_2 b_3]
\end{align*}

所以
\begin{align*}
  A b_1 = 0 \\
  A b_2 = 0 \\
  A b_3 = 0
\end{align*}

对于线性方程组$AX = 0$,
由于$rank(A) = 2$,所以方程组的基础解系存在,
且基础解系中向量个数为$3 - 2 = 1$,此时已经确定了$B$的存在性,
接下来,就是计算$AX$的基础解系,这个步骤略。

\section*{10}

由命题4.4可知
\begin{align*}
  rank(A + B) \leq rank(A) + rank(B) < n
\end{align*}
记
\begin{align*}
  C = [c_1 \, c_2 \, c_3 \, \cdots \, c_n]
\end{align*}
所以
\begin{align*}
  (A + B)C = 0
\end{align*}
可以表示成
\begin{align*}
  (A + B) c_1 = 0 \\
  (A + B) c_2 = 0 \\
  \vdots          \\
  (A + B) c_n = 0
\end{align*}
问题转变成线性方程组$(A+B)X = 0$是否有解,因为$rank(A+B) < n$,
于是方程组的基础解系存在,
所以,
可以通过基础解系中向量的线性组合构成$C$,使得$(A + B)C = 0$。

\section*{11}

因为存在非零的$C$使得$AC = 0$可得
$rank(A) < n$,否则$C$只能是零矩阵。
于是可得$rank(AB) \leq min\{rank(A), rank(B)\} < n$。

接下来的证明与习题10类似,这里不做赘述。

\section*{12}

设$D = AB$,
考虑矩阵的齐次线性方程组:
\begin{align*}
  A x = 0 \\
  B x = 0 \\
  C x = 0 \\
  D x = AB x = BA x = 0
\end{align*}

于是,他们的基础解系的秩,可以分别设为
\begin{align*}
  s = n - rank(A) \\
  t = n - rank(B) \\
  u = n - rank(C) \\
  v = n - rank(AB)
\end{align*}

且由基础解系构成的解集,分别设为:
\begin{align*}
  S_A \\
  S_B \\
  S_C \\
  S_D
\end{align*}
因为$Cx = 0$,是$Ax = 0, Bx = 0$的联立方程组,于是有
\begin{align*}
  S_C = S_A \cap S_B
\end{align*}
因为
\begin{align*}
  Dx = A(Bx) = B(Ax) = 0
\end{align*}
所以$S_A \subseteq S_D, S_B \subseteq S_D$,于是有
\begin{align*}
  (S_A \cup S_B) \subseteq S_D
\end{align*}

对于$Cx = 0$的基础解系$\gamma_1,\gamma_2,\cdots,\gamma_u$,
由$S_C = S_A \cap S_B$,可知$\gamma_1,\gamma_2,\cdots,\gamma_u$
都是$Ax = 0, Bx = 0$的解。

由$\S 3$习题7可知,可以在$\gamma_1,\gamma_2,\cdots,\gamma_u$基础上
扩充成$Ax = 0, Bx = 0$的基础解系:
\begin{align*}
  \alpha_{u + 1}, \cdots, \alpha_s, \gamma_1, \gamma_2, \cdots, \gamma_u \\
  \gamma_1, \gamma_2, \cdots, \gamma_u, \beta_{u + 1}, \cdots, \beta_t
\end{align*}
现在,我们证明
\begin{align*}
  \alpha_{u + 1}, \cdots, \alpha_s, \gamma_1, \gamma_2, \cdots, \gamma_u, \beta_{u + 1}, \cdots, \beta_t
  \ \ \ (I)
\end{align*}
是线性无关的。
\begin{align*}
  a_{u + 1}\alpha_{u + 1} + \cdots + a_s \alpha_s
  + b_1 \gamma_1 + \cdots + b_u \gamma_u + c_{u + 1}\beta_{u + 1} + \cdots + c_t\beta_t = 0 \\
  a_{u + 1}\alpha_{u + 1} + \cdots + a_s \alpha_s + b_1 \gamma_1 + \cdots + b_u \gamma_u
  = - (c_{u + 1}\beta_{u + 1} + \cdots + c_t\beta_t)
\end{align*}
又我们有
\begin{align*}
  a_{u + 1}\alpha_{u + 1} + \cdots + a_s \alpha_s + b_1 \gamma_1 + \cdots + b_u \gamma_u \in S_A \\
  - (c_{u + 1}\beta_{u + 1} + \cdots + c_t\beta_t) \in S_B
\end{align*}
所以
\begin{align*}
  a_{u + 1}\alpha_{u + 1} + \cdots + a_s \alpha_s + b_1 \gamma_1 + \cdots + b_u \gamma_u \in S_C \\
  - (c_{u + 1}\beta_{u + 1} + \cdots + c_t\beta_t) \in S_C
\end{align*}
于是存在
\begin{align*}
  c_{u + 1}\beta_{u + 1} + \cdots + c_t\beta_t = d_1 \gamma_1 + \cdots + d_u \gamma_u
\end{align*}
所以
\begin{align*}
  - (c_{u + 1}\beta_{u + 1} + \cdots + c_t\beta_t) + d_1 \gamma_1 + \cdots + d_u \gamma_u = 0
\end{align*}
由于这里使用的向量都是$Bx = 0$基础解系中的向量,于是可得
\begin{align*}
  d_1 = d_2 = \cdots = d_u = 0 \\
  c_{u + 1} = c_{u + 2} = \cdots = c_t = 0
\end{align*}
于是我们有
\begin{align*}
  a_{u + 1}\alpha_{u + 1} + \cdots + a_s \alpha_s + b_1 \gamma_1 + \cdots + b_u \gamma_u
   & = - (c_{u + 1}\beta_{u + 1} + \cdots + c_t\beta_t) \\
   & = 0
\end{align*}
这个左侧都是$Ax = 0$的基础解系中的向量,所以有
\begin{align*}
  a_{u + 1} = \cdots = a_s = 0 \\
  b_1 = \cdots = b_u = 0
\end{align*}
综上可得,
$(I)$是线性无关的。

由于$(S_A \cup S_B) \subseteq S_D$,
所以$(I)$可以被$Dx = 0$的基础解系表示,
利用替换定理,我们有
\begin{align*}
  v \geq s + t - u                                          \\
  n - rank(AB) \geq n - rank(A) + n - rank(B) - n + rank(C) \\
  rank(A) + rank(B) \geq rank(C) + rank(AB)
\end{align*}

\section*{13}

记
\begin{align*}
  A = \begin{bmatrix}
        \eta_1^T \\
        \eta_1^T \\
        \vdots   \\
        \eta_s^T
      \end{bmatrix}
\end{align*}
考虑线性方程组$Ax = 0$的解。
因为$rank(A) = s \neq n$,所以方程组的基础解系存在,
且基础解系中向量个数为$n - s$,不妨设
\begin{align*}
  \alpha_1, \alpha_2, \cdots, \alpha_{n - s}
\end{align*}
是方程组的一个基础解系。
记
\begin{align*}
  B = \begin{bmatrix}
        \alpha_1 & \alpha_2 & \cdots & \alpha_{n - s}
      \end{bmatrix}
\end{align*}
于是,我们有
\begin{align*}
  A B = 0
\end{align*}
又
\begin{align*}
  (AB)^T = B^T A^T = 0
\end{align*}
这里
\begin{align*}
  A^T = \begin{bmatrix}
          \eta_1 & \eta_2 & \cdots & \eta_s
        \end{bmatrix}
\end{align*}
因为$rank(A^T) = s, rank(B^T) = n - s$,$A^T$中的$s$个列向量都是线性无关的,
且$A^T$的任意列向量$\eta$都有$B^T \eta = 0$,
于是由基础解系的定义可知$A^T$的列向量组是$B^T$对应的齐次线性方程的基础解系。

\section*{14}

设
\begin{align*}
  \gamma_0, \gamma_1, \gamma_2, \cdots \gamma_s \ \ \ (I)
\end{align*}
令
\begin{align*}
  \eta_1 = \gamma_1 - \gamma_0 \\
  \eta_2 = \gamma_2 - \gamma_0 \\
  \vdots                       \\
  \eta_s = \gamma_s - \gamma_0
\end{align*}
易得向量组
\begin{align*}
  \gamma_0, \eta_1, \eta_2, \cdots \eta_s \ \ \ (II)
\end{align*}
与$(I)$是线性等价的。
由$(I)$是线性无关的,可得$(II)$是线性无关的,
于是可得
\begin{align*}
  \eta_1, \eta_2, \cdots \eta_s \ \ \ (III)
\end{align*}
是线性无关的。

利用习题13可得,存在一个$K$上一个齐次线性方程组以$(III)$
为基础解系。设该齐次线性方程组的系数矩阵为$B$,并令
\begin{align*}
  \beta = B \gamma_0
\end{align*}
先证明$\beta \neq 0$,
如果$B \gamma_0 = 0$,那么$\gamma_0$应该可以被$Bx = 0$的基础解系
$(III)$表示,这与$(II)$是线性无关向量组矛盾。
到此,得到非齐次线性方程组
\begin{align*}
  Bx = \beta
\end{align*}

接下来,证明该非齐次线性方程组满足题设要求。
通过构造过程,我们知道$\gamma_0$是一个特解,
$(III)$是其基础解系。并且对任意$1 \leq j \leq s$,我们有
\begin{align*}
  B\eta_j   & = 0                      \\
            & = B(\gamma_j - \gamma_0) \\
            & = B\gamma_j - B\gamma_0  \\
            & = B\gamma_j - \beta      \\
  \implies                             \\
  B\gamma_j & = \beta
\end{align*}
可得$\gamma_j$是非齐次线性方程组的解,
所以这个非齐次线性方程组满足了题设条件(1);

非齐次线性方程组的解都可以表示成
\begin{align*}
  \gamma & = \gamma_0 + k_1 \eta_1 + k_2 \eta_2 + \cdots + k_s \eta_s
\end{align*}
$(I)(II)$线性等价,于是可得题设条件(2)成立。

\section*{15}
 (1)

由基础解系与秩的关系,我们有
\begin{align*}
  rank(B) = s - k \\
  rank(AB) = s - l
\end{align*}
于是可得
\begin{align*}
  rank(B) - rank(AB) = l - k
\end{align*}
于是问题等价于证明$B\eta_{k + 1}, \cdots, B\eta_l$是线性无关的。

设$a_{k+1}, \cdots, a_l \in K$使得
\begin{align*}
  a_{k + 1}B\eta_{k + 1} + \cdots + a_lB\eta_l = 0 \\
  B(a_{k + 1} \eta_{k + 1} + \cdots + a_l \eta_l) = 0
\end{align*}
于是可得$a_{k + 1} \eta_{k + 1} + \cdots + a_l \eta_l$是
$BX$的一个解,所以可以被$\eta_1, \cdots, \eta_k$线性表示,
系数不妨设为$a_1, \cdots, a_k$,即
\begin{align*}
  a_{k + 1} \eta_{k + 1} + \cdots + a_l \eta_l
  = a_1 \eta_1 + \cdots + a_k \eta_k \\
  a_1 \eta_1 + \cdots + a_k \eta_k + a_{k + 1} \eta_{k + 1} + \cdots + a_l \eta_l = 0
\end{align*}
因为$\eta_1, \cdots, \eta_k, \eta_{k + 1}, \cdots, \eta_l$是线性无关的,
于是可得
\begin{align*}
  a_1 = a_2 = \cdots = a_k = a_{k + 1} = \cdots = a_l = 0
\end{align*}
从而
\begin{align*}
  a_{k + 1}B\eta_{k + 1} + \cdots + a_lB\eta_l = 0
\end{align*}
只有零解,所以$B\eta_{k + 1}, \cdots, B\eta_l$是线性无关的。

(2) 证明命题4.6

我们有
\begin{align*}
  (AB)X = A(BX) = 0
\end{align*}
因为$\eta_{k + 1}, \cdots , \eta_l$是$(AB)X = 0$的解,
所以$B\eta_{k + 1}, \cdots, B\eta_l$是$A(BX) = 0$的解。

设$(I)$是$A(BX) = 0$的一个基础解系,秩为$n - rank(A)$,
于是可得$(I)$可以线性表示$B\eta_{k + 1}, \cdots, B\eta_l$,
又因为$B\eta_{k + 1}, \cdots, B\eta_l$是线性无关的,
所以
\begin{align*}
  n - rank(A) \geq l - k = rank(B) - rank(AB) \\
  \implies                                    \\
  rank(AB) \geq rank(A) + rank(B) - n
\end{align*}

\section*{16}

先考虑$C$的秩。
我们有
\begin{align*}
  rank(A) + rank(B) - n \leq rank(C) \leq min\{rank(A), rank(B)\} \\
  rank(B) \leq rank(C) \leq rank(B)                               \\
  r \leq rank(C) \leq r                                           \\
  rank(C) = r
\end{align*}

设
\begin{align*}
  c_{i_1} = B \beta_{i_1} \\
  c_{i_2} = B \beta_{i_2} \\
  \vdots                  \\
  c_{i_r} = B \beta_{i_r}
\end{align*}
验证
\begin{align*}
  c_{i_1}, c_{i_2}, \cdots, c_{i_r} \ \ \ (I)
\end{align*}
是否为$C$的极大线性无关部分组。

我们只需证明$(I)$的极大性,即$C$中的任意列向量都可以被$(I)$线性表示,
就可以证明$(I)$是$C$的极大线性无关部分组($\S 2$习题15)。

$C$任意列向量$c_j(1 \leq j \leq s)$为
\begin{align*}
  c_j = A \beta_j
\end{align*}
由于$\beta_{i_1}, \beta_{i_2}, \cdots, \beta_{i_r}$是$B$的极大线性无关部分组,
所以$\beta_j$可以被其线性表示:
\begin{align*}
  \beta_j = k_1 \beta_{i_1} + k_2 \beta_{i_2} + \cdots + k_r \beta_{i_r}
\end{align*}
于是可得
\begin{align*}
  c_j & = A(k_1 \beta_{i_1} + k_2 \beta_{i_2} + \cdots + k_r \beta_{i_r})    \\
      & = k_1 A \beta_{i_1} + k_2 A \beta_{i_2} + \cdots + k_r A \beta_{i_r} \\
      & = k_1 c_{i_1} + k_2 c_{i_2} + \cdots + k_r c_{i_r}
\end{align*}
所以,$c_j$可以被$(I)$线性表示,极大性得证。

\end{document}
