\documentclass{article}
\usepackage{mathtools} 
\usepackage{fontspec}
\usepackage[UTF8]{ctex}
\usepackage{amsthm}
\usepackage{mdframed}
\usepackage{xcolor}
\usepackage{amssymb}
\usepackage{amsmath}
\usepackage{hyperref}


% 定义新的带灰色背景的说明环境 zremark
\newmdtheoremenv[
  backgroundcolor=gray!10,
  % 边框与背景一致,边框线会消失
  linecolor=gray!10
]{zremark}{注释}

% 通用矩阵命令: \flexmatrix{矩阵名}{元素符号}{行数}{列数}
\newcommand{\flexmatrix}[4]{
  \[
  #1 = \begin{pmatrix}
    #2_{11}     & #2_{12}     & \cdots & #2_{1#4}   \\
    #2_{21}     & #2_{22}     & \cdots & #2_{2#4}   \\
    \vdots      & \vdots      & \ddots & \vdots     \\
    #2_{#31}    & #2_{#32}    & \cdots & #2_{#3#4}
  \end{pmatrix}
  \]
}

% 简化版命令(默认矩阵名为A,元素符号为a): \quickmatrix{行数}{列数}
\newcommand{\quickmatrix}[2]{\flexmatrix{A}{a}{#1}{#2}}


\begin{document}
\title{2.6}
\author{张志聪}
\maketitle

\section*{5}

以$rank(A) = m$为例。证明与书中命题6.1一致,
把$A$化作标准型$D_1$,因为$rank(A) = m$,与
$C$的行数一致,于是可以通过$D_1$把$C$化作零矩阵,于是我们有
\begin{align*}
  \begin{bmatrix}
    D_1 & 0 \\
    0   & B
  \end{bmatrix}
\end{align*}
通过行列变换把$B$化作标准型$D_2$,因为$B$的左侧和上方都是$0$,于是这些操作对
其他方块没有影响,于是我们有
\begin{align*}
  \begin{bmatrix}
    D_1 & 0   \\
    0   & D_2
  \end{bmatrix}
\end{align*}
所以
\begin{align*}
  rank(M) = rank(D_1) + rank(D_2) = rank(A) + rank(B)
\end{align*}

\section*{9}

我们有
\begin{align*}
  f(J) = \begin{bmatrix}
           f(J_1) &        & 0      \\
                  & f(J_2) &        \\
           0      &        & f(J_3)
         \end{bmatrix}
\end{align*}
令
\begin{align*}
  J_1 = \begin{bmatrix}
          -2 & 1  \\
          0  & -2
        \end{bmatrix}
  = E_2 + \begin{bmatrix}
            0 & 1 \\
            0 & 0
          \end{bmatrix}
\end{align*}
我们有
\begin{align*}
  \begin{bmatrix}
    0 & 1 \\
    0 & 0
  \end{bmatrix} \begin{bmatrix}
                  0 & 1 \\
                  0 & 0
                \end{bmatrix} = 0
\end{align*}
$J_2, J_3$有类似的情况,
于是设
\begin{align*}
  g(x) = (x - 2)^3 \\
  h(x) = (x - 3)^2
\end{align*}
我们有
\begin{align*}
  g(J_1) = 0 \\
  g(J_2) = 0 \\
  h(J_3) = 0
\end{align*}
于是可得
\begin{align*}
  f(x) = g(x) h(x)
\end{align*}
此时
\begin{align*}
  f(J_1) = 0 \\
  f(J_2) = 0 \\
  f(J_3) = 0
\end{align*}
即
\begin{align*}
  f(J) = \begin{bmatrix}
           f(J_1) &        & 0      \\
                  & f(J_2) &        \\
           0      &        & f(J_3)
         \end{bmatrix}
  = 0
\end{align*}

\section*{12}
因为$A, A^T$可交换,所以
\begin{align*}
  A A^T = A^T A
\end{align*}
于是
\begin{align*}
  A A^T =
  \begin{bmatrix}
    A_1 & A_2 \\
    0   & A_3
  \end{bmatrix}
  \begin{bmatrix}
    A_1^T & 0     \\
    A_2^T & A_3^T
  \end{bmatrix}
  =
  \begin{bmatrix}
    A_1 A_1^T + A_2 A_2^T & * \\
    *                     & *
  \end{bmatrix}
\end{align*}
又
\begin{align*}
  \begin{bmatrix}
    A_1^T & 0     \\
    A_2^T & A_3^T
  \end{bmatrix}
  \begin{bmatrix}
    A_1 & A_2 \\
    0   & A_3
  \end{bmatrix}
  =
  \begin{bmatrix}
    A_1^T A_1 & * \\
    *         & *
  \end{bmatrix}
\end{align*}
于是我们有
\begin{align*}
  A_1 A_1^T + A_2 A_2^T = A_1^T A_1
\end{align*}
由$\S 5$习题23-(1)(准确说是一般形式无需是方阵),我们有
\begin{align*}
  Tr(A_1 A_1^T + A_2 A_2^T) = Tr(A_1^T A_1)     \\
  Tr(A_1 A_1^T) + Tr(A_2 A_2^T) = Tr(A_1^T A_1) \\
  Tr(A_2 A_2^T) = 0
\end{align*}
由$\S 5$习题23-(2),可知
\begin{align*}
  A_2 = 0
\end{align*}

\end{document}
