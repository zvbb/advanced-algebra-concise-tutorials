\documentclass{article}
\usepackage{mathtools} 
\usepackage{fontspec}
\usepackage[UTF8]{ctex}
\usepackage{amsthm}
\usepackage{mdframed}
\usepackage{xcolor}
\usepackage{amssymb}
\usepackage{amsmath}
\usepackage{hyperref}


% 定义新的带灰色背景的说明环境 zremark
\newmdtheoremenv[
  backgroundcolor=gray!10,
  % 边框与背景一致,边框线会消失
  linecolor=gray!10
]{zremark}{注释}

% 通用矩阵命令: \flexmatrix{矩阵名}{元素符号}{行数}{列数}
\newcommand{\flexmatrix}[4]{
  \[
  #1 = \begin{pmatrix}
    #2_{11}     & #2_{12}     & \cdots & #2_{1#4}   \\
    #2_{21}     & #2_{22}     & \cdots & #2_{2#4}   \\
    \vdots      & \vdots      & \ddots & \vdots     \\
    #2_{#31}    & #2_{#32}    & \cdots & #2_{#3#4}
  \end{pmatrix}
  \]
}

% 简化版命令(默认矩阵名为A,元素符号为a): \quickmatrix{行数}{列数}
\newcommand{\quickmatrix}[2]{\flexmatrix{A}{a}{#1}{#2}}


\begin{document}
\title{2.4 注释}
\author{张志聪}
\maketitle

\begin{zremark}
  对某种映射的讨论:

  在数域$K$,对任意$m,n \in \mathbb{N}^+$
  定义映射
  \begin{align*}
     & \theta     :  \{A: A \in M_{m,n}(K)\} \to \{K^n \to K^m \text{的映射}\} \\
     & \theta(A)  = f_A
  \end{align*}
  它是单射、满射?
\end{zremark}

\textbf{证明:}

\begin{itemize}
  \item (1) 是单射;

        对任意$A, B \in M_{m,n}(K)$且$A \neq B$,
        所存在某列$col_j(A) \neq col_j(B)$($1 \leq j \leq n$)。
        于是取$K^n$中坐标向量
        \begin{align*}
          x_j^T = \begin{bmatrix}
                    0      \\
                    \vdots \\
                    0      \\
                    1      \\
                    0      \\
                    \vdots
                    \\ 0
                  \end{bmatrix}
          \cdots \ \cdots j
        \end{align*}
        此时
        \begin{align*}
          \theta(A)(x_j^T) = f_A(x_j^T) = col_j(A) \\
          \theta(B)(x_j^T) = f_B(x_j^T) = col_j(B) \\
          \implies                                 \\
          \theta(A) \neq \theta(B)
        \end{align*}
        所以,$\theta$是单射。
  \item (2) 不是满射。

        举一个反例,设映射$f: K^n \to K^m $,对任意$x \in K^n$,$f(x) = [1, 1, \cdots, 1]^T$。
        但对任意矩阵$A \in M_{m,n}(K)$,我们有
        \begin{align*}
          \theta(A)(0) = f_A(0) = 0
        \end{align*}
        即:在$\theta$中找不到原像$A$,使得$\theta(A) = f$。
        故不是满射。
\end{itemize}


\begin{zremark}
  命题4.4(ii)的扩展。
  \begin{align*}
    |rank(A) - rank(B)| \leq rank(A + B) \leq rank(A) + rank(B)
  \end{align*}
\end{zremark}

\textbf{证明:}

因为
\begin{align*}
  A = A + B - B
\end{align*}

通过(ii),我们有
\begin{align*}
  rank(A)           & \leq rank(A + B) + rank(-B) \\
                    & = rank(A + B) + rank(B)     \\
                    & \implies                    \\
  rank(A) - rank(B) & \leq rank(A + B)
\end{align*}

同理,我们有
\begin{align*}
  rank(B) - rank(A) \leq rank(A + B)
\end{align*}

综上
\begin{align*}
  |rank(A) - rank(B)| \leq rank(A + B)
\end{align*}
结合已知的(ii),我们有
\begin{align*}
  |rank(A) - rank(B)| \leq rank(A + B) \leq rank(A) + rank(B)
\end{align*}
\end{document}