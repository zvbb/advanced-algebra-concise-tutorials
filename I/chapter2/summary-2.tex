\documentclass{article}
\usepackage{mathtools} 
\usepackage{fontspec}
\usepackage[UTF8]{ctex}
\usepackage{amsthm}
\usepackage{mdframed}
\usepackage{xcolor}
\usepackage{amssymb}
\usepackage{amsmath}
\usepackage{hyperref}


% 定义新的带灰色背景的说明环境 zremark
\newmdtheoremenv[
  backgroundcolor=gray!10,
  % 边框与背景一致,边框线会消失
  linecolor=gray!10
]{zremark}{注释}

% 通用矩阵命令: \flexmatrix{矩阵名}{元素符号}{行数}{列数}
\newcommand{\flexmatrix}[4]{
  \[
  #1 = \begin{pmatrix}
    #2_{11}     & #2_{12}     & \cdots & #2_{1#4}   \\
    #2_{21}     & #2_{22}     & \cdots & #2_{2#4}   \\
    \vdots      & \vdots      & \ddots & \vdots     \\
    #2_{#31}    & #2_{#32}    & \cdots & #2_{#3#4}
  \end{pmatrix}
  \]
}

% 简化版命令(默认矩阵名为A,元素符号为a): \quickmatrix{行数}{列数}
\newcommand{\quickmatrix}[2]{\flexmatrix{A}{a}{#1}{#2}}


\begin{document}
\title{第二章总结}
\author{张志聪}
\maketitle

\begin{zremark}
  矩阵消去律成立的充分条件:

  矩阵$A$满秩,那么
  \begin{align*}
    AB = AC               \\
    A^{-1} A = A^{-1} A C \\
    B = C
  \end{align*}
\end{zremark}

\begin{zremark}
  矩阵 $A, B$满足
  \begin{align*}
    AB = BA
  \end{align*}
  的充分条件有:
  \begin{itemize}
    \item $A,B$都是对角矩阵。
    \item 还有些其他条件。(持续补充)
  \end{itemize}
\end{zremark}


\end{document}
