\documentclass{article}
\usepackage{mathtools} 
\usepackage{fontspec}
\usepackage[UTF8]{ctex}
\usepackage{amsthm}
\usepackage{mdframed}
\usepackage{xcolor}
\usepackage{amssymb}
\usepackage{amsmath}
\usepackage{hyperref}


% 定义新的带灰色背景的说明环境 zremark
\newmdtheoremenv[
  backgroundcolor=gray!10,
  % 边框与背景一致,边框线会消失
  linecolor=gray!10
]{zremark}{注释}

% 通用矩阵命令: \flexmatrix{矩阵名}{元素符号}{行数}{列数}
\newcommand{\flexmatrix}[4]{
  \[
  #1 = \begin{pmatrix}
    #2_{11}     & #2_{12}     & \cdots & #2_{1#4}   \\
    #2_{21}     & #2_{22}     & \cdots & #2_{2#4}   \\
    \vdots      & \vdots      & \ddots & \vdots     \\
    #2_{#31}    & #2_{#32}    & \cdots & #2_{#3#4}
  \end{pmatrix}
  \]
}

% 简化版命令(默认矩阵名为A,元素符号为a): \quickmatrix{行数}{列数}
\newcommand{\quickmatrix}[2]{\flexmatrix{A}{a}{#1}{#2}}


\begin{document}
\title{2.3 注释}
\author{张志聪}
\maketitle

\begin{zremark}
  $K^m$内向量方程
  \begin{align*}
    x_1 \alpha_1 + x_2 \alpha_2 + \cdots + x_m \alpha_m = 0
  \end{align*}
  等价于$K$上的齐次线性方程组,它的解集
  \begin{align*}
    S = \{
    \begin{bmatrix}
      x_1    \\
      x_2    \\
      \vdots \\
      x_m
    \end{bmatrix}
    \in K^m : x_1 \alpha_1 + x_2 \alpha_2 + \cdots + x_m \alpha_m = 0\}
  \end{align*}
  对于加法、乘法封闭。
  即:\\
  $\forall \eta_1, \eta_2 \in S$,有$\eta_1 + \eta_2 \in S$;\\
  $\forall \eta \in S, k \in K$,有$k \eta \in S$。
\end{zremark}

\begin{zremark}
  形如
  \begin{align*}
    A = \begin{bmatrix}
          b      & a      & \cdots & a      \\
          a      & b      & \cdots & a      \\
          \vdots & \vdots & \ddots & \vdots \\
          a      & a      & \cdots & b
        \end{bmatrix}
  \end{align*}
  的$n \times n$系数矩阵,秩的情况。
\end{zremark}

所有行都加到第一行(初等行变换),我们有
\begin{align*}
  \begin{bmatrix}
    (n - 1)a + b & (n - 1)a + b & \cdots & (n - 1)a + b \\
    a            & b            & \cdots & a            \\
    \vdots       & \vdots       & \ddots & \vdots       \\
    a            & a            & \cdots & b
  \end{bmatrix}
\end{align*}

于是可以分两种情况讨论:
\begin{itemize}
  \item $(n - 1)a + b \neq 0$;

        把第一行除以$\frac{1}{(n - 1)a + b}$得到
        \begin{align*}
          \begin{bmatrix}
            1      & 1      & \cdots & 1      \\
            a      & b      & \cdots & a      \\
            \vdots & \vdots & \ddots & \vdots \\
            a      & a      & \cdots & b
          \end{bmatrix}
          \xrightarrow{\text{从第二行开始减去$a$倍的第一行}}
          \begin{bmatrix}
            1      & 1      & \cdots & 1      \\
            0      & b - a  & \cdots & 0      \\
            \vdots & \vdots & \ddots & \vdots \\
            0      & 0      & \cdots & b - a
          \end{bmatrix}
        \end{align*}

        可见,除在主对角线上是$b - a$,其余位置都是$0$。

        所以,我们有
        \begin{equation*}
          rank(A) =
          \begin{cases*}
            1, a = b \\
            n, a \neq b
          \end{cases*}
        \end{equation*}

  \item $(n - 1)a + b = 0$;

        我们有
        \begin{align*}
          \begin{bmatrix}
            0      & 0      & \cdots & 0      \\
            a      & b      & \cdots & a      \\
            \vdots & \vdots & \ddots & \vdots \\
            a      & a      & \cdots & b
          \end{bmatrix}
          \xrightarrow{\text{从第二列开始减去第一列}}
          \begin{bmatrix}
            0      & 0      & \cdots & 0      \\
            a      & b - a  & \cdots & 0      \\
            \vdots & \vdots & \ddots & \vdots \\
            a      & 0      & \cdots & b - a
          \end{bmatrix}
        \end{align*}

        如果$b - a = 0$,那么由$(n - 1)a + b = 0$可知$a = b = 0$,
        此时矩阵是零矩阵,$rank(A) = 0$。

        如果$b - a \neq 0$,那么$rank(A) = n - 1$。
\end{itemize}

综上,
\begin{equation*}
  rank(A) = \begin{cases}
    0     & a = b = 0           \\
    1     & a = b \neq 0        \\
    n - 1 & (n - 1)a + b = 0    \\
    n     & (n - 1)a + b \neq 0
  \end{cases}
\end{equation*}

\begin{zremark}
  数域$K_1$上含有$n$个未知量的非齐次线性方程组$(I)$,扩大数域到$K_2(K_1 \subset K_2)$,
  对解的影响。
\end{zremark}

\textbf{证明:}

设$(I)$的系数矩阵和增广矩阵分别为$A, \overline{A}$,常数列设为$\beta$。

通过初等变换把$A$化作标准型,
(由书中的讨论可知,在初等变换中可只使用$K_1$的数,化为标准型),
最后,确定$A$的秩。

分情况讨论:
\begin{itemize}
  \item 无解;

        即$rank(A) \neq rank(\overline{A})$,
        把数域扩展到$K_2$,因为$K_1 \subset K_2$,
        标准型可以通过相同的初等变换得到(即:只用$K_1$中的数),
        % 标准型有其特殊性,元素只有1和0,即使数域改变,还是要满足数域的运算法则,线性关系无法改变
        所以,秩的关系不会因为数域的扩大而改变。
        因此,在数域$K_2$也是无解的。

  \item 唯一解;

        即$rank(A) = rank(\overline{A}) = n$,
        在$K_1$下,$(I)$存在唯一解。

        在$K_2$下,秩的情况是不变的,于是在$K_2$下,
        $(I)$也存在唯一解。

        这唯一的解在$K_1$中已经被唯一确定。
        所以在$K_2$中的解也只会是这个在$K_1$中的解,
        即:解是相同的。

  \item 无数解;

        即$rank(A) = rank(\overline{A}) = r < n$,
        在$K_1$下,$(I)$的解可以表示成
        \begin{align*}
          \gamma_0 + k_1 \eta_1 + k_2 \eta_2 + \cdots + k_{n - r} \eta_{n - r}
        \end{align*}
        其中$\gamma_0$是特解,
        \begin{align*}
          \eta_1, \eta_2, \cdots, \eta_{n - r}
        \end{align*}
        是$(I)$导出方程组的基础解系,且
        \begin{align*}
          k_1, k_2, \cdots, k_{n - r} \in K_1
        \end{align*}

        因为在$K_2$下,$A$的秩是不变的,
        于是导出方程组的基础解系向量个数也是$n - r$,
        显然$\eta_1, \eta_2, \cdots, \eta_{n - r}$
        也是$K_2$下的解,于是由习题3,可知该基础解系也是
        $K_2$下导出方程组的基础解系。

        特解$\gamma_0$也是$K_2$下的特解,
        于是,在$K_2$下,$(I)$的解可以表示成
        \begin{align*}
          \gamma_0 + s_1 \eta_1 + s_2 \eta_2 + \cdots + s_{n - r} \eta_{n - r}
        \end{align*}
        其中
        \begin{align*}
          s_1, s_2, \cdots, s_{n - r} \in K_2
        \end{align*}
        因为系数的不同,在$K_2$下,解是不同的。
\end{itemize}

\end{document}