\documentclass{article}
\usepackage{mathtools} 
\usepackage{fontspec}
\usepackage[UTF8]{ctex}
\usepackage{amsthm}
\usepackage{mdframed}
\usepackage{xcolor}
\usepackage{amssymb}
\usepackage{amsmath}
\usepackage{hyperref}


% 定义新的带灰色背景的说明环境 zremark
\newmdtheoremenv[
  backgroundcolor=gray!10,
  % 边框与背景一致,边框线会消失
  linecolor=gray!10
]{zremark}{注释}

% 通用矩阵命令: \flexmatrix{矩阵名}{元素符号}{行数}{列数}
\newcommand{\flexmatrix}[4]{
  \[
  #1 = \begin{pmatrix}
    #2_{11}     & #2_{12}     & \cdots & #2_{1#4}   \\
    #2_{21}     & #2_{22}     & \cdots & #2_{2#4}   \\
    \vdots      & \vdots      & \ddots & \vdots     \\
    #2_{#31}    & #2_{#32}    & \cdots & #2_{#3#4}
  \end{pmatrix}
  \]
}

% 简化版命令(默认矩阵名为A,元素符号为a): \quickmatrix{行数}{列数}
\newcommand{\quickmatrix}[2]{\flexmatrix{A}{a}{#1}{#2}}


\begin{document}
\title{2.2 注释}
\author{张志聪}
\maketitle

\begin{zremark}
  问题2:

  求$K^n$内下面向量组(以行向量为例)的极大线性无关部分组:
  \begin{align*}
    \alpha_1, \alpha_2, \cdots, \alpha_m \ \ \ (I)
  \end{align*}

  操作方法:

  操作方式和例2.2相同,只是额外支持初等列变换,在进行该操作时,
  希腊字母表示的向量\textbf{不}跟着变,最终把$m \times n$矩阵化为阶梯型。

  试着完成该操作方法的理论支持部分。
\end{zremark}

\textbf{证明:}

矩阵$A$通过上述“操作方式”我们得到矩阵$B$,并的到一个极大线性无关部分组:
\begin{align}
  \alpha_{i_1}, \alpha_{i_2}, \cdots, \alpha_{i_r}
\end{align}
我们需要证明这确实是矩阵$A$的极大线性无关部分组。


\end{document}