\documentclass{article}
\usepackage{mathtools} 
\usepackage{fontspec}
\usepackage[UTF8]{ctex}
\usepackage{amsthm}
\usepackage{mdframed}
\usepackage{xcolor}
\usepackage{amssymb}
\usepackage{amsmath}
\usepackage{hyperref}
\usepackage{tikz}
\usetikzlibrary{shapes.geometric, arrows}
\usetikzlibrary{positioning}


% 定义新的带灰色背景的说明环境 zremark
\newmdtheoremenv[
  backgroundcolor=gray!10,
  % 边框与背景一致,边框线会消失
  linecolor=gray!10
]{zremark}{注释}

% 通用矩阵命令: \flexmatrix{矩阵名}{元素符号}{行数}{列数}
\newcommand{\flexmatrix}[4]{
  \[
  #1 = \begin{pmatrix}
    #2_{11}     & #2_{12}     & \cdots & #2_{1#4}   \\
    #2_{21}     & #2_{22}     & \cdots & #2_{2#4}   \\
    \vdots      & \vdots      & \ddots & \vdots     \\
    #2_{#31}    & #2_{#32}    & \cdots & #2_{#3#4}
  \end{pmatrix}
  \]
}

% 简化版命令(默认矩阵名为A,元素符号为a): \quickmatrix{行数}{列数}
\newcommand{\quickmatrix}[2]{\flexmatrix{A}{a}{#1}{#2}}


\begin{document}
\title{2.2 注释}
\author{张志聪}
\maketitle

\begin{zremark}
  $A$经过若干次初等列变换化为矩阵$B$,“推论”可以推广到$A$的行向量组。
\end{zremark}

\textbf{证明:}

设$A$的行向量组是$\alpha_1^T, \alpha_2^T, \cdots, \alpha_m^T$,$B$
的行向量组是$\beta_1^T, \beta_2^T, \cdots, \beta_m^T$。

考察$A^T$。$A$的行向量组是$A^T$的列向量组,对$A$做初等列变换等价于
对$A^T$做初等行变换。

结论(i):\\
如果$\alpha_{i1}^T, \alpha_{i2}^T, \cdots, \alpha_{ir}^T$是$A$的行向量组的一个极大线性
无关部分组,那么$\alpha_{i1}, \alpha_{i2}, \cdots, \alpha_{ir}$
是矩阵$A^T$的列向量组的一个极大线性无关部分组,
利用推理(i)可知,$\beta_{i1}, \beta_{i2}, \cdots, \beta_{ir}$
是$B^T$的列向量组的一个极大线性无关部分组,而且,当
\begin{align*}
  \alpha_i = k_1 \alpha_{i1} + k_2 \alpha_{i2} + \cdots + k_r \alpha_{ir}
\end{align*}
时,有$\beta_i = k_1 \beta_{i1} + k_2 \beta_{i2} + \cdots + k_r \beta_{ir}$,
于是
\begin{align*}
  \alpha_i^T = k_1 \alpha_{i1}^T + k_2 \alpha_{i2}^T + \cdots + k_r \alpha_{ir}^T
\end{align*}
时,有$\beta_i^T = k_1 \beta_{i1}^T + k_2 \beta_{i2}^T + \cdots + k_r \beta_{ir}^T$。

结论(ii): \\
证明类似,不做赘述。

\begin{zremark}
  问题2:

  求$K^n$内下面向量组(以行向量为例)的极大线性无关部分组:
  \begin{align*}
    \alpha_1, \alpha_2, \cdots, \alpha_m \ \ \ (I)
  \end{align*}

  操作方法:

  操作方式和例2.2相同,只是额外支持初等列变换,在进行该操作时,
  希腊字母表示的向量\textbf{不}跟着变,最终把$m \times n$矩阵化为阶梯型。

  试着完成该操作方法的理论支持部分。
\end{zremark}

\textbf{证明:}

矩阵$A$通过上述“操作方式”我们得到矩阵$B$,并的到一个极大线性无关部分组:
\begin{align}
  \alpha_{i_1}, \alpha_{i_2}, \cdots, \alpha_{i_r}
\end{align}
我们需要证明这确实是矩阵$A$的极大线性无关部分组。

\begin{zremark}
  标准型的唯一性。
\end{zremark}

\textbf{证明:}

任意$m \times n$矩阵$A$,设$A$的行秩为$r$,
通过初等变换化为两个标准型$D_1, D_2$,
如果$D_1 \neq D_2$,由于标准型的特殊性,只能是$D_1$与$D_2$
中$1$的个数不同,而$1$的个数决定了$D_1, D_2$的行秩,
由命题2.1和2.2可知,初等变换是不改变$A$的行秩的,
那么$D_1, D_2$的行秩都等于$A$的行秩,即
$D_1$与$D_2$中$1$的个数相等,
于是存在矛盾,所以必有$D_1 = D_2$。

\begin{zremark}
  任意矩阵$A$,$A$的行秩 = $A$的列秩,书中是如何保证的。
\end{zremark}
\textbf{如图:}

\begin{tikzpicture}[node distance=2cm, font=\small]
  % 定义样式
  \tikzstyle{condition} = [rectangle, draw, fill=blue!20, text centered, minimum height=1cm, minimum width=3cm]
  \tikzstyle{conclusion} = [rectangle, draw, fill=yellow!20, text centered, minimum height=1cm, minimum width=3cm]
  \tikzstyle{arrow} = [->, thick]

  % 结论节点在上方
  \node (concl) [conclusion] {$A$行秩 = $A$列秩};

  % 条件节点在下方一行排列
  \node (cond1) [condition, below left=2cm and 3cm of concl] {初等变换不改变行秩、列秩};
  \node (cond2) [condition, below=2cm of concl] {$A$经初等变换化为标准型$D$};
  \node (cond3) [condition, below right=2cm and 3cm of concl] {$D$行秩$=D$列秩};

  % 箭头连接:从条件指向结论
  \draw [arrow] (cond1) -- (concl);
  \draw [arrow] (cond2) -- (concl);
  \draw [arrow] (cond3) -- (concl);
\end{tikzpicture}

\begin{zremark}
  例2.2和例2.3思想的结合,
  求向量组的极大线性无关部分组。
\end{zremark}

\textbf{证明:}

\textbf{操作方式:}

以行向量组为例,设任意行向量组为
\begin{align*}
  \alpha_1^T, \alpha_2^T, \cdots, \alpha_n^T
\end{align*}
把向量组作为行排成一个矩阵$A$,
在做初等行变换时,如例2.2把该行向量的变换过程记录下来,
在做初等列变换时,无需做任何记录。

最后化为(行)阶梯型,剩余的操作与例2.2相同:
确定秩以及通过线性表示关系确定极大线性无关部分组。

注意:如果是列向量组,则最后要化为列形式的阶梯型,并记录下列变换过程,
行变换无需记录。

\textbf{理论支持:}

设$A$经过$n$次初等变换化为阶梯性$B$,其中
有$k$次列变换,$n - k$次行变换。

$k = 0$时,就是例2.2的情况。

$k = n$时,就是例2.3的情况。

$0 < k < n$时,由推论可知,即:初等列变换,
是不改变矩阵行向量组的线性表示关系的。

具体情况如下:
通过阶梯型$B$我们确定$A$的行秩,因为初等变换
对矩阵的行秩没有影响,这一点我们不做考虑。

通过线性表示关系,我们确定$A$的极大线性无关部分组,
又初等列变换,不改变矩阵行向量组的线性表示关系,
所以,不会对行向量组的极大线性无关部分组的确定产生影响。

\end{document}