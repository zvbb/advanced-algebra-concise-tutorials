\documentclass{article}
\usepackage{mathtools} 
\usepackage{fontspec}
\usepackage[UTF8]{ctex}
\usepackage{amsthm}
\usepackage{mdframed}
\usepackage{xcolor}
\usepackage{amssymb}
\usepackage{amsmath}
\usepackage{hyperref}


% 定义新的带灰色背景的说明环境 zremark
\newmdtheoremenv[
  backgroundcolor=gray!10,
  % 边框与背景一致,边框线会消失
  linecolor=gray!10
]{zremark}{注释}

% 通用矩阵命令: \flexmatrix{矩阵名}{元素符号}{行数}{列数}
\newcommand{\flexmatrix}[4]{
  \[
  #1 = \begin{pmatrix}
    #2_{11}     & #2_{12}     & \cdots & #2_{1#4}   \\
    #2_{21}     & #2_{22}     & \cdots & #2_{2#4}   \\
    \vdots      & \vdots      & \ddots & \vdots     \\
    #2_{#31}    & #2_{#32}    & \cdots & #2_{#3#4}
  \end{pmatrix}
  \]
}

% 简化版命令(默认矩阵名为A,元素符号为a): \quickmatrix{行数}{列数}
\newcommand{\quickmatrix}[2]{\flexmatrix{A}{a}{#1}{#2}}


\begin{document}
\title{5.3}
\author{张志聪}
\maketitle

\section*{3}

设二次型为$f(x_1, x_2, \cdots, x_n) = \sum_{i=1}^n \sum_{j=1}^n a_{ij}x_ix_j$。

\begin{itemize}
  \item 必要性

        设
        \begin{align*}
          f = (a_1 x_1 + a_2 x_2 + \cdots + a_n x_n) (b_1 x_1 + b_2 x_2 + \cdots + b_n x_n)
        \end{align*}

        接下来要做可逆的线性变数替换,而线性变数替换其实就是换基操作,
        所以我们要讨论两个一次多项式的线性关系。

        (i) $a_1 x_1 + a_2 x_2 + \cdots + a_n x_n, b_1 x_1 + b_2 x_2 + \cdots + b_n x_n$
        线性相关,即存在$k$使得
        \begin{align*}
          a_1 x_1 + a_2 x_2 + \cdots + a_n x_n = k(b_1 x_1 + b_2 x_2 + \cdots + b_n x_n)
        \end{align*}
        于是存在可逆矩阵$C$使得
        \begin{align*}
          Y = C X
        \end{align*}
        其中$C$的第一行为$b_1, b_2, \cdots, b_n$,即
        \begin{align*}
          y_1 = b_1 x_1 + b_2 x_2 + \cdots + b_n x_n
        \end{align*}
        (注意:这里的$y_2, y_3, \cdots, y_n$只是没有列出,因为它们在$f$中没有出现,具体的值不重要。)\\
        $\therefore f = k y_1^2$\\
        由规范性的性质可知(更严格的说,这里需要再进行一次可逆线性变数替换),$f$的秩为1。

        (ii) $a_1 x_1 + a_2 x_2 + \cdots + a_n x_n, b_1 x_1 + b_2 x_2 + \cdots + b_n x_n$
        线性无关。
        于是存在可逆矩阵$C$使得
        \begin{align*}
          Y = C X
        \end{align*}
        其中$C$的第一行为$a_1, a_2, \cdots, a_n$,第二行为$b_1, b_2, \cdots, b_n$,即
        \begin{align*}
          y_1 = a_1 x_1 + a_2 x_2 + \cdots + a_n x_n \\
          y_2 = b_1 x_1 + b_2 x_2 + \cdots + b_n x_n
        \end{align*}
        $\therefore f = y_1 y_2$。\\
        作可逆线性变数替换,
        \begin{align*}
          y_1 & = z_1 + z_2 \\
          y_2 & = z_1 - z_2 \\
          y_3 & = z_3       \\
              & \vdots      \\
          y_n & = z_n
        \end{align*}
        $\therefore f = z_1^2 - z_2^2$\\
        $\therefore f$的秩为2,符号差为0

  \item 充分性

        (i)$f$的秩为2,符号差为0,于是$f$可作可逆线性变数替换
        \begin{align*}
          Y = C X
        \end{align*}
        使得
        \begin{align*}
          f = y_1^2 - y_2^2
        \end{align*}
        (或符号相反)

        设$C$的第一行为$a_1, a_2, \cdots, a_n$,第二行为$b_1, b_2, \cdots, b_n$,即
        \begin{align*}
          y_1 = a_1 x_1 + a_2 x_2 + \cdots + a_n x_n \\
          y_2 = b_1 x_1 + b_2 x_2 + \cdots + b_n x_n
        \end{align*}
        于是可得
        \begin{align*}
          f & = (a_1 x_1 + a_2 x_2 + \cdots + a_n x_n)^2 - (b_1 x_1 + b_2 x_2 + \cdots + b_n x_n)^2                                         \\
            & = [(a_1 x_1 + \cdots + a_n x_n) + (b_1 x_1 + \cdots + b_n x_n)] [(a_1 x_1 + \cdots + a_n x_n) - (b_1 x_1 + \cdots + b_n x_n)]
        \end{align*}
        命题得证。

        (ii) $f$的秩为1,同理可证。

\end{itemize}

\section*{4}
设$f$通过可逆线性变数替换(矩阵为$C$)
\begin{align*}
  X = C Y
\end{align*}
把$f$化为规范形。

由题设可知,$f$的正负惯性指数都不为1。
反证法,假设$f$的正惯性指数为0,秩为$r$,
那么$f$通过可逆线性变数替换可得
\begin{align*}
  f = - u_1^2 - u_2^2 - \cdots - u_r^2
\end{align*}
这表明在$\mathbb{R}$上$n$维线性空间$V$内存在一组基$\eta_1, \eta_2, \cdots, \eta_n$,对任意
$\alpha \in V$,$\alpha = u_1 \eta_1 + u_2 \eta_2 + \cdots + u_n \eta_n$时
\begin{align*}
  Q_f(\alpha) = - u_1^2 - u_2^2 - \cdots - u_r^2 < 0
\end{align*}
与题设矛盾,故$f$的正惯性指数不为0。同理可证,$f$的正负惯性指数都不为0。

于是可设$f$的正惯性指数不为$p > 0$,负关系指数为$r - p > 0$,即
\begin{align*}
  f = u_1^2 + u_2^2 + \cdots + u_p^2 - u_{p+1}^2 - u_{p+2}^2 - \cdots - u_r^2
\end{align*}
设$Y_0 = (1, 0, 0, \cdots, 1, 0, \cdots, 0)$(第1个和第$p+1$个分量为1,其余为0),
此时
\begin{align*}
  f(Y_0) = 1 - 1 = 0
\end{align*}
于是,我们有
\begin{align*}
  f = f(CY_0) = (CY_0)^T A (CY_0) = 0
\end{align*}

\section*{5}

设$f$经过可逆线性变数替换(矩阵为$C$)
\begin{align*}
  Y = C X
\end{align*}
化作规范形,
\begin{align*}
  f(X) = g(Y) = y_1^2 + y_2^2 + \cdots + y_t^2 - y_{t+1}^2 - \cdots - y_{t+s}^2
\end{align*}

(i)假设$s > q$,考虑关于$x_1, x_2, \cdots, x_n$的方程组
\begin{equation*}
  \begin{cases}
    y_1 = row_{1}(C) X = 0 \\
    y_2 = row_{2}(C) X = 0 \\
    \vdots                 \\
    y_t = row_{t}(C) X = 0 \\
    l_{p+1} = 0            \\
    l_{p+2} = 0            \\
    \vdots                 \\
    l_{p+q} = 0
  \end{cases}
\end{equation*}

以上齐次线性方程组方程的个数为$t + q < t + s < n$,所以以上齐次线性方程组存在非零解,
设为\begin{align*}
  X_0 = \begin{bmatrix}
          k_1    \\
          k_2    \\
          \vdots \\
          k_n
        \end{bmatrix} \\
  Y_0 = C X_0
\end{align*}
由于$C$是可逆的,且$X_0 \neq 0$,所以$Y_0 \neq 0$,即存在$y_i \neq 0 (i > t)$,
于是我们有
\begin{align*}
  f(X_0) \geq 0 \\
  g(Y_0) < 0
\end{align*}
存在矛盾,故$s \leq q$。

(ii)假设$t > p$,考虑关于$x_1, x_2, \cdots, x_n$的方程组
\begin{equation*}
  \begin{cases}
    y_{t+1} = row_{t + 1}(C) X = 0 \\
    y_{t+2} = row_{t + 2}(C) X = 0 \\
    \vdots                         \\
    y_{t+s} = row_{t + s}(C) X = 0 \\
    l_{1} = 0                      \\
    l_{2} = 0                      \\
    \vdots                         \\
    l_{p} = 0
  \end{cases}
\end{equation*}

以上齐次线性方程组方程的个数为$s + p < s + t < n$,
所以以上齐次线性方程组存在非零解,
设为\begin{align*}
  X_0 = \begin{bmatrix}
          k_1    \\
          k_2    \\
          \vdots \\
          k_n
        \end{bmatrix} \\
  Y_0 = C X_0
\end{align*}
由于$C$是可逆的,且$X_0 \neq 0$,所以$Y_0 \neq 0$,即存在$y_i \neq 0 (i < t+1)$,
于是我们有
\begin{align*}
  f(X_0) \leq 0 \\
  g(Y_0) > 0
\end{align*}
存在矛盾,故$t \leq p$。

\section*{6}

(i)设$f$经过可逆线性变数替换(矩阵为$C$)化作规范形,因为负惯性指数为0,所以可表示为
\begin{align*}
  f(X) = g(Y) = y_1^2 + y_2^2 + \cdots + y_r^2
\end{align*}
其中$r$是$f$的秩。
这表明在$\mathbb{R}$上$n$维线性空间V内存在一组基
$\eta_1, \eta_2, \cdots, \eta_n$,使当$\alpha = y_1 \eta_1 + y_2 \eta_2 + \cdots + y_n \eta_n$时
\begin{align*}
  Q_f(\alpha) = y_1^2 + y_2^2 + \cdots + y_r^2
\end{align*}

任意$\alpha, \beta \in M$有$f(\alpha, \alpha) = 0$,
设
\begin{align*}
  \alpha = a_1 \eta_1 + a_2 \eta_2 + \cdots + a_n \eta_n \\
  \beta = b_1 \eta_1 + b_2 \eta_2 + \cdots + b_n \eta_n
\end{align*}
则有
\begin{align*}
  f(\alpha, \alpha) = a_1^2 + a_2^2 + \cdots + a_r^2 = 0 \\
  f(\beta, \beta) = b_1^2 + b_2^2 + \cdots + b_r^2 = 0
\end{align*}
可得
\begin{align*}
  a_1 = a_2 = \cdots = a_r = 0 \\
  b_1 = b_2 = \cdots = b_r = 0
\end{align*}
于是
\begin{align*}
  f(k\alpha + l\beta)
   & = (k a_1 + l b_1)^2 + (k a_2 + l b_2)^2 + \cdots + (k a_r + l b_r)^2 \\
   & = 0
\end{align*}
所以$k \alpha + l \beta \in M$,可得$M$是V的子空间。

(ii) 下面通过证明$M = L(\eta_{r+1}, \eta_{r+2}, \cdots, \eta_n)$确定$dim M$。

任意$\alpha \in M$,由(i)中的讨论可知,$\alpha \in L(\eta_{r+1}, \eta_{r+2}, \cdots, \eta_n)$,
所以$M \subseteq L(\eta_{r+1}, \eta_{r+2}, \cdots, \eta_n)$。

对任意$\alpha \in L(\eta_{r+1}, \eta_{r+2}, \cdots, \eta_n)$,有 
\begin{align*}
  \alpha = a_{r+1} \eta_{r+1} + a_{r+2} \eta_{r+2} + \cdots + a_n \eta_n
\end{align*}
于是
\begin{align*}
  f(\alpha) = 0
\end{align*}
可得$L(\eta_{r+1}, \eta_{r+2}, \cdots, \eta_n) \subseteq M$。

综上$M = L(\eta_{r+1}, \eta_{r+2}, \cdots, \eta_n)$,
所以
\begin{align*}
  dim M = dim L(\eta_{r+1}, \eta_{r+2}, \cdots, \eta_n) = n - r
\end{align*}

\section*{7}


\end{document}
