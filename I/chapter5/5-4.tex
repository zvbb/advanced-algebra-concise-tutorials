\documentclass{article}
\usepackage{mathtools} 
\usepackage{fontspec}
\usepackage[UTF8]{ctex}
\usepackage{amsthm}
\usepackage{mdframed}
\usepackage{xcolor}
\usepackage{amssymb}
\usepackage{amsmath}
\usepackage{hyperref}


% 定义新的带灰色背景的说明环境 zremark
\newmdtheoremenv[
  backgroundcolor=gray!10,
  % 边框与背景一致,边框线会消失
  linecolor=gray!10
]{zremark}{注释}

% 通用矩阵命令: \flexmatrix{矩阵名}{元素符号}{行数}{列数}
\newcommand{\flexmatrix}[4]{
  \[
  #1 = \begin{pmatrix}
    #2_{11}     & #2_{12}     & \cdots & #2_{1#4}   \\
    #2_{21}     & #2_{22}     & \cdots & #2_{2#4}   \\
    \vdots      & \vdots      & \ddots & \vdots     \\
    #2_{#31}    & #2_{#32}    & \cdots & #2_{#3#4}
  \end{pmatrix}
  \]
}

% 简化版命令(默认矩阵名为A,元素符号为a): \quickmatrix{行数}{列数}
\newcommand{\quickmatrix}[2]{\flexmatrix{A}{a}{#1}{#2}}


\begin{document}
\title{5.4}
\author{张志聪}
\maketitle

\section*{3}

解题思路:通过调整$C^T A C = B$,把主子式改为顺序主子式。

设$A$的$r$阶主子式为
\begin{align*}
  A
  \begin{Bmatrix}
    i_1, i_2, \cdots, i_r \\
    i_1, i_2, \cdots, i_r
  \end{Bmatrix}
\end{align*}
把$i_1$行,$i_1$列,调整到第一行,第一列,
通过左右乘上初阶矩阵完成:
\begin{align*}
  P_n(1, i_1)^T A P_n(1, i_1)
\end{align*}
因为$P_n(1, i_1)^T = P_n(1, i_1)$,且
\begin{align*}
  |P_n(1, i_1)| = - 1
\end{align*}
于是可得
\begin{align*}
  |P_n(1, i_1)^T A P_n(1, i_1)| = |P_n(1, i_1)| |A| |P_n(1, i_1)| = |A|
\end{align*}
所以,最多经过$r$次调整的到矩阵$B$,
由具体的操作可知$A, B$是合同的,且$|A| = |B|$,
所以$B$也是正定矩阵(合同关系是等价关系,等价关系具有传递性,利用命题4.1(ii)可得$B$是正定矩阵)。

至此,A中的$r$阶主子式会被调整为$B$中的顺序主子式,
但子式本身是不变的,由定理4.1可知,正定矩阵$B$的各阶顺序主子式都大于零,
命题得证。

\section*{4}

提示:$t$充分大时,$tE + A$是对角严格占优矩阵,
通过证明各阶顺序主子式大于零,来完成证明。$|tE_k - (- A_k)|$可以看做关于t的特征多项式,
通过$t \to +\infty$,可得$|tE_k - (- A_k)| > 0$,完成证明。

\section*{5}

因为$A$是正定矩阵,所有存在可逆矩阵$T$使得
\begin{align*}
  T^T A T = E                    \\
  (T^T A T)^{-1} = E^{-1}        \\
  T^{-1} A^{-1} (T^T)^{-1} = E   \\
  T^{-1} A^{-1} (T^{-1})^{T} = E \\
\end{align*}
因为$T^{-1}$是可逆矩阵,所以$A^{-1}, E$合同,
由命题4.1(ii)可知,$A^{-1}$是正定矩阵,

\section*{6}

由习题4可知,存在$c_1, c_2$使得
\begin{align*}
  c_1 E + A \\
  c_2 E - A
\end{align*}
都是正定矩阵,取$c = max(c_1, c_2)$,于是
\begin{align*}
  c E + A \\
  c E - A
\end{align*}
都是正定矩阵。

由正定矩阵的性质可得,对任意$X$,我们有
\begin{align*}
  X^T (c E + A) X \geq 0    \\
  cX^T E X + X^T A X \geq 0 \\
  cX^T  X \geq - X^T A X
\end{align*}
同理
\begin{align*}
  X^T (c E - A) X \geq 0 \\
  cX^T X \geq X^T A X
\end{align*}
综上,
\begin{align*}
  |X^T A X| \leq c X^T X
\end{align*}

\section*{7}

$|A| < 0$,由命题4.1的推论(逆否命题)可知,$A$不是正定矩阵,
于是命题4.1(iii)不成立,即存在$X \neq 0$使得
\begin{align*}
  X^T A X < 0
\end{align*}

\section*{8}

因为$A, B$是正定矩阵,于是对任意$X \neq 0$,都有
\begin{align*}
  X^T A X > 0 \\
  X^T B X > 0
\end{align*}
两式相加得
\begin{align*}
  X^T (A + B) X > 0
\end{align*}
所以,$A + B$是正定矩阵。

\section*{9}

提示:对$n$进行归纳,证明$f \geq 0$。

\section*{10}

二次型表示成矩阵乘积的形式:$f = X^T A X$,
因为$f$是正定二次型,所以$|A|>0$。

\begin{itemize}
  \item (1)

        设$A$的列向量组为$\alpha_1, \alpha_2, \cdots, \alpha_n$,
        $\beta = \begin{bmatrix}
            y_1    \\
            y_2    \\
            \vdots \\
            y_n
          \end{bmatrix}$
        因为$A$是满秩的,所以$\beta$可以被$\alpha_1, \alpha_2, \cdots, \alpha_n$线性表示,
        设为
        \begin{align*}
          \beta & = c_1 \alpha_1 + c_2 \alpha_2 + \cdots + c_n \alpha_n \\
          C     & = \begin{bmatrix}
                      c_1    \\
                      c_2    \\
                      \vdots \\
                      c_n
                    \end{bmatrix}                                      \\
          \beta & = A C
        \end{align*}
        于是,对任意$y_1, y_2, \cdots, y_n$,有
        \begin{align*}
          g(y_1, y_2, \cdots, y_n) & = \begin{vmatrix}
                                         \alpha_1 & \alpha_2 & \cdots & \alpha_n & \beta \\
                                         y_1      & y_2      & \cdots & y_n      & 0
                                       \end{vmatrix}                                  \\
                                   & =\begin{vmatrix}
                                        \alpha_1 & \alpha_2 & \cdots & \alpha_n & 0                                       \\
                                        y_1      & y_2      & \cdots & y_n      & -(c_1 y_1 + c_2 y_2 + \cdots + c_n y_n)
                                      \end{vmatrix} \\
                                   & = |A|[-(c_1 y_1 + c_2 y_2 + \cdots + c_n y_n)]                                     \\
                                   & = - |A| (C^T A C)
        \end{align*}
        因为$f$是正定二次型,所有$\beta = (y_1, y_2, \cdots, y_n)\neq 0$时,
        $C \neq 0$,$C^T A C > 0$,所以
        \begin{align*}
          - |A| (C^T A C) < 0
        \end{align*}
        可得$g$是负定二次型。

  \item (2)

        由行列式的性质,我们有
        \begin{align*}
          |A| & = \begin{vmatrix}
                    a_{11} & a_{12} & \cdots & a_{1n} \\
                    a_{21} & a_{22} & \cdots & a_{2n} \\
                    \vdots & \vdots & \ddots & \vdots \\
                    a_{n1} & a_{n2} & \cdots & a_{nn}
                  \end{vmatrix}                 \\
              & = \begin{vmatrix}
                    a_{11} & a_{12} & \cdots & a_{1n} \\
                    a_{21} & a_{22} & \cdots & a_{2n} \\
                    \vdots & \vdots & \ddots & \vdots \\
                    a_{n1} & a_{n2} & \cdots & 0
                  \end{vmatrix} + \begin{vmatrix}
                                    a_{11} & a_{12} & \cdots & 0      \\
                                    a_{21} & a_{22} & \cdots & 0      \\
                                    \vdots & \vdots & \ddots & \vdots \\
                                    a_{n1} & a_{n2} & \cdots & a_{nn}
                                  \end{vmatrix} \\
              & = \begin{vmatrix}
                    a_{11} & a_{12} & \cdots & a_{1n} \\
                    a_{21} & a_{22} & \cdots & a_{2n} \\
                    \vdots & \vdots & \ddots & \vdots \\
                    a_{n1} & a_{n2} & \cdots & 0
                  \end{vmatrix}
          + a_{nn} \cdot P_{n - 1}
        \end{align*}
        利用(1)可知
        \begin{align*}
          \begin{vmatrix}
            a_{11} & a_{12} & \cdots & a_{1n} \\
            a_{21} & a_{22} & \cdots & a_{2n} \\
            \vdots & \vdots & \ddots & \vdots \\
            a_{n1} & a_{n2} & \cdots & 0
          \end{vmatrix} \leq 0
        \end{align*}
        所以
        \begin{align*}
          |A| \leq a_{nn} \cdot P_{n - 1}
        \end{align*}

  \item (3)

        多次利用(2):
        \begin{align*}
          |A| \leq a_{nn} \cdot P_{n - 1} \leq a_{nn}  a_{n-1 \  n-1} P_{n - 2} \leq \cdots \leq a_{nn} \cdots a_{22} a_{11}
        \end{align*}

  \item (4)

        因为$T$是可逆的,所以对任意$x \neq 0$,我们有
        \begin{align*}
          x^T T^T T x = (Tx)^2 (T x) > 0
        \end{align*}
        所以,$T^T T$是正定矩阵。

        又
        \begin{align*}
          T^T T = \begin{bmatrix}
                    t_{11}^2 + t_{21}^2 + \cdots + t_{n1}^2 &                                                                                            \\
                                                            & t_{12}^2 + t_{22}^2 + \cdots + t_{n2}^2 &                                                  \\
                                                            &                                         & \ddots                                           \\
                                                            &                                         &        & t_{1n}^2 + t_{2n}^2 + \cdots + t_{nn}^2
                  \end{bmatrix}
        \end{align*}
        其中未写的部分是任意值。

        由(2)可知,
        \begin{align*}
          |T^T T| = |T|^2 \leq \prod\limits_{i = 1}^n (t_{1i}^2 + t_{2i}^2 + \cdots + t_{ni}^2)
        \end{align*}
\end{itemize}

\section*{11}

设$\mathbb{R}$上n维线性空间V上的对称双线性函数$f(\alpha, \beta)$的二次型函数$Q_f(\alpha)$
在基$\epsilon_1, \epsilon_2, \cdots, \epsilon_n$下解析式表达式为
$f$,矩阵为$A$。

\begin{itemize}
  \item 必要性

        令$M = L(\epsilon_{i_1}, \epsilon_{i_2}, \cdots, \epsilon_{i_k})$。把$f(\alpha, \beta)$限制在
        $M$内,在$M$的基$\epsilon_{i_1}, \epsilon_{i_2}, \cdots, \epsilon_{i_k}$下它的矩阵
        \begin{align}
          A_k = \begin{bmatrix}
                  a_{i_1 \ i_1} & a_{i_1 \ i_2} & \cdots & a_{i_1 \ i_k} \\
                  a_{i_2 \ i_1} & a_{i_2 \ i_2} & \cdots & a_{i_2 \ i_k} \\
                  \vdots        & \vdots        & \ddots & \vdots        \\
                  a_{i_k \ i_1} & a_{i_k \ i_2} & \cdots & a_{i_k \ i_k}
                \end{bmatrix}
        \end{align}
        任意$\alpha \in M$(设坐标为$X$)即$\alpha \in M$,由于$f$是半正定二次型,所以$Q_f(\alpha) = X^T A_k X > 0$,
        所以限制在$M$内的$f$也是半正定二次型,于是根据半正定二次型的根据定义,$A_k$在$\mathbb{R}$内合同于标准矩阵$D$。
        所以,存在可逆矩阵$T$使得
        \begin{align}
          |A_k| & = |T^T D T|     \\
                & = |T^T| |D| |T| \\
                & = |D| |T|^2     \\
                & \geq 0
        \end{align}

  \item 充分性

        todo

        $A$所有主子式包括$|A|$,所以
        \begin{align*}
          |A| \geq 0
        \end{align*}

\end{itemize}

\section*{12}

\begin{itemize}
  \item (1)

        $f, g$的二次型矩阵分别为:
        \begin{align*}
          A = \begin{bmatrix}
                -1 & 0 \\
                0  & 2
              \end{bmatrix} \\
          B = \begin{bmatrix}
                2 & 0  \\
                0 & -1
              \end{bmatrix}
        \end{align*}
        而$f + g$的矩阵为
        \begin{align*}
          C & = A + B \\
            & = E
        \end{align*}

  \item (2)

        设$f$在基$\epsilon_1, \epsilon_2, \cdots, \epsilon_n$下为规范型
        \begin{align*}
          u_1^2 + \cdots + u_{p_{1}}^2 - u_{p_{1} + 1}^2 - \cdots - u_{r_1}^2
        \end{align*}

        $g$在基$\eta_1, \eta_2, \cdots, \eta_n$下为规范型
        \begin{align*}
          w_1^2 + \cdots + w_{p_{2}}^2 - w_{p_{2} + 1}^2 - \cdots - w_{r_2}^2
        \end{align*}

        由题设可知$p_1, p_2 < \frac{n}{2}$,
        所以$\epsilon_{p_1 + 1}, \cdots, \epsilon_n$与$\eta_1, \eta_2, \cdots, \eta_{p_2}$不会线性等价(向量数不相同),
        即存在$\epsilon_i (i > p_1)$,无法被$\eta_1, \eta_2, \cdots, \eta_{p_2}$不会线性表示,
        我们有
        \begin{align}
          f(\epsilon_i) \leq 0 \\
          g(\epsilon_i) \leq 0
        \end{align}
        所以
        \begin{align}
          (f + g)(\epsilon_i) \leq 0
        \end{align}
        所以,$f + g$一定不是正定二次型。
\end{itemize}



\end{document}
