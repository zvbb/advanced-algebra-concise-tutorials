\documentclass{article}
\usepackage{mathtools} 
\usepackage{fontspec}
\usepackage[UTF8]{ctex}
\usepackage{amsthm}
\usepackage{mdframed}
\usepackage{xcolor}
\usepackage{amssymb}
\usepackage{amsmath}
\usepackage{hyperref}


% 定义新的带灰色背景的说明环境 zremark
\newmdtheoremenv[
  backgroundcolor=gray!10,
  % 边框与背景一致,边框线会消失
  linecolor=gray!10
]{zremark}{注释}

% 通用矩阵命令: \flexmatrix{矩阵名}{元素符号}{行数}{列数}
\newcommand{\flexmatrix}[4]{
  \[
  #1 = \begin{pmatrix}
    #2_{11}     & #2_{12}     & \cdots & #2_{1#4}   \\
    #2_{21}     & #2_{22}     & \cdots & #2_{2#4}   \\
    \vdots      & \vdots      & \ddots & \vdots     \\
    #2_{#31}    & #2_{#32}    & \cdots & #2_{#3#4}
  \end{pmatrix}
  \]
}

% 简化版命令(默认矩阵名为A,元素符号为a): \quickmatrix{行数}{列数}
\newcommand{\quickmatrix}[2]{\flexmatrix{A}{a}{#1}{#2}}


\begin{document}
\title{5.1}
\author{张志聪}
\maketitle

\section*{9}

提示:反证法可能更方便。考虑$Ax = 0$解的情况。

把$A$看做$K$上n维线性空间$V$内双线性函数$f(\alpha, \beta)$在
基$\epsilon_1, \epsilon_2, \cdots, \epsilon_n$下的矩阵。

\begin{itemize}
  \item 必要性

        反证法,假设存在$\alpha \neq 0$,对一切$\beta \in V$有
        $f(\alpha, \beta) = 0$。
        不妨设$\alpha, \beta$在$\epsilon_1, \epsilon_2, \cdots, \epsilon_n$的基下的坐标为
        $X, Y$,且$Y \neq 0$,使得
        \begin{align*}
          f(\alpha, \beta) = X^T A Y = 0
        \end{align*}
        因为$f(\alpha, \beta)$满秩,即$A$是满秩的,
        于是$Ax = 0$没有非零解,所以$A Y \neq 0$,
        所以$\alpha = 0$,
        存在矛盾。

  \item 充分性

        反证法,假设$f(\alpha, \beta)$不满秩,即$A$不满秩,
        于是$A$的行向量组是线性相关的,即存在$X^T \neq 0$使得
        \begin{align*}
          X^T A = 0
        \end{align*}
        此时$\forall Y \in \mathbb{K}^n$有
        \begin{align*}
          X^T A Y = 0
        \end{align*}
        与题设矛盾。
\end{itemize}

\section*{10}

利用习题9完成证明。

证明:对任意$B \in M_n(K)$都有
\begin{align*}
  f(A, B) = Tr(AB) = 0
\end{align*}
则必有$A = 0$。

设$B = E_{ij} (1 \leq i, j \leq n)$,$A = (a_{ij})$
于是
\begin{align*}
  f(A, B) & = Tr(A E_{ij})
\end{align*}
其中$A E_{ij}$的第$j$列 $ = $ 矩阵$A$的第$i$列,其余列全是0。
于是在主对角线上,只有$A E_{ij}$的$j$行,$j$列可能有值,即
\begin{align*}
  Tr(A E_{ij}) = a_{ji}
\end{align*}
因为
\begin{align*}
  Tr(A E_{ij}) = 0
\end{align*}
所以$a_{ji} = 0$。

综上,$A = 0$。

\section*{13}

我们可以通过初等矩阵,把对角线上的元素进行调整。

任意$1 \leq i, j \leq n$,希望互换对角线上第$i$个元素和第$j$个元素。
设
\begin{align*}
  A = \begin{bmatrix}
        \lambda_1                         \\
         & \lambda_2                      \\
         &           & \ddots             \\
         &           &        & \lambda_n
      \end{bmatrix} \\
  B = \begin{bmatrix}
        \lambda_{i_1}                             \\
         & \lambda_{i_2}                          \\
         &               & \ddots                 \\
         &               &        & \lambda_{i_n}
      \end{bmatrix}
\end{align*}
于是
\begin{align*}
  (P_n(i, j))^T A P_n(i, j)
   & = P_n(i, j) A P_n(i, j)                                                       \\
   & = \begin{bmatrix}
         \lambda_1                                                                   \\
          & \lambda_2                                                                \\
          &           & \ddots                                                       \\
          &           &        & \lambda_j                                           \\
          &           &        &           & \ddots                                  \\
          &           &        &           &        & \lambda_i                      \\
          &           &        &           &        &           & \ddots             \\
          &           &        &           &        &           &        & \lambda_n \\
       \end{bmatrix}
\end{align*}
即交换了$i, j$行和$i, j$列。
所以,存在$T$使得
\begin{align*}
  B = T^T A T
\end{align*}
所以,$A, B$合同。

\section*{14}

 (1)
\begin{itemize}
  \item 必要性

        因为$A$是反对称矩阵,
        所以
        \begin{align*}
          A^T = -A
        \end{align*}
        于是,对任意$n$维列向量$x$,有
        \begin{align*}
          (x^T A x)^T
           & = x^T A^T x  \\
           & = x^T (-A) x \\
           & = -x^T A x
        \end{align*}
        因为$x^T A x$是标量,所以
        \begin{align*}
          (x^T A x)^T & = x^T A x \\
          -x^T A x    & = x^T A x \\
          2 x^T A x   & = 0       \\
          x^T A x     & = 0
        \end{align*}

  \item 充分性

        对$n$维列向量$x + y$,我们有
        \begin{align*}
          (x + y)^T A (x + y)
           & = (x^T + y^T) A (x + y)                 \\
           & = (x^T A + y^T A)(x + y)                \\
           & = x^T A x + x^T A y + y^T A x + y^T A y \\
           & = 0 + x^T A y + y^T A x + 0             \\
           & = x^T A y + y^T A x
        \end{align*}

        令列项量$x = e_i$(第$i$个分量为1,其他分量为0),$y = e_j$,
        设$A = (a_{ij})$,于是
        \begin{align*}
          e_i^T A e_j + e_j^T A e_i = 0 \\
          a_{ij} + a_{ji} = 0           \\
          a_{ij} = -a_{ji}
        \end{align*}
        所以,$A$是反对称矩阵。

\end{itemize}

(2)

利用(1)可知,$A$是反对称矩阵,另外由题设可知$A$是对称矩阵,
所以,$A = 0$。


\section*{17}

对$V$的维数进行归纳。

$n = 1$时,$f(\alpha, \beta)$在任意基下都是0,命题成立。

归纳假设$< n$时,命题成立。

$f(\alpha, \beta) \equiv 0$,命题显然成立;
$f(\alpha, \beta) \not \equiv 0$,于是存在$d = f(\epsilon_1, \epsilon_2) \neq 0$,
其中$\epsilon_1, \epsilon_2 \neq 0$(按照双线性函数的定义总有$f(\epsilon_1, 0) = 0, f(0, \epsilon_2) = 0$)。

于是,把$\epsilon_1, \epsilon_2 $,扩充成$V$的一组基
\begin{align*}
  \frac{1}{d} \epsilon_1, \epsilon_2, \epsilon_3, \cdots, \epsilon_n
\end{align*}
于是,$f(\alpha, \beta)$在这组基下的矩阵形如:
\begin{align*}
  \begin{bmatrix}
    0   & 1      & * \\
    - 1 & 0      & * \\
    *   & \cdots & *
  \end{bmatrix}
\end{align*}
即只能确定左上角的$2 \times 2$部分,和主对角线是0。

接下来,通过调整基,把前两行和前两列相关元素置为0。
令
\begin{equation*}
  \begin{cases*}
    \eta_1 = \frac{1}{d} \epsilon_1 \\
    \eta_2 = \epsilon_2             \\
    \eta_i = \epsilon_i - f(\frac{1}{d} \epsilon_1, \epsilon_2) \epsilon_i
  \end{cases*}
\end{equation*}
显然,$\eta_1, \eta_2, \cdots, \eta_n$是$V$的一组基,且
\begin{align*}
  f(\eta_1, \eta_i)
   & = f(\frac{1}{d} \epsilon_1, \epsilon_i - f(\frac{1}{d} \epsilon_1, \epsilon_2) \epsilon_i)                            \\
   & = f(\frac{1}{d} \epsilon_1, \epsilon_i) - f(\frac{1}{d} \epsilon_1, \epsilon_2) f(\frac{1}{d} \epsilon_1, \epsilon_i) \\
   & = f(\frac{1}{d} \epsilon_1, \epsilon_i) - 1 f(\frac{1}{d} \epsilon_1, \epsilon_i)                                     \\
   & = 0
\end{align*}
通过换基,此时的矩阵形如:
\begin{align*}
  \begin{bmatrix}
    0      & 1      & 0      & \cdots & 0 \\
    - 1    & 0      & *      & \cdots & * \\
    0      & *      & *      & \cdots & * \\
    \vdots & \vdots & \vdots & \cdots & * \\
    0      & *      & *      & \cdots & *
  \end{bmatrix}
\end{align*}
注意:由于$f$是反对称双线性函数,在任意基下的矩阵都是对称矩阵。

再次换基,把第二行和第二列相关元素置为0。
令
\begin{equation*}
  \begin{cases*}
    \eta_1' = \eta_1 \\
    \eta_2' = \eta_2 \\
    \eta_i' = \eta_i + f(\eta_2, \eta_1) \eta_i
  \end{cases*}
\end{equation*}
显然,$\eta_1', \eta_2', \cdots, \eta_n'$是$V$的一组基,且
\begin{align*}
  f(\eta_1', \eta_i')
   & = f(\eta_1, \eta_i + f(\eta_2, \eta_1) \eta_i)            \\
   & = f(\eta_1, \eta_i) + f(\eta_2, \eta_1) f(\eta_1, \eta_i) \\
   & = f(\eta_1, \eta_i) - 1 \cdot f(\eta_1, \eta_i)           \\
   & = 0
\end{align*}
通过换基,此时的矩阵形如:
\begin{align*}
  \begin{bmatrix}
    0      & 1      & 0      & \cdots & 0 \\
    - 1    & 0      & 0      & \cdots & 0 \\
    0      & 0      & *      & \cdots & * \\
    \vdots & \vdots & \vdots & \cdots & * \\
    0      & 0      & *      & \cdots & *
  \end{bmatrix}
\end{align*}
由归纳假设可知,右下角部分可以化为准对角形。

归纳完成。

\section*{18}

\begin{itemize}
  \item (1) $L(M), R(M)$是V的子空间;

        略

  \item (2) $dim L(M) = dim R(M) = n - dim$;

        $L(M)$与$R(M)$是对称的,我们以$R(M)$为例。

        取$M$的一组基$\epsilon_1, \epsilon_2, \cdots, \epsilon_r$,并扩充成$V$的一组基
        $\epsilon_1, \epsilon_2, \cdots, \epsilon_n$。
        设$f(\alpha, \beta)$在这组基下的矩阵为$A$,由题设可知$A$是满秩矩阵。

        考虑$R(M)$的定义,$\forall \alpha \in R(M)$
        当且仅当$f(\epsilon_i, \alpha) = 0 \ (i = 1, 2, \cdots, r)$,
        当且仅当
        \begin{align*}
          crd(\epsilon_i)^T A crd(\alpha) = 0 \\
          e_i^T A crd(\alpha) = 0             \\
          row_i(A)^T crd(\alpha) = 0
        \end{align*}
        因为$i = 1, 2, \cdots, r$,所以
        就是$A$的前$r$行矩阵$A_r$,使得
        \begin{align*}
          A_r crd(\alpha) = 0
        \end{align*}
        因为$A$是满秩的,所以$rank(A_r) = r$,于是可得
        $A_r x = 0$这个线性方程组的解空间是$n - r$维的,
        因为取坐标会保证坐标与$V$中向量同构,所以$dim R(M) = n - r$,即
        \begin{align*}
          dim R(M) = n - r = n - dim M
        \end{align*}

  \item (3) $R(L(M)) = L(R(M)) = M$。

        以$L(R(M))$为例。

        (i)$M \subseteq L(R(M))$。

        因为$R(M)$是子空间,所以$R(M) \neq \varnothing$,
        所以,存在$\beta \in R(M)$,由$L(R(M))$的定义可知,
        任意$\alpha \in M$,都有
        \begin{align*}
          \alpha \in L(R(M))
        \end{align*}
        所以,$M \subseteq L(R(M))$。

        (ii)利用(2)的结论:
        \begin{align*}
          dim L(R(M)) = n - dim R(M) = n - (n - dim M) = dim M
        \end{align*}
        综上,$M = L(R(M))$。
\end{itemize}

\section*{19}

\begin{itemize}
  \item (1) $L(M + N) = L(M) \cap L(N), R(M + N) = R(M) \cap R(N)$。

        以$L(M + N) = L(M) \cap L(N)$为例。

        (i) $L(M + N) \subseteq L(M) \cap L(N)$;

        任意$\alpha \in L(M + N)$,即$\forall \beta_1 + \beta_2 \in M + N (\beta_1 \in M, \beta_2 \in N)$,有
        \begin{align*}
          f(\alpha, \beta_1 + \beta_2) = 0
        \end{align*}
        于是,令$\beta_2 = 0$,可得$\forall \beta_1 \in M$有
        \begin{align*}
          f(\alpha, \beta_1) = 0
        \end{align*}
        即$\alpha \in L(M)$,同理可证$\alpha \in L(N)$。
        所以$L(M + N) \subseteq L(M) \cap L(N)$。

        (ii) $L(M) \cap L(N) \subseteq L(M + N)$。

        $\alpha \in L(M) \cap L(N)$,即
        $\forall \beta_1 \in M, \forall \beta_2 \in N$有
        \begin{equation*}
          \begin{cases*}
            f(\alpha, \beta_1) = 0 \\
            f(\alpha, \beta_2) = 0
          \end{cases*}
        \end{equation*}
        因为$f$是双线性函数,所以
        \begin{align*}
          f(\alpha, \beta_1) + f(\alpha, \beta_2)
           & = f(\alpha, \beta_1 + \beta_2) \\
           & = 0
        \end{align*}
        所以,$\alpha \in L(M + N)$。

        综上,$L(M + N) = L(M) \cap L(N)$。

  \item (2)$f(\alpha, \beta)$满秩,则$L(M \cap N) = L(M) + L(N), R(M \cap N) = R(M) + R(N)$。

        (i)任意$\alpha \in L(M) + L(N)$,设
        \begin{align*}
          \alpha = \alpha_M + \alpha_N
        \end{align*}
        于是有
        \begin{equation*}
          \begin{cases*}
            f(\alpha_M, \beta_1) = 0, \ \forall \beta_1 \in M \\
            f(\alpha_N, \beta_2) = 0, \ \forall \beta_2 \in N
          \end{cases*}
        \end{equation*}
        所以对$\forall \beta \in M \cap N$,我们有$\beta \in M, \beta \in N$,
        并由$f(\alpha, \beta)$是双线性函数,两式相加得
        \begin{align*}
          f(\alpha_M, \beta) + f(\alpha_N, \beta)
           & = f(\alpha_M + \alpha_N, \beta) \\
           & = f(\alpha, \beta)
        \end{align*}
        所以,$\alpha \in L(M \cap N)$,即
        \begin{align*}
          L(M) + L(N) \subseteq L(M \cap N)
        \end{align*}

        (ii)由习题18可知,
        \begin{align*}
          dim L(M \cap N) & = n - dim M \cap N \\
        \end{align*}
        利用维数公式和已知的等价关系
        \begin{align*}
          dim L (M) + L(N)
           & = dim L(M) + dim L(N) - dim L (M) \cap L(N)                    \\
           & = dim L(M) + dim L(N) - dim L(M + N)                           \\
           & = n - dim M + n - dim N - (n - dim M + N)                      \\
           & = n - dim M + n - dim N - [n - (dim M + dim N - dim M \cap N)] \\
           & = n -  dim M \cap N
        \end{align*}

        综上,由包含关系和维数相等可知,
        \begin{align*}
          L(M) + L(N) = L(M \cap N)
        \end{align*}
\end{itemize}

\section*{20}

反证法,假设$f(\alpha) \not \equiv 0, g(\alpha) \not \equiv 0$,
即存在$\alpha, \beta \in V$,使得
\begin{align*}
  f(\alpha) \neq 0 \\
  g(\beta) \neq 0
\end{align*}
由题设,我们有
\begin{align*}
  f(\alpha ) g(\alpha) = 0
\end{align*}
所以$g(\alpha) = 0$,同理,$f(\beta) = 0$。

又
\begin{align*}
  f(\alpha + \beta) g(\alpha + \beta)
   & = [f(\alpha) + f(\beta)][g(\alpha) + g(\beta)]                                      \\
   & = f(\alpha) g(\alpha) + f(\alpha) g(\beta) + f(\beta) g(\alpha) + f(\beta) g(\beta) \\
   & = f(\alpha) g(\beta) + f(\beta) g(\alpha)                                           \\
   & = f(\alpha) g(\beta)                                                                \\
   & \neq 0
\end{align*}
与题设$f(\alpha + \beta) g(\alpha + \beta) = 0$矛盾。

\section*{21}

解题思路:通过一个点,确定比例常数。

 (i)$g \equiv 0$,$f(\alpha, \beta) \equiv 0$,于是令
\begin{align*}
  \lambda = 0 \\
  l \equiv 0
\end{align*}
即可完成命题要求。

(ii)$g \not \equiv 0$,于是$\exists v \in V, g(v) \neq 0$,
由题设可知
\begin{align*}
  f(v, \beta) = g(v) h(\beta)
\end{align*}
由$f$是对称双线性函数可知
\begin{align*}
  f(v, \beta) & = f(\beta, v)   \\
              & = g(\beta) h(v)
\end{align*}
所以
\begin{align*}
  g(v) h(\beta) = g(\beta) h(v) \\
  h(\beta) = \frac{h(v)}{g(v)} g(\beta)
\end{align*}
令
\begin{align*}
  \lambda & = \frac{h(v)}{g(v)} \\
  l       & = \lambda g
\end{align*}
于是,
\begin{align*}
  f(\alpha, \beta)
   & = g(\alpha) h(\beta)                   \\
   & = g(\alpha) \frac{h(v)}{g(v)} g(\beta) \\
   & = \frac{h(v)}{g(v)} g(\alpha) g(\beta) \\
   & = \lambda g(\alpha) g(\beta)
\end{align*}


\end{document}
