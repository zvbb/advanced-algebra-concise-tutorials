\documentclass{article}
\usepackage{mathtools} 
\usepackage{fontspec}
\usepackage[UTF8]{ctex}
\usepackage{amsthm}
\usepackage{mdframed}
\usepackage{xcolor}
\usepackage{amssymb}
\usepackage{amsmath}
\usepackage{hyperref}


% 定义新的带灰色背景的说明环境 zremark
\newmdtheoremenv[
  backgroundcolor=gray!10,
  % 边框与背景一致,边框线会消失
  linecolor=gray!10
]{zremark}{注释}

% 通用矩阵命令: \flexmatrix{矩阵名}{元素符号}{行数}{列数}
\newcommand{\flexmatrix}[4]{
  \[
  #1 = \begin{pmatrix}
    #2_{11}     & #2_{12}     & \cdots & #2_{1#4}   \\
    #2_{21}     & #2_{22}     & \cdots & #2_{2#4}   \\
    \vdots      & \vdots      & \ddots & \vdots     \\
    #2_{#31}    & #2_{#32}    & \cdots & #2_{#3#4}
  \end{pmatrix}
  \]
}

% 简化版命令(默认矩阵名为A,元素符号为a): \quickmatrix{行数}{列数}
\newcommand{\quickmatrix}[2]{\flexmatrix{A}{a}{#1}{#2}}


\begin{document}
\title{5.2}
\author{张志聪}
\maketitle

\section*{6}

设$V$对称双线性函数$f(\alpha, \beta)$在基$\epsilon_1, \epsilon_2, \cdots, \epsilon_n$下的矩阵为
$A$,由定理1.1可知,$V$内存在一组基,使$f(\alpha, \beta)$在这组基下的矩阵成对角型,设对角矩阵为$D$,
因为$A,D$是同一个双线性函数$f(\alpha, \beta)$在两组基下的矩阵,所以存在可逆的n阶方阵$T$,使得
\begin{align*}
  A = T^{T} D T
\end{align*}
由题设可知$A$的秩为$r$,而$T$是可逆的矩阵,所以$D$的秩也是$r$,
于是可得$D$在主对角线上又$r$个非零元素,为了讨论的方便,不妨设主对角线中前$r$个元素不为零,
于是写成单个非零对角元的和:
\begin{align*}
  D = \sum \limits_{j = 1}^r d_j E_{jj}
\end{align*}
于是
\begin{align*}
  A & = T^{T} D T                                 \\
    & = T^T (\sum \limits_{j = 1}^r d_j E_{jj}) T \\
    & = \sum \limits_{j = 1}^r T^T (d_j E_{jj}) T
\end{align*}
显然,$M_j = d_j E_{jj}$是秩等于1的对角矩阵,且有
\begin{align*}
  (T^T M_j T)^T
   & = T^T M_j^T T \\
   & = T^T M_j T
\end{align*}
所以$T^T M_j T$是对称矩阵。

综上可得:秩等于$r$的对称矩阵可以表成$r$个秩等于1的对称矩阵之和。

\section*{7}

设二次型
\begin{align*}
  g = y_1^2 + y_2^2 + \cdots + y_n^2
\end{align*}
其系数矩阵为
\begin{align*}
  E_n = \begin{bmatrix}
          1                 \\
           & 1              \\
           &   & \ddots     \\
           &   &        & 1
        \end{bmatrix}
\end{align*}
于是$g$写成矩阵形式为
\begin{align*}
  g = Y^T E_n Y
\end{align*}
令
\begin{align*}
  X = \begin{bmatrix}
        x_1    \\
        x_2    \\
        \vdots \\
        x_n
      \end{bmatrix},
  Y = \begin{bmatrix}
        y_1    \\
        y_2    \\
        \vdots \\
        y_n
      \end{bmatrix},
\end{align*}
于是,有
\begin{align*}
  Y = A X
\end{align*}
则代入得
\begin{align*}
  g & = Y^T E_n Y         \\
    & = (A X)^T E_n (A X) \\
    & = X^T A^T E_n A X   \\
    & = X^T (A^T A) X     \\
    & = f
\end{align*}
于是可得$A^T A$是$f$的秩,又因为
\begin{align*}
  rank(A) = rank(A^T A)
\end{align*}
命题得证。

\section*{9}

\begin{itemize}
  \item (1)

        \begin{itemize}
          \item 必要性

                $g$写成矩阵形式为
                \begin{align*}
                  g = Y^T D Y
                \end{align*}
                其中
                \begin{align*}
                  D = \begin{bmatrix}
                        \lambda_1                                          \\
                         & \lambda_2                                       \\
                         &           & \ddots                              \\
                         &           &        & \lambda_r                  \\
                         &           &        &           & 0              \\
                         &           &        &           &   & \ddots     \\
                         &           &        &           &   &        & 0 \\
                      \end{bmatrix}
                \end{align*}

                由习题8可知
                \begin{align*}
                  D_k = A \begin{Bmatrix}
                            1 & 2 & \cdots & k \\
                            1 & 2 & \cdots & k
                          \end{Bmatrix}
                   & = D \begin{Bmatrix}
                           1 & 2 & \cdots & k \\
                           1 & 2 & \cdots & k
                         \end{Bmatrix}                 \\
                   & = \lambda_1 \lambda_2 \cdots \lambda_k
                  \ \ \ (k = 1, 2, \cdots, r)
                \end{align*}
                又因为$\lambda_i \neq 0$,所以
                \begin{align*}
                  D_k \neq 0
                \end{align*}

                进而可知
                \begin{align*}
                  A \begin{Bmatrix}
                      1 & 2 & \cdots & k \\
                      1 & 2 & \cdots & k
                    \end{Bmatrix}
                  = 0 \ \ \ (k = r + 1, r + 2, \cdots, n)
                \end{align*}

          \item 充分性

                % 由题设可知$rank(A) = r$,
                % 于是可经过可逆线性变数替换$Y = T X$,化作规范型
                % \begin{align*}
                %   u_1^2 + u_2^2 + \cdots + u_r^2
                % \end{align*}
                % $T$是满秩的,可以秩通过初等列变换$Q$化为$E_n$,于是
                % \begin{align*}
                %   Y & = T X            \\
                %     & = (E_n Q^{-1}) X
                % \end{align*}
                todo


        \end{itemize}
  \item (2)

        利用(1)
        \begin{align*}
          \lambda_k & = \frac{D \begin{Bmatrix}
                                    1 & 2 & \cdots & k \\
                                    1 & 2 & \cdots & k
                                  \end{Bmatrix}}{D \begin{Bmatrix}
                                                     1 & 2 & \cdots & k-1 \\
                                                     1 & 2 & \cdots & k-1
                                                   \end{Bmatrix}} \\
                    & =\frac{D_k}{D_{k-1}}
        \end{align*}
\end{itemize}


\end{document}
