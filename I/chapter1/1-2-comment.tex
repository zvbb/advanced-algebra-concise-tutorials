\documentclass{article}
\usepackage{mathtools} 
\usepackage{fontspec}
\usepackage[UTF8]{ctex}
\usepackage{amsthm}
\usepackage{mdframed}
\usepackage{xcolor}
\usepackage{amssymb}
\usepackage{amsmath}


% 定义新的带灰色背景的说明环境 zremark
\newmdtheoremenv[
  backgroundcolor=gray!10,
  % 边框与背景一致,边框线会消失
  linecolor=gray!10
]{zremark}{注释}

% 通用矩阵命令: \flexmatrix{矩阵名}{元素符号}{行数}{列数}
\newcommand{\flexmatrix}[4]{
  \[
  #1 = \begin{pmatrix}
    #2_{11}     & #2_{12}     & \cdots & #2_{1#4}   \\
    #2_{21}     & #2_{22}     & \cdots & #2_{2#4}   \\
    \vdots      & \vdots      & \ddots & \vdots     \\
    #2_{#31}    & #2_{#32}    & \cdots & #2_{#3#4}
  \end{pmatrix}
  \]
}

% 简化版命令(默认矩阵名为A,元素符号为a): \quickmatrix{行数}{列数}
\newcommand{\quickmatrix}[2]{\flexmatrix{A}{a}{#1}{#2}}


\begin{document}
\title{1.2 注释}
\author{张志聪}
\maketitle

\begin{zremark}
      文中:在数域$K$上的一元一次方程
      \begin{align*}
            a x = b (a,b \in K, a \neq 0)
      \end{align*}
      因为$K$内可以做除法,立即得$x = \frac{b}{a}$是它的唯一的一个根,
      证明这个根是唯一的。
\end{zremark}

\textbf{证明:}

设$x_1, x_2 \in K$是方程的解,那么
\begin{equation*}
      \begin{cases*}
            ax_1 = b \\
            ax_2 = b
      \end{cases*}
\end{equation*}
(从逻辑学的角度,相等需要满足四条相等公理:自反性,对称公理,传递公理,替换公理,否则就不是相等关系。
这里使用了传递公理,
即$x = y, y = z$,那么$x = z$。)

于是,我们有(利用了$a \neq 0$)
\begin{align*}
      ax_1              & = ax_2              \\
      \frac{1}{a} a x_1 & = \frac{1}{a} a x_2 \\
      x_1               & = x_2
\end{align*}
这里第二个等式,使用了相等公理中的替换公理:\\
“对于两个同类型的对象$x, y$,如果$x = y$,那么对任意一个函数或者运算$f$都有$f(x) = f(y)$”。

本题可看做$f(t) = \frac{1}{a} t$。

\begin{zremark}
      文中:
      \begin{align*}
            x^k - a^k = (x - a)(x^{k - 1} + a x^{k - 2} + \cdots + a^{k - 1})
      \end{align*}
      是如何得到的。
\end{zremark}

右侧乘开,我们有
\begin{align*}
       & (x - a)(x^{k - 1} + a x^{k - 2} + \cdots + a^{k - 1}) \\
       & = x^k + ax^{k - 1} + \cdots + a^{k - 1} x
      - ax^{k - 1} - a^2 x^{k - 2} - \cdots - a^k              \\
       & = x^k - a^k
\end{align*}

\begin{zremark}
      推论1的证明过程中,隐含使用了如下命题:\\
      “$a, b \neq 0$,那么$ab = 0$”。\\
      是否可以使用9条运算法则(更准确的说是复数版本)推导出?
\end{zremark}

\textbf{证明:}

反证法,假设$ab = 0$。
于是由例1.2可知,对任意复数都有$x$,都有$0 x = 0$。
我们令$x = \frac{1}{a} \frac{1}{b}$,
那么,$ab x = 0$,这与
\begin{align*}
      ab \frac{1}{a} \frac{1}{b} = 1
\end{align*}
矛盾,假设不成立,命题成立。

\begin{zremark}
      预备命题的证明中,等式:
      \begin{align*}
            \sigma_i(\alpha_1,\cdots,\alpha_k)
            +
            \sigma_{i - 1}(\alpha_1, \cdots, \alpha_k)\alpha_{k + 1}
            =
            \sigma_i(\alpha_1, \cdots, \alpha_k, \alpha_{k + 1})
      \end{align*}
      如何证明?
\end{zremark}

\textbf{证明:}

我们从右边拆解,选取有两种方式:\\
1.选取的结果中没有$\sigma_{k + 1}$,那么就是所有项都是从
前$k$个数中选,选$i$个数,即:
\begin{align*}
      \sigma_{i}(\alpha_1,\cdots,\alpha_k)
\end{align*}

2.选取的结果中包含$\sigma_{k + 1}$,那么需要先从$k$个中选$i - 1$个,
然后乘以$\alpha_{k + 1}$,即:
\begin{align*}
      \sigma_{i - 1}(\alpha_1, \cdots, \alpha_k)\alpha_{k + 1}
\end{align*}
这就完成了证明。

\begin{zremark}
      通过命题2.3,推导出高中的韦达定理:\\
      \textbf{“
            设$f(x) = ax^2 + bx + c \ (a \neq 0)$”,
            有两个根$x_1, x_2$,那么,
            \begin{align*}
                  x_1 + x_2 = - \frac{b}{a} \\
                  x_1 x_2 = \frac{c}{a}
            \end{align*}
      }
\end{zremark}

\textbf{证明:}

其实韦达定理说的就是根与系数的关系,只是命题2.3的特例。

由命题2.3,我们有
\begin{align*}
      \frac{b}{a} = \frac{a_1}{a_0} & = (-1)^1 \sigma_1(x_1, x_2) \\
                                    & = - (x_1 + x_2)
\end{align*}
同理
\begin{align*}
      \frac{c}{a} = \frac{a_2}{a_0} & = (-1)^2 \sigma_2(x_1, x_2) \\
                                    & = x_1x_2
\end{align*}


\end{document}