\documentclass{article}
\usepackage{mathtools} 
\usepackage{fontspec}
\usepackage[UTF8]{ctex}
\usepackage{amsthm}
\usepackage{mdframed}
\usepackage{xcolor}
\usepackage{amssymb}
\usepackage{amsmath}
\usepackage{hyperref}


% 定义新的带灰色背景的说明环境 zremark
\newmdtheoremenv[
  backgroundcolor=gray!10,
  % 边框与背景一致,边框线会消失
  linecolor=gray!10
]{zremark}{注释}

% 通用矩阵命令: \flexmatrix{矩阵名}{元素符号}{行数}{列数}
\newcommand{\flexmatrix}[4]{
  \[
  #1 = \begin{pmatrix}
    #2_{11}     & #2_{12}     & \cdots & #2_{1#4}   \\
    #2_{21}     & #2_{22}     & \cdots & #2_{2#4}   \\
    \vdots      & \vdots      & \ddots & \vdots     \\
    #2_{#31}    & #2_{#32}    & \cdots & #2_{#3#4}
  \end{pmatrix}
  \]
}

% 简化版命令(默认矩阵名为A,元素符号为a): \quickmatrix{行数}{列数}
\newcommand{\quickmatrix}[2]{\flexmatrix{A}{a}{#1}{#2}}


\begin{document}
\title{1.3}
\author{张志聪}
\maketitle

\section*{1}
只做第一题(处理过程很机械)。

写成方程的的增广矩阵,并进行初等变换:
\begin{align*}
  \begin{bmatrix}
    2 & -1 & 1  & -1 & 1  \\
    2 & -1 & 0  & -3 & 2  \\
    3 & 0  & -1 & 1  & -3 \\
    2 & 2  & -2 & 5  & -6
  \end{bmatrix}
  \xrightarrow{\frac{1}{2} R_1}
  \begin{bmatrix}
    1 & \frac{-1}{2} & \frac{1}{2} & \frac{-1}{2} & \frac{1}{2} \\
    2 & -1           & 0           & -3           & 2           \\
    3 & 0            & -1          & 1            & -3          \\
    2 & 2            & -2          & 5            & -6
  \end{bmatrix}
  \xrightarrow{R_2 - 2 R_1}
  \begin{bmatrix}
    1 & \frac{-1}{2} & \frac{1}{2} & \frac{-1}{2} & \frac{1}{2} \\
    0 & -1           & 0           & -3           & 2           \\
    3 & 0            & -1          & 1            & -3          \\
    2 & 2            & -2          & 5            & -6
  \end{bmatrix}   \\
  \xrightarrow{R_3 - 3 R_1}
  \begin{bmatrix}
    1 & \frac{-1}{2} & \frac{1}{2}  & \frac{-1}{2} & \frac{1}{2}  \\
    0 & -1           & 0            & -3           & 2            \\
    0 & \frac{3}{2}  & \frac{-5}{2} & \frac{-1}{2} & \frac{-9}{2} \\
    2 & 2            & -2           & 5            & -6
  \end{bmatrix}
  \xrightarrow{R_4 - 2 R_1}
  \begin{bmatrix}
    1 & \frac{-1}{2} & \frac{1}{2}  & \frac{-1}{2} & \frac{1}{2}  \\
    0 & -1           & 0            & -3           & 2            \\
    0 & \frac{3}{2}  & \frac{-5}{2} & \frac{-1}{2} & \frac{-9}{2} \\
    0 & 3            & -3           & 6            & -7
  \end{bmatrix}
  \xrightarrow{R_3 + \frac{3}{2} R_2}
  \begin{bmatrix}
    1 & \frac{-1}{2} & \frac{1}{2}  & \frac{-1}{2} & \frac{1}{2}  \\
    0 & -1           & 0            & -3           & 2            \\
    0 & 0            & \frac{-5}{2} & -5           & \frac{-3}{2} \\
    0 & 3            & -3           & 6            & -7
  \end{bmatrix} \\
  \xrightarrow{R_4 + 3 R_2}
  \begin{bmatrix}
    1 & \frac{-1}{2} & \frac{1}{2}  & \frac{-1}{2} & \frac{1}{2}  \\
    0 & -1           & 0            & -3           & 2            \\
    0 & 0            & \frac{-5}{2} & -5           & \frac{-3}{2} \\
    0 & 0            & -3           & -3           & -1
  \end{bmatrix}
  \xrightarrow{R_3 \times (-2)}
  \begin{bmatrix}
    1 & \frac{-1}{2} & \frac{1}{2} & \frac{-1}{2} & \frac{1}{2} \\
    0 & -1           & 0           & -3           & 2           \\
    0 & 0            & 5           & 10           & 3           \\
    0 & 0            & -3          & -3           & -1
  \end{bmatrix}
  \xrightarrow{R_4 + \frac{3}{5} R_3}
  \begin{bmatrix}
    1 & \frac{-1}{2} & \frac{1}{2} & \frac{-1}{2} & \frac{1}{2} \\
    0 & -1           & 0           & -3           & 2           \\
    0 & 0            & 5           & 10           & 3           \\
    0 & 0            & 0           & 3            & \frac{4}{5}
  \end{bmatrix}
\end{align*}
写出对应方程组
\begin{equation*}
  \begin{cases*}
    x_1 - \frac{1}{2} x_2 + \frac{1}{2} x_3 - \frac{1}{2} x_4 = \frac{1}{2} \\
    -x_2  - 3 x_4 = 2                                                       \\
    5 x_3 + 10 x_4 = 3                                                      \\
    3 x_4 = \frac{4}{5}                                                     \\
  \end{cases*}
\end{equation*}
然后自下而上逐次求$x_4, x_3, x_2, x_1$,最后得
\begin{align*}
  x_1 & = - \frac{4}{5}  \\
  x_2 & = - \frac{14}{5} \\
  x_3 & = \frac{1}{15}   \\
  x_4 & = \frac{4}{15}
\end{align*}

\section*{2}

先证明极端情况,$a_{11} = a_{21} = 0$(或$a_{12} = a_{22} = 0$),
这种情况$a_{11}a_{22} - a_{12} a_{21} = 0$是必然的,无需讨论。

写出系数矩阵,并做初等变换,完成消元:
\begin{align*}
  \begin{bmatrix}
    a_{11} & a_{12} \\
    a_{21} & a_{22}
  \end{bmatrix}
  \xrightarrow{R_2 - \frac{a_{21}}{a_{11}} R_1}
  \begin{bmatrix}
    a_{11} & a_{12}                                \\
    0      & a_{22} - \frac{a_{12} a_{21}}{a_{11}}
  \end{bmatrix}
\end{align*}
(这里假设了$a_{11} \neq 0$,$a_{12} \neq 0$证明方式类似,只需调整方程顺序即可。)

\begin{itemize}

  \item 充分性

        当$a_{11}a_{22} - a_{12} a_{21} = 0$时
        \begin{align*}
          a_{22} - \frac{a_{12} a_{21}}{a_{11}} = 0
        \end{align*}
        此时,矩阵如下:
        \begin{align*}
          \begin{bmatrix}
            a_{11} & a_{12} \\
            0      & 0
          \end{bmatrix}
        \end{align*}
        于是第二行是无效行,
        由命题3.2可知,方程组一定有非零解。

  \item 必要性

        反证法,假设$a_{11}a_{22} - a_{12} a_{21} \neq 0$,

        于是,我们有
        \begin{equation*}
          \begin{cases*}
            a_{11}x_1 + a_{12}x_2 = 0 \\
            \frac{a_{11}a_{22} - a_{12} a_{21}}{a_{11}} x_2 = 0
          \end{cases*}
        \end{equation*}
        因为,$a_{11}a_{22} - a_{12} a_{21} \neq 0$,所有
        \begin{align*}
          x_2 = 0
        \end{align*}
        进而
        \begin{align*}
          x_1 = 0
        \end{align*}
        也就是说,此时方程没有非零解,与题设矛盾,假设不成立,命题得证。

\end{itemize}

\section*{3}

有以下情况:
\begin{itemize}
  \item 可以直接解出来;
  \item 存在矛盾的等式,无解;
  \item 存在无效行,使用命题3.2,判断是否有非零解。
\end{itemize}

\section*{4}

先做矩阵消元
\begin{align*}
  \begin{bmatrix}
    2  & 1 & 1  \\
    a  & 0 & -1 \\
    -1 & 0 & 3
  \end{bmatrix}
  \xrightarrow{R_2 + \frac{-a}{2}R_1}
  \begin{bmatrix}
    2  & 1            & 1                \\
    0  & \frac{-a}{2} & -1 - \frac{a}{2} \\
    -1 & 0            & 3
  \end{bmatrix}
  \xrightarrow{R_3 + \frac{1}{2}R_1}
  \begin{bmatrix}
    2 & 1            & 1                \\
    0 & \frac{-a}{2} & -1 - \frac{a}{2} \\
    0 & \frac{1}{2}  & \frac{7}{2}
  \end{bmatrix} \\
  \xrightarrow{}
  \begin{bmatrix}
    2 & 1            & 1                \\
    0 & \frac{1}{2}  & \frac{7}{2}      \\
    0 & \frac{-a}{2} & -1 - \frac{a}{2}
  \end{bmatrix}
  \xrightarrow{R_3 + a R_2}
  \begin{bmatrix}
    2 & 1           & 1           \\
    0 & \frac{1}{2} & \frac{7}{2} \\
    0 & 0           & -1 + 3a
  \end{bmatrix}
\end{align*}
写出对应的方程组
\begin{equation*}
  \begin{cases*}
    2x + y + z = 0                  \\
    \frac{1}{2}y + \frac{7}{2}z = 0 \\
    (-1 + 3a) z = 0
  \end{cases*}
\end{equation*}
假设$(-1 + 3a) \neq 0$,于是可得$z = 0$,进而得$y = 0, x = 0$。
此时,方程没有非零解。

假设$(-1 + 3a) = 0$(即$a = \frac{1}{3}$),于是方程组成为
\begin{equation*}
  \begin{cases*}
    2x + y + z = 0                \\
    \frac{1}{2}y = - \frac{7}{2}z \\
  \end{cases*}
\end{equation*}
可得
\begin{equation*}
  \begin{cases*}
    x = 3 z \\
    y = -7 z
  \end{cases*}
\end{equation*}
$z$是自由变量。

\section*{5}

只做第一题

对增广矩阵进行消元
\begin{align*}
  \begin{bmatrix}
    \lambda & 1       & 1       & 1         \\
    1       & \lambda & 1       & \lambda   \\
    1       & 1       & \lambda & \lambda^2
  \end{bmatrix}
  \xrightarrow{}
  \begin{bmatrix}
    1       & 1       & \lambda & \lambda^2 \\
    1       & \lambda & 1       & \lambda   \\
    \lambda & 1       & 1       & 1
  \end{bmatrix}
  \xrightarrow{R_2 - R_1}
  \begin{bmatrix}
    1       & 1           & \lambda     & \lambda^2           \\
    0       & \lambda - 1 & 1 - \lambda & \lambda - \lambda^2 \\
    \lambda & 1           & 1           & 1
  \end{bmatrix}                       \\
  \xrightarrow{R_3 - \lambda R_1}
  \begin{bmatrix}
    1 & 1           & \lambda       & \lambda^2           \\
    0 & \lambda - 1 & 1 - \lambda   & \lambda - \lambda^2 \\
    0 & 1 - \lambda & 1 - \lambda^2 & 1 - \lambda^3
  \end{bmatrix}
  \xrightarrow{R_3 + R_2}
  \begin{bmatrix}
    1 & 1           & \lambda                 & \lambda^2                           \\
    0 & \lambda - 1 & 1 - \lambda             & \lambda - \lambda^2                 \\
    0 & 0           & 2 - \lambda - \lambda^2 & 1 + \lambda - \lambda^2 - \lambda^3
  \end{bmatrix} \\
  \xrightarrow{(1 - \lambda)R_2}
  \begin{bmatrix}
    1 & 1   & \lambda                 & \lambda^2                           \\
    0 & - 1 & 1                       & \lambda                             \\
    0 & 0   & 2 - \lambda - \lambda^2 & 1 + \lambda - \lambda^2 - \lambda^3
  \end{bmatrix}
\end{align*}
写出对应方程组
\begin{equation*}
  \begin{cases}
    x_1 + x_2 +\lambda x_3 = \lambda^2 \\
    - x_2 + x_3 = \lambda              \\
    (2 - \lambda - \lambda^2)x_3 = 1 + \lambda - \lambda^2 - \lambda^3
  \end{cases}
\end{equation*}
调整下
\begin{equation*}
  \begin{cases}
    x_1 + x_2 +\lambda x_3 = \lambda^2 \\
    - x_2 + x_3 = \lambda              \\
    (- \lambda  + 1)(\lambda + 2)x_3 = 1 + \lambda - \lambda^2 - \lambda^3
  \end{cases}
\end{equation*}
考虑$(- \lambda  + 1)(\lambda + 2) = 0$时,方程组的求解情况:

\begin{itemize}
  \item $\lambda = -2$。

        第三个方程会出现如下矛盾
        \begin{align*}
          0 = 3
        \end{align*}
        此时,方程无解。

  \item $\lambda = 1$。

        $x_3$将是自由变量(答案中说$x_2$也是自由的,个人判断是错误的),
        \begin{align*}
          x_1 & = -2x_3 + 2 \\
          x_2 & = x_3 - 1
        \end{align*}
\end{itemize}

考虑$(- \lambda  + 1)(\lambda + 2) \neq 0$,
我们自下而上逐次求$x_3,x_2,x_1$,最后得
\begin{align*}
  x_1 & = - \frac{\lambda + 1}{\lambda + 2}   \\
  x_2 & = \frac{1}{\lambda + 2}               \\
  x_3 & = \frac{(1 + \lambda)^2}{\lambda + 2}
\end{align*}

\section*{6}

先进行增广矩阵消元:
\begin{align*}
  \begin{bmatrix}
    1  & -1 & 0  & 0  & 0  & a_1 \\
    0  & 1  & -1 & 0  & 0  & a_2 \\
    0  & 0  & 1  & -1 & 0  & a_3 \\
    0  & 0  & 0  & 1  & -1 & a_4 \\
    -1 & 0  & 0  & 0  & 1  & a_5 \\
  \end{bmatrix}
  \xrightarrow{R_5 + R_1}
  \begin{bmatrix}
    1 & -1 & 0  & 0  & 0  & a_1       \\
    0 & 1  & -1 & 0  & 0  & a_2       \\
    0 & 0  & 1  & -1 & 0  & a_3       \\
    0 & 0  & 0  & 1  & -1 & a_4       \\
    0 & -1 & 0  & 0  & 1  & a_5 + a_1 \\
  \end{bmatrix}                   \\
  \xrightarrow{R_5 + R_2}
  \begin{bmatrix}
    1 & -1 & 0  & 0  & 0  & a_1             \\
    0 & 1  & -1 & 0  & 0  & a_2             \\
    0 & 0  & 1  & -1 & 0  & a_3             \\
    0 & 0  & 0  & 1  & -1 & a_4             \\
    0 & 0  & -1 & 0  & 1  & a_5 + a_1 + a_2 \\
  \end{bmatrix}
  \xrightarrow{R_5 + R_3}
  \begin{bmatrix}
    1 & -1 & 0  & 0  & 0  & a_1                   \\
    0 & 1  & -1 & 0  & 0  & a_2                   \\
    0 & 0  & 1  & -1 & 0  & a_3                   \\
    0 & 0  & 0  & 1  & -1 & a_4                   \\
    0 & 0  & 0  & -1 & 1  & a_5 + a_1 + a_2 + a_3 \\
  \end{bmatrix}       \\
  \xrightarrow{R_5 + R_4}
  \begin{bmatrix}
    1 & -1 & 0  & 0  & 0  & a_1                         \\
    0 & 1  & -1 & 0  & 0  & a_2                         \\
    0 & 0  & 1  & -1 & 0  & a_3                         \\
    0 & 0  & 0  & 1  & -1 & a_4                         \\
    0 & 0  & 0  & 0  & 0  & a_5 + a_1 + a_2 + a_3 + a_4 \\
  \end{bmatrix} \\
\end{align*}

\begin{itemize}
  \item 必要性

        显然,如果$a_5 + a_1 + a_2 + a_3 + a_4 \neq 0$,那么第5个方程存在矛盾,
        则线性方程组无解,可知$a_5 + a_1 + a_2 + a_3 + a_4 = 0$是有解的必要条件。

  \item 充分性

        已知,$a_5 + a_1 + a_2 + a_3 + a_4 = 0$。

        写出消元后的矩阵,对应的方程组
        \begin{equation*}
          \begin{cases*}
            x_1 - x_2 = a_1 \\
            x_2 - x_3 = a_2 \\
            x_3 - x_4 = a_3 \\
            x_4 - x_5 = a_4
          \end{cases*}
        \end{equation*}
        此时,$x_5$是自由变量,可得
        \begin{align*}
          x_4 = a_4 + x_5             \\
          x_3 = a_3 + a_4 + x_5       \\
          x_2 = a_2 + a_3 + a_4 + x_5 \\
          x_1 = a_1 + a_2 + a_3 + a_4 + x_5
        \end{align*}
        由于第5行变为 $0 = 0$,不会引入矛盾,所构造的解满足全部方程。

        于是,我们找到了满足所有方程的解,充分性证明完成。
\end{itemize}

\section*{7}

自醒:不要看到方程组的解,就做矩阵消元法,不看当前的形式能不能直接处理,
因为矩阵消元法的目的,也只是为了方便方程组解的处理的。

由方程组前$n - 1$个方程可得:
\begin{align*}
  x_1 = - x_n       \\
  x_2 = -x_1 = x_n  \\
  x_3 = -x_2 = -x_n \\
  \vdots            \\
  x_{n - 1} = (-1)^{n - 1} x_n
\end{align*}

$n$是奇数时,
\begin{align*}
  x_{n - 1} = (-1)^{n - 1} x_n = x_n
\end{align*}
为了保证最后一个方程
\begin{align*}
  x_{n - 1} + x_n = 0
\end{align*}
成立,此时只能$x_n = 0$,于是所有的$x_i \ (1 \leq i \leq n - 1)$都为0。

$n$是偶数时,
\begin{align*}
  x_{n - 1} = (-1)^{n - 1} x_n = -x_n
\end{align*}
满足最后一个方程
\begin{align*}
  x_{n - 1} + x_n = 0
\end{align*}
于是,我们找到了以$x_n$为自由变量的解。

\section*{8}

二次曲线的一般形式为
\begin{align*}
  Ax^2 + Bxy + Cy^2 + Dx + Ey + F = 0
\end{align*}
我们需要通过五个点,确定系数$A,B,C,D,E,F$的值。

设任意不在同一条直线上的五个点为
\begin{align*}
  (x_1, y_1), (x_2, y_2), (x_3, y_3), (x_4, y_4), (x_5, y_5)
\end{align*}
于是,我们有
\begin{equation*}
  \begin{cases*}
    Ax_1^2 + Bx_1y_1 + Cy_1^2 + Dx_1 + Ey_1 + F & = 0 \\
    Ax_2^2 + Bx_2y_2 + Cy_2^2 + Dx_2 + Ey_2 + F & = 0 \\
    Ax_3^2 + Bx_3y_3 + Cy_3^2 + Dx_3 + Ey_3 + F & = 0 \\
    Ax_4^2 + Bx_4y_4 + Cy_4^2 + Dx_4 + Ey_4 + F & = 0 \\
    Ax_5^2 + Bx_5y_5 + Cy_5^2 + Dx_5 + Ey_5 + F & = 0
  \end{cases*}
\end{equation*}
我们得到了一个齐次方程组,对应的系数矩阵如下:
\begin{align*}
  \begin{bmatrix}
    x_1^2 & x_1y_1 & y_1^2 & x_1 & y_1 & F \\
    x_2^2 & x_2y_2 & y_2^2 & x_2 & y_2 & F \\
    x_3^2 & x_3y_3 & y_3^2 & x_3 & y_3 & F \\
    x_4^2 & x_4y_4 & y_4^2 & x_4 & y_4 & F \\
    x_5^2 & x_5y_5 & y_5^2 & x_5 & y_5 & F
  \end{bmatrix}
\end{align*}
由命题3.2可知,他必定存在非零解。

这个非零解会不会使得$A=B=C=0$(即二次项全部为零),
此时得到的不在是二次曲线,而是一条直线,这与题设
五个点不在同一条直线上矛盾,故不会存在这种情况。

(2)给定四个点,这部分就是矩阵的计算问题了,就不做解答了。

\section*{9}

所有方程相加,我们有
\begin{align*}
   & (n-1)x_1 + (a(n-2) + 1)x_2 + (a(n-2) + 1)x_3 + (a(n-2) + 1)x_4 + \cdots (a(n-2) + 1)x_n \\
   & = b_1 + b_2 + b_3 + b_4 + \cdots + b_n \textcircled{1}
\end{align*}

第一个方程乘以$(a(n-2) + 1)$,我们有
\begin{align*}
   & (a(n-2) + 1)x_2 + (a(n-2) + 1)x_3 + (a(n-2) + 1)x_4 + \cdots (a(n-2) + 1)x_n \\
   & = (a(n-2) + 1)b_1 \textcircled{2}
\end{align*}

$\textcircled{1} - \textcircled{2}$,我们有
\begin{align*}
  (n-1)x_1 & = b_1 + b_2 + b_3 + b_4 + \cdots + b_n - (a(n-2) + 1)b_1               \\
  x_1      & = \frac{b_1 + b_2 + b_3 + b_4 + \cdots + b_n - (a(n-2) + 1)b_1}{n - 1} \\
  x_1      & = \frac{b_1 + b_2 + b_3 + b_4 + \cdots + b_n - a(n-2)b_1}{n - 1}
\end{align*}


第一个方程乘以$a$,我们有
\begin{align*}
  a x_2 + ax_3 + ax_4 + \cdots + ax_n = ab_1 \textcircled{3}
\end{align*}

式$\textcircled{3}$减去方程二,我们有
\begin{align*}
  -x_1 + ax_2 & = ab_1 - b_2                 \\
  x_2         & = \frac{ab_1 - b_2 + x_1}{a}
\end{align*}
以此类推,对任意$x_i(2 \leq i \leq n)$都有
\begin{align*}
  x_i & = \frac{ab_1 - b_i + x_1}{a}
\end{align*}

综上,我们有
\begin{align*}
  x_1 & = \frac{b_1 + b_2 + b_3 + b_4 + \cdots + b_n - a(n-2)b_1}{n - 1} \\
  x_i & = \frac{ab_1 - b_i + x_1}{a} (2 \leq i \leq n)
\end{align*}

把结果代入方程组验证(只需代入方程1和方程2,其他方程同理),
方程组成立,所以这就是方程组的解。\\
(这一步必不可少,因为我们假设了这个方程组有解,才能把所有的方程相加,
具体可参考这个视频:
\url{https://www.bilibili.com/video/BV13q4y1E75c/?spm_id_from=333.1391.0.0&vd_source=d155cb524fdd3dbd0fd732cc3ae44dff}
)

\section*{10}

如果$a_1,a_2, \cdots, a_n$都是$0$,于是未知量可以取任意值。

设存在一个$a_t \neq 0 (1 \leq t \leq n) $,于是,利用题设,有:
\begin{align*}
  a_ta_t x_{jk} - a_j a_k x_{tt} & = 0                            \\
  x_{jk}                         & = \frac{a_j a_k}{a_t^2} x_{tt}
\end{align*}

所以,我们得到$x_{tt}$为自由变量的解
\begin{align*}
  x_{jk} = \frac{a_j a_k}{a_t^2} x_{tt}
\end{align*}

与9题一样,需要验证该解是否满足所有方程组。
对任意方程
\begin{align*}
  a_ia_l x_{jk} - a_ja_k x_{il} & = a_ia_l \frac{a_j a_k}{a_t^2} x_{tt} - a_ja_k \frac{a_i a_l}{a_t^2} x_{tt} \\
                                & = 0
\end{align*}
验证完毕。

\end{document}