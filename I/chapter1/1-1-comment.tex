\documentclass{article}
\usepackage{mathtools} 
\usepackage{fontspec}
\usepackage[UTF8]{ctex}
\usepackage{amsthm}
\usepackage{mdframed}
\usepackage{xcolor}
\usepackage{amssymb}
\usepackage{amsmath}


% 定义新的带灰色背景的说明环境 zremark
\newmdtheoremenv[
  backgroundcolor=gray!10,
  % 边框与背景一致,边框线会消失
  linecolor=gray!10
]{zremark}{注释}

% 通用矩阵命令: \flexmatrix{矩阵名}{元素符号}{行数}{列数}
\newcommand{\flexmatrix}[4]{
  \[
  #1 = \begin{pmatrix}
    #2_{11}     & #2_{12}     & \cdots & #2_{1#4}   \\
    #2_{21}     & #2_{22}     & \cdots & #2_{2#4}   \\
    \vdots      & \vdots      & \ddots & \vdots     \\
    #2_{#31}    & #2_{#32}    & \cdots & #2_{#3#4}
  \end{pmatrix}
  \]
}

% 简化版命令(默认矩阵名为A,元素符号为a): \quickmatrix{行数}{列数}
\newcommand{\quickmatrix}[2]{\flexmatrix{A}{a}{#1}{#2}}


\begin{document}
\title{1.1 注释}
\author{张志聪}
\maketitle

\begin{zremark}
  $f$是可逆映射,则其逆映射$g$是唯一确定的。
\end{zremark}

\textbf{证明:}

假设逆映射不是唯一的,即存在$h$也是
$f$的逆映射,且$g \neq h$。

因为$g \neq h$,所以存在$b \in B$使得
\begin{align*}
  g(b) \neq h(b)
\end{align*}

又因为
\begin{align*}
  fg = id_{A} \\
  fh = id_{A}
\end{align*}
所以,我们有
\begin{align*}
  f(g(b)) = f(h(b)) = b
\end{align*}
这与$f$是双射矛盾(利用了命题1.3)。


\end{document}