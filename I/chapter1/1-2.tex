\documentclass{article}
\usepackage{mathtools} 
\usepackage{fontspec}
\usepackage[UTF8]{ctex}
\usepackage{amsthm}
\usepackage{mdframed}
\usepackage{xcolor}
\usepackage{amssymb}
\usepackage{amsmath}


% 定义新的带灰色背景的说明环境 zremark
\newmdtheoremenv[
  backgroundcolor=gray!10,
  % 边框与背景一致,边框线会消失
  linecolor=gray!10
]{zremark}{注释}

% 通用矩阵命令: \flexmatrix{矩阵名}{元素符号}{行数}{列数}
\newcommand{\flexmatrix}[4]{
  \[
  #1 = \begin{pmatrix}
    #2_{11}     & #2_{12}     & \cdots & #2_{1#4}   \\
    #2_{21}     & #2_{22}     & \cdots & #2_{2#4}   \\
    \vdots      & \vdots      & \ddots & \vdots     \\
    #2_{#31}    & #2_{#32}    & \cdots & #2_{#3#4}
  \end{pmatrix}
  \]
}

% 简化版命令(默认矩阵名为A,元素符号为a): \quickmatrix{行数}{列数}
\newcommand{\quickmatrix}[2]{\flexmatrix{A}{a}{#1}{#2}}


\begin{document}
\title{1.2}
\author{张志聪}
\maketitle

\section*{1}

$\mathbb{Q}(\sqrt{5})$

\section*{2}

令$f(x) = 2x^4 - 3x^2 + x - 1$。

由命题2.1,令$a = 1, f(1) = -1$,于是我们有
\begin{align*}
  f(x) & = q(x)(x - 1) + f(1) \\
       & = q(x)(x - 1) - 1
\end{align*}
其中
\begin{align*}
  q(x) & = 2 q_4(x) - 3 q_2(x) + 1                 \\
       & = 2(x^3 + x^2 + x + 1^1) - 3(x^1 + 1) + 1 \\
       & = 2x^3 + 2x^2 + 2x + 2 - 3x - 3 + 1       \\
       & = 2x^3 + 2x^2 - x
\end{align*}

\section*{3}

由命题2.1可知
\begin{align*}
  q(x) = a_0q_n(x) + a_1q_{n-1}(x) + \cdots + a_{n - 1}
\end{align*}

对任意$2 \leq k \leq n$,我们有
\begin{align*}
  q_k(x) = x^{k - 1} + ax^{k - 2} + \cdots + a^{k - 1}
\end{align*}
因为$a \in K$,所以$q_k(x)$的系数也都属于$K$。
进而,$q(x)$的系数也都属于$K$。

\section*{4}

我们考虑变量代换$y = x - a$,于是$x = y + a$,我们有
\begin{align*}
  f(x) = f(y + a)
\end{align*}
$f(y + a)$是一个以$y$为变量的新的多项式$g$,次数为$n$,且系数任然属于数域$K$,
设它为
\begin{align*}
  g(y) = b_0 + b_1 y + b_2 y^2 + \cdots + b_{n} y^{n}
\end{align*}
代回$y = x - a$,我们有
\begin{align*}
  g(y) = b_0 + b_1 (x - a) + b_2 (x - a)^2 + \cdots + b_{n} (x - a)^n
\end{align*}
又因为,我们有
\begin{align*}
  g(y) = f(y + a) = f(x)
\end{align*}
所以
\begin{align*}
  f(x) = b_0 + b_1 (x - a) + b_2 (x - a)^2 + \cdots + b_{n} (x - a)^n
\end{align*}

(另一种方法就是以数学分析的角度考虑——泰勒公式的有限版本)

\section*{5}

由推论1可知,$f(x)$有$n$个复数根,不妨设为
\begin{align*}
  \alpha_1, \alpha_2, \cdots, \alpha_n
\end{align*}
由命题2.2可知,$f(x)$可表示成
\begin{align*}
  f(x) = a_0(x - \alpha_1)(x - \alpha_2) \cdots (x - \alpha_n)
\end{align*}

对$\alpha_1, \alpha_2, \cdots, \alpha_n$可以分成两类:\\
一类是实数根,设为$R = \{b_i \in \mathbb{R}, i = 1, 2, \cdots, k\}$;\\
一类是复数根,设为$C = \{c_i \in \mathbb{C}, i = 1, 2, \cdots, m\}$;\\
于是,我们有
\begin{align*}
  f(x) = a_0(x - b_1)(x - b_2) \cdots (x - b_k)(x - c_1)(x - c_2) \cdots (x - c_m)
\end{align*}
由命题2.4可知,对任意$c_j \in C$,由对应的共轭复数也在$C$中,
于是,我们有
\begin{align*}
  f_j(x) & = (x - c_j) (x - \overline{c_j})                    \\
         & = x^2 - (c_j + \overline{c_j})x + c_j\overline{c_j}
\end{align*}
令$p_j = - (c_j + \overline{c_j})$, $q_j = c_j\overline{c_j}$,
此时$p_j, q_j$都是实数,所以$f_j(x)$是一个实数系上的二次多项式。
又因为$f_j(x) = 0$没有实数解,所以
\begin{align*}
  p_j^2 - 4q_j < 0
\end{align*}


综上,$f(x)$可表示成:
\begin{align*}
  f(x) = a_0\left(\prod\limits_{i = 1}^k (x - b_i)\right)
  \left(\prod\limits_{j = 1}^{l = m/2} (x^2 + p_j x + q_j)\right)
\end{align*}
其中$p_j^2 - 4q_j < 0$,$(b_i \in \mathbb{R}, i = 1, 2, \cdots, k)$。

\section*{6}

由命题2.1可知,
\begin{align*}
  f(x) = q(x)(x - 1) + f(1)
\end{align*}
因为$f(a) = 0$,即
\begin{align*}
  q(a)(a - 1) + f(1) = 0                     \\
  q(a)(a - 1) + a_0 + a_1 + \cdots + a_n = 0 \\
  - q(a)(a - 1) = a_0 + a_1 + \cdots + a_n
\end{align*}
由$q(x)$的构造方式可知,$q(a)$是整数,于是
$-q(a)(a - 1)$也是整数,可得$a - 1$整除$a_0 + a_1 + \cdots + a_n$。
(注意:代数学中的整除结果不一定非要是正的,负的也可以)

类似地,
\begin{align*}
  f(x) = q(x)(x - (-1)) + f(-1)
\end{align*}
因为$f(a) = 0$,即
\begin{align*}
  q(a)(a-(-1)) + f(-1) = 0                                \\
  q(a)(a + 1) + a_0 - a_1 + a_2 + \cdots + (-1)^n a_n = 0 \\
  - q(a)(a + 1) = a_0 - a_1 + a_2 + \cdots + (-1)^n a_n
\end{align*}
于是$a+1$整除$a_0 - a_1 + a_2 + \cdots + (-1)^n a_n$,也就
整除$(-1)^n (a_0 - a_1 + a_2 + \cdots + (-1)^n a_n)$。

\section*{7}
只做第一题,别的类似。

多项式的$a_0 = 1, a_4 = -14$。

设有理数的零点表示成$\frac{m}{k}$。

于是,我们要保证以下结果都是整数:
\begin{align*}
  \frac{a_0}{k} = \frac{1}{k} \\
  \frac{a_4}{m} = \frac{-14}{m}
\end{align*}
于是可能的结果是
\begin{align*}
  k \in \{1, -1\} \\
  m \in \{1, -1, 2, -2, 7, -7, 14, -14\}
\end{align*}
他们的组合是否为零点,就要一个个试验了。

\begin{itemize}
  \item $\frac{m}{k} = \frac{1}{1} = 1$。

        \begin{align*}
          x^3 - 6x^2 + 15x - 14 = 1 - 6 + 15 - 14 = -4
        \end{align*}
        不是零点。

  \item $\frac{m}{k} = \frac{-1}{1} = -1$。

        \begin{align*}
          x^3 - 6x^2 + 15x - 14 = (-1) - 6 + (-15) - 14 = -46
        \end{align*}
        不是零点。

  \item $\frac{m}{k} = \frac{2}{1} = 2$。
        \begin{align*}
          x^3 - 6x^2 + 15x - 14 = 8 - 24 + 30 - 14 = 0
        \end{align*}
        是零点。

  \item 以此类推


        最后,有唯一的理数根:2
\end{itemize}

\section*{8}

我们有
\begin{align*}
  \sum\limits_{i = 1}^n \alpha_i^2
   & = (\sum\limits_{i = 1}^n \alpha_i)^2
  - \sum\limits_{i \neq j} \alpha_i \alpha_j \\
   & = (\sum\limits_{i = 1}^n \alpha_i)^2
  - 2 \sum\limits_{1 \leq i < j \leq n} \alpha_i \alpha_j
\end{align*}

由命题2.3可知
\begin{align*}
  \sum\limits_{i = 1}^n \alpha_i = \sigma_1(\alpha_1, \cdots, \alpha_n) = - \frac{a_1}{a_0} \\
  \sum\limits_{1 \leq i < j \leq n} \alpha_i \alpha_j = \sigma_2(\alpha_1, \cdots, \alpha_n) = \frac{a_2}{a_0}
\end{align*}
综上,
\begin{align*}
  \sum\limits_{i = 1}^n \alpha_i^2
   & = (- \frac{a_1}{a_0})^2 - 2 \frac{a_2}{a_0}
\end{align*}
因为$a_1, a_2, \cdots , a_n$属于$K$,
所以$(- \frac{a_1}{a_0})^2 - 2 \frac{a_2}{a_0} \in K$。

\section*{9}

% 这道题,个人感觉挺难的,主要在于$n,k$都是可变的,导致情况比较复杂,
% 我会把思考过程写进去,这样方便下次查看。

% $e^{\frac{2\pi i}{n}}$在P8页处有提及,并说明
% $e^{\frac{2k\pi i}{n}} \ (k \in 0, 1, \cdots, n-1)$都是代数方程
% $x^n - 1 = 0$在复数系$\mathbb{C}$内的$n$个根,
% 这也就意味着$\epsilon^k, \epsilon^{2k}, \cdot, \epsilon^{nk}$都是代数方程
% $x^n - 1 = 0$的根。
% 不妨设这些根为$\alpha_1, \alpha_2, \cdots, \alpha_n$。

% 现在的问题是$nk > n - 1$是可能的,但有一点是明确的:
% $\epsilon^{jk} \ 1 \leq j \leq n$,都是$x^n - 1 = 0$的根,
% 因为都是$\frac{2\pi i}{n}$的整数倍。

% 到这里,问题就转变成,在$\alpha_1, \alpha_2, \cdots, \alpha_n$中,
% 中选取$n$个值的问题(有可能重复),一种最简单的情况是:
% 选取的都是不同的值,也就是把所有的根$\alpha_1, \alpha_2, \cdots, \alpha_n$都选了,
% 此时
% \begin{align*}
%   \epsilon^k + \epsilon^{2k} + \cdot + \epsilon^{nk}
%    & = \alpha_1 + \alpha_2 + \cdot + \alpha_n \\
%    & = - \frac{a_1}{a_0}
% \end{align*}
% 现在,讨论什么情况下,会全选。先考虑简单的清空,$k = 1$,此时步长为$1$,
% 选$n$次,显然可以选到所有的根。

% 那么如果$k = 2 > 1$呢?

% 此时,会每隔$2$个选1个(步长为$2$),
% 也就是,第一遍选的都是$\alpha$中下标都是偶数的根,现在有两种情况:\\
% \begin{itemize}
%   \item 总的根有偶数个,即$n$是偶数。

%         那么,$\alpha_n$的下标是偶数,第二遍和第一遍选取的会是同样的根,
%         以此内推,直到选取$k = nk / n$遍。
%         以上情况,显然值选取了偶数根,只选取了部分根。

%   \item 总的根是奇数个,比如说是$3$。

%         第一遍选取的是$\alpha_2$,第二遍选取$\alpha_1, \alpha_3$,
%         选取完毕,此时,选取了所有根。
% \end{itemize}
% 我们希望从特例中找到规律,直觉告诉我们,$n / k$是整数,也就是第一次循环选取后,
% 没有余数,第二遍选取将会和第一遍选取是一样的。

% % 可以把所有的根,看做一个首尾相连的环,对任意位置的根$\alpha_j$,有如下规律:
% % \begin{align*}
% %   first : \alpha_j        \\
% %   second : \alpha_{j + n} \\
% %   third: \alpha_{j + 2n}  \\
% %   \vdots                  \\
% %   k: \alpha_{j + kn}
% % \end{align*}
% % 于是,$\frac{j + xn}{k} \ (1 \leq x \leq k)$是整数,就会被选中。

% 余数,在于他们$n,k$有没有公因子,
todo




\end{document}