\documentclass{article}
\usepackage{mathtools} 
\usepackage{fontspec}
\usepackage[UTF8]{ctex}
\usepackage{amsthm}
\usepackage{mdframed}
\usepackage{xcolor}
\usepackage{amssymb}
\usepackage{amsmath}


% 定义新的带灰色背景的说明环境 zremark
\newmdtheoremenv[
  backgroundcolor=gray!10,
  % 边框与背景一致,边框线会消失
  linecolor=gray!10
]{zremark}{注释}

% 通用矩阵命令: \flexmatrix{矩阵名}{元素符号}{行数}{列数}
\newcommand{\flexmatrix}[4]{
  \[
  #1 = \begin{pmatrix}
    #2_{11}     & #2_{12}     & \cdots & #2_{1#4}   \\
    #2_{21}     & #2_{22}     & \cdots & #2_{2#4}   \\
    \vdots      & \vdots      & \ddots & \vdots     \\
    #2_{#31}    & #2_{#32}    & \cdots & #2_{#3#4}
  \end{pmatrix}
  \]
}

% 简化版命令(默认矩阵名为A,元素符号为a): \quickmatrix{行数}{列数}
\newcommand{\quickmatrix}[2]{\flexmatrix{A}{a}{#1}{#2}}


\begin{document}
\title{1.2}
\author{张志聪}
\maketitle

\section*{1}

$\mathbb{Q}(\sqrt{-5}) = \{a + b\sqrt{5}i \ | \  a, b \in \mathbb{Q}\}$

以扩张的角度看待这个问题:
首先,$\mathbb{Q}$是最小数域,且
\begin{align*}
  x^2 + 5 = 0 \\
  x = \pm \sqrt{5}i
\end{align*}
于是,我们的目标数域$K$必须有如下性质:
\begin{align*}
  \mathbb{Q} \cup \{\pm \sqrt{5}i\} \subseteq K
\end{align*}
数域$K$要保持四则运算的封闭性,我们最终得到
\begin{align*}
  K = \mathbb{Q}(\sqrt{-5})
\end{align*}
那么,这里的$K$是满足条件的最小数域么?\\
假设存在另一个更小的数域$G$,也满足
\begin{align*}
  \mathbb{Q} \cup \{\pm \sqrt{5}i\} \subseteq G
\end{align*}
且对四则运算的封闭性。

因为$G \subset K$,所以存在$\alpha \in K$,使得
\begin{align*}
  \alpha \notin G
\end{align*}
因为
\begin{align*}
  \mathbb{Q} \cup \{\pm \sqrt{5}i\} \subseteq K \\
  \mathbb{Q} \cup \{\pm \sqrt{5}i\} \subseteq G 
\end{align*}
可得
\begin{align*}
  \alpha \notin \mathbb{Q}
\end{align*}
于是,$\alpha$只能是如下形式的无理数
\begin{align*}
  \alpha = a + b\sqrt{5}i \, (b \neq 0)
\end{align*}
因为
\begin{align*}
  b\sqrt{5}i = \sqrt{5}i + \sqrt{5}i + \cdots + \sqrt{5}i 
\end{align*}
由于$G$要保持四则运算的封闭性,所以
\begin{align*}
  \alpha \in G
\end{align*}
存在矛盾。


\section*{2}

令$f(x) = 2x^4 - 3x^2 + x - 1$。

由命题2.1,令$a = 1, f(1) = -1$,于是我们有
\begin{align*}
  f(x) & = q(x)(x - 1) + f(1) \\
       & = q(x)(x - 1) - 1
\end{align*}
其中
\begin{align*}
  q(x) & = 2 q_4(x) - 3 q_2(x) + 1                 \\
       & = 2(x^3 + x^2 + x + 1^1) - 3(x^1 + 1) + 1 \\
       & = 2x^3 + 2x^2 + 2x + 2 - 3x - 3 + 1       \\
       & = 2x^3 + 2x^2 - x
\end{align*}

\section*{3}

由命题2.1可知
\begin{align*}
  q(x) = a_0q_n(x) + a_1q_{n-1}(x) + \cdots + a_{n - 1}
\end{align*}

对任意$2 \leq k \leq n$,我们有
\begin{align*}
  q_k(x) = x^{k - 1} + ax^{k - 2} + \cdots + a^{k - 1}
\end{align*}
因为$a \in K$,所以$q_k(x)$的系数也都属于$K$。
进而,$q(x)$的系数也都属于$K$。

\section*{4}

我们考虑变量代换$y = x - a$,于是$x = y + a$,我们有
\begin{align*}
  f(x) = f(y + a)
\end{align*}
$f(y + a)$是一个以$y$为变量的新的多项式$g$,次数为$n$,且系数任然属于数域$K$,
设它为
\begin{align*}
  g(y) = b_0 + b_1 y + b_2 y^2 + \cdots + b_{n} y^{n}
\end{align*}
代回$y = x - a$,我们有
\begin{align*}
  g(y) = b_0 + b_1 (x - a) + b_2 (x - a)^2 + \cdots + b_{n} (x - a)^n
\end{align*}
又因为,我们有
\begin{align*}
  g(y) = f(y + a) = f(x)
\end{align*}
所以
\begin{align*}
  f(x) = b_0 + b_1 (x - a) + b_2 (x - a)^2 + \cdots + b_{n} (x - a)^n
\end{align*}

(另一种方法就是以数学分析的角度考虑——泰勒公式的有限版本)

\section*{5}

由推论1可知,$f(x)$有$n$个复数根,不妨设为
\begin{align*}
  \alpha_1, \alpha_2, \cdots, \alpha_n
\end{align*}
由命题2.2可知,$f(x)$可表示成
\begin{align*}
  f(x) = a_0(x - \alpha_1)(x - \alpha_2) \cdots (x - \alpha_n)
\end{align*}

对$\alpha_1, \alpha_2, \cdots, \alpha_n$可以分成两类:\\
一类是实数根,设为$R = \{b_i \in \mathbb{R}, i = 1, 2, \cdots, k\}$;\\
一类是复数根,设为$C = \{c_i \in \mathbb{C}, i = 1, 2, \cdots, m\}$;\\
于是,我们有
\begin{align*}
  f(x) = a_0(x - b_1)(x - b_2) \cdots (x - b_k)(x - c_1)(x - c_2) \cdots (x - c_m)
\end{align*}
由命题2.4可知,对任意$c_j \in C$,由对应的共轭复数也在$C$中,
于是,我们有
\begin{align*}
  f_j(x) & = (x - c_j) (x - \overline{c_j})                    \\
         & = x^2 - (c_j + \overline{c_j})x + c_j\overline{c_j}
\end{align*}
令$p_j = - (c_j + \overline{c_j})$, $q_j = c_j\overline{c_j}$,
此时$p_j, q_j$都是实数,所以$f_j(x)$是一个实数系上的二次多项式。
又因为$f_j(x) = 0$没有实数解,所以
\begin{align*}
  p_j^2 - 4q_j < 0
\end{align*}


综上,$f(x)$可表示成:
\begin{align*}
  f(x) = a_0\left(\prod\limits_{i = 1}^k (x - b_i)\right)
  \left(\prod\limits_{j = 1}^{l = m/2} (x^2 + p_j x + q_j)\right)
\end{align*}
其中$p_j^2 - 4q_j < 0$,$(b_i \in \mathbb{R}, i = 1, 2, \cdots, k)$。

\section*{6}

由命题2.1可知,
\begin{align*}
  f(x) = q(x)(x - 1) + f(1)
\end{align*}
因为$f(a) = 0$,即
\begin{align*}
  q(a)(a - 1) + f(1) = 0                     \\
  q(a)(a - 1) + a_0 + a_1 + \cdots + a_n = 0 \\
  - q(a)(a - 1) = a_0 + a_1 + \cdots + a_n
\end{align*}
由$q(x)$的构造方式可知,$q(a)$是整数,于是
$-q(a)(a - 1)$也是整数,可得$a - 1$整除$a_0 + a_1 + \cdots + a_n$。
(注意:代数学中的整除结果不一定非要是正的,负的也可以)

类似地,
\begin{align*}
  f(x) = q(x)(x - (-1)) + f(-1)
\end{align*}
因为$f(a) = 0$,即
\begin{align*}
  q(a)(a-(-1)) + f(-1) = 0                                \\
  q(a)(a + 1) + a_0 - a_1 + a_2 + \cdots + (-1)^n a_n = 0 \\
  - q(a)(a + 1) = a_0 - a_1 + a_2 + \cdots + (-1)^n a_n
\end{align*}
于是$a+1$整除$a_0 - a_1 + a_2 + \cdots + (-1)^n a_n$,也就
整除$(-1)^n (a_0 - a_1 + a_2 + \cdots + (-1)^n a_n)$。

\section*{7}
只做第一题,别的类似。

多项式的$a_0 = 1, a_4 = -14$。

设有理数的零点表示成$\frac{m}{k}$。

于是,我们要保证以下结果都是整数:
\begin{align*}
  \frac{a_0}{k} = \frac{1}{k} \\
  \frac{a_4}{m} = \frac{-14}{m}
\end{align*}
于是可能的结果是
\begin{align*}
  k \in \{1, -1\} \\
  m \in \{1, -1, 2, -2, 7, -7, 14, -14\}
\end{align*}
他们的组合是否为零点,就要一个个试验了。

\begin{itemize}
  \item $\frac{m}{k} = \frac{1}{1} = 1$。

        \begin{align*}
          x^3 - 6x^2 + 15x - 14 = 1 - 6 + 15 - 14 = -4
        \end{align*}
        不是零点。

  \item $\frac{m}{k} = \frac{-1}{1} = -1$。

        \begin{align*}
          x^3 - 6x^2 + 15x - 14 = (-1) - 6 + (-15) - 14 = -46
        \end{align*}
        不是零点。

  \item $\frac{m}{k} = \frac{2}{1} = 2$。
        \begin{align*}
          x^3 - 6x^2 + 15x - 14 = 8 - 24 + 30 - 14 = 0
        \end{align*}
        是零点。

  \item 以此类推


        最后,有唯一的理数根:2
\end{itemize}

\section*{8}

我们有
\begin{align*}
  \sum\limits_{i = 1}^n \alpha_i^2
   & = (\sum\limits_{i = 1}^n \alpha_i)^2
  - \sum\limits_{i \neq j} \alpha_i \alpha_j \\
   & = (\sum\limits_{i = 1}^n \alpha_i)^2
  - 2 \sum\limits_{1 \leq i < j \leq n} \alpha_i \alpha_j
\end{align*}

由命题2.3可知
\begin{align*}
  \sum\limits_{i = 1}^n \alpha_i = \sigma_1(\alpha_1, \cdots, \alpha_n) = - \frac{a_1}{a_0} \\
  \sum\limits_{1 \leq i < j \leq n} \alpha_i \alpha_j = \sigma_2(\alpha_1, \cdots, \alpha_n) = \frac{a_2}{a_0}
\end{align*}
综上,
\begin{align*}
  \sum\limits_{i = 1}^n \alpha_i^2
   & = (- \frac{a_1}{a_0})^2 - 2 \frac{a_2}{a_0}
\end{align*}
因为$a_1, a_2, \cdots , a_n$属于$K$,
所以$(- \frac{a_1}{a_0})^2 - 2 \frac{a_2}{a_0} \in K$。

\section*{9}

先讨论$k > 0$的情况。

讨论,$n, k$互为质数。

我们有
\begin{align*}
  \epsilon^k + \epsilon^{2k} + \cdots + \epsilon^{nk}
   & = e^{i 2\pi \frac{k}{n}} + e^{i 2\pi \frac{2k}{n}} + \cdots + e^{i 2\pi \frac{nk}{n}}
\end{align*}
现在我们证明,任意$\epsilon^{jk} \neq \epsilon^{lk}(1 \leq j < l \leq n)$,即两两不相等。

反证法,假设存在$1 \leq j < l \leq n$,使得
\begin{align*}
  \epsilon^{jk}           & = \epsilon^{lk}           \\
  e^{i 2\pi \frac{jk}{n}} & = e^{i 2\pi \frac{lk}{n}}
\end{align*}
因为等式两边复数的模都为$1$,所以只能是复数的辐角相同,相差$2\pi$的正整数倍(设为$m$),即
\begin{align*}
  \frac{lk}{n} / \frac{jk}{n} = m \\
  \frac{l}{j} = m
\end{align*}
这与$1 \leq j < l \leq n$矛盾,假设不成立,两两不相等得证。

因为$\epsilon^{jk} (1 \leq j \leq n)$都是$x^n - 1 = 0$的根,
且两两不相等,个数为$n$,于是可得$\epsilon^{k}, \epsilon^{2k}, \cdots, \epsilon^{nk}$是
$x^n - 1 = 0$的$n$个根,由根与系数的关系(命题2.3),我们有
\begin{align*}
  \epsilon^{k} + \epsilon^{2k} + \cdots + \epsilon^{nk} = 0
\end{align*}

特别地,$n = 1$,我们有
\begin{align*}
  \epsilon^k + \epsilon^{2k} + \cdots + \epsilon^{nk} = n
\end{align*}

设
\begin{align*}
  (n, k) = d
\end{align*}
($n, k$的最大公因子)。

于是,存在$p, q \in \mathbb{N}^+$且$(p,q) = 1$,使得
\begin{align*}
  n = dp \\
  k = dq
\end{align*}
我们有
\begin{align*}
  \epsilon^k + \epsilon^{2k} + \cdots + \epsilon^{nk}
   & = e^{i 2\pi \frac{k}{n}} + e^{i 2\pi \frac{2k}{n}} + \cdots + e^{i 2\pi \frac{nk}{n}}  \\
   & = e^{i 2\pi \frac{q}{p}} + e^{i 2\pi \frac{q}{p} 2} + \cdots + e^{i 2\pi \frac{q}{p}n}
\end{align*}
由(1)中的讨论,$p > 1$时,我们有
\begin{align*}
  e^{i 2\pi \frac{q}{p}} + e^{i 2\pi \frac{q}{p} 2} + \cdots + e^{i 2\pi \frac{q}{p}n}
   & = (e^{i 2\pi \frac{q}{p}} + e^{i 2\pi \frac{q}{p} 2} + \cdots + e^{i 2\pi \frac{q}{p}p})                    \\
   & +
  (e^{i 2\pi \frac{q}{p}(p+1)} + e^{i 2\pi \frac{q}{p} (p+2)} + \cdots + e^{i 2\pi \frac{q}{p}2p})               \\
   & +                                                                                                           \\
   & \vdots                                                                                                      \\
   & +
  (e^{i 2\pi \frac{q}{p}[(d-1)p + 1]} + e^{i 2\pi \frac{q}{p} [(d-1)p + 2]} + \cdots + e^{i 2\pi \frac{q}{p}dp}) \\
   & = 0 + 0 + \cdots + 0                                                                                        \\
   & = 0
\end{align*}

$p = 1$时,我们有
\begin{align*}
  e^{i 2\pi \frac{q}{p}} + e^{i 2\pi \frac{q}{p} 2} + \cdots + e^{i 2\pi \frac{q}{p}n} = n
\end{align*}

由于$p = 1$,可知$n = d$,即$n$可以整除$k$($n | k$);
$p > 1$时,我们有$n$不可以整除$k$($n \nmid k$)。

当$k < 0$时,我们有
\begin{align*}
  e^{i 2\pi \frac{-k}{n}} = cos (2\pi \frac{k}{n}) - i sin (2\pi \frac{k}{n})
\end{align*}
所以,$e^{i 2\pi \frac{-k}{n}}, e^{i 2\pi \frac{k}{n}}$两者共轭,
于是
\begin{align*}
  \epsilon^k + \epsilon^{2k} + \cdots + \epsilon^{nk} & = \overline{\epsilon^{-k} + \epsilon^{-2k} + \cdots + \epsilon^{-nk}}
\end{align*}
又$\epsilon^{-k} + \epsilon^{-2k} + \cdots + \epsilon^{-nk}$都为实数,所以
\begin{align*}
  \epsilon^k + \epsilon^{2k} + \cdots + \epsilon^{nk} & = \epsilon^{-k} + \epsilon^{-2k} + \cdots + \epsilon^{-nk}
\end{align*}

当$k = 0$时,我们有
\begin{align*}
  \epsilon^k + \epsilon^{2k} + \cdots + \epsilon^{nk} = n
\end{align*}

综上
\begin{equation*}
  \epsilon^k + \epsilon^{2k} + \cdots + \epsilon^{nk} = \begin{cases*}
    0 & n | k     \\
    n & n $\nmid$ k
  \end{cases*}
\end{equation*}
\end{document}