\documentclass{article}
\usepackage{mathtools} 
\usepackage{fontspec}
\usepackage[UTF8]{ctex}
\usepackage{amsthm}
\usepackage{mdframed}
\usepackage{xcolor}
\usepackage{amssymb}
\usepackage{amsmath}


% 定义新的带灰色背景的说明环境 zremark
\newmdtheoremenv[
  backgroundcolor=gray!10,
  % 边框与背景一致,边框线会消失
  linecolor=gray!10
]{zremark}{说明}

% 通用矩阵命令: \flexmatrix{矩阵名}{元素符号}{行数}{列数}
\newcommand{\flexmatrix}[4]{
  \[
  #1 = \begin{pmatrix}
    #2_{11}     & #2_{12}     & \cdots & #2_{1#4}   \\
    #2_{21}     & #2_{22}     & \cdots & #2_{2#4}   \\
    \vdots      & \vdots      & \ddots & \vdots     \\
    #2_{#31}    & #2_{#32}    & \cdots & #2_{#3#4}
  \end{pmatrix}
  \]
}

% 简化版命令(默认矩阵名为A,元素符号为a): \quickmatrix{行数}{列数}
\newcommand{\quickmatrix}[2]{\flexmatrix{A}{a}{#1}{#2}}


\begin{document}
\title{1.1 习题}
\author{张志聪}
\maketitle

\section*{1}

\begin{itemize}
    \item (1)

          是数域,证明略。
    \item (2)

          是数域。

          设$a, b, c, d \in \mathbb{Q}$,于是
          \begin{align*}
              (a + b\sqrt{3}i) + (c + d\sqrt{3}i) = (a + c) + (b + d)\sqrt{3}i \in \mathbb{Q}(-\sqrt{3}) \\
              (a + b\sqrt{3}i) (c + d\sqrt{3}i) = (ac - 3bd) + (ac + bd)\sqrt{3}i \in \mathbb{Q}(-\sqrt{3})
          \end{align*}
          当$c + d\sqrt{3}i \neq 0$,即$c^2 + 3d^2 \neq 0$时,有
          \begin{align*}
              \frac{a + b\sqrt{3}i}{c + d\sqrt{3}i}
               & = \frac{(a + b\sqrt{3}i)(c - d\sqrt{3}i)}{c^2 + 3d^2}                           \\
               & = \frac{(ac - 3bd) + (-ad + bc)\sqrt{3}i}{c^2 + 3d^2} \in \mathbb{Q}(-\sqrt{3})
          \end{align*}
          $\mathbb{Q}(-\sqrt{3})$对复数的四则运算粉笔,所以它是一个数域。

    \item (3)

          不是数域。举一个反例。

          


\end{itemize}

\end{document}