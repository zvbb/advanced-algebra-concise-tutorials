\documentclass{article}
\usepackage{mathtools} 
\usepackage{fontspec}
\usepackage[UTF8]{ctex}
\usepackage{amsthm}
\usepackage{mdframed}
\usepackage{xcolor}
\usepackage{amssymb}
\usepackage{amsmath}
\usepackage{hyperref}


% 定义新的带灰色背景的说明环境 zremark
\newmdtheoremenv[
  backgroundcolor=gray!10,
  % 边框与背景一致,边框线会消失
  linecolor=gray!10
]{zremark}{注释}

% 通用矩阵命令: \flexmatrix{矩阵名}{元素符号}{行数}{列数}
\newcommand{\flexmatrix}[4]{
  \[
  #1 = \begin{pmatrix}
    #2_{11}     & #2_{12}     & \cdots & #2_{1#4}   \\
    #2_{21}     & #2_{22}     & \cdots & #2_{2#4}   \\
    \vdots      & \vdots      & \ddots & \vdots     \\
    #2_{#31}    & #2_{#32}    & \cdots & #2_{#3#4}
  \end{pmatrix}
  \]
}

% 简化版命令(默认矩阵名为A,元素符号为a): \quickmatrix{行数}{列数}
\newcommand{\quickmatrix}[2]{\flexmatrix{A}{a}{#1}{#2}}


\begin{document}
\title{第一章 总结}
\author{张志聪}
\maketitle

\begin{zremark}
  扩展出的知识点:\\
  第一节
  \begin{itemize}
    \item 常见数域间(不一定是标准的)的数量关系(基数关系)

          \begin{align*}
            \mathbb{N} = \mathbb{Z} = \mathbb{Q} < \mathbb{R} = \mathbb{C}
          \end{align*}

  \end{itemize}
  第二节
  \begin{itemize}
    \item 扩展的方式构造最小数域($\S2$习题1)
    \item “综合除法”得到的$n-1$多项式的系数,不会超出原始数域($\S2$习题3)
  \end{itemize}

  第三节
  \begin{itemize}
    \item 求线性方程组的解, 可用的方法。
          \begin{itemize}
            \item 先矩阵消元,再求解。
            \item 一开始,就把某个未知量作为自由未知量,通过已有的方程用自由未知量表示出其他未知量;
          \end{itemize}
  \end{itemize}
\end{zremark}


\end{document}